% This file was generated by Nim.
% Generated: 2020-08-22 06:37:24 UTC
\documentclass[a4paper]{article}
\usepackage[left=2cm,right=3cm,top=3cm,bottom=3cm]{geometry}
\usepackage[utf8]{inputenc}
\usepackage[T1]{fontenc}
\usepackage{graphicx}
\usepackage{lmodern}
\usepackage{fancyvrb, courier}
\usepackage{tabularx}
\usepackage{hyperref}

\begin{document}
\title{Hot code reloading }
\author{}

\tolerance 1414 
\hbadness 1414 
\emergencystretch 1.5em 
\hfuzz 0.3pt 
\widowpenalty=10000 
\vfuzz \hfuzz 
\raggedbottom 

\maketitle

\newenvironment{rstpre}{\VerbatimEnvironment\begingroup\begin{Verbatim}[fontsize=\footnotesize , commandchars=\\\{\}]}{\end{Verbatim}\endgroup}

% to pack tabularx into a new environment, special syntax is needed :-(
\newenvironment{rsttab}[1]{\tabularx{\linewidth}{#1}}{\endtabularx}

\newcommand{\rstsub}[1]{\raisebox{-0.5ex}{\scriptsize{#1}}}
\newcommand{\rstsup}[1]{\raisebox{0.5ex}{\scriptsize{#1}}}

\newcommand{\rsthA}[1]{\section{#1}}
\newcommand{\rsthB}[1]{\subsection{#1}}
\newcommand{\rsthC}[1]{\subsubsection{#1}}
\newcommand{\rsthD}[1]{\paragraph{#1}}
\newcommand{\rsthE}[1]{\paragraph{#1}}

\newcommand{\rstovA}[1]{\section*{#1}}
\newcommand{\rstovB}[1]{\subsection*{#1}}
\newcommand{\rstovC}[1]{\subsubsection*{#1}}
\newcommand{\rstovD}[1]{\paragraph*{#1}}
\newcommand{\rstovE}[1]{\paragraph*{#1}}

% Syntax highlighting:
\newcommand{\spanDecNumber}[1]{#1}
\newcommand{\spanBinNumber}[1]{#1}
\newcommand{\spanHexNumber}[1]{#1}
\newcommand{\spanOctNumber}[1]{#1}
\newcommand{\spanFloatNumber}[1]{#1}
\newcommand{\spanIdentifier}[1]{#1}
\newcommand{\spanKeyword}[1]{\textbf{#1}}
\newcommand{\spanStringLit}[1]{#1}
\newcommand{\spanLongStringLit}[1]{#1}
\newcommand{\spanCharLit}[1]{#1}
\newcommand{\spanEscapeSequence}[1]{#1}
\newcommand{\spanOperator}[1]{#1}
\newcommand{\spanPunctuation}[1]{#1}
\newcommand{\spanComment}[1]{\emph{#1}}
\newcommand{\spanLongComment}[1]{\emph{#1}}
\newcommand{\spanRegularExpression}[1]{#1}
\newcommand{\spanTagStart}[1]{#1}
\newcommand{\spanTagEnd}[1]{#1}
\newcommand{\spanKey}[1]{#1}
\newcommand{\spanValue}[1]{#1}
\newcommand{\spanRawData}[1]{#1}
\newcommand{\spanAssembler}[1]{#1}
\newcommand{\spanPreprocessor}[1]{#1}
\newcommand{\spanDirective}[1]{#1}
\newcommand{\spanCommand}[1]{#1}
\newcommand{\spanRule}[1]{#1}
\newcommand{\spanHyperlink}[1]{#1}
\newcommand{\spanLabel}[1]{#1}
\newcommand{\spanReference}[1]{#1}
\newcommand{\spanOther}[1]{#1}
\newcommand{\spantok}[1]{\frame{#1}}

The hotCodeReloading\label{hotcodereloading_1} option enables special compilation mode where changes in the code can be applied automatically to a running program. The code reloading happens at the granularity of an individual module. When a module is reloaded, any newly added global variables will be initialized, but all other top-level code appearing in the module won't be re-executed and the state of all existing global variables will be preserved.

\rsthA{Basic workflow}\label{basic-workflow}
Currently hot code reloading does not work for the main module itself, so we have to use a helper module where the major logic we want to change during development resides.

In this example we use SDL2 to create a window and we reload the logic code when \texttt{F9} is pressed. The important lines are marked with \texttt{\#***}. To install SDL2 you can use \texttt{nimble install sdl2}.

\begin{rstpre}
\spanComment{\# logic.nim}
\spanKeyword{import} \spanIdentifier{sdl2}

\spanComment{\#*** import the hotcodereloading stdlib module ***}
\spanKeyword{import} \spanIdentifier{hotcodereloading}

\spanKeyword{var} \spanIdentifier{runGame}\spanOperator{*:} \spanIdentifier{bool} \spanOperator{=} \spanIdentifier{true}
\spanKeyword{var} \spanIdentifier{window}\spanPunctuation{:} \spanIdentifier{WindowPtr}
\spanKeyword{var} \spanIdentifier{renderer}\spanPunctuation{:} \spanIdentifier{RendererPtr}
\spanKeyword{var} \spanIdentifier{evt} \spanOperator{=} \spanIdentifier{sdl2}\spanOperator{.}\spanIdentifier{defaultEvent}

\spanKeyword{proc} \spanIdentifier{init}\spanOperator{*}\spanPunctuation{(}\spanPunctuation{)} \spanOperator{=}
  \spanKeyword{discard} \spanIdentifier{sdl2}\spanOperator{.}\spanIdentifier{init}\spanPunctuation{(}\spanIdentifier{INIT\_EVERYTHING}\spanPunctuation{)}
  \spanIdentifier{window} \spanOperator{=} \spanIdentifier{createWindow}\spanPunctuation{(}\spanStringLit{"testing"}\spanPunctuation{,} \spanIdentifier{SDL\_WINDOWPOS\_UNDEFINED}\spanOperator{.}\spanIdentifier{cint}\spanPunctuation{,} \spanIdentifier{SDL\_WINDOWPOS\_UNDEFINED}\spanOperator{.}\spanIdentifier{cint}\spanPunctuation{,} \spanDecNumber{640}\spanPunctuation{,} \spanDecNumber{480}\spanPunctuation{,} \spanDecNumber{0'}\spanIdentifier{u32}\spanPunctuation{)}
  \spanIdentifier{assert}\spanPunctuation{(}\spanIdentifier{window} \spanOperator{!=} \spanKeyword{nil}\spanPunctuation{,} \spanOperator{\$}\spanIdentifier{sdl2}\spanOperator{.}\spanIdentifier{getError}\spanPunctuation{(}\spanPunctuation{)}\spanPunctuation{)}
  \spanIdentifier{renderer} \spanOperator{=} \spanIdentifier{createRenderer}\spanPunctuation{(}\spanIdentifier{window}\spanPunctuation{,} \spanOperator{-}\spanDecNumber{1}\spanPunctuation{,} \spanIdentifier{RENDERER\_SOFTWARE}\spanPunctuation{)}
  \spanIdentifier{assert}\spanPunctuation{(}\spanIdentifier{renderer} \spanOperator{!=} \spanKeyword{nil}\spanPunctuation{,} \spanOperator{\$}\spanIdentifier{sdl2}\spanOperator{.}\spanIdentifier{getError}\spanPunctuation{(}\spanPunctuation{)}\spanPunctuation{)}

\spanKeyword{proc} \spanIdentifier{destroy}\spanOperator{*}\spanPunctuation{(}\spanPunctuation{)} \spanOperator{=}
  \spanIdentifier{destroyRenderer}\spanPunctuation{(}\spanIdentifier{renderer}\spanPunctuation{)}
  \spanIdentifier{destroyWindow}\spanPunctuation{(}\spanIdentifier{window}\spanPunctuation{)}

\spanKeyword{var} \spanIdentifier{posX}\spanPunctuation{:} \spanIdentifier{cint} \spanOperator{=} \spanDecNumber{1}
\spanKeyword{var} \spanIdentifier{posY}\spanPunctuation{:} \spanIdentifier{cint} \spanOperator{=} \spanDecNumber{0}
\spanKeyword{var} \spanIdentifier{dX}\spanPunctuation{:} \spanIdentifier{cint} \spanOperator{=} \spanDecNumber{1}
\spanKeyword{var} \spanIdentifier{dY}\spanPunctuation{:} \spanIdentifier{cint} \spanOperator{=} \spanDecNumber{1}

\spanKeyword{proc} \spanIdentifier{update}\spanOperator{*}\spanPunctuation{(}\spanPunctuation{)} \spanOperator{=}
  \spanKeyword{while} \spanIdentifier{pollEvent}\spanPunctuation{(}\spanIdentifier{evt}\spanPunctuation{)}\spanPunctuation{:}
    \spanKeyword{if} \spanIdentifier{evt}\spanOperator{.}\spanIdentifier{kind} \spanOperator{==} \spanIdentifier{QuitEvent}\spanPunctuation{:}
      \spanIdentifier{runGame} \spanOperator{=} \spanIdentifier{false}
      \spanKeyword{break}
    \spanKeyword{if} \spanIdentifier{evt}\spanOperator{.}\spanIdentifier{kind} \spanOperator{==} \spanIdentifier{KeyDown}\spanPunctuation{:}
      \spanKeyword{if} \spanIdentifier{evt}\spanOperator{.}\spanIdentifier{key}\spanOperator{.}\spanIdentifier{keysym}\spanOperator{.}\spanIdentifier{scancode} \spanOperator{==} \spanIdentifier{SDL\_SCANCODE\_ESCAPE}\spanPunctuation{:} \spanIdentifier{runGame} \spanOperator{=} \spanIdentifier{false}
      \spanKeyword{elif} \spanIdentifier{evt}\spanOperator{.}\spanIdentifier{key}\spanOperator{.}\spanIdentifier{keysym}\spanOperator{.}\spanIdentifier{scancode} \spanOperator{==} \spanIdentifier{SDL\_SCANCODE\_F9}\spanPunctuation{:}
        \spanComment{\#*** reload this logic.nim module on the F9 keypress ***}
        \spanIdentifier{performCodeReload}\spanPunctuation{(}\spanPunctuation{)}
  
  \spanComment{\# draw a bouncing rectangle:}
  \spanIdentifier{posX} \spanOperator{+=} \spanIdentifier{dX}
  \spanIdentifier{posY} \spanOperator{+=} \spanIdentifier{dY}
  
  \spanKeyword{if} \spanIdentifier{posX} \spanOperator{>=} \spanDecNumber{640}\spanPunctuation{:} \spanIdentifier{dX} \spanOperator{=} \spanOperator{-}\spanDecNumber{2}
  \spanKeyword{if} \spanIdentifier{posX} \spanOperator{<=} \spanDecNumber{0}\spanPunctuation{:} \spanIdentifier{dX} \spanOperator{=} \spanOperator{+}\spanDecNumber{2}
  \spanKeyword{if} \spanIdentifier{posY} \spanOperator{>=} \spanDecNumber{480}\spanPunctuation{:} \spanIdentifier{dY} \spanOperator{=} \spanOperator{-}\spanDecNumber{2}
  \spanKeyword{if} \spanIdentifier{posY} \spanOperator{<=} \spanDecNumber{0}\spanPunctuation{:} \spanIdentifier{dY} \spanOperator{=} \spanOperator{+}\spanDecNumber{2}
  
  \spanKeyword{discard} \spanIdentifier{renderer}\spanOperator{.}\spanIdentifier{setDrawColor}\spanPunctuation{(}\spanDecNumber{0}\spanPunctuation{,} \spanDecNumber{0}\spanPunctuation{,} \spanDecNumber{255}\spanPunctuation{,} \spanDecNumber{255}\spanPunctuation{)}
  \spanKeyword{discard} \spanIdentifier{renderer}\spanOperator{.}\spanIdentifier{clear}\spanPunctuation{(}\spanPunctuation{)}
  \spanKeyword{discard} \spanIdentifier{renderer}\spanOperator{.}\spanIdentifier{setDrawColor}\spanPunctuation{(}\spanDecNumber{255}\spanPunctuation{,} \spanDecNumber{128}\spanPunctuation{,} \spanDecNumber{128}\spanPunctuation{,} \spanDecNumber{0}\spanPunctuation{)}
  
  \spanKeyword{var} \spanIdentifier{rect} \spanOperator{=} \spanIdentifier{Rect}\spanPunctuation{(}\spanIdentifier{x}\spanPunctuation{:} \spanIdentifier{posX} \spanOperator{-} \spanDecNumber{25}\spanPunctuation{,} \spanIdentifier{y}\spanPunctuation{:} \spanIdentifier{posY} \spanOperator{-} \spanDecNumber{25}\spanPunctuation{,} \spanIdentifier{w}\spanPunctuation{:} \spanFloatNumber{50.}\spanIdentifier{cint}\spanPunctuation{,} \spanIdentifier{h}\spanPunctuation{:} \spanFloatNumber{50.}\spanIdentifier{cint}\spanPunctuation{)}
  \spanKeyword{discard} \spanIdentifier{renderer}\spanOperator{.}\spanIdentifier{fillRect}\spanPunctuation{(}\spanIdentifier{rect}\spanPunctuation{)}
  \spanIdentifier{delay}\spanPunctuation{(}\spanDecNumber{16}\spanPunctuation{)}
  \spanIdentifier{renderer}\spanOperator{.}\spanIdentifier{present}\spanPunctuation{(}\spanPunctuation{)}
\end{rstpre}
\begin{rstpre}
\spanComment{\# mymain.nim}
\spanKeyword{import} \spanIdentifier{logic}

\spanKeyword{proc} \spanIdentifier{main}\spanPunctuation{(}\spanPunctuation{)} \spanOperator{=}
  \spanIdentifier{init}\spanPunctuation{(}\spanPunctuation{)}
  \spanKeyword{while} \spanIdentifier{runGame}\spanPunctuation{:}
    \spanIdentifier{update}\spanPunctuation{(}\spanPunctuation{)}
  \spanIdentifier{destroy}\spanPunctuation{(}\spanPunctuation{)}

\spanIdentifier{main}\spanPunctuation{(}\spanPunctuation{)}
\end{rstpre}
Compile this example via:\begin{rstpre}

nim c --hotcodereloading:on mymain.nim
\end{rstpre}


Now start the program and KEEP it running!

\begin{rstpre}

\# Unix:
mymain \&
\# or Windows (click on the .exe)
mymain.exe
\# edit
\end{rstpre}
For example, change the line:\begin{rstpre}

discard renderer.setDrawColor(255, 128, 128, 0)
\end{rstpre}


into:\begin{rstpre}

discard renderer.setDrawColor(255, 255, 128, 0)
\end{rstpre}


(This will change the color of the rectangle.)

Then recompile the project, but do not restart or quit the mymain.exe program!

\begin{rstpre}

nim c --hotcodereloading:on mymain.nim
\end{rstpre}
Now give the \texttt{mymain} SDL window the focus, press F9 and watch the updated version of the program.

\rsthA{Reloading API}\label{reloading-api}
One can use the special event handlers \texttt{beforeCodeReload} and \texttt{afterCodeReload} to reset the state of a particular variable or to force the execution of certain statements:

\begin{rstpre}
\spanKeyword{var}
 \spanIdentifier{settings} \spanOperator{=} \spanIdentifier{initTable}\spanPunctuation{\symbol{91}}\spanIdentifier{string}\spanPunctuation{,} \spanIdentifier{string}\spanPunctuation{\symbol{93}}\spanPunctuation{(}\spanPunctuation{)}
 \spanIdentifier{lastReload}\spanPunctuation{:} \spanIdentifier{Time}

\spanKeyword{for} \spanIdentifier{k}\spanPunctuation{,} \spanIdentifier{v} \spanKeyword{in} \spanIdentifier{loadSettings}\spanPunctuation{(}\spanPunctuation{)}\spanPunctuation{:}
  \spanIdentifier{settings}\spanPunctuation{\symbol{91}}\spanIdentifier{k}\spanPunctuation{\symbol{93}} \spanOperator{=} \spanIdentifier{v}

\spanIdentifier{initProgram}\spanPunctuation{(}\spanPunctuation{)}

\spanIdentifier{afterCodeReload}\spanPunctuation{:}
  \spanIdentifier{lastReload} \spanOperator{=} \spanIdentifier{now}\spanPunctuation{(}\spanPunctuation{)}
  \spanIdentifier{resetProgramState}\spanPunctuation{(}\spanPunctuation{)}
\end{rstpre}
On each code reload, Nim will first execute all beforeCodeReload\label{beforecodereload_1} handlers registered in the previous version of the program and then all afterCodeReload\label{aftercodereload_1} handlers appearing in the newly loaded code. Please note that any handlers appearing in modules that weren't reloaded will also be executed. To prevent this behavior, one can guard the code with the hasModuleChanged()\label{hasmodulechanged_1} API:

\begin{rstpre}
\spanKeyword{import} \spanIdentifier{mydb}

\spanKeyword{var} \spanIdentifier{myCache} \spanOperator{=} \spanIdentifier{initTable}\spanPunctuation{\symbol{91}}\spanIdentifier{Key}\spanPunctuation{,} \spanIdentifier{Value}\spanPunctuation{\symbol{93}}\spanPunctuation{(}\spanPunctuation{)}

\spanIdentifier{afterCodeReload}\spanPunctuation{:}
  \spanKeyword{if} \spanIdentifier{hasModuleChanged}\spanPunctuation{(}\spanIdentifier{mydb}\spanPunctuation{)}\spanPunctuation{:}
    \spanIdentifier{resetCache}\spanPunctuation{(}\spanIdentifier{myCache}\spanPunctuation{)}
\end{rstpre}
The hot code reloading is based on dynamic library hot swapping in the native targets and direct manipulation of the global namespace in the JavaScript target. The Nim compiler does not specify the mechanism for detecting the conditions when the code must be reloaded. Instead, the program code is expected to call performCodeReload()\label{performcodereload_1} every time it wishes to reload its code.

It's expected that most projects will implement the reloading with a suitable build-system triggered IPC notification mechanism, but a polling solution is also possible through the provided hasAnyModuleChanged()\label{hasanymodulechanged_1} API.

In order to access \texttt{beforeCodeReload}, \texttt{afterCodeReload}, \texttt{hasModuleChanged} or \texttt{hasAnyModuleChanged} one must import the hotcodereloading\label{hotcodereloading_2} module.

\rsthA{Native code targets}\label{native-code-targets}
Native projects using the hot code reloading option will be implicitly compiled with the \texttt{-d:useNimRtl} option and they will depend on both the \texttt{nimrtl} library and the \texttt{nimhcr} library which implements the hot code reloading run-time. Both libraries can be found in the \texttt{lib} folder of Nim and can be compiled into dynamic libraries to satisfy runtime demands of the example code above. An example of compiling \texttt{nimhcr.nim} and \texttt{nimrtl.nim} when the source dir of Nim is installed with choosenim follows.

\begin{rstpre}

\# Unix/MacOS
\# Make sure you are in the directory containing your .nim files
\$ cd your-source-directory

\# Compile two required files and set their output directory to current dir
\$ nim c --outdir:\$PWD \symbol{126}/.choosenim/toolchains/nim-\#devel/lib/nimhcr.nim
\$ nim c --outdir:\$PWD \symbol{126}/.choosenim/toolchains/nim-\#devel/lib/nimrtl.nim

\# verify that you have two files named libnimhcr and libnimrtl in your
\# source directory (.dll for Windows, .so for Unix, .dylib for MacOS)
\end{rstpre}
All modules of the project will be compiled to separate dynamic link libraries placed in the \texttt{nimcache} directory. Please note that during the execution of the program, the hot code reloading run-time will load only copies of these libraries in order to not interfere with any newly issued build commands.

The main module of the program is considered non-reloadable. Please note that procs from reloadable modules should not appear in the call stack of program while \texttt{performCodeReload} is being called. Thus, the main module is a suitable place for implementing a program loop capable of calling \texttt{performCodeReload}.

Please note that reloading won't be possible when any of the type definitions in the program has been changed. When closure iterators are used (directly or through async code), the reloaded definitions will affect only newly created instances. Existing iterator instances will execute their original code to completion.

\rsthA{JavaScript target}\label{javascript-target}
Once your code is compiled for hot reloading, a convenient solution for implementing the actual reloading in the browser using a framework such as \href{http://livereload.com/}{LiveReload} or \href{https://browsersync.io/}{BrowserSync}. 




\end{document}
