% This file was generated by Nim.
% Generated: 2020-08-22 06:37:34 UTC
\documentclass[a4paper]{article}
\usepackage[left=2cm,right=3cm,top=3cm,bottom=3cm]{geometry}
\usepackage[utf8]{inputenc}
\usepackage[T1]{fontenc}
\usepackage{graphicx}
\usepackage{lmodern}
\usepackage{fancyvrb, courier}
\usepackage{tabularx}
\usepackage{hyperref}

\begin{document}
\title{niminst User's manual 1.3.5}
\author{Andreas Rumpf}

\tolerance 1414 
\hbadness 1414 
\emergencystretch 1.5em 
\hfuzz 0.3pt 
\widowpenalty=10000 
\vfuzz \hfuzz 
\raggedbottom 

\maketitle

\newenvironment{rstpre}{\VerbatimEnvironment\begingroup\begin{Verbatim}[fontsize=\footnotesize , commandchars=\\\{\}]}{\end{Verbatim}\endgroup}

% to pack tabularx into a new environment, special syntax is needed :-(
\newenvironment{rsttab}[1]{\tabularx{\linewidth}{#1}}{\endtabularx}

\newcommand{\rstsub}[1]{\raisebox{-0.5ex}{\scriptsize{#1}}}
\newcommand{\rstsup}[1]{\raisebox{0.5ex}{\scriptsize{#1}}}

\newcommand{\rsthA}[1]{\section{#1}}
\newcommand{\rsthB}[1]{\subsection{#1}}
\newcommand{\rsthC}[1]{\subsubsection{#1}}
\newcommand{\rsthD}[1]{\paragraph{#1}}
\newcommand{\rsthE}[1]{\paragraph{#1}}

\newcommand{\rstovA}[1]{\section*{#1}}
\newcommand{\rstovB}[1]{\subsection*{#1}}
\newcommand{\rstovC}[1]{\subsubsection*{#1}}
\newcommand{\rstovD}[1]{\paragraph*{#1}}
\newcommand{\rstovE}[1]{\paragraph*{#1}}

% Syntax highlighting:
\newcommand{\spanDecNumber}[1]{#1}
\newcommand{\spanBinNumber}[1]{#1}
\newcommand{\spanHexNumber}[1]{#1}
\newcommand{\spanOctNumber}[1]{#1}
\newcommand{\spanFloatNumber}[1]{#1}
\newcommand{\spanIdentifier}[1]{#1}
\newcommand{\spanKeyword}[1]{\textbf{#1}}
\newcommand{\spanStringLit}[1]{#1}
\newcommand{\spanLongStringLit}[1]{#1}
\newcommand{\spanCharLit}[1]{#1}
\newcommand{\spanEscapeSequence}[1]{#1}
\newcommand{\spanOperator}[1]{#1}
\newcommand{\spanPunctuation}[1]{#1}
\newcommand{\spanComment}[1]{\emph{#1}}
\newcommand{\spanLongComment}[1]{\emph{#1}}
\newcommand{\spanRegularExpression}[1]{#1}
\newcommand{\spanTagStart}[1]{#1}
\newcommand{\spanTagEnd}[1]{#1}
\newcommand{\spanKey}[1]{#1}
\newcommand{\spanValue}[1]{#1}
\newcommand{\spanRawData}[1]{#1}
\newcommand{\spanAssembler}[1]{#1}
\newcommand{\spanPreprocessor}[1]{#1}
\newcommand{\spanDirective}[1]{#1}
\newcommand{\spanCommand}[1]{#1}
\newcommand{\spanRule}[1]{#1}
\newcommand{\spanHyperlink}[1]{#1}
\newcommand{\spanLabel}[1]{#1}
\newcommand{\spanReference}[1]{#1}
\newcommand{\spanOther}[1]{#1}
\newcommand{\spantok}[1]{\frame{#1}}

\tableofcontents \newpage
\rsthA{Introduction}\label{introduction}
niminst is a tool to generate an installer for a Nim program. Currently it can create an installer for Windows via \href{http://www.jrsoftware.org/isinfo.php}{Inno Setup} as well as installation/deinstallation scripts for UNIX. Later versions will support Linux' package management systems.

niminst works by reading a configuration file that contains all the information that it needs to generate an installer for the different operating systems.

\rsthA{Configuration file}\label{configuration-file}
niminst uses the Nim \href{parsecfg.html}{parsecfg} module to parse the configuration file. Here's an example of how the syntax looks like:

\begin{rstpre}
\# This is a comment.
; this too.

\symbol{91}Common\symbol{93}
cc=gcc     \# '=' and ':' are the same
--foo="bar"   \# '--cc' and 'cc' are the same, 'bar' and '"bar"' are the same (except for '\#')
macrosym: "\#"  \# Note that '\#' is interpreted as a comment without the quotation
--verbose

\symbol{91}Windows\symbol{93}
isConsoleApplication=False ; another comment

\symbol{91}Posix\symbol{93}
isConsoleApplication=True

key1: "in this string backslash escapes are interpreted\symbol{92}n"
key2: r"in this string not"
key3: """triple quotes strings
are also supported. They may span
multiple lines."""

--"long option with spaces": r"c:\symbol{92}myfiles\symbol{92}test.txt" 

\end{rstpre}
The value of a key-value pair can reference user-defined variables via the \texttt{\$variable} notation: They can be defined in the command line with the \texttt{--var:name=value} switch. This is useful to not hard-coding the program's version number into the configuration file, for instance.

It follows a description of each possible section and how it affects the generated installers.

\rsthB{Project section}\label{configuration-file-project-section}
The project section gathers general information about your project. It must contain the following key-value pairs:

\begin{table}\begin{rsttab}{|X|X|X|X|X|X|X|X|X|X|}
\hline
\textbf{Key} & \textbf{description}\\
\hline
\texttt{Name} & the project's name; this needs to be a single word\\
\hline
\texttt{DisplayName} & the project's long name; this can contain spaces. If not specified, this is the same as \texttt{Name}.\\
\hline
\texttt{Version} & the project's version\\
\hline
\texttt{OS} & the OSes to generate C code for; for example: \texttt{"windows;linux;macosx"}\\
\hline
\texttt{CPU} & the CPUs to generate C code for; for example: \texttt{"i386;amd64;powerpc"}\\
\hline
\texttt{Authors} & the project's authors\\
\hline
\texttt{Description} & the project's description\\
\hline
\texttt{App} & the application's type: "Console" or "GUI". If "Console", niminst generates a special batch file for Windows to open up the command line shell.\\
\hline
\texttt{License} & the filename of the application's license\\
\hline
\end{rsttab}\end{table}\rsthB{\texttt{files} key}\label{configuration-file-files-key}
Many sections support the \texttt{files} key. Listed filenames can be separated by semicolon or the \texttt{files} key can be repeated. Wildcards in filenames are supported. If it is a directory name, all files in the directory are used:\begin{rstpre}

\symbol{91}Config\symbol{93}
Files: "configDir"
Files: "otherconfig/*.conf;otherconfig/*.cfg"
\end{rstpre}


\rsthB{Config section}\label{configuration-file-config-section}
The \texttt{config} section currently only supports the \texttt{files} key. Listed files will be installed into the OS's configuration directory.

\rsthB{Documentation section}\label{configuration-file-documentation-section}
The \texttt{documentation} section supports the \texttt{files} key. Listed files will be installed into the OS's native documentation directory (which might be \texttt{\$appdir/doc}).

There is a \texttt{start} key which determines whether the Windows installer generates a link to e.g. the \texttt{index.html} of your documentation.

\rsthB{Other section}\label{configuration-file-other-section}
The \texttt{other} section currently only supports the \texttt{files} key. Listed files will be installed into the application installation directory (\texttt{\$appdir}).

\rsthB{Lib section}\label{configuration-file-lib-section}
The \texttt{lib} section currently only supports the \texttt{files} key. Listed files will be installed into the OS's native library directory (which might be \texttt{\$appdir/lib}).

\rsthB{Windows section}\label{configuration-file-windows-section}
The \texttt{windows} section supports the \texttt{files} key for Windows specific files. Listed files will be installed into the application installation directory (\texttt{\$appdir}).

Other possible options are:

\begin{table}\begin{rsttab}{|X|X|X|}
\hline
\textbf{Key} & \textbf{description}\\
\hline
\texttt{BinPath} & paths to add to the Windows \texttt{\%PATH\%} environment variable. Example: \texttt{BinPath: r"bin;dist\symbol{92}mingw\symbol{92}bin"}\\
\hline
\texttt{InnoSetup} & boolean flag whether an Inno Setup installer should be generated for Windows. Example: \texttt{InnoSetup: "Yes"}\\
\hline
\end{rsttab}\end{table}\rsthB{UnixBin section}\label{configuration-file-unixbin-section}
The \texttt{UnixBin} section currently only supports the \texttt{files} key. Listed files will be installed into the OS's native bin directory (e.g. \texttt{/usr/local/bin}). The exact location depends on the installation path the user specifies when running the \texttt{install.sh} script.

\rsthB{Unix section}\label{configuration-file-unix-section}
Possible options are:

\begin{table}\begin{rsttab}{|X|X|X|}
\hline
\textbf{Key} & \textbf{description}\\
\hline
\texttt{InstallScript} & boolean flag whether an installation shell script should be generated. Example: \texttt{InstallScript: "Yes"}\\
\hline
\texttt{UninstallScript} & boolean flag whether a deinstallation shell script should be generated. Example: \texttt{UninstallScript: "Yes"}\\
\hline
\end{rsttab}\end{table}\rsthB{InnoSetup section}\label{configuration-file-innosetup-section}
Possible options are:

\begin{table}\begin{rsttab}{|X|X|X|}
\hline
\textbf{Key} & \textbf{description}\\
\hline
\texttt{path} & Path to Inno Setup. Example: \texttt{path = r"c:\symbol{92}inno setup 5\symbol{92}iscc.exe"}\\
\hline
\texttt{flags} & Flags to pass to Inno Setup. Example: \texttt{flags = "/Q"}\\
\hline
\end{rsttab}\end{table}\rsthB{C\_Compiler section}\label{configuration-file-c-compiler-section}
Possible options are:

\begin{table}\begin{rsttab}{|X|X|X|}
\hline
\textbf{Key} & \textbf{description}\\
\hline
\texttt{path} & Path to the C compiler.\\
\hline
\texttt{flags} & Flags to pass to the C Compiler. Example: \texttt{flags = "-w"}\\
\hline
\end{rsttab}\end{table}\rsthA{Real world example}\label{real-world-example}
The installers for the Nim compiler itself are generated by niminst. Have a look at its configuration file:

\begin{rstpre}
; This config file holds configuration information about the Nim compiler
; and project.

\symbol{91}Project\symbol{93}
Name: "Nim"
Version: "\$version"
Platforms: """
  windows: i386;amd64
  linux: i386;hppa;ia64;alpha;amd64;powerpc64;arm;sparc;sparc64;m68k;mips;mipsel;mips64;mips64el;powerpc;powerpc64el;arm64;riscv64
  macosx: i386;amd64;powerpc64
  solaris: i386;amd64;sparc;sparc64
  freebsd: i386;amd64;powerpc64;arm;arm64;riscv64;sparc64;mips;mipsel;mips64;mips64el;powerpc
  netbsd: i386;amd64
  openbsd: i386;amd64;arm;arm64
  dragonfly: i386;amd64
  haiku: i386;amd64
  android: i386;arm;arm64
  nintendoswitch: arm64
"""

Authors: "Andreas Rumpf"
Description: """This is the Nim Compiler. Nim is a new statically typed,
imperative programming language, that supports procedural, functional, object
oriented and generic programming styles while remaining simple and efficient.
A special feature that Nim inherited from Lisp is that Nim's abstract
syntax tree (AST) is part of the specification - this allows a powerful macro
system which can be used to create domain specific languages.

Nim is a compiled, garbage-collected systems programming language
which has an excellent productivity/performance ratio. Nim's design
focuses on the 3E: efficiency, expressiveness, elegance (in the order of
priority)."""

App: Console
License: "copying.txt"

\symbol{91}Config\symbol{93}
Files: "config/*.cfg"
Files: "config/config.nims"

\symbol{91}Documentation\symbol{93}
; Files: "doc/*.html"
; Files: "doc/*.cfg"
; Files: "doc/*.pdf"
; Files: "doc/*.ini"
Files: "doc/html/overview.html"
Start: "doc/html/overview.html"


\symbol{91}Other\symbol{93}
Files: "copying.txt"
Files: "koch.nim"

Files: "icons/nim.ico"
Files: "icons/nim.rc"
Files: "icons/nim.res"
Files: "icons/nim\_icon.o"
Files: "icons/koch.ico"
Files: "icons/koch.rc"
Files: "icons/koch.res"
Files: "icons/koch\_icon.o"

Files: "compiler"
Files: "doc"
Files: "doc/html"
Files: "tools"
Files: "tools/nim-gdb.py"
Files: "nimpretty"
Files: "testament"
Files: "nimsuggest"
Files: "nimsuggest/tests/*.nim"

\symbol{91}Lib\symbol{93}
Files: "lib"

\symbol{91}Other\symbol{93}
Files: "examples"
Files: "dist/nimble"

Files: "tests"

\symbol{91}Windows\symbol{93}
Files: "bin/nim.exe"
Files: "bin/nimgrep.exe"
Files: "bin/nimsuggest.exe"
Files: "bin/nimble.exe"
Files: "bin/vccexe.exe"
Files: "bin/nimgrab.exe"
Files: "bin/nimpretty.exe"
Files: "bin/testament.exe"
Files: "bin/nim-gdb.bat"

Files: "koch.exe"
Files: "finish.exe"
; Files: "bin/downloader.exe"

; Files: "dist/mingw"
Files: r"tools\symbol{92}start.bat"
BinPath: r"bin;dist\symbol{92}mingw\symbol{92}bin;dist"

;           Section | dir | zipFile | size hint (in KB) | url | exe start menu entry
Download: r"Documentation|doc|docs.zip|13824|https://nim-lang.org/download/docs-\$\symbol{123}version\symbol{125}.zip|overview.html"
Download: r"C Compiler (MingW)|dist|mingw.zip|82944|https://nim-lang.org/download/\$\symbol{123}mingw\symbol{125}.zip"
Download: r"Support DLLs|bin|nim\_dlls.zip|479|https://nim-lang.org/download/dlls.zip"
Download: r"Aporia Text Editor|dist|aporia.zip|97997|https://nim-lang.org/download/aporia-0.4.0.zip|aporia-0.4.0\symbol{92}bin\symbol{92}aporia.exe"
; for now only NSIS supports optional downloads

\symbol{91}WinBin\symbol{93}
Files: "bin/makelink.exe"
Files: "bin/7zG.exe"
Files: "bin/*.dll"

\symbol{91}UnixBin\symbol{93}
Files: "bin/nim"


\symbol{91}Unix\symbol{93}
InstallScript: "yes"
UninstallScript: "yes"
Files: "bin/nim-gdb"
Files: "bin/nim-gdb.bash"


\symbol{91}InnoSetup\symbol{93}
path = r"c:\symbol{92}Program Files (x86)\symbol{92}Inno Setup 5\symbol{92}iscc.exe"
flags = "/Q"

\symbol{91}NSIS\symbol{93}
flags = "/V0"

\symbol{91}C\_Compiler\symbol{93}
path = r""
flags = "-w"


\symbol{91}deb\symbol{93}
buildDepends: "gcc (>= 4:4.3.2)"
pkgDepends: "gcc (>= 4:4.3.2)"
shortDesc: "The Nim Compiler"
licenses: "bin/nim,MIT;lib/*,MIT;"

\symbol{91}nimble\symbol{93}
pkgName: "compiler"
pkgFiles: "compiler/*;doc/basicopt.txt;doc/advopt.txt;doc/nimdoc.css"

\end{rstpre}



\end{document}
