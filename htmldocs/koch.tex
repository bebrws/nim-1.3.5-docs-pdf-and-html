% This file was generated by Nim.
% Generated: 2020-08-22 06:37:24 UTC
\documentclass[a4paper]{article}
\usepackage[left=2cm,right=3cm,top=3cm,bottom=3cm]{geometry}
\usepackage[utf8]{inputenc}
\usepackage[T1]{fontenc}
\usepackage{graphicx}
\usepackage{lmodern}
\usepackage{fancyvrb, courier}
\usepackage{tabularx}
\usepackage{hyperref}

\begin{document}
\title{Nim maintenance script 1.3.5}
\author{}

\tolerance 1414 
\hbadness 1414 
\emergencystretch 1.5em 
\hfuzz 0.3pt 
\widowpenalty=10000 
\vfuzz \hfuzz 
\raggedbottom 

\maketitle

\newenvironment{rstpre}{\VerbatimEnvironment\begingroup\begin{Verbatim}[fontsize=\footnotesize , commandchars=\\\{\}]}{\end{Verbatim}\endgroup}

% to pack tabularx into a new environment, special syntax is needed :-(
\newenvironment{rsttab}[1]{\tabularx{\linewidth}{#1}}{\endtabularx}

\newcommand{\rstsub}[1]{\raisebox{-0.5ex}{\scriptsize{#1}}}
\newcommand{\rstsup}[1]{\raisebox{0.5ex}{\scriptsize{#1}}}

\newcommand{\rsthA}[1]{\section{#1}}
\newcommand{\rsthB}[1]{\subsection{#1}}
\newcommand{\rsthC}[1]{\subsubsection{#1}}
\newcommand{\rsthD}[1]{\paragraph{#1}}
\newcommand{\rsthE}[1]{\paragraph{#1}}

\newcommand{\rstovA}[1]{\section*{#1}}
\newcommand{\rstovB}[1]{\subsection*{#1}}
\newcommand{\rstovC}[1]{\subsubsection*{#1}}
\newcommand{\rstovD}[1]{\paragraph*{#1}}
\newcommand{\rstovE}[1]{\paragraph*{#1}}

% Syntax highlighting:
\newcommand{\spanDecNumber}[1]{#1}
\newcommand{\spanBinNumber}[1]{#1}
\newcommand{\spanHexNumber}[1]{#1}
\newcommand{\spanOctNumber}[1]{#1}
\newcommand{\spanFloatNumber}[1]{#1}
\newcommand{\spanIdentifier}[1]{#1}
\newcommand{\spanKeyword}[1]{\textbf{#1}}
\newcommand{\spanStringLit}[1]{#1}
\newcommand{\spanLongStringLit}[1]{#1}
\newcommand{\spanCharLit}[1]{#1}
\newcommand{\spanEscapeSequence}[1]{#1}
\newcommand{\spanOperator}[1]{#1}
\newcommand{\spanPunctuation}[1]{#1}
\newcommand{\spanComment}[1]{\emph{#1}}
\newcommand{\spanLongComment}[1]{\emph{#1}}
\newcommand{\spanRegularExpression}[1]{#1}
\newcommand{\spanTagStart}[1]{#1}
\newcommand{\spanTagEnd}[1]{#1}
\newcommand{\spanKey}[1]{#1}
\newcommand{\spanValue}[1]{#1}
\newcommand{\spanRawData}[1]{#1}
\newcommand{\spanAssembler}[1]{#1}
\newcommand{\spanPreprocessor}[1]{#1}
\newcommand{\spanDirective}[1]{#1}
\newcommand{\spanCommand}[1]{#1}
\newcommand{\spanRule}[1]{#1}
\newcommand{\spanHyperlink}[1]{#1}
\newcommand{\spanLabel}[1]{#1}
\newcommand{\spanReference}[1]{#1}
\newcommand{\spanOther}[1]{#1}
\newcommand{\spantok}[1]{\frame{#1}}

\tableofcontents \newpage
\rsthA{Introduction}\label{introduction}
The koch\label{koch_1} program is Nim's maintenance script. It is a replacement for make and shell scripting with the advantage that it is much more portable. The word \emph{koch} means \emph{cook} in German. \texttt{koch} is used mainly to build the Nim compiler, but it can also be used for other tasks. This document describes the supported commands and their options.

\rsthA{Commands}\label{commands}
\rsthB{boot command}\label{commands-boot-command}
The boot\label{boot_1} command bootstraps the compiler, and it accepts different options:

\begin{description}\item[-d:release] By default a debug version is created, passing this option will force a release build, which is much faster and should be preferred unless you are debugging the compiler.
\item[-d:nimUseLinenoise] Use the linenoise library for interactive mode (not needed on Windows).
\end{description}
After compilation is finished you will hopefully end up with the nim compiler in the \texttt{bin} directory. You can add Nim's \texttt{bin} directory to your \texttt{\$PATH} or use the install command to place it where it will be found.

\rsthB{csource command}\label{commands-csource-command}
The csource\label{csource_1} command builds the C sources for installation. It accepts the same options as you would pass to the \href{\#commands-boot-command}{boot command}.

\rsthB{temp command}\label{commands-temp-command}
The temp command builds the Nim compiler but with a different final name (\texttt{nim\_temp}), so it doesn't overwrite your normal compiler. You can use this command to test different options, the same you would issue for the \href{\#commands-boot-command}{boot command}.

\rsthB{test command}\label{commands-test-command}
The test\label{test_1} command can also be invoked with the alias \texttt{tests}. This command will compile and run \texttt{testament/tester.nim}, which is the main driver of Nim's test suite. You can pass options to the \texttt{test} command, they will be forwarded to the tester. See its source code for available options.

\rsthB{web command}\label{commands-web-command}
The web\label{web_1} command converts the documentation in the \texttt{doc} directory from rst to HTML. It also repeats the same operation but places the result in the \texttt{web/upload} which can be used to update the website at \href{https://nim-lang.org}{https://nim-lang.org}.

By default the documentation will be built in parallel using the number of available CPU cores. If any documentation build sub commands fail, they will be rerun in serial fashion so that meaningful error output can be gathered for inspection. The \texttt{--parallelBuild:n} switch or configuration option can be used to force a specific number of parallel jobs or run everything serially from the start (\texttt{n == 1}). 




\end{document}
