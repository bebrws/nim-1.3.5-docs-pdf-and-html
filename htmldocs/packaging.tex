% This file was generated by Nim.
% Generated: 2020-08-22 06:37:34 UTC
\documentclass[a4paper]{article}
\usepackage[left=2cm,right=3cm,top=3cm,bottom=3cm]{geometry}
\usepackage[utf8]{inputenc}
\usepackage[T1]{fontenc}
\usepackage{graphicx}
\usepackage{lmodern}
\usepackage{fancyvrb, courier}
\usepackage{tabularx}
\usepackage{hyperref}

\begin{document}
\title{Packaging Nim }
\author{}

\tolerance 1414 
\hbadness 1414 
\emergencystretch 1.5em 
\hfuzz 0.3pt 
\widowpenalty=10000 
\vfuzz \hfuzz 
\raggedbottom 

\maketitle

\newenvironment{rstpre}{\VerbatimEnvironment\begingroup\begin{Verbatim}[fontsize=\footnotesize , commandchars=\\\{\}]}{\end{Verbatim}\endgroup}

% to pack tabularx into a new environment, special syntax is needed :-(
\newenvironment{rsttab}[1]{\tabularx{\linewidth}{#1}}{\endtabularx}

\newcommand{\rstsub}[1]{\raisebox{-0.5ex}{\scriptsize{#1}}}
\newcommand{\rstsup}[1]{\raisebox{0.5ex}{\scriptsize{#1}}}

\newcommand{\rsthA}[1]{\section{#1}}
\newcommand{\rsthB}[1]{\subsection{#1}}
\newcommand{\rsthC}[1]{\subsubsection{#1}}
\newcommand{\rsthD}[1]{\paragraph{#1}}
\newcommand{\rsthE}[1]{\paragraph{#1}}

\newcommand{\rstovA}[1]{\section*{#1}}
\newcommand{\rstovB}[1]{\subsection*{#1}}
\newcommand{\rstovC}[1]{\subsubsection*{#1}}
\newcommand{\rstovD}[1]{\paragraph*{#1}}
\newcommand{\rstovE}[1]{\paragraph*{#1}}

% Syntax highlighting:
\newcommand{\spanDecNumber}[1]{#1}
\newcommand{\spanBinNumber}[1]{#1}
\newcommand{\spanHexNumber}[1]{#1}
\newcommand{\spanOctNumber}[1]{#1}
\newcommand{\spanFloatNumber}[1]{#1}
\newcommand{\spanIdentifier}[1]{#1}
\newcommand{\spanKeyword}[1]{\textbf{#1}}
\newcommand{\spanStringLit}[1]{#1}
\newcommand{\spanLongStringLit}[1]{#1}
\newcommand{\spanCharLit}[1]{#1}
\newcommand{\spanEscapeSequence}[1]{#1}
\newcommand{\spanOperator}[1]{#1}
\newcommand{\spanPunctuation}[1]{#1}
\newcommand{\spanComment}[1]{\emph{#1}}
\newcommand{\spanLongComment}[1]{\emph{#1}}
\newcommand{\spanRegularExpression}[1]{#1}
\newcommand{\spanTagStart}[1]{#1}
\newcommand{\spanTagEnd}[1]{#1}
\newcommand{\spanKey}[1]{#1}
\newcommand{\spanValue}[1]{#1}
\newcommand{\spanRawData}[1]{#1}
\newcommand{\spanAssembler}[1]{#1}
\newcommand{\spanPreprocessor}[1]{#1}
\newcommand{\spanDirective}[1]{#1}
\newcommand{\spanCommand}[1]{#1}
\newcommand{\spanRule}[1]{#1}
\newcommand{\spanHyperlink}[1]{#1}
\newcommand{\spanLabel}[1]{#1}
\newcommand{\spanReference}[1]{#1}
\newcommand{\spanOther}[1]{#1}
\newcommand{\spantok}[1]{\frame{#1}}

This page provide hints on distributing Nim using OS packages.

See \href{distros.html}{distros} for tools to detect Linux distribution at runtime.

\rsthA{Supported architectures}\label{supported-architectures}
Nim runs on a wide variety of platforms. Support on amd64 and i386 is tested regularly, while less popular platforms are tested by the community.

\begin{itemize}\item amd64
\item arm64 (aka aarch64)
\item armel
\item armhf
\item i386
\item m68k
\item mips64el
\item mipsel
\item powerpc
\item ppc64
\item ppc64el (aka ppc64le)
\item riscv64
\end{itemize}
The following platforms are seldomly tested:

\begin{itemize}\item alpha
\item hppa
\item ia64
\item mips
\item s390x
\item sparc64
\end{itemize}
\rsthA{Packaging for Linux}\label{packaging-for-linux}
See \href{https://github.com/nim-lang/Nim/labels/Installation}{https://github.com/nim-lang/Nim/labels/Installation} for installation-related bugs.

Build Nim from the released tarball at \href{https://nim-lang.org/install\_unix.html}{https://nim-lang.org/install\_unix.html} It is different from the GitHub sources as it contains Nimble, C sources \& other tools.

The Debian package ships bash and ksh completion and manpages that can be reused.

Hints on the build process:

\begin{rstpre}

\# build from C sources and then using koch
./build.sh --os \$os\_type --cpu \$cpu\_arch
./bin/nim c koch
./koch boot -d:release

\# optionally generate docs into doc/html
./koch docs

./koch tools -d:release

\# extract files to be really installed
./install.sh <tempdir>

\# also include the tools
for fn in nimble nimsuggest nimgrep; do cp ./bin/\$fn <tempdir>/nim/bin/; done
\end{rstpre}
What to install:

\begin{itemize}\item The expected stdlib location is /usr/lib/nim
\item Global configuration files under /etc/nim
\item Optionally: manpages, documentation, shell completion
\item When installing documentation, .idx files are not required
\item The "compiler" directory contains compiler sources and should not be part of the compiler binary package
\end{itemize}



\end{document}
