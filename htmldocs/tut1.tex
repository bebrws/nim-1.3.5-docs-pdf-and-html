% This file was generated by Nim.
% Generated: 2020-08-22 06:37:46 UTC
\documentclass[a4paper]{article}
\usepackage[left=2cm,right=3cm,top=3cm,bottom=3cm]{geometry}
\usepackage[utf8]{inputenc}
\usepackage[T1]{fontenc}
\usepackage{graphicx}
\usepackage{lmodern}
\usepackage{fancyvrb, courier}
\usepackage{tabularx}
\usepackage{hyperref}

\begin{document}
\title{Nim Tutorial (Part I) 1.3.5}
\author{Andreas Rumpf}

\tolerance 1414 
\hbadness 1414 
\emergencystretch 1.5em 
\hfuzz 0.3pt 
\widowpenalty=10000 
\vfuzz \hfuzz 
\raggedbottom 

\maketitle

\newenvironment{rstpre}{\VerbatimEnvironment\begingroup\begin{Verbatim}[fontsize=\footnotesize , commandchars=\\\{\}]}{\end{Verbatim}\endgroup}

% to pack tabularx into a new environment, special syntax is needed :-(
\newenvironment{rsttab}[1]{\tabularx{\linewidth}{#1}}{\endtabularx}

\newcommand{\rstsub}[1]{\raisebox{-0.5ex}{\scriptsize{#1}}}
\newcommand{\rstsup}[1]{\raisebox{0.5ex}{\scriptsize{#1}}}

\newcommand{\rsthA}[1]{\section{#1}}
\newcommand{\rsthB}[1]{\subsection{#1}}
\newcommand{\rsthC}[1]{\subsubsection{#1}}
\newcommand{\rsthD}[1]{\paragraph{#1}}
\newcommand{\rsthE}[1]{\paragraph{#1}}

\newcommand{\rstovA}[1]{\section*{#1}}
\newcommand{\rstovB}[1]{\subsection*{#1}}
\newcommand{\rstovC}[1]{\subsubsection*{#1}}
\newcommand{\rstovD}[1]{\paragraph*{#1}}
\newcommand{\rstovE}[1]{\paragraph*{#1}}

% Syntax highlighting:
\newcommand{\spanDecNumber}[1]{#1}
\newcommand{\spanBinNumber}[1]{#1}
\newcommand{\spanHexNumber}[1]{#1}
\newcommand{\spanOctNumber}[1]{#1}
\newcommand{\spanFloatNumber}[1]{#1}
\newcommand{\spanIdentifier}[1]{#1}
\newcommand{\spanKeyword}[1]{\textbf{#1}}
\newcommand{\spanStringLit}[1]{#1}
\newcommand{\spanLongStringLit}[1]{#1}
\newcommand{\spanCharLit}[1]{#1}
\newcommand{\spanEscapeSequence}[1]{#1}
\newcommand{\spanOperator}[1]{#1}
\newcommand{\spanPunctuation}[1]{#1}
\newcommand{\spanComment}[1]{\emph{#1}}
\newcommand{\spanLongComment}[1]{\emph{#1}}
\newcommand{\spanRegularExpression}[1]{#1}
\newcommand{\spanTagStart}[1]{#1}
\newcommand{\spanTagEnd}[1]{#1}
\newcommand{\spanKey}[1]{#1}
\newcommand{\spanValue}[1]{#1}
\newcommand{\spanRawData}[1]{#1}
\newcommand{\spanAssembler}[1]{#1}
\newcommand{\spanPreprocessor}[1]{#1}
\newcommand{\spanDirective}[1]{#1}
\newcommand{\spanCommand}[1]{#1}
\newcommand{\spanRule}[1]{#1}
\newcommand{\spanHyperlink}[1]{#1}
\newcommand{\spanLabel}[1]{#1}
\newcommand{\spanReference}[1]{#1}
\newcommand{\spanOther}[1]{#1}
\newcommand{\spantok}[1]{\frame{#1}}

\tableofcontents \newpage
\rsthA{Introduction}\label{introduction}
This document is a tutorial for the programming language \emph{Nim}. This tutorial assumes that you are familiar with basic programming concepts like variables, types or statements but is kept very basic. The \href{manual.html}{manual} contains many more examples of the advanced language features. All code examples in this tutorial, as well as the ones found in the rest of Nim's documentation, follow the \href{nep1.html}{Nim style guide}.

\rsthA{The first program}\label{the-first-program}
We start the tour with a modified "hello world" program:

\begin{rstpre}
\spanComment{\# This is a comment}
\spanIdentifier{echo} \spanStringLit{"What's your name? "}
\spanKeyword{var} \spanIdentifier{name}\spanPunctuation{:} \spanIdentifier{string} \spanOperator{=} \spanIdentifier{readLine}\spanPunctuation{(}\spanIdentifier{stdin}\spanPunctuation{)}
\spanIdentifier{echo} \spanStringLit{"Hi, "}\spanPunctuation{,} \spanIdentifier{name}\spanPunctuation{,} \spanStringLit{"!"}
\end{rstpre}
Save this code to the file "greetings.nim". Now compile and run it:\begin{rstpre}

nim compile --run greetings.nim
\end{rstpre}


With the \texttt{--run} \href{nimc.html\#compiler-usage-command-line-switches}{switch} Nim executes the file automatically after compilation. You can give your program command line arguments by appending them after the filename:\begin{rstpre}

nim compile --run greetings.nim arg1 arg2
\end{rstpre}


Commonly used commands and switches have abbreviations, so you can also use:\begin{rstpre}

nim c -r greetings.nim
\end{rstpre}


To compile a release version use:\begin{rstpre}

nim c -d:release greetings.nim
\end{rstpre}


By default the Nim compiler generates a large amount of runtime checks aiming for your debugging pleasure. With \texttt{-d:release} some checks are \href{nimc.html\#compiler-usage-compile-time-symbols}{turned off and optimizations are turned on}.

Though it should be pretty obvious what the program does, I will explain the syntax: statements which are not indented are executed when the program starts. Indentation is Nim's way of grouping statements. Indentation is done with spaces only, tabulators are not allowed.

String literals are enclosed in double quotes. The \texttt{var} statement declares a new variable named \texttt{name} of type \texttt{string} with the value that is returned by the \href{io.html\#readLine,File}{readLine} procedure. Since the compiler knows that \href{io.html\#readLine,File}{readLine} returns a string, you can leave out the type in the declaration (this is called local type inference\label{local-type-inference_1}). So this will work too:

\begin{rstpre}
\spanKeyword{var} \spanIdentifier{name} \spanOperator{=} \spanIdentifier{readLine}\spanPunctuation{(}\spanIdentifier{stdin}\spanPunctuation{)}
\end{rstpre}
Note that this is basically the only form of type inference that exists in Nim: it is a good compromise between brevity and readability.

The "hello world" program contains several identifiers that are already known to the compiler: \texttt{echo}, \href{io.html\#readLine,File}{readLine}, etc. These built-ins are declared in the \href{system.html}{system} module which is implicitly imported by any other module.

\rsthA{Lexical elements}\label{lexical-elements}
Let us look at Nim's lexical elements in more detail: like other programming languages Nim consists of (string) literals, identifiers, keywords, comments, operators, and other punctuation marks.

\rsthB{String and character literals}\label{lexical-elements-string-and-character-literals}
String literals are enclosed in double quotes; character literals in single quotes. Special characters are escaped with \texttt{\symbol{92}}: \texttt{\symbol{92}n} means newline, \texttt{\symbol{92}t} means tabulator, etc. There are also \emph{raw} string literals:

\begin{rstpre}
\spanRawData{r"C:\symbol{92}program files\symbol{92}nim"}
\end{rstpre}
In raw literals the backslash is not an escape character.

The third and last way to write string literals are \emph{long string literals}. They are written with three quotes: \texttt{""" ... """}; they can span over multiple lines and the \texttt{\symbol{92}} is not an escape character either. They are very useful for embedding HTML code templates for example.

\rsthB{Comments}\label{lexical-elements-comments}
Comments start anywhere outside a string or character literal with the hash character \texttt{\#}. Documentation comments start with \texttt{\#\#}:

\begin{rstpre}
\spanComment{\# A comment.}

\spanKeyword{var} \spanIdentifier{myVariable}\spanPunctuation{:} \spanIdentifier{int} \spanComment{\#\# a documentation comment}
\end{rstpre}
Documentation comments are tokens; they are only allowed at certain places in the input file as they belong to the syntax tree! This feature enables simpler documentation generators.

Multiline comments are started with \texttt{\#\symbol{91}} and terminated with \texttt{\symbol{93}\#}.  Multiline comments can also be nested.

\begin{rstpre}
\spanLongComment{\#\symbol{91}
You can have any Nim code text commented
out inside this with no indentation restrictions.
      yes("May I ask a pointless question?")
  \#\symbol{91}
     Note: these can be nested!!
  \symbol{93}\#
\symbol{93}\#}
\end{rstpre}
\rsthB{Numbers}\label{lexical-elements-numbers}
Numerical literals are written as in most other languages. As a special twist, underscores are allowed for better readability: \texttt{1\_000\_000} (one million). A number that contains a dot (or 'e' or 'E') is a floating point literal: \texttt{1.0e9} (one billion). Hexadecimal literals are prefixed with \texttt{0x}, binary literals with \texttt{0b} and octal literals with \texttt{0o}. A leading zero alone does not produce an octal.

\rsthA{The var statement}\label{the-var-statement}
The var statement declares a new local or global variable:

\begin{rstpre}
\spanKeyword{var} \spanIdentifier{x}\spanPunctuation{,} \spanIdentifier{y}\spanPunctuation{:} \spanIdentifier{int} \spanComment{\# declares x and y to have the type \symbol{96}\symbol{96}int\symbol{96}\symbol{96}}
\end{rstpre}
Indentation can be used after the \texttt{var} keyword to list a whole section of variables:

\begin{rstpre}
\spanKeyword{var}
  \spanIdentifier{x}\spanPunctuation{,} \spanIdentifier{y}\spanPunctuation{:} \spanIdentifier{int}
  \spanComment{\# a comment can occur here too}
  \spanIdentifier{a}\spanPunctuation{,} \spanIdentifier{b}\spanPunctuation{,} \spanIdentifier{c}\spanPunctuation{:} \spanIdentifier{string}
\end{rstpre}
\rsthA{The assignment statement}\label{the-assignment-statement}
The assignment statement assigns a new value to a variable or more generally to a storage location:

\begin{rstpre}
\spanKeyword{var} \spanIdentifier{x} \spanOperator{=} \spanStringLit{"abc"} \spanComment{\# introduces a new variable \symbol{96}x\symbol{96} and assigns a value to it}
\spanIdentifier{x} \spanOperator{=} \spanStringLit{"xyz"}     \spanComment{\# assigns a new value to \symbol{96}x\symbol{96}}
\end{rstpre}
\texttt{=} is the \emph{assignment operator}. The assignment operator can be overloaded. You can declare multiple variables with a single assignment statement and all the variables will have the same value:

\begin{rstpre}
\spanKeyword{var} \spanIdentifier{x}\spanPunctuation{,} \spanIdentifier{y} \spanOperator{=} \spanDecNumber{3}  \spanComment{\# assigns 3 to the variables \symbol{96}x\symbol{96} and \symbol{96}y\symbol{96}}
\spanIdentifier{echo} \spanStringLit{"x "}\spanPunctuation{,} \spanIdentifier{x}  \spanComment{\# outputs "x 3"}
\spanIdentifier{echo} \spanStringLit{"y "}\spanPunctuation{,} \spanIdentifier{y}  \spanComment{\# outputs "y 3"}
\spanIdentifier{x} \spanOperator{=} \spanDecNumber{42}        \spanComment{\# changes \symbol{96}x\symbol{96} to 42 without changing \symbol{96}y\symbol{96}}
\spanIdentifier{echo} \spanStringLit{"x "}\spanPunctuation{,} \spanIdentifier{x}  \spanComment{\# outputs "x 42"}
\spanIdentifier{echo} \spanStringLit{"y "}\spanPunctuation{,} \spanIdentifier{y}  \spanComment{\# outputs "y 3"}
\end{rstpre}
Note that declaring multiple variables with a single assignment which calls a procedure can have unexpected results: the compiler will \emph{unroll} the assignments and end up calling the procedure several times. If the result of the procedure depends on side effects, your variables may end up having different values! For safety use side-effect free procedures if making multiple assignments.

\rsthA{Constants}\label{constants}
Constants are symbols which are bound to a value. The constant's value cannot change. The compiler must be able to evaluate the expression in a constant declaration at compile time:

\begin{rstpre}
\spanKeyword{const} \spanIdentifier{x} \spanOperator{=} \spanStringLit{"abc"} \spanComment{\# the constant x contains the string "abc"}
\end{rstpre}
Indentation can be used after the \texttt{const} keyword to list a whole section of constants:

\begin{rstpre}
\spanKeyword{const}
  \spanIdentifier{x} \spanOperator{=} \spanDecNumber{1}
  \spanComment{\# a comment can occur here too}
  \spanIdentifier{y} \spanOperator{=} \spanDecNumber{2}
  \spanIdentifier{z} \spanOperator{=} \spanIdentifier{y} \spanOperator{+} \spanDecNumber{5} \spanComment{\# computations are possible}
\end{rstpre}
\rsthA{The let statement}\label{the-let-statement}
The \texttt{let} statement works like the \texttt{var} statement but the declared symbols are \emph{single assignment} variables: After the initialization their value cannot change:

\begin{rstpre}
\spanKeyword{let} \spanIdentifier{x} \spanOperator{=} \spanStringLit{"abc"} \spanComment{\# introduces a new variable \symbol{96}x\symbol{96} and binds a value to it}
\spanIdentifier{x} \spanOperator{=} \spanStringLit{"xyz"}     \spanComment{\# Illegal: assignment to \symbol{96}x\symbol{96}}
\end{rstpre}
The difference between \texttt{let} and \texttt{const} is: \texttt{let} introduces a variable that can not be re-assigned, \texttt{const} means "enforce compile time evaluation and put it into a data section":

\begin{rstpre}
\spanKeyword{const} \spanIdentifier{input} \spanOperator{=} \spanIdentifier{readLine}\spanPunctuation{(}\spanIdentifier{stdin}\spanPunctuation{)} \spanComment{\# Error: constant expression expected}
\end{rstpre}
\begin{rstpre}
\spanKeyword{let} \spanIdentifier{input} \spanOperator{=} \spanIdentifier{readLine}\spanPunctuation{(}\spanIdentifier{stdin}\spanPunctuation{)}   \spanComment{\# works}
\end{rstpre}
\rsthA{Control flow statements}\label{control-flow-statements}
The greetings program consists of 3 statements that are executed sequentially. Only the most primitive programs can get away with that: branching and looping are needed too.

\rsthB{If statement}\label{control-flow-statements-if-statement}
The if statement is one way to branch the control flow:

\begin{rstpre}
\spanKeyword{let} \spanIdentifier{name} \spanOperator{=} \spanIdentifier{readLine}\spanPunctuation{(}\spanIdentifier{stdin}\spanPunctuation{)}
\spanKeyword{if} \spanIdentifier{name} \spanOperator{==} \spanStringLit{""}\spanPunctuation{:}
  \spanIdentifier{echo} \spanStringLit{"Poor soul, you lost your name?"}
\spanKeyword{elif} \spanIdentifier{name} \spanOperator{==} \spanStringLit{"name"}\spanPunctuation{:}
  \spanIdentifier{echo} \spanStringLit{"Very funny, your name is name."}
\spanKeyword{else}\spanPunctuation{:}
  \spanIdentifier{echo} \spanStringLit{"Hi, "}\spanPunctuation{,} \spanIdentifier{name}\spanPunctuation{,} \spanStringLit{"!"}
\end{rstpre}
There can be zero or more \texttt{elif} parts, and the \texttt{else} part is optional. The keyword \texttt{elif} is short for \texttt{else if}, and is useful to avoid excessive indentation. (The \texttt{""} is the empty string. It contains no characters.)

\rsthB{Case statement}\label{control-flow-statements-case-statement}
Another way to branch is provided by the case statement. A case statement is a multi-branch:

\begin{rstpre}
\spanKeyword{let} \spanIdentifier{name} \spanOperator{=} \spanIdentifier{readLine}\spanPunctuation{(}\spanIdentifier{stdin}\spanPunctuation{)}
\spanKeyword{case} \spanIdentifier{name}
\spanKeyword{of} \spanStringLit{""}\spanPunctuation{:}
  \spanIdentifier{echo} \spanStringLit{"Poor soul, you lost your name?"}
\spanKeyword{of} \spanStringLit{"name"}\spanPunctuation{:}
  \spanIdentifier{echo} \spanStringLit{"Very funny, your name is name."}
\spanKeyword{of} \spanStringLit{"Dave"}\spanPunctuation{,} \spanStringLit{"Frank"}\spanPunctuation{:}
  \spanIdentifier{echo} \spanStringLit{"Cool name!"}
\spanKeyword{else}\spanPunctuation{:}
  \spanIdentifier{echo} \spanStringLit{"Hi, "}\spanPunctuation{,} \spanIdentifier{name}\spanPunctuation{,} \spanStringLit{"!"}
\end{rstpre}
As it can be seen, for an \texttt{of} branch a comma separated list of values is also allowed.

The case statement can deal with integers, other ordinal types and strings. (What an ordinal type is will be explained soon.) For integers or other ordinal types value ranges are also possible:

\begin{rstpre}
\spanComment{\# this statement will be explained later:}
\spanKeyword{from} \spanIdentifier{strutils} \spanKeyword{import} \spanIdentifier{parseInt}

\spanIdentifier{echo} \spanStringLit{"A number please: "}
\spanKeyword{let} \spanIdentifier{n} \spanOperator{=} \spanIdentifier{parseInt}\spanPunctuation{(}\spanIdentifier{readLine}\spanPunctuation{(}\spanIdentifier{stdin}\spanPunctuation{)}\spanPunctuation{)}
\spanKeyword{case} \spanIdentifier{n}
\spanKeyword{of} \spanFloatNumber{0.}\spanOperator{.}\spanDecNumber{2}\spanPunctuation{,} \spanFloatNumber{4.}\spanOperator{.}\spanDecNumber{7}\spanPunctuation{:} \spanIdentifier{echo} \spanStringLit{"The number is in the set: \symbol{123}0, 1, 2, 4, 5, 6, 7\symbol{125}"}
\spanKeyword{of} \spanDecNumber{3}\spanPunctuation{,} \spanDecNumber{8}\spanPunctuation{:} \spanIdentifier{echo} \spanStringLit{"The number is 3 or 8"}
\end{rstpre}
However, the above code does not compile: the reason is that you have to cover every value that \texttt{n} may contain, but the code only handles the values \texttt{0..8}. Since it is not very practical to list every other possible integer (though it is possible thanks to the range notation), we fix this by telling the compiler that for every other value nothing should be done:

\begin{rstpre}
\spanOperator{...}
\spanKeyword{case} \spanIdentifier{n}
\spanKeyword{of} \spanFloatNumber{0.}\spanOperator{.}\spanDecNumber{2}\spanPunctuation{,} \spanFloatNumber{4.}\spanOperator{.}\spanDecNumber{7}\spanPunctuation{:} \spanIdentifier{echo} \spanStringLit{"The number is in the set: \symbol{123}0, 1, 2, 4, 5, 6, 7\symbol{125}"}
\spanKeyword{of} \spanDecNumber{3}\spanPunctuation{,} \spanDecNumber{8}\spanPunctuation{:} \spanIdentifier{echo} \spanStringLit{"The number is 3 or 8"}
\spanKeyword{else}\spanPunctuation{:} \spanKeyword{discard}
\end{rstpre}
The empty \href{\#procedures-discard-statement}{discard statement} is a \emph{do nothing} statement. The compiler knows that a case statement with an else part cannot fail and thus the error disappears. Note that it is impossible to cover all possible string values: that is why string cases always need an \texttt{else} branch.

In general the case statement is used for subrange types or enumerations where it is of great help that the compiler checks that you covered any possible value.

\rsthB{While statement}\label{control-flow-statements-while-statement}
The while statement is a simple looping construct:

\begin{rstpre}
\spanIdentifier{echo} \spanStringLit{"What's your name? "}
\spanKeyword{var} \spanIdentifier{name} \spanOperator{=} \spanIdentifier{readLine}\spanPunctuation{(}\spanIdentifier{stdin}\spanPunctuation{)}
\spanKeyword{while} \spanIdentifier{name} \spanOperator{==} \spanStringLit{""}\spanPunctuation{:}
  \spanIdentifier{echo} \spanStringLit{"Please tell me your name: "}
  \spanIdentifier{name} \spanOperator{=} \spanIdentifier{readLine}\spanPunctuation{(}\spanIdentifier{stdin}\spanPunctuation{)}
  \spanComment{\# no \symbol{96}\symbol{96}var\symbol{96}\symbol{96}, because we do not declare a new variable here}
\end{rstpre}
The example uses a while loop to keep asking the users for their name, as long as the user types in nothing (only presses RETURN).

\rsthB{For statement}\label{control-flow-statements-for-statement}
The \texttt{for} statement is a construct to loop over any element an \emph{iterator} provides. The example uses the built-in \href{system.html\#countup.i,T,T,Positive}{countup} iterator:

\begin{rstpre}
\spanIdentifier{echo} \spanStringLit{"Counting to ten: "}
\spanKeyword{for} \spanIdentifier{i} \spanKeyword{in} \spanIdentifier{countup}\spanPunctuation{(}\spanDecNumber{1}\spanPunctuation{,} \spanDecNumber{10}\spanPunctuation{)}\spanPunctuation{:}
  \spanIdentifier{echo} \spanIdentifier{i}
\spanComment{\# --> Outputs 1 2 3 4 5 6 7 8 9 10 on different lines}
\end{rstpre}
The variable \texttt{i} is implicitly declared by the \texttt{for} loop and has the type \texttt{int}, because that is what \href{system.html\#countup.i,T,T,Positive}{countup} returns. \texttt{i} runs through the values 1, 2, .., 10. Each value is \texttt{echo}-ed. This code does the same:

\begin{rstpre}
\spanIdentifier{echo} \spanStringLit{"Counting to 10: "}
\spanKeyword{var} \spanIdentifier{i} \spanOperator{=} \spanDecNumber{1}
\spanKeyword{while} \spanIdentifier{i} \spanOperator{<=} \spanDecNumber{10}\spanPunctuation{:}
  \spanIdentifier{echo} \spanIdentifier{i}
  \spanIdentifier{inc}\spanPunctuation{(}\spanIdentifier{i}\spanPunctuation{)} \spanComment{\# increment i by 1}
\spanComment{\# --> Outputs 1 2 3 4 5 6 7 8 9 10 on different lines}
\end{rstpre}
Counting down can be achieved as easily (but is less often needed):

\begin{rstpre}
\spanIdentifier{echo} \spanStringLit{"Counting down from 10 to 1: "}
\spanKeyword{for} \spanIdentifier{i} \spanKeyword{in} \spanIdentifier{countdown}\spanPunctuation{(}\spanDecNumber{10}\spanPunctuation{,} \spanDecNumber{1}\spanPunctuation{)}\spanPunctuation{:}
  \spanIdentifier{echo} \spanIdentifier{i}
\spanComment{\# --> Outputs 10 9 8 7 6 5 4 3 2 1 on different lines}
\end{rstpre}
Since counting up occurs so often in programs, Nim also has a \href{system.html\#...i,T,T}{..} iterator that does the same:

\begin{rstpre}
\spanKeyword{for} \spanIdentifier{i} \spanKeyword{in} \spanDecNumber{1} \spanOperator{..} \spanDecNumber{10}\spanPunctuation{:}
  \spanOperator{...}
\end{rstpre}
Zero-indexed counting has two shortcuts \texttt{..<} and \texttt{.. \symbol{94}1} (\href{system.html\#\symbol{94}.t\%2Cint}{backwards index operator}) to simplify counting to one less than the higher index:

\begin{rstpre}
\spanKeyword{for} \spanIdentifier{i} \spanKeyword{in} \spanDecNumber{0} \spanOperator{..<} \spanDecNumber{10}\spanPunctuation{:}
  \spanOperator{...}  \spanComment{\# 0 .. 9}
\end{rstpre}
or

\begin{rstpre}
\spanKeyword{var} \spanIdentifier{s} \spanOperator{=} \spanStringLit{"some string"}
\spanKeyword{for} \spanIdentifier{i} \spanKeyword{in} \spanDecNumber{0} \spanOperator{..<} \spanIdentifier{s}\spanOperator{.}\spanIdentifier{len}\spanPunctuation{:}
  \spanOperator{...}
\end{rstpre}
or

\begin{rstpre}
\spanKeyword{var} \spanIdentifier{s} \spanOperator{=} \spanStringLit{"some string"}
\spanKeyword{for} \spanIdentifier{idx}\spanPunctuation{,} \spanIdentifier{c} \spanKeyword{in} \spanIdentifier{s}\spanPunctuation{\symbol{91}}\spanDecNumber{0} \spanOperator{..} \spanOperator{\symbol{94}}\spanDecNumber{1}\spanPunctuation{\symbol{93}}\spanPunctuation{:}
  \spanOperator{...}
\end{rstpre}
Other useful iterators for collections (like arrays and sequences) are

\begin{itemize}\item \texttt{items} and \texttt{mitems}, which provides immutable and mutable elements respectively, and
\item \texttt{pairs} and \texttt{mpairs} which provides the element and an index number (immutable and mutable respectively)
\end{itemize}
\begin{rstpre}
\spanKeyword{for} \spanIdentifier{index}\spanPunctuation{,} \spanIdentifier{item} \spanKeyword{in} \spanPunctuation{\symbol{91}}\spanStringLit{"a"}\spanPunctuation{,}\spanStringLit{"b"}\spanPunctuation{\symbol{93}}\spanOperator{.}\spanIdentifier{pairs}\spanPunctuation{:}
  \spanIdentifier{echo} \spanIdentifier{item}\spanPunctuation{,} \spanStringLit{" at index "}\spanPunctuation{,} \spanIdentifier{index}
\spanComment{\# => a at index 0}
\spanComment{\# => b at index 1}
\end{rstpre}
\rsthB{Scopes and the block statement}\label{control-flow-statements-scopes-and-the-block-statement}
Control flow statements have a feature not covered yet: they open a new scope. This means that in the following example, \texttt{x} is not accessible outside the loop:

\begin{rstpre}
\spanKeyword{while} \spanIdentifier{false}\spanPunctuation{:}
  \spanKeyword{var} \spanIdentifier{x} \spanOperator{=} \spanStringLit{"hi"}
\spanIdentifier{echo} \spanIdentifier{x} \spanComment{\# does not work}
\end{rstpre}
A while (for) statement introduces an implicit block. Identifiers are only visible within the block they have been declared. The \texttt{block} statement can be used to open a new block explicitly:

\begin{rstpre}
\spanKeyword{block} \spanIdentifier{myblock}\spanPunctuation{:}
  \spanKeyword{var} \spanIdentifier{x} \spanOperator{=} \spanStringLit{"hi"}
\spanIdentifier{echo} \spanIdentifier{x} \spanComment{\# does not work either}
\end{rstpre}
The block's \emph{label} (\texttt{myblock} in the example) is optional.

\rsthB{Break statement}\label{control-flow-statements-break-statement}
A block can be left prematurely with a \texttt{break} statement. The break statement can leave a \texttt{while}, \texttt{for}, or a \texttt{block} statement. It leaves the innermost construct, unless a label of a block is given:

\begin{rstpre}
\spanKeyword{block} \spanIdentifier{myblock}\spanPunctuation{:}
  \spanIdentifier{echo} \spanStringLit{"entering block"}
  \spanKeyword{while} \spanIdentifier{true}\spanPunctuation{:}
    \spanIdentifier{echo} \spanStringLit{"looping"}
    \spanKeyword{break} \spanComment{\# leaves the loop, but not the block}
  \spanIdentifier{echo} \spanStringLit{"still in block"}

\spanKeyword{block} \spanIdentifier{myblock2}\spanPunctuation{:}
  \spanIdentifier{echo} \spanStringLit{"entering block"}
  \spanKeyword{while} \spanIdentifier{true}\spanPunctuation{:}
    \spanIdentifier{echo} \spanStringLit{"looping"}
    \spanKeyword{break} \spanIdentifier{myblock2} \spanComment{\# leaves the block (and the loop)}
  \spanIdentifier{echo} \spanStringLit{"still in block"}
\end{rstpre}
\rsthB{Continue statement}\label{control-flow-statements-continue-statement}
Like in many other programming languages, a \texttt{continue} statement starts the next iteration immediately:

\begin{rstpre}
\spanKeyword{while} \spanIdentifier{true}\spanPunctuation{:}
  \spanKeyword{let} \spanIdentifier{x} \spanOperator{=} \spanIdentifier{readLine}\spanPunctuation{(}\spanIdentifier{stdin}\spanPunctuation{)}
  \spanKeyword{if} \spanIdentifier{x} \spanOperator{==} \spanStringLit{""}\spanPunctuation{:} \spanKeyword{continue}
  \spanIdentifier{echo} \spanIdentifier{x}
\end{rstpre}
\rsthB{When statement}\label{control-flow-statements-when-statement}
Example:

\begin{rstpre}
\spanKeyword{when} \spanIdentifier{system}\spanOperator{.}\spanIdentifier{hostOS} \spanOperator{==} \spanStringLit{"windows"}\spanPunctuation{:}
  \spanIdentifier{echo} \spanStringLit{"running on Windows!"}
\spanKeyword{elif} \spanIdentifier{system}\spanOperator{.}\spanIdentifier{hostOS} \spanOperator{==} \spanStringLit{"linux"}\spanPunctuation{:}
  \spanIdentifier{echo} \spanStringLit{"running on Linux!"}
\spanKeyword{elif} \spanIdentifier{system}\spanOperator{.}\spanIdentifier{hostOS} \spanOperator{==} \spanStringLit{"macosx"}\spanPunctuation{:}
  \spanIdentifier{echo} \spanStringLit{"running on Mac OS X!"}
\spanKeyword{else}\spanPunctuation{:}
  \spanIdentifier{echo} \spanStringLit{"unknown operating system"}
\end{rstpre}
The \texttt{when} statement is almost identical to the \texttt{if} statement, but with these differences:

\begin{itemize}\item Each condition must be a constant expression since it is evaluated by the compiler.
\item The statements within a branch do not open a new scope.
\item The compiler checks the semantics and produces code \emph{only} for the statements that belong to the first condition that evaluates to \texttt{true}.
\end{itemize}
The \texttt{when} statement is useful for writing platform specific code, similar to the \texttt{\#ifdef} construct in the C programming language.

\rsthA{Statements and indentation}\label{statements-and-indentation}
Now that we covered the basic control flow statements, let's return to Nim indentation rules.

In Nim there is a distinction between \emph{simple statements} and \emph{complex statements}. \emph{Simple statements} cannot contain other statements: Assignment, procedure calls or the \texttt{return} statement belong to the simple statements. \emph{Complex statements} like \texttt{if}, \texttt{when}, \texttt{for}, \texttt{while} can contain other statements. To avoid ambiguities, complex statements must always be indented, but single simple statements do not:

\begin{rstpre}
\spanComment{\# no indentation needed for single assignment statement:}
\spanKeyword{if} \spanIdentifier{x}\spanPunctuation{:} \spanIdentifier{x} \spanOperator{=} \spanIdentifier{false}

\spanComment{\# indentation needed for nested if statement:}
\spanKeyword{if} \spanIdentifier{x}\spanPunctuation{:}
  \spanKeyword{if} \spanIdentifier{y}\spanPunctuation{:}
    \spanIdentifier{y} \spanOperator{=} \spanIdentifier{false}
  \spanKeyword{else}\spanPunctuation{:}
    \spanIdentifier{y} \spanOperator{=} \spanIdentifier{true}

\spanComment{\# indentation needed, because two statements follow the condition:}
\spanKeyword{if} \spanIdentifier{x}\spanPunctuation{:}
  \spanIdentifier{x} \spanOperator{=} \spanIdentifier{false}
  \spanIdentifier{y} \spanOperator{=} \spanIdentifier{false}
\end{rstpre}
\emph{Expressions} are parts of a statement which usually result in a value. The condition in an if statement is an example for an expression. Expressions can contain indentation at certain places for better readability:

\begin{rstpre}
\spanKeyword{if} \spanIdentifier{thisIsaLongCondition}\spanPunctuation{(}\spanPunctuation{)} \spanKeyword{and}
    \spanIdentifier{thisIsAnotherLongCondition}\spanPunctuation{(}\spanDecNumber{1}\spanPunctuation{,}
       \spanDecNumber{2}\spanPunctuation{,} \spanDecNumber{3}\spanPunctuation{,} \spanDecNumber{4}\spanPunctuation{)}\spanPunctuation{:}
  \spanIdentifier{x} \spanOperator{=} \spanIdentifier{true}
\end{rstpre}
As a rule of thumb, indentation within expressions is allowed after operators, an open parenthesis and after commas.

With parenthesis and semicolons \texttt{(;)} you can use statements where only an expression is allowed:

\begin{rstpre}
\spanComment{\# computes fac(4) at compile time:}
\spanKeyword{const} \spanIdentifier{fac4} \spanOperator{=} \spanPunctuation{(}\spanKeyword{var} \spanIdentifier{x} \spanOperator{=} \spanDecNumber{1}\spanPunctuation{;} \spanKeyword{for} \spanIdentifier{i} \spanKeyword{in} \spanFloatNumber{1.}\spanOperator{.}\spanDecNumber{4}\spanPunctuation{:} \spanIdentifier{x} \spanOperator{*=} \spanIdentifier{i}\spanPunctuation{;} \spanIdentifier{x}\spanPunctuation{)}
\end{rstpre}
\rsthA{Procedures}\label{procedures}
To define new commands like \href{system.html\#echo,varargs\symbol{91}typed,\symbol{93}}{echo} and \href{io.html\#readLine,File}{readLine} in the examples, the concept of a \texttt{procedure} is needed. (Some languages call them \emph{methods} or \emph{functions}.) In Nim new procedures are defined with the \texttt{proc} keyword:

\begin{rstpre}
\spanKeyword{proc} \spanIdentifier{yes}\spanPunctuation{(}\spanIdentifier{question}\spanPunctuation{:} \spanIdentifier{string}\spanPunctuation{)}\spanPunctuation{:} \spanIdentifier{bool} \spanOperator{=}
  \spanIdentifier{echo} \spanIdentifier{question}\spanPunctuation{,} \spanStringLit{" (y/n)"}
  \spanKeyword{while} \spanIdentifier{true}\spanPunctuation{:}
    \spanKeyword{case} \spanIdentifier{readLine}\spanPunctuation{(}\spanIdentifier{stdin}\spanPunctuation{)}
    \spanKeyword{of} \spanStringLit{"y"}\spanPunctuation{,} \spanStringLit{"Y"}\spanPunctuation{,} \spanStringLit{"yes"}\spanPunctuation{,} \spanStringLit{"Yes"}\spanPunctuation{:} \spanKeyword{return} \spanIdentifier{true}
    \spanKeyword{of} \spanStringLit{"n"}\spanPunctuation{,} \spanStringLit{"N"}\spanPunctuation{,} \spanStringLit{"no"}\spanPunctuation{,} \spanStringLit{"No"}\spanPunctuation{:} \spanKeyword{return} \spanIdentifier{false}
    \spanKeyword{else}\spanPunctuation{:} \spanIdentifier{echo} \spanStringLit{"Please be clear: yes or no"}

\spanKeyword{if} \spanIdentifier{yes}\spanPunctuation{(}\spanStringLit{"Should I delete all your important files?"}\spanPunctuation{)}\spanPunctuation{:}
  \spanIdentifier{echo} \spanStringLit{"I'm sorry Dave, I'm afraid I can't do that."}
\spanKeyword{else}\spanPunctuation{:}
  \spanIdentifier{echo} \spanStringLit{"I think you know what the problem is just as well as I do."}
\end{rstpre}
This example shows a procedure named \texttt{yes} that asks the user a \texttt{question} and returns true if they answered "yes" (or something similar) and returns false if they answered "no" (or something similar). A \texttt{return} statement leaves the procedure (and therefore the while loop) immediately. The \texttt{(question: string): bool} syntax describes that the procedure expects a parameter named \texttt{question} of type \texttt{string} and returns a value of type \texttt{bool}. The \texttt{bool} type is built-in: the only valid values for \texttt{bool} are \texttt{true} and \texttt{false}. The conditions in if or while statements must be of type \texttt{bool}.

Some terminology: in the example \texttt{question} is called a (formal) \emph{parameter}, \texttt{"Should I..."} is called an \emph{argument} that is passed to this parameter.

\rsthB{Result variable}\label{procedures-result-variable}
A procedure that returns a value has an implicit \texttt{result} variable declared that represents the return value. A \texttt{return} statement with no expression is a shorthand for \texttt{return result}. The \texttt{result} value is always returned automatically at the end of a procedure if there is no \texttt{return} statement at the exit.

\begin{rstpre}
\spanKeyword{proc} \spanIdentifier{sumTillNegative}\spanPunctuation{(}\spanIdentifier{x}\spanPunctuation{:} \spanIdentifier{varargs}\spanPunctuation{\symbol{91}}\spanIdentifier{int}\spanPunctuation{\symbol{93}}\spanPunctuation{)}\spanPunctuation{:} \spanIdentifier{int} \spanOperator{=}
  \spanKeyword{for} \spanIdentifier{i} \spanKeyword{in} \spanIdentifier{x}\spanPunctuation{:}
    \spanKeyword{if} \spanIdentifier{i} \spanOperator{<} \spanDecNumber{0}\spanPunctuation{:}
      \spanKeyword{return}
    \spanIdentifier{result} \spanOperator{=} \spanIdentifier{result} \spanOperator{+} \spanIdentifier{i}

\spanIdentifier{echo} \spanIdentifier{sumTillNegative}\spanPunctuation{(}\spanPunctuation{)} \spanComment{\# echos 0}
\spanIdentifier{echo} \spanIdentifier{sumTillNegative}\spanPunctuation{(}\spanDecNumber{3}\spanPunctuation{,} \spanDecNumber{4}\spanPunctuation{,} \spanDecNumber{5}\spanPunctuation{)} \spanComment{\# echos 12}
\spanIdentifier{echo} \spanIdentifier{sumTillNegative}\spanPunctuation{(}\spanDecNumber{3}\spanPunctuation{,} \spanDecNumber{4} \spanPunctuation{,} \spanOperator{-}\spanDecNumber{1} \spanPunctuation{,} \spanDecNumber{6}\spanPunctuation{)} \spanComment{\# echos 7}
\end{rstpre}
The \texttt{result} variable is already implicitly declared at the start of the function, so declaring it again with 'var result', for example, would shadow it with a normal variable of the same name. The result variable is also already initialised with the type's default value. Note that referential data types will be \texttt{nil} at the start of the procedure, and thus may require manual initialisation.

\rsthB{Parameters}\label{procedures-parameters}
Parameters are immutable in the procedure body. By default, their value cannot be changed because this allows the compiler to implement parameter passing in the most efficient way. If a mutable variable is needed inside the procedure, it has to be declared with \texttt{var} in the procedure body. Shadowing the parameter name is possible, and actually an idiom:

\begin{rstpre}
\spanKeyword{proc} \spanIdentifier{printSeq}\spanPunctuation{(}\spanIdentifier{s}\spanPunctuation{:} \spanIdentifier{seq}\spanPunctuation{,} \spanIdentifier{nprinted}\spanPunctuation{:} \spanIdentifier{int} \spanOperator{=} \spanOperator{-}\spanDecNumber{1}\spanPunctuation{)} \spanOperator{=}
  \spanKeyword{var} \spanIdentifier{nprinted} \spanOperator{=} \spanKeyword{if} \spanIdentifier{nprinted} \spanOperator{==} \spanOperator{-}\spanDecNumber{1}\spanPunctuation{:} \spanIdentifier{s}\spanOperator{.}\spanIdentifier{len} \spanKeyword{else}\spanPunctuation{:} \spanIdentifier{min}\spanPunctuation{(}\spanIdentifier{nprinted}\spanPunctuation{,} \spanIdentifier{s}\spanOperator{.}\spanIdentifier{len}\spanPunctuation{)}
  \spanKeyword{for} \spanIdentifier{i} \spanKeyword{in} \spanDecNumber{0} \spanOperator{..<} \spanIdentifier{nprinted}\spanPunctuation{:}
    \spanIdentifier{echo} \spanIdentifier{s}\spanPunctuation{\symbol{91}}\spanIdentifier{i}\spanPunctuation{\symbol{93}}
\end{rstpre}
If the procedure needs to modify the argument for the caller, a \texttt{var} parameter can be used:

\begin{rstpre}
\spanKeyword{proc} \spanIdentifier{divmod}\spanPunctuation{(}\spanIdentifier{a}\spanPunctuation{,} \spanIdentifier{b}\spanPunctuation{:} \spanIdentifier{int}\spanPunctuation{;} \spanIdentifier{res}\spanPunctuation{,} \spanIdentifier{remainder}\spanPunctuation{:} \spanKeyword{var} \spanIdentifier{int}\spanPunctuation{)} \spanOperator{=}
  \spanIdentifier{res} \spanOperator{=} \spanIdentifier{a} \spanKeyword{div} \spanIdentifier{b}        \spanComment{\# integer division}
  \spanIdentifier{remainder} \spanOperator{=} \spanIdentifier{a} \spanKeyword{mod} \spanIdentifier{b}  \spanComment{\# integer modulo operation}

\spanKeyword{var}
  \spanIdentifier{x}\spanPunctuation{,} \spanIdentifier{y}\spanPunctuation{:} \spanIdentifier{int}
\spanIdentifier{divmod}\spanPunctuation{(}\spanDecNumber{8}\spanPunctuation{,} \spanDecNumber{5}\spanPunctuation{,} \spanIdentifier{x}\spanPunctuation{,} \spanIdentifier{y}\spanPunctuation{)} \spanComment{\# modifies x and y}
\spanIdentifier{echo} \spanIdentifier{x}
\spanIdentifier{echo} \spanIdentifier{y}
\end{rstpre}
In the example, \texttt{res} and \texttt{remainder} are \texttt{var parameters}. Var parameters can be modified by the procedure and the changes are visible to the caller. Note that the above example would better make use of a tuple as a return value instead of using var parameters.

\rsthB{Discard statement}\label{procedures-discard-statement}
To call a procedure that returns a value just for its side effects and ignoring its return value, a \texttt{discard} statement \textbf{must} be used. Nim does not allow silently throwing away a return value:

\begin{rstpre}
\spanKeyword{discard} \spanIdentifier{yes}\spanPunctuation{(}\spanStringLit{"May I ask a pointless question?"}\spanPunctuation{)}
\end{rstpre}
The return value can be ignored implicitly if the called proc/iterator has been declared with the \texttt{discardable} pragma:

\begin{rstpre}
\spanKeyword{proc} \spanIdentifier{p}\spanPunctuation{(}\spanIdentifier{x}\spanPunctuation{,} \spanIdentifier{y}\spanPunctuation{:} \spanIdentifier{int}\spanPunctuation{)}\spanPunctuation{:} \spanIdentifier{int} \spanPunctuation{\symbol{123}}\spanOperator{.}\spanIdentifier{discardable}\spanOperator{.}\spanPunctuation{\symbol{125}} \spanOperator{=}
  \spanKeyword{return} \spanIdentifier{x} \spanOperator{+} \spanIdentifier{y}

\spanIdentifier{p}\spanPunctuation{(}\spanDecNumber{3}\spanPunctuation{,} \spanDecNumber{4}\spanPunctuation{)} \spanComment{\# now valid}
\end{rstpre}
\rsthB{Named arguments}\label{procedures-named-arguments}
Often a procedure has many parameters and it is not clear in which order the parameters appear. This is especially true for procedures that construct a complex data type. Therefore the arguments to a procedure can be named, so that it is clear which argument belongs to which parameter:

\begin{rstpre}
\spanKeyword{proc} \spanIdentifier{createWindow}\spanPunctuation{(}\spanIdentifier{x}\spanPunctuation{,} \spanIdentifier{y}\spanPunctuation{,} \spanIdentifier{width}\spanPunctuation{,} \spanIdentifier{height}\spanPunctuation{:} \spanIdentifier{int}\spanPunctuation{;} \spanIdentifier{title}\spanPunctuation{:} \spanIdentifier{string}\spanPunctuation{;}
                  \spanIdentifier{show}\spanPunctuation{:} \spanIdentifier{bool}\spanPunctuation{)}\spanPunctuation{:} \spanIdentifier{Window} \spanOperator{=}
   \spanOperator{...}

\spanKeyword{var} \spanIdentifier{w} \spanOperator{=} \spanIdentifier{createWindow}\spanPunctuation{(}\spanIdentifier{show} \spanOperator{=} \spanIdentifier{true}\spanPunctuation{,} \spanIdentifier{title} \spanOperator{=} \spanStringLit{"My Application"}\spanPunctuation{,}
                     \spanIdentifier{x} \spanOperator{=} \spanDecNumber{0}\spanPunctuation{,} \spanIdentifier{y} \spanOperator{=} \spanDecNumber{0}\spanPunctuation{,} \spanIdentifier{height} \spanOperator{=} \spanDecNumber{600}\spanPunctuation{,} \spanIdentifier{width} \spanOperator{=} \spanDecNumber{800}\spanPunctuation{)}
\end{rstpre}
Now that we use named arguments to call \texttt{createWindow} the argument order does not matter anymore. Mixing named arguments with ordered arguments is also possible, but not very readable:

\begin{rstpre}
\spanKeyword{var} \spanIdentifier{w} \spanOperator{=} \spanIdentifier{createWindow}\spanPunctuation{(}\spanDecNumber{0}\spanPunctuation{,} \spanDecNumber{0}\spanPunctuation{,} \spanIdentifier{title} \spanOperator{=} \spanStringLit{"My Application"}\spanPunctuation{,}
                     \spanIdentifier{height} \spanOperator{=} \spanDecNumber{600}\spanPunctuation{,} \spanIdentifier{width} \spanOperator{=} \spanDecNumber{800}\spanPunctuation{,} \spanIdentifier{true}\spanPunctuation{)}
\end{rstpre}
The compiler checks that each parameter receives exactly one argument.

\rsthB{Default values}\label{procedures-default-values}
To make the \texttt{createWindow} proc easier to use it should provide \texttt{default values}; these are values that are used as arguments if the caller does not specify them:

\begin{rstpre}
\spanKeyword{proc} \spanIdentifier{createWindow}\spanPunctuation{(}\spanIdentifier{x} \spanOperator{=} \spanDecNumber{0}\spanPunctuation{,} \spanIdentifier{y} \spanOperator{=} \spanDecNumber{0}\spanPunctuation{,} \spanIdentifier{width} \spanOperator{=} \spanDecNumber{500}\spanPunctuation{,} \spanIdentifier{height} \spanOperator{=} \spanDecNumber{700}\spanPunctuation{,}
                  \spanIdentifier{title} \spanOperator{=} \spanStringLit{"unknown"}\spanPunctuation{,}
                  \spanIdentifier{show} \spanOperator{=} \spanIdentifier{true}\spanPunctuation{)}\spanPunctuation{:} \spanIdentifier{Window} \spanOperator{=}
   \spanOperator{...}

\spanKeyword{var} \spanIdentifier{w} \spanOperator{=} \spanIdentifier{createWindow}\spanPunctuation{(}\spanIdentifier{title} \spanOperator{=} \spanStringLit{"My Application"}\spanPunctuation{,} \spanIdentifier{height} \spanOperator{=} \spanDecNumber{600}\spanPunctuation{,} \spanIdentifier{width} \spanOperator{=} \spanDecNumber{800}\spanPunctuation{)}
\end{rstpre}
Now the call to \texttt{createWindow} only needs to set the values that differ from the defaults.

Note that type inference works for parameters with default values; there is no need to write \texttt{title: string = "unknown"}, for example.

\rsthB{Overloaded procedures}\label{procedures-overloaded-procedures}
Nim provides the ability to overload procedures similar to C++:

\begin{rstpre}
\spanKeyword{proc} \spanIdentifier{toString}\spanPunctuation{(}\spanIdentifier{x}\spanPunctuation{:} \spanIdentifier{int}\spanPunctuation{)}\spanPunctuation{:} \spanIdentifier{string} \spanOperator{=}
  \spanIdentifier{result} \spanOperator{=}
    \spanKeyword{if} \spanIdentifier{x} \spanOperator{<} \spanDecNumber{0}\spanPunctuation{:} \spanStringLit{"negative"}
    \spanKeyword{elif} \spanIdentifier{x} \spanOperator{>} \spanDecNumber{0}\spanPunctuation{:} \spanStringLit{"positive"}
    \spanKeyword{else}\spanPunctuation{:} \spanStringLit{"zero"}

\spanKeyword{proc} \spanIdentifier{toString}\spanPunctuation{(}\spanIdentifier{x}\spanPunctuation{:} \spanIdentifier{bool}\spanPunctuation{)}\spanPunctuation{:} \spanIdentifier{string} \spanOperator{=}
  \spanIdentifier{result} \spanOperator{=}
    \spanKeyword{if} \spanIdentifier{x}\spanPunctuation{:} \spanStringLit{"yep"}
    \spanKeyword{else}\spanPunctuation{:} \spanStringLit{"nope"}

\spanIdentifier{assert} \spanIdentifier{toString}\spanPunctuation{(}\spanDecNumber{13}\spanPunctuation{)} \spanOperator{==} \spanStringLit{"positive"} \spanComment{\# calls the toString(x: int) proc}
\spanIdentifier{assert} \spanIdentifier{toString}\spanPunctuation{(}\spanIdentifier{true}\spanPunctuation{)} \spanOperator{==} \spanStringLit{"yep"}    \spanComment{\# calls the toString(x: bool) proc}
\end{rstpre}
(Note that \texttt{toString} is usually the \href{dollars.html}{\$} operator in Nim.) The compiler chooses the most appropriate proc for the \texttt{toString} calls. How this overloading resolution algorithm works exactly is not discussed here (it will be specified in the manual soon).  However, it does not lead to nasty surprises and is based on a quite simple unification algorithm. Ambiguous calls are reported as errors.

\rsthB{Operators}\label{procedures-operators}
The Nim library makes heavy use of overloading - one reason for this is that each operator like \texttt{+} is just an overloaded proc. The parser lets you use operators in \texttt{infix notation} (\texttt{a + b}) or \texttt{prefix notation} (\texttt{+ a}). An infix operator always receives two arguments, a prefix operator always one. (Postfix operators are not possible, because this would be ambiguous: does \texttt{a \symbol{64} \symbol{64} b} mean \texttt{(a) \symbol{64} (\symbol{64}b)} or \texttt{(a\symbol{64}) \symbol{64} (b)}? It always means \texttt{(a) \symbol{64} (\symbol{64}b)}, because there are no postfix operators in Nim.)

Apart from a few built-in keyword operators such as \texttt{and}, \texttt{or}, \texttt{not}, operators always consist of these characters: \texttt{+ - * \symbol{92} / < > = \symbol{64} \$ \symbol{126} \& \% ! ? \symbol{94} . |}

User defined operators are allowed. Nothing stops you from defining your own \texttt{\symbol{64}!?+\symbol{126}} operator, but doing so may reduce readability.

The operator's precedence is determined by its first character. The details can be found in the manual.

To define a new operator enclose the operator in backticks "\symbol{96}\symbol{96}":

\begin{rstpre}
\spanKeyword{proc} \spanPunctuation{\symbol{96}}\spanOperator{\$}\spanPunctuation{\symbol{96}} \spanPunctuation{(}\spanIdentifier{x}\spanPunctuation{:} \spanIdentifier{myDataType}\spanPunctuation{)}\spanPunctuation{:} \spanIdentifier{string} \spanOperator{=} \spanOperator{...}
\spanComment{\# now the \$ operator also works with myDataType, overloading resolution}
\spanComment{\# ensures that \$ works for built-in types just like before}
\end{rstpre}
The "\symbol{96}\symbol{96}" notation can also be used to call an operator just like any other procedure:

\begin{rstpre}
\spanKeyword{if} \spanPunctuation{\symbol{96}}\spanOperator{==}\spanPunctuation{\symbol{96}}\spanPunctuation{(} \spanPunctuation{\symbol{96}}\spanOperator{+}\spanPunctuation{\symbol{96}}\spanPunctuation{(}\spanDecNumber{3}\spanPunctuation{,} \spanDecNumber{4}\spanPunctuation{)}\spanPunctuation{,} \spanDecNumber{7}\spanPunctuation{)}\spanPunctuation{:} \spanIdentifier{echo} \spanStringLit{"True"}
\end{rstpre}
\rsthB{Forward declarations}\label{procedures-forward-declarations}
Every variable, procedure, etc. needs to be declared before it can be used. (The reason for this is that it is non-trivial to avoid this need in a language that supports meta programming as extensively as Nim does.) However, this cannot be done for mutually recursive procedures:

\begin{rstpre}
\spanComment{\# forward declaration:}
\spanKeyword{proc} \spanIdentifier{even}\spanPunctuation{(}\spanIdentifier{n}\spanPunctuation{:} \spanIdentifier{int}\spanPunctuation{)}\spanPunctuation{:} \spanIdentifier{bool}
\end{rstpre}
\begin{rstpre}
\spanKeyword{proc} \spanIdentifier{odd}\spanPunctuation{(}\spanIdentifier{n}\spanPunctuation{:} \spanIdentifier{int}\spanPunctuation{)}\spanPunctuation{:} \spanIdentifier{bool} \spanOperator{=}
  \spanIdentifier{assert}\spanPunctuation{(}\spanIdentifier{n} \spanOperator{>=} \spanDecNumber{0}\spanPunctuation{)} \spanComment{\# makes sure we don't run into negative recursion}
  \spanKeyword{if} \spanIdentifier{n} \spanOperator{==} \spanDecNumber{0}\spanPunctuation{:} \spanIdentifier{false}
  \spanKeyword{else}\spanPunctuation{:}
    \spanIdentifier{n} \spanOperator{==} \spanDecNumber{1} \spanKeyword{or} \spanIdentifier{even}\spanPunctuation{(}\spanIdentifier{n}\spanOperator{-}\spanDecNumber{1}\spanPunctuation{)}

\spanKeyword{proc} \spanIdentifier{even}\spanPunctuation{(}\spanIdentifier{n}\spanPunctuation{:} \spanIdentifier{int}\spanPunctuation{)}\spanPunctuation{:} \spanIdentifier{bool} \spanOperator{=}
  \spanIdentifier{assert}\spanPunctuation{(}\spanIdentifier{n} \spanOperator{>=} \spanDecNumber{0}\spanPunctuation{)} \spanComment{\# makes sure we don't run into negative recursion}
  \spanKeyword{if} \spanIdentifier{n} \spanOperator{==} \spanDecNumber{1}\spanPunctuation{:} \spanIdentifier{false}
  \spanKeyword{else}\spanPunctuation{:}
    \spanIdentifier{n} \spanOperator{==} \spanDecNumber{0} \spanKeyword{or} \spanIdentifier{odd}\spanPunctuation{(}\spanIdentifier{n}\spanOperator{-}\spanDecNumber{1}\spanPunctuation{)}
\end{rstpre}
Here \texttt{odd} depends on \texttt{even} and vice versa. Thus \texttt{even} needs to be introduced to the compiler before it is completely defined. The syntax for such a forward declaration is simple: just omit the \texttt{=} and the procedure's body. The \texttt{assert} just adds border conditions, and will be covered later in Modules\ref{modules} section.

Later versions of the language will weaken the requirements for forward declarations.

The example also shows that a proc's body can consist of a single expression whose value is then returned implicitly.

\rsthA{Iterators}\label{iterators}
Let's return to the simple counting example:

\begin{rstpre}
\spanIdentifier{echo} \spanStringLit{"Counting to ten: "}
\spanKeyword{for} \spanIdentifier{i} \spanKeyword{in} \spanIdentifier{countup}\spanPunctuation{(}\spanDecNumber{1}\spanPunctuation{,} \spanDecNumber{10}\spanPunctuation{)}\spanPunctuation{:}
  \spanIdentifier{echo} \spanIdentifier{i}
\end{rstpre}
Can a \href{system.html\#countup.i,T,T,Positive}{countup} proc be written that supports this loop? Lets try:

\begin{rstpre}
\spanKeyword{proc} \spanIdentifier{countup}\spanPunctuation{(}\spanIdentifier{a}\spanPunctuation{,} \spanIdentifier{b}\spanPunctuation{:} \spanIdentifier{int}\spanPunctuation{)}\spanPunctuation{:} \spanIdentifier{int} \spanOperator{=}
  \spanKeyword{var} \spanIdentifier{res} \spanOperator{=} \spanIdentifier{a}
  \spanKeyword{while} \spanIdentifier{res} \spanOperator{<=} \spanIdentifier{b}\spanPunctuation{:}
    \spanKeyword{return} \spanIdentifier{res}
    \spanIdentifier{inc}\spanPunctuation{(}\spanIdentifier{res}\spanPunctuation{)}
\end{rstpre}
However, this does not work. The problem is that the procedure should not only \texttt{return}, but return and \textbf{continue} after an iteration has finished. This \emph{return and continue} is called a \texttt{yield} statement. Now the only thing left to do is to replace the \texttt{proc} keyword by \texttt{iterator} and here it is - our first iterator:

\begin{rstpre}
\spanKeyword{iterator} \spanIdentifier{countup}\spanPunctuation{(}\spanIdentifier{a}\spanPunctuation{,} \spanIdentifier{b}\spanPunctuation{:} \spanIdentifier{int}\spanPunctuation{)}\spanPunctuation{:} \spanIdentifier{int} \spanOperator{=}
  \spanKeyword{var} \spanIdentifier{res} \spanOperator{=} \spanIdentifier{a}
  \spanKeyword{while} \spanIdentifier{res} \spanOperator{<=} \spanIdentifier{b}\spanPunctuation{:}
    \spanKeyword{yield} \spanIdentifier{res}
    \spanIdentifier{inc}\spanPunctuation{(}\spanIdentifier{res}\spanPunctuation{)}
\end{rstpre}
Iterators look very similar to procedures, but there are several important differences:

\begin{itemize}\item Iterators can only be called from for loops.
\item Iterators cannot contain a \texttt{return} statement (and procs cannot contain a \texttt{yield} statement).
\item Iterators have no implicit \texttt{result} variable.
\item Iterators do not support recursion.
\item Iterators cannot be forward declared, because the compiler must be able to inline an iterator. (This restriction will be gone in a future version of the compiler.)
\end{itemize}
However, you can also use a \texttt{closure} iterator to get a different set of restrictions. See \href{manual.html\#iterators-and-the-for-statement-first-class-iterators}{first class iterators} for details. Iterators can have the same name and parameters as a proc, since essentially they have their own namespaces. Therefore it is common practice to wrap iterators in procs of the same name which accumulate the result of the iterator and return it as a sequence, like \texttt{split} from the \href{strutils.html}{strutils module}.

\rsthA{Basic types}\label{basic-types}
This section deals with the basic built-in types and the operations that are available for them in detail.

\rsthB{Booleans}\label{basic-types-booleans}
Nim's boolean type is called \texttt{bool} and consists of the two pre-defined values \texttt{true} and \texttt{false}. Conditions in while, if, elif, and when statements must be of type bool.

The operators \texttt{not, and, or, xor, <, <=, >, >=, !=, ==} are defined for the bool type. The \texttt{and} and \texttt{or} operators perform short-circuit evaluation. For example:

\begin{rstpre}
\spanKeyword{while} \spanIdentifier{p} \spanOperator{!=} \spanKeyword{nil} \spanKeyword{and} \spanIdentifier{p}\spanOperator{.}\spanIdentifier{name} \spanOperator{!=} \spanStringLit{"xyz"}\spanPunctuation{:}
  \spanComment{\# p.name is not evaluated if p == nil}
  \spanIdentifier{p} \spanOperator{=} \spanIdentifier{p}\spanOperator{.}\spanIdentifier{next}
\end{rstpre}
\rsthB{Characters}\label{basic-types-characters}
The \texttt{character type} is called \texttt{char}. Its size is always one byte, so it cannot represent most UTF-8 characters; but it \emph{can} represent one of the bytes that makes up a multi-byte UTF-8 character. The reason for this is efficiency: for the overwhelming majority of use-cases, the resulting programs will still handle UTF-8 properly as UTF-8 was specially designed for this. Character literals are enclosed in single quotes.

Chars can be compared with the \texttt{==}, \texttt{<}, \texttt{<=}, \texttt{>}, \texttt{>=} operators. The \texttt{\$} operator converts a \texttt{char} to a \texttt{string}. Chars cannot be mixed with integers; to get the ordinal value of a \texttt{char} use the \texttt{ord} proc. Converting from an integer to a \texttt{char} is done with the \texttt{chr} proc.

\rsthB{Strings}\label{basic-types-strings}
String variables are \textbf{mutable}, so appending to a string is possible, and quite efficient. Strings in Nim are both zero-terminated and have a length field. A string's length can be retrieved with the builtin \texttt{len} procedure; the length never counts the terminating zero. Accessing the terminating zero is an error, it only exists so that a Nim string can be converted to a \texttt{cstring} without doing a copy.

The assignment operator for strings copies the string. You can use the \texttt{\&} operator to concatenate strings and \texttt{add} to append to a string.

Strings are compared using their lexicographical order. All the comparison operators are supported. By convention, all strings are UTF-8 encoded, but this is not enforced. For example, when reading strings from binary files, they are merely a sequence of bytes. The index operation \texttt{s\symbol{91}i\symbol{93}} means the i-th \emph{char} of \texttt{s}, not the i-th \emph{unichar}.

A string variable is initialized with the empty string \texttt{""}.

\rsthB{Integers}\label{basic-types-integers}
Nim has these integer types built-in: \texttt{int int8 int16 int32 int64 uint uint8 uint16 uint32 uint64}.

The default integer type is \texttt{int}. Integer literals can have a \emph{type suffix} to specify a non-default integer type:

\begin{rstpre}
\spanKeyword{let}
  \spanIdentifier{x} \spanOperator{=} \spanDecNumber{0}     \spanComment{\# x is of type \symbol{96}\symbol{96}int\symbol{96}\symbol{96}}
  \spanIdentifier{y} \spanOperator{=} \spanDecNumber{0'i8}  \spanComment{\# y is of type \symbol{96}\symbol{96}int8\symbol{96}\symbol{96}}
  \spanIdentifier{z} \spanOperator{=} \spanDecNumber{0'i64} \spanComment{\# z is of type \symbol{96}\symbol{96}int64\symbol{96}\symbol{96}}
  \spanIdentifier{u} \spanOperator{=} \spanDecNumber{0'}\spanIdentifier{u}   \spanComment{\# u is of type \symbol{96}\symbol{96}uint\symbol{96}\symbol{96}}
\end{rstpre}
Most often integers are used for counting objects that reside in memory, so \texttt{int} has the same size as a pointer.

The common operators \texttt{+ - * div mod < <= == != > >=} are defined for integers. The \texttt{and or xor not} operators are also defined for integers, and provide \emph{bitwise} operations. Left bit shifting is done with the \texttt{shl}, right shifting with the \texttt{shr} operator. Bit shifting operators always treat their arguments as \emph{unsigned}. For arithmetic bit shifts\label{arithmetic-bit-shifts_1} ordinary multiplication or division can be used.

Unsigned operations all wrap around; they cannot lead to over- or under-flow errors.

Lossless Automatic type conversion\label{automatic-type-conversion_1} is performed in expressions where different kinds of integer types are used. However, if the type conversion would cause loss of information, the EOutOfRange\label{eoutofrange_1} exception is raised (if the error cannot be detected at compile time).

\rsthB{Floats}\label{basic-types-floats}
Nim has these floating point types built-in: \texttt{float float32 float64}.

The default float type is \texttt{float}. In the current implementation, \texttt{float} is always 64-bits.

Float literals can have a \emph{type suffix} to specify a non-default float type:

\begin{rstpre}
\spanKeyword{var}
  \spanIdentifier{x} \spanOperator{=} \spanFloatNumber{0.0}      \spanComment{\# x is of type \symbol{96}\symbol{96}float\symbol{96}\symbol{96}}
  \spanIdentifier{y} \spanOperator{=} \spanFloatNumber{0.0'f32}  \spanComment{\# y is of type \symbol{96}\symbol{96}float32\symbol{96}\symbol{96}}
  \spanIdentifier{z} \spanOperator{=} \spanFloatNumber{0.0'f64}  \spanComment{\# z is of type \symbol{96}\symbol{96}float64\symbol{96}\symbol{96}}
\end{rstpre}
The common operators \texttt{+ - * / < <= == != > >=} are defined for floats and follow the IEEE-754 standard.

Automatic type conversion in expressions with different kinds of floating point types is performed: the smaller type is converted to the larger. Integer types are \textbf{not} converted to floating point types automatically, nor vice versa. Use the \href{system.html\#toInt,float}{toInt} and \href{system.html\#toFloat,int}{toFloat} procs for these conversions.

\rsthB{Type Conversion}\label{basic-types-type-conversion}
Conversion between numerical types is performed by using the type as a function:

\begin{rstpre}
\spanKeyword{var}
  \spanIdentifier{x}\spanPunctuation{:} \spanIdentifier{int32} \spanOperator{=} \spanFloatNumber{1.}\spanIdentifier{int32}   \spanComment{\# same as calling int32(1)}
  \spanIdentifier{y}\spanPunctuation{:} \spanIdentifier{int8}  \spanOperator{=} \spanIdentifier{int8}\spanPunctuation{(}\spanCharLit{'a'}\spanPunctuation{)} \spanComment{\# 'a' == 97'i8}
  \spanIdentifier{z}\spanPunctuation{:} \spanIdentifier{float} \spanOperator{=} \spanFloatNumber{2.5}       \spanComment{\# int(2.5) rounds down to 2}
  \spanIdentifier{sum}\spanPunctuation{:} \spanIdentifier{int} \spanOperator{=} \spanIdentifier{int}\spanPunctuation{(}\spanIdentifier{x}\spanPunctuation{)} \spanOperator{+} \spanIdentifier{int}\spanPunctuation{(}\spanIdentifier{y}\spanPunctuation{)} \spanOperator{+} \spanIdentifier{int}\spanPunctuation{(}\spanIdentifier{z}\spanPunctuation{)} \spanComment{\# sum == 100}
\end{rstpre}
\rsthA{Internal type representation}\label{internal-type-representation}
As mentioned earlier, the built-in \href{dollars.html}{\$} (stringify) operator turns any basic type into a string, which you can then print to the console using the \texttt{echo} proc. However, advanced types, and your own custom types, won't work with the \texttt{\$} operator until you define it for them. Sometimes you just want to debug the current value of a complex type without having to write its \texttt{\$} operator.  You can use then the \href{system.html\#repr,T}{repr} proc which works with any type and even complex data graphs with cycles. The following example shows that even for basic types there is a difference between the \texttt{\$} and \texttt{repr} outputs:

\begin{rstpre}
\spanKeyword{var}
  \spanIdentifier{myBool} \spanOperator{=} \spanIdentifier{true}
  \spanIdentifier{myCharacter} \spanOperator{=} \spanCharLit{'n'}
  \spanIdentifier{myString} \spanOperator{=} \spanStringLit{"nim"}
  \spanIdentifier{myInteger} \spanOperator{=} \spanDecNumber{42}
  \spanIdentifier{myFloat} \spanOperator{=} \spanFloatNumber{3.14}
\spanIdentifier{echo} \spanIdentifier{myBool}\spanPunctuation{,} \spanStringLit{":"}\spanPunctuation{,} \spanIdentifier{repr}\spanPunctuation{(}\spanIdentifier{myBool}\spanPunctuation{)}
\spanComment{\# --> true:true}
\spanIdentifier{echo} \spanIdentifier{myCharacter}\spanPunctuation{,} \spanStringLit{":"}\spanPunctuation{,} \spanIdentifier{repr}\spanPunctuation{(}\spanIdentifier{myCharacter}\spanPunctuation{)}
\spanComment{\# --> n:'n'}
\spanIdentifier{echo} \spanIdentifier{myString}\spanPunctuation{,} \spanStringLit{":"}\spanPunctuation{,} \spanIdentifier{repr}\spanPunctuation{(}\spanIdentifier{myString}\spanPunctuation{)}
\spanComment{\# --> nim:0x10fa8c050"nim"}
\spanIdentifier{echo} \spanIdentifier{myInteger}\spanPunctuation{,} \spanStringLit{":"}\spanPunctuation{,} \spanIdentifier{repr}\spanPunctuation{(}\spanIdentifier{myInteger}\spanPunctuation{)}
\spanComment{\# --> 42:42}
\spanIdentifier{echo} \spanIdentifier{myFloat}\spanPunctuation{,} \spanStringLit{":"}\spanPunctuation{,} \spanIdentifier{repr}\spanPunctuation{(}\spanIdentifier{myFloat}\spanPunctuation{)}
\spanComment{\# --> 3.1400000000000001e+00:3.1400000000000001e+00}
\end{rstpre}
\rsthA{Advanced types}\label{advanced-types}
In Nim new types can be defined within a \texttt{type} statement:

\begin{rstpre}
\spanKeyword{type}
  \spanIdentifier{biggestInt} \spanOperator{=} \spanIdentifier{int64}      \spanComment{\# biggest integer type that is available}
  \spanIdentifier{biggestFloat} \spanOperator{=} \spanIdentifier{float64}  \spanComment{\# biggest float type that is available}
\end{rstpre}
Enumeration and object types may only be defined within a \texttt{type} statement.

\rsthB{Enumerations}\label{advanced-types-enumerations}
A variable of an enumeration type can only be assigned one of the enumeration's specified values. These values are a set of ordered symbols. Each symbol is mapped to an integer value internally. The first symbol is represented at runtime by 0, the second by 1 and so on. For example:

\begin{rstpre}
\spanKeyword{type}
  \spanIdentifier{Direction} \spanOperator{=} \spanKeyword{enum}
    \spanIdentifier{north}\spanPunctuation{,} \spanIdentifier{east}\spanPunctuation{,} \spanIdentifier{south}\spanPunctuation{,} \spanIdentifier{west}

\spanKeyword{var} \spanIdentifier{x} \spanOperator{=} \spanIdentifier{south}     \spanComment{\# \symbol{96}x\symbol{96} is of type \symbol{96}Direction\symbol{96}; its value is \symbol{96}south\symbol{96}}
\spanIdentifier{echo} \spanIdentifier{x}            \spanComment{\# writes "south" to \symbol{96}stdout\symbol{96}}
\end{rstpre}
All the comparison operators can be used with enumeration types.

An enumeration's symbol can be qualified to avoid ambiguities: \texttt{Direction.south}.

The \texttt{\$} operator can convert any enumeration value to its name, and the \texttt{ord} proc can convert it to its underlying integer value.

For better interfacing to other programming languages, the symbols of enum types can be assigned an explicit ordinal value. However, the ordinal values must be in ascending order.

\rsthB{Ordinal types}\label{advanced-types-ordinal-types}
Enumerations, integer types, \texttt{char} and \texttt{bool} (and subranges) are called ordinal types. Ordinal types have quite a few special operations:

\begin{table}\begin{rsttab}{|X|X|X|X|X|X|X|X|X|X|}
\hline
\textbf{Operation} & \textbf{Comment}\\
\hline
\texttt{ord(x)} & returns the integer value that is used to represent \texttt{x}'s value\\
\hline
\texttt{inc(x)} & increments \texttt{x} by one\\
\hline
\texttt{inc(x, n)} & increments \texttt{x} by \texttt{n}; \texttt{n} is an integer\\
\hline
\texttt{dec(x)} & decrements \texttt{x} by one\\
\hline
\texttt{dec(x, n)} & decrements \texttt{x} by \texttt{n}; \texttt{n} is an integer\\
\hline
\texttt{succ(x)} & returns the successor of \texttt{x}\\
\hline
\texttt{succ(x, n)} & returns the \texttt{n}'th successor of \texttt{x}\\
\hline
\texttt{pred(x)} & returns the predecessor of \texttt{x}\\
\hline
\texttt{pred(x, n)} & returns the \texttt{n}'th predecessor of \texttt{x}\\
\hline
\end{rsttab}\end{table}The \href{system.html\#inc,T,int}{inc}, \href{system.html\#dec,T,int}{dec}, \href{system.html\#succ,T,int}{succ} and \href{system.html\#pred,T,int}{pred} operations can fail by raising an \texttt{EOutOfRange} or \texttt{EOverflow} exception. (If the code has been compiled with the proper runtime checks turned on.)

\rsthB{Subranges}\label{advanced-types-subranges}
A subrange type is a range of values from an integer or enumeration type (the base type). Example:

\begin{rstpre}
\spanKeyword{type}
  \spanIdentifier{MySubrange} \spanOperator{=} \spanIdentifier{range}\spanPunctuation{\symbol{91}}\spanFloatNumber{0.}\spanOperator{.}\spanDecNumber{5}\spanPunctuation{\symbol{93}}
\end{rstpre}
\texttt{MySubrange} is a subrange of \texttt{int} which can only hold the values 0 to 5. Assigning any other value to a variable of type \texttt{MySubrange} is a compile-time or runtime error. Assignments from the base type to one of its subrange types (and vice versa) are allowed.

The \texttt{system} module defines the important \href{system.html\#Natural}{Natural} type as \texttt{range\symbol{91}0..high(int)\symbol{93}} (\href{system.html\#high,typedesc\symbol{91}T\symbol{93}}{high} returns the maximal value). Other programming languages may suggest the use of unsigned integers for natural numbers. This is often \textbf{unwise}: you don't want unsigned arithmetic (which wraps around) just because the numbers cannot be negative. Nim's \texttt{Natural} type helps to avoid this common programming error.

\rsthB{Sets}\label{advanced-types-sets}
The set type models the mathematical notion of a set. The set's basetype can only be an ordinal type of a certain size, namely:\begin{itemize}\item \texttt{int8}-\texttt{int16}
\item \texttt{uint8}/\texttt{byte}-\texttt{uint16}
\item \texttt{char}
\item \texttt{enum}
\end{itemize}
or equivalent. For signed integers the set's base type is defined to be in the range \texttt{0 .. MaxSetElements-1} where \texttt{MaxSetElements} is currently always 2\symbol{94}16.

The reason is that sets are implemented as high performance bit vectors. Attempting to declare a set with a larger type will result in an error:

\begin{rstpre}
\spanKeyword{var} \spanIdentifier{s}\spanPunctuation{:} \spanIdentifier{set}\spanPunctuation{\symbol{91}}\spanIdentifier{int64}\spanPunctuation{\symbol{93}} \spanComment{\# Error: set is too large}
\end{rstpre}
Sets can be constructed via the set constructor: \texttt{\symbol{123}\symbol{125}} is the empty set. The empty set is type compatible with any concrete set type. The constructor can also be used to include elements (and ranges of elements):

\begin{rstpre}
\spanKeyword{type}
  \spanIdentifier{CharSet} \spanOperator{=} \spanIdentifier{set}\spanPunctuation{\symbol{91}}\spanIdentifier{char}\spanPunctuation{\symbol{93}}
\spanKeyword{var}
  \spanIdentifier{x}\spanPunctuation{:} \spanIdentifier{CharSet}
\spanIdentifier{x} \spanOperator{=} \spanPunctuation{\symbol{123}}\spanCharLit{'a'}\spanOperator{..}\spanCharLit{'z'}\spanPunctuation{,} \spanCharLit{'0'}\spanOperator{..}\spanCharLit{'9'}\spanPunctuation{\symbol{125}} \spanComment{\# This constructs a set that contains the}
                         \spanComment{\# letters from 'a' to 'z' and the digits}
                         \spanComment{\# from '0' to '9'}
\end{rstpre}
These operations are supported by sets:

\begin{table}\begin{rsttab}{|X|X|X|X|X|X|X|X|X|X|X|X|X|}
\hline
\textbf{operation} & \textbf{meaning}\\
\hline
\texttt{A + B} & union of two sets\\
\hline
\texttt{A * B} & intersection of two sets\\
\hline
\texttt{A - B} & difference of two sets (A without B's elements)\\
\hline
\texttt{A == B} & set equality\\
\hline
\texttt{A <= B} & subset relation (A is subset of B or equal to B)\\
\hline
\texttt{A < B} & strict subset relation (A is a proper subset of B)\\
\hline
\texttt{e in A} & set membership (A contains element e)\\
\hline
\texttt{e notin A} & A does not contain element e\\
\hline
\texttt{contains(A, e)} & A contains element e\\
\hline
\texttt{card(A)} & the cardinality of A (number of elements in A)\\
\hline
\texttt{incl(A, elem)} & same as \texttt{A = A + \symbol{123}elem\symbol{125}}\\
\hline
\texttt{excl(A, elem)} & same as \texttt{A = A - \symbol{123}elem\symbol{125}}\\
\hline
\end{rsttab}\end{table}\rsthC{Bit fields}\label{sets-bit-fields}
Sets are often used to define a type for the \emph{flags} of a procedure. This is a cleaner (and type safe) solution than defining integer constants that have to be \texttt{or}'ed together.

Enum, sets and casting can be used together as in:

\begin{rstpre}
\spanKeyword{type}
  \spanIdentifier{MyFlag}\spanOperator{*} \spanPunctuation{\symbol{123}}\spanOperator{.}\spanIdentifier{size}\spanPunctuation{:} \spanIdentifier{sizeof}\spanPunctuation{(}\spanIdentifier{cint}\spanPunctuation{)}\spanOperator{.}\spanPunctuation{\symbol{125}} \spanOperator{=} \spanKeyword{enum}
    \spanIdentifier{A}
    \spanIdentifier{B}
    \spanIdentifier{C}
    \spanIdentifier{D}
  \spanIdentifier{MyFlags} \spanOperator{=} \spanIdentifier{set}\spanPunctuation{\symbol{91}}\spanIdentifier{MyFlag}\spanPunctuation{\symbol{93}}

\spanKeyword{proc} \spanIdentifier{toNum}\spanPunctuation{(}\spanIdentifier{f}\spanPunctuation{:} \spanIdentifier{MyFlags}\spanPunctuation{)}\spanPunctuation{:} \spanIdentifier{int} \spanOperator{=} \spanKeyword{cast}\spanPunctuation{\symbol{91}}\spanIdentifier{cint}\spanPunctuation{\symbol{93}}\spanPunctuation{(}\spanIdentifier{f}\spanPunctuation{)}
\spanKeyword{proc} \spanIdentifier{toFlags}\spanPunctuation{(}\spanIdentifier{v}\spanPunctuation{:} \spanIdentifier{int}\spanPunctuation{)}\spanPunctuation{:} \spanIdentifier{MyFlags} \spanOperator{=} \spanKeyword{cast}\spanPunctuation{\symbol{91}}\spanIdentifier{MyFlags}\spanPunctuation{\symbol{93}}\spanPunctuation{(}\spanIdentifier{v}\spanPunctuation{)}

\spanIdentifier{assert} \spanIdentifier{toNum}\spanPunctuation{(}\spanPunctuation{\symbol{123}}\spanPunctuation{\symbol{125}}\spanPunctuation{)} \spanOperator{==} \spanDecNumber{0}
\spanIdentifier{assert} \spanIdentifier{toNum}\spanPunctuation{(}\spanPunctuation{\symbol{123}}\spanIdentifier{A}\spanPunctuation{\symbol{125}}\spanPunctuation{)} \spanOperator{==} \spanDecNumber{1}
\spanIdentifier{assert} \spanIdentifier{toNum}\spanPunctuation{(}\spanPunctuation{\symbol{123}}\spanIdentifier{D}\spanPunctuation{\symbol{125}}\spanPunctuation{)} \spanOperator{==} \spanDecNumber{8}
\spanIdentifier{assert} \spanIdentifier{toNum}\spanPunctuation{(}\spanPunctuation{\symbol{123}}\spanIdentifier{A}\spanPunctuation{,} \spanIdentifier{C}\spanPunctuation{\symbol{125}}\spanPunctuation{)} \spanOperator{==} \spanDecNumber{5}
\spanIdentifier{assert} \spanIdentifier{toFlags}\spanPunctuation{(}\spanDecNumber{0}\spanPunctuation{)} \spanOperator{==} \spanPunctuation{\symbol{123}}\spanPunctuation{\symbol{125}}
\spanIdentifier{assert} \spanIdentifier{toFlags}\spanPunctuation{(}\spanDecNumber{7}\spanPunctuation{)} \spanOperator{==} \spanPunctuation{\symbol{123}}\spanIdentifier{A}\spanPunctuation{,} \spanIdentifier{B}\spanPunctuation{,} \spanIdentifier{C}\spanPunctuation{\symbol{125}}
\end{rstpre}
Note how the set turns enum values into powers of 2.

If using enums and sets with C, use distinct cint.

For interoperability with C see also the \href{manual.html\#implementation-specific-pragmas-bitsize-pragma}{bitsize pragma}.

\rsthB{Arrays}\label{advanced-types-arrays}
An array is a simple fixed length container. Each element in an array has the same type. The array's index type can be any ordinal type.

Arrays can be constructed using \texttt{\symbol{91}\symbol{93}}:

\begin{rstpre}
\spanKeyword{type}
  \spanIdentifier{IntArray} \spanOperator{=} \spanIdentifier{array}\spanPunctuation{\symbol{91}}\spanFloatNumber{0.}\spanOperator{.}\spanDecNumber{5}\spanPunctuation{,} \spanIdentifier{int}\spanPunctuation{\symbol{93}} \spanComment{\# an array that is indexed with 0..5}
\spanKeyword{var}
  \spanIdentifier{x}\spanPunctuation{:} \spanIdentifier{IntArray}
\spanIdentifier{x} \spanOperator{=} \spanPunctuation{\symbol{91}}\spanDecNumber{1}\spanPunctuation{,} \spanDecNumber{2}\spanPunctuation{,} \spanDecNumber{3}\spanPunctuation{,} \spanDecNumber{4}\spanPunctuation{,} \spanDecNumber{5}\spanPunctuation{,} \spanDecNumber{6}\spanPunctuation{\symbol{93}}
\spanKeyword{for} \spanIdentifier{i} \spanKeyword{in} \spanIdentifier{low}\spanPunctuation{(}\spanIdentifier{x}\spanPunctuation{)}\spanOperator{..}\spanIdentifier{high}\spanPunctuation{(}\spanIdentifier{x}\spanPunctuation{)}\spanPunctuation{:}
  \spanIdentifier{echo} \spanIdentifier{x}\spanPunctuation{\symbol{91}}\spanIdentifier{i}\spanPunctuation{\symbol{93}}
\end{rstpre}
The notation \texttt{x\symbol{91}i\symbol{93}} is used to access the i-th element of \texttt{x}. Array access is always bounds checked (at compile-time or at runtime). These checks can be disabled via pragmas or invoking the compiler with the \texttt{--bound\_checks:off} command line switch.

Arrays are value types, like any other Nim type. The assignment operator copies the whole array contents.

The built-in \href{system.html\#len,TOpenArray}{len} proc returns the array's length. \href{system.html\#low,openArray\symbol{91}T\symbol{93}}{low(a)} returns the lowest valid index for the array \texttt{a} and \href{system.html\#high,openArray\symbol{91}T\symbol{93}}{high(a)} the highest valid index.

\begin{rstpre}
\spanKeyword{type}
  \spanIdentifier{Direction} \spanOperator{=} \spanKeyword{enum}
    \spanIdentifier{north}\spanPunctuation{,} \spanIdentifier{east}\spanPunctuation{,} \spanIdentifier{south}\spanPunctuation{,} \spanIdentifier{west}
  \spanIdentifier{BlinkLights} \spanOperator{=} \spanKeyword{enum}
    \spanIdentifier{off}\spanPunctuation{,} \spanIdentifier{on}\spanPunctuation{,} \spanIdentifier{slowBlink}\spanPunctuation{,} \spanIdentifier{mediumBlink}\spanPunctuation{,} \spanIdentifier{fastBlink}
  \spanIdentifier{LevelSetting} \spanOperator{=} \spanIdentifier{array}\spanPunctuation{\symbol{91}}\spanIdentifier{north}\spanOperator{..}\spanIdentifier{west}\spanPunctuation{,} \spanIdentifier{BlinkLights}\spanPunctuation{\symbol{93}}
\spanKeyword{var}
  \spanIdentifier{level}\spanPunctuation{:} \spanIdentifier{LevelSetting}
\spanIdentifier{level}\spanPunctuation{\symbol{91}}\spanIdentifier{north}\spanPunctuation{\symbol{93}} \spanOperator{=} \spanIdentifier{on}
\spanIdentifier{level}\spanPunctuation{\symbol{91}}\spanIdentifier{south}\spanPunctuation{\symbol{93}} \spanOperator{=} \spanIdentifier{slowBlink}
\spanIdentifier{level}\spanPunctuation{\symbol{91}}\spanIdentifier{east}\spanPunctuation{\symbol{93}} \spanOperator{=} \spanIdentifier{fastBlink}
\spanIdentifier{echo} \spanIdentifier{repr}\spanPunctuation{(}\spanIdentifier{level}\spanPunctuation{)}  \spanComment{\# --> \symbol{91}on, fastBlink, slowBlink, off\symbol{93}}
\spanIdentifier{echo} \spanIdentifier{low}\spanPunctuation{(}\spanIdentifier{level}\spanPunctuation{)}   \spanComment{\# --> north}
\spanIdentifier{echo} \spanIdentifier{len}\spanPunctuation{(}\spanIdentifier{level}\spanPunctuation{)}   \spanComment{\# --> 4}
\spanIdentifier{echo} \spanIdentifier{high}\spanPunctuation{(}\spanIdentifier{level}\spanPunctuation{)}  \spanComment{\# --> west}
\end{rstpre}
The syntax for nested arrays (multidimensional) in other languages is a matter of appending more brackets because usually each dimension is restricted to the same index type as the others. In Nim you can have different dimensions with different index types, so the nesting syntax is slightly different. Building on the previous example where a level is defined as an array of enums indexed by yet another enum, we can add the following lines to add a light tower type subdivided in height levels accessed through their integer index:

\begin{rstpre}
\spanKeyword{type}
  \spanIdentifier{LightTower} \spanOperator{=} \spanIdentifier{array}\spanPunctuation{\symbol{91}}\spanFloatNumber{1.}\spanOperator{.}\spanDecNumber{10}\spanPunctuation{,} \spanIdentifier{LevelSetting}\spanPunctuation{\symbol{93}}
\spanKeyword{var}
  \spanIdentifier{tower}\spanPunctuation{:} \spanIdentifier{LightTower}
\spanIdentifier{tower}\spanPunctuation{\symbol{91}}\spanDecNumber{1}\spanPunctuation{\symbol{93}}\spanPunctuation{\symbol{91}}\spanIdentifier{north}\spanPunctuation{\symbol{93}} \spanOperator{=} \spanIdentifier{slowBlink}
\spanIdentifier{tower}\spanPunctuation{\symbol{91}}\spanDecNumber{1}\spanPunctuation{\symbol{93}}\spanPunctuation{\symbol{91}}\spanIdentifier{east}\spanPunctuation{\symbol{93}} \spanOperator{=} \spanIdentifier{mediumBlink}
\spanIdentifier{echo} \spanIdentifier{len}\spanPunctuation{(}\spanIdentifier{tower}\spanPunctuation{)}     \spanComment{\# --> 10}
\spanIdentifier{echo} \spanIdentifier{len}\spanPunctuation{(}\spanIdentifier{tower}\spanPunctuation{\symbol{91}}\spanDecNumber{1}\spanPunctuation{\symbol{93}}\spanPunctuation{)}  \spanComment{\# --> 4}
\spanIdentifier{echo} \spanIdentifier{repr}\spanPunctuation{(}\spanIdentifier{tower}\spanPunctuation{)}    \spanComment{\# --> \symbol{91}\symbol{91}slowBlink, mediumBlink, ...more output..}
\spanComment{\# The following lines don't compile due to type mismatch errors}
\spanComment{\#tower\symbol{91}north\symbol{93}\symbol{91}east\symbol{93} = on}
\spanComment{\#tower\symbol{91}0\symbol{93}\symbol{91}1\symbol{93} = on}
\end{rstpre}
Note how the built-in \texttt{len} proc returns only the array's first dimension length.  Another way of defining the \texttt{LightTower} to better illustrate its nested nature would be to omit the previous definition of the \texttt{LevelSetting} type and instead write it embedded directly as the type of the first dimension:

\begin{rstpre}
\spanKeyword{type}
  \spanIdentifier{LightTower} \spanOperator{=} \spanIdentifier{array}\spanPunctuation{\symbol{91}}\spanFloatNumber{1.}\spanOperator{.}\spanDecNumber{10}\spanPunctuation{,} \spanIdentifier{array}\spanPunctuation{\symbol{91}}\spanIdentifier{north}\spanOperator{..}\spanIdentifier{west}\spanPunctuation{,} \spanIdentifier{BlinkLights}\spanPunctuation{\symbol{93}}\spanPunctuation{\symbol{93}}
\end{rstpre}
It is quite common to have arrays start at zero, so there's a shortcut syntax to specify a range from zero to the specified index minus one:

\begin{rstpre}
\spanKeyword{type}
  \spanIdentifier{IntArray} \spanOperator{=} \spanIdentifier{array}\spanPunctuation{\symbol{91}}\spanFloatNumber{0.}\spanOperator{.}\spanDecNumber{5}\spanPunctuation{,} \spanIdentifier{int}\spanPunctuation{\symbol{93}} \spanComment{\# an array that is indexed with 0..5}
  \spanIdentifier{QuickArray} \spanOperator{=} \spanIdentifier{array}\spanPunctuation{\symbol{91}}\spanDecNumber{6}\spanPunctuation{,} \spanIdentifier{int}\spanPunctuation{\symbol{93}}  \spanComment{\# an array that is indexed with 0..5}
\spanKeyword{var}
  \spanIdentifier{x}\spanPunctuation{:} \spanIdentifier{IntArray}
  \spanIdentifier{y}\spanPunctuation{:} \spanIdentifier{QuickArray}
\spanIdentifier{x} \spanOperator{=} \spanPunctuation{\symbol{91}}\spanDecNumber{1}\spanPunctuation{,} \spanDecNumber{2}\spanPunctuation{,} \spanDecNumber{3}\spanPunctuation{,} \spanDecNumber{4}\spanPunctuation{,} \spanDecNumber{5}\spanPunctuation{,} \spanDecNumber{6}\spanPunctuation{\symbol{93}}
\spanIdentifier{y} \spanOperator{=} \spanIdentifier{x}
\spanKeyword{for} \spanIdentifier{i} \spanKeyword{in} \spanIdentifier{low}\spanPunctuation{(}\spanIdentifier{x}\spanPunctuation{)}\spanOperator{..}\spanIdentifier{high}\spanPunctuation{(}\spanIdentifier{x}\spanPunctuation{)}\spanPunctuation{:}
  \spanIdentifier{echo} \spanIdentifier{x}\spanPunctuation{\symbol{91}}\spanIdentifier{i}\spanPunctuation{\symbol{93}}\spanPunctuation{,} \spanIdentifier{y}\spanPunctuation{\symbol{91}}\spanIdentifier{i}\spanPunctuation{\symbol{93}}
\end{rstpre}
\rsthB{Sequences}\label{advanced-types-sequences}
Sequences are similar to arrays but of dynamic length which may change during runtime (like strings). Since sequences are resizable they are always allocated on the heap and garbage collected.

Sequences are always indexed with an \texttt{int} starting at position 0.  The \href{system.html\#len,seq\symbol{91}T\symbol{93}}{len}, \href{system.html\#low,openArray\symbol{91}T\symbol{93}}{low} and \href{system.html\#high,openArray\symbol{91}T\symbol{93}}{high} operations are available for sequences too. The notation \texttt{x\symbol{91}i\symbol{93}} can be used to access the i-th element of \texttt{x}.

Sequences can be constructed by the array constructor \texttt{\symbol{91}\symbol{93}} in conjunction with the array to sequence operator \texttt{\symbol{64}}. Another way to allocate space for a sequence is to call the built-in \href{system.html\#newSeq}{newSeq} procedure.

A sequence may be passed to an openarray parameter.

Example:

\begin{rstpre}
\spanKeyword{var}
  \spanIdentifier{x}\spanPunctuation{:} \spanIdentifier{seq}\spanPunctuation{\symbol{91}}\spanIdentifier{int}\spanPunctuation{\symbol{93}} \spanComment{\# a reference to a sequence of integers}
\spanIdentifier{x} \spanOperator{=} \spanOperator{\symbol{64}}\spanPunctuation{\symbol{91}}\spanDecNumber{1}\spanPunctuation{,} \spanDecNumber{2}\spanPunctuation{,} \spanDecNumber{3}\spanPunctuation{,} \spanDecNumber{4}\spanPunctuation{,} \spanDecNumber{5}\spanPunctuation{,} \spanDecNumber{6}\spanPunctuation{\symbol{93}} \spanComment{\# the \symbol{64} turns the array into a sequence allocated on the heap}
\end{rstpre}
Sequence variables are initialized with \texttt{\symbol{64}\symbol{91}\symbol{93}}.

The \texttt{for} statement can be used with one or two variables when used with a sequence. When you use the one variable form, the variable will hold the value provided by the sequence. The \texttt{for} statement is looping over the results from the \href{system.html\#items.i,seq\symbol{91}T\symbol{93}}{items()} iterator from the \href{system.html}{system} module.  But if you use the two variable form, the first variable will hold the index position and the second variable will hold the value. Here the \texttt{for} statement is looping over the results from the \href{system.html\#pairs.i,seq\symbol{91}T\symbol{93}}{pairs()} iterator from the \href{system.html}{system} module.  Examples:

\begin{rstpre}
\spanKeyword{for} \spanIdentifier{value} \spanKeyword{in} \spanOperator{\symbol{64}}\spanPunctuation{\symbol{91}}\spanDecNumber{3}\spanPunctuation{,} \spanDecNumber{4}\spanPunctuation{,} \spanDecNumber{5}\spanPunctuation{\symbol{93}}\spanPunctuation{:}
  \spanIdentifier{echo} \spanIdentifier{value}
\spanComment{\# --> 3}
\spanComment{\# --> 4}
\spanComment{\# --> 5}

\spanKeyword{for} \spanIdentifier{i}\spanPunctuation{,} \spanIdentifier{value} \spanKeyword{in} \spanOperator{\symbol{64}}\spanPunctuation{\symbol{91}}\spanDecNumber{3}\spanPunctuation{,} \spanDecNumber{4}\spanPunctuation{,} \spanDecNumber{5}\spanPunctuation{\symbol{93}}\spanPunctuation{:}
  \spanIdentifier{echo} \spanStringLit{"index: "}\spanPunctuation{,} \spanOperator{\$}\spanIdentifier{i}\spanPunctuation{,} \spanStringLit{", value:"}\spanPunctuation{,} \spanOperator{\$}\spanIdentifier{value}
\spanComment{\# --> index: 0, value:3}
\spanComment{\# --> index: 1, value:4}
\spanComment{\# --> index: 2, value:5}
\end{rstpre}
\rsthB{Open arrays}\label{advanced-types-open-arrays}
\textbf{Note}: Openarrays can only be used for parameters.

Often fixed size arrays turn out to be too inflexible; procedures should be able to deal with arrays of different sizes. The openarray\label{openarray_1} type allows this. Openarrays are always indexed with an \texttt{int} starting at position 0. The \href{system.html\#len,TOpenArray}{len}, \href{system.html\#low,openArray\symbol{91}T\symbol{93}}{low} and \href{system.html\#high,openArray\symbol{91}T\symbol{93}}{high} operations are available for open arrays too.  Any array with a compatible base type can be passed to an openarray parameter, the index type does not matter.

\begin{rstpre}
\spanKeyword{var}
  \spanIdentifier{fruits}\spanPunctuation{:}   \spanIdentifier{seq}\spanPunctuation{\symbol{91}}\spanIdentifier{string}\spanPunctuation{\symbol{93}}       \spanComment{\# reference to a sequence of strings that is initialized with '\symbol{64}\symbol{91}\symbol{93}'}
  \spanIdentifier{capitals}\spanPunctuation{:} \spanIdentifier{array}\spanPunctuation{\symbol{91}}\spanDecNumber{3}\spanPunctuation{,} \spanIdentifier{string}\spanPunctuation{\symbol{93}}  \spanComment{\# array of strings with a fixed size}

\spanIdentifier{capitals} \spanOperator{=} \spanPunctuation{\symbol{91}}\spanStringLit{"New York"}\spanPunctuation{,} \spanStringLit{"London"}\spanPunctuation{,} \spanStringLit{"Berlin"}\spanPunctuation{\symbol{93}}   \spanComment{\# array 'capitals' allows assignment of only three elements}
\spanIdentifier{fruits}\spanOperator{.}\spanIdentifier{add}\spanPunctuation{(}\spanStringLit{"Banana"}\spanPunctuation{)}          \spanComment{\# sequence 'fruits' is dynamically expandable during runtime}
\spanIdentifier{fruits}\spanOperator{.}\spanIdentifier{add}\spanPunctuation{(}\spanStringLit{"Mango"}\spanPunctuation{)}

\spanKeyword{proc} \spanIdentifier{openArraySize}\spanPunctuation{(}\spanIdentifier{oa}\spanPunctuation{:} \spanIdentifier{openArray}\spanPunctuation{\symbol{91}}\spanIdentifier{string}\spanPunctuation{\symbol{93}}\spanPunctuation{)}\spanPunctuation{:} \spanIdentifier{int} \spanOperator{=}
  \spanIdentifier{oa}\spanOperator{.}\spanIdentifier{len}

\spanIdentifier{assert} \spanIdentifier{openArraySize}\spanPunctuation{(}\spanIdentifier{fruits}\spanPunctuation{)} \spanOperator{==} \spanDecNumber{2}     \spanComment{\# procedure accepts a sequence as parameter}
\spanIdentifier{assert} \spanIdentifier{openArraySize}\spanPunctuation{(}\spanIdentifier{capitals}\spanPunctuation{)} \spanOperator{==} \spanDecNumber{3}   \spanComment{\# but also an array type}
\end{rstpre}
The openarray type cannot be nested: multidimensional openarrays are not supported because this is seldom needed and cannot be done efficiently.

\rsthB{Varargs}\label{advanced-types-varargs}
A \texttt{varargs} parameter is like an openarray parameter. However, it is also a means to implement passing a variable number of arguments to a procedure. The compiler converts the list of arguments to an array automatically:

\begin{rstpre}
\spanKeyword{proc} \spanIdentifier{myWriteln}\spanPunctuation{(}\spanIdentifier{f}\spanPunctuation{:} \spanIdentifier{File}\spanPunctuation{,} \spanIdentifier{a}\spanPunctuation{:} \spanIdentifier{varargs}\spanPunctuation{\symbol{91}}\spanIdentifier{string}\spanPunctuation{\symbol{93}}\spanPunctuation{)} \spanOperator{=}
  \spanKeyword{for} \spanIdentifier{s} \spanKeyword{in} \spanIdentifier{items}\spanPunctuation{(}\spanIdentifier{a}\spanPunctuation{)}\spanPunctuation{:}
    \spanIdentifier{write}\spanPunctuation{(}\spanIdentifier{f}\spanPunctuation{,} \spanIdentifier{s}\spanPunctuation{)}
  \spanIdentifier{write}\spanPunctuation{(}\spanIdentifier{f}\spanPunctuation{,} \spanStringLit{"}\spanEscapeSequence{\symbol{92}n}\spanStringLit{"}\spanPunctuation{)}

\spanIdentifier{myWriteln}\spanPunctuation{(}\spanIdentifier{stdout}\spanPunctuation{,} \spanStringLit{"abc"}\spanPunctuation{,} \spanStringLit{"def"}\spanPunctuation{,} \spanStringLit{"xyz"}\spanPunctuation{)}
\spanComment{\# is transformed by the compiler to:}
\spanIdentifier{myWriteln}\spanPunctuation{(}\spanIdentifier{stdout}\spanPunctuation{,} \spanPunctuation{\symbol{91}}\spanStringLit{"abc"}\spanPunctuation{,} \spanStringLit{"def"}\spanPunctuation{,} \spanStringLit{"xyz"}\spanPunctuation{\symbol{93}}\spanPunctuation{)}
\end{rstpre}
This transformation is only done if the varargs parameter is the last parameter in the procedure header. It is also possible to perform type conversions in this context:

\begin{rstpre}
\spanKeyword{proc} \spanIdentifier{myWriteln}\spanPunctuation{(}\spanIdentifier{f}\spanPunctuation{:} \spanIdentifier{File}\spanPunctuation{,} \spanIdentifier{a}\spanPunctuation{:} \spanIdentifier{varargs}\spanPunctuation{\symbol{91}}\spanIdentifier{string}\spanPunctuation{,} \spanPunctuation{\symbol{96}}\spanOperator{\$}\spanPunctuation{\symbol{96}}\spanPunctuation{\symbol{93}}\spanPunctuation{)} \spanOperator{=}
  \spanKeyword{for} \spanIdentifier{s} \spanKeyword{in} \spanIdentifier{items}\spanPunctuation{(}\spanIdentifier{a}\spanPunctuation{)}\spanPunctuation{:}
    \spanIdentifier{write}\spanPunctuation{(}\spanIdentifier{f}\spanPunctuation{,} \spanIdentifier{s}\spanPunctuation{)}
  \spanIdentifier{write}\spanPunctuation{(}\spanIdentifier{f}\spanPunctuation{,} \spanStringLit{"}\spanEscapeSequence{\symbol{92}n}\spanStringLit{"}\spanPunctuation{)}

\spanIdentifier{myWriteln}\spanPunctuation{(}\spanIdentifier{stdout}\spanPunctuation{,} \spanDecNumber{123}\spanPunctuation{,} \spanStringLit{"abc"}\spanPunctuation{,} \spanFloatNumber{4.0}\spanPunctuation{)}
\spanComment{\# is transformed by the compiler to:}
\spanIdentifier{myWriteln}\spanPunctuation{(}\spanIdentifier{stdout}\spanPunctuation{,} \spanPunctuation{\symbol{91}}\spanOperator{\$}\spanDecNumber{123}\spanPunctuation{,} \spanOperator{\$}\spanStringLit{"abc"}\spanPunctuation{,} \spanOperator{\$}\spanFloatNumber{4.0}\spanPunctuation{\symbol{93}}\spanPunctuation{)}
\end{rstpre}
In this example \href{dollars.html}{\$} is applied to any argument that is passed to the parameter \texttt{a}. Note that \href{dollars.html}{\$} applied to strings is a nop.

\rsthB{Slices}\label{advanced-types-slices}
Slices look similar to subranges types in syntax but are used in a different context. A slice is just an object of type Slice which contains two bounds, \texttt{a} and \texttt{b}. By itself a slice is not very useful, but other collection types define operators which accept Slice objects to define ranges.

\begin{rstpre}
\spanKeyword{var}
  \spanIdentifier{a} \spanOperator{=} \spanStringLit{"Nim is a programming language"}
  \spanIdentifier{b} \spanOperator{=} \spanStringLit{"Slices are useless."}

\spanIdentifier{echo} \spanIdentifier{a}\spanPunctuation{\symbol{91}}\spanDecNumber{7} \spanOperator{..} \spanDecNumber{12}\spanPunctuation{\symbol{93}} \spanComment{\# --> 'a prog'}
\spanIdentifier{b}\spanPunctuation{\symbol{91}}\spanDecNumber{11} \spanOperator{..} \spanOperator{\symbol{94}}\spanDecNumber{2}\spanPunctuation{\symbol{93}} \spanOperator{=} \spanStringLit{"useful"}
\spanIdentifier{echo} \spanIdentifier{b} \spanComment{\# --> 'Slices are useful.'}
\end{rstpre}
In the previous example slices are used to modify a part of a string. The slice's bounds can hold any value supported by their type, but it is the proc using the slice object which defines what values are accepted.

To understand some of the different ways of specifying the indices of strings, arrays, sequences, etc., it must be remembered that Nim uses zero-based indices.

So the string \texttt{b} is of length 19, and two different ways of specifying the indices are

\begin{rstpre}
\spanStringLit{"Slices are useless."}
 \spanOperator{|}          \spanOperator{|}     \spanOperator{|}
 \spanDecNumber{0}         \spanDecNumber{11}    \spanDecNumber{17}   \spanKeyword{using} \spanIdentifier{indices}
\spanOperator{\symbol{94}}\spanDecNumber{19}        \spanOperator{\symbol{94}}\spanDecNumber{8}    \spanOperator{\symbol{94}}\spanDecNumber{2}   \spanKeyword{using} \spanOperator{\symbol{94}} \spanIdentifier{syntax}
\end{rstpre}
where \texttt{b\symbol{91}0 .. \symbol{94}1\symbol{93}} is equivalent to \texttt{b\symbol{91}0 .. b.len-1\symbol{93}} and \texttt{b\symbol{91}0 ..< b.len\symbol{93}}, and it can be seen that the \texttt{\symbol{94}1} provides a short-hand way of specifying the \texttt{b.len-1}. See the \href{system.html\#\symbol{94}.t\%2Cint}{backwards index operator}.

In the above example, because the string ends in a period, to get the portion of the string that is "useless" and replace it with "useful".

\texttt{b\symbol{91}11 .. \symbol{94}2\symbol{93}} is the portion "useless", and \texttt{b\symbol{91}11 .. \symbol{94}2\symbol{93} = "useful"} replaces the "useless" portion with "useful", giving the result "Slices are useful."

Note 1: alternate ways of writing this are \texttt{b\symbol{91}\symbol{94}8 .. \symbol{94}2\symbol{93} = "useful"} or as \texttt{b\symbol{91}11 .. b.len-2\symbol{93} = "useful"} or as \texttt{b\symbol{91}11 ..< b.len-1\symbol{93} = "useful"}.

Note 2: As the \texttt{\symbol{94}} template returns a \href{manual.html\#types-distinct-type}{distinct int} of type \texttt{BackwardsIndex}, we can have a \texttt{lastIndex} constant defined as \texttt{const lastIndex = \symbol{94}1}, and later used as \texttt{b\symbol{91}0 .. lastIndex\symbol{93}}.

\rsthB{Objects}\label{advanced-types-objects}
The default type to pack different values together in a single structure with a name is the object type. An object is a value type, which means that when an object is assigned to a new variable all its components are copied as well.

Each object type \texttt{Foo} has a constructor \texttt{Foo(field: value, ...)} where all of its fields can be initialized. Unspecified fields will get their default value.

\begin{rstpre}
\spanKeyword{type}
  \spanIdentifier{Person} \spanOperator{=} \spanKeyword{object}
    \spanIdentifier{name}\spanPunctuation{:} \spanIdentifier{string}
    \spanIdentifier{age}\spanPunctuation{:} \spanIdentifier{int}

\spanKeyword{var} \spanIdentifier{person1} \spanOperator{=} \spanIdentifier{Person}\spanPunctuation{(}\spanIdentifier{name}\spanPunctuation{:} \spanStringLit{"Peter"}\spanPunctuation{,} \spanIdentifier{age}\spanPunctuation{:} \spanDecNumber{30}\spanPunctuation{)}

\spanIdentifier{echo} \spanIdentifier{person1}\spanOperator{.}\spanIdentifier{name} \spanComment{\# "Peter"}
\spanIdentifier{echo} \spanIdentifier{person1}\spanOperator{.}\spanIdentifier{age}  \spanComment{\# 30}

\spanKeyword{var} \spanIdentifier{person2} \spanOperator{=} \spanIdentifier{person1} \spanComment{\# copy of person 1}

\spanIdentifier{person2}\spanOperator{.}\spanIdentifier{age} \spanOperator{+=} \spanDecNumber{14}

\spanIdentifier{echo} \spanIdentifier{person1}\spanOperator{.}\spanIdentifier{age} \spanComment{\# 30}
\spanIdentifier{echo} \spanIdentifier{person2}\spanOperator{.}\spanIdentifier{age} \spanComment{\# 44}


\spanComment{\# the order may be changed}
\spanKeyword{let} \spanIdentifier{person3} \spanOperator{=} \spanIdentifier{Person}\spanPunctuation{(}\spanIdentifier{age}\spanPunctuation{:} \spanDecNumber{12}\spanPunctuation{,} \spanIdentifier{name}\spanPunctuation{:} \spanStringLit{"Quentin"}\spanPunctuation{)}

\spanComment{\# not every member needs to be specified}
\spanKeyword{let} \spanIdentifier{person4} \spanOperator{=} \spanIdentifier{Person}\spanPunctuation{(}\spanIdentifier{age}\spanPunctuation{:} \spanDecNumber{3}\spanPunctuation{)}
\spanComment{\# unspecified members will be initialized with their default}
\spanComment{\# values. In this case it is the empty string.}
\spanIdentifier{doAssert} \spanIdentifier{person4}\spanOperator{.}\spanIdentifier{name} \spanOperator{==} \spanStringLit{""}
\end{rstpre}
Object fields that should be visible from outside the defining module have to be marked with \texttt{*}.

\begin{rstpre}
\spanKeyword{type}
  \spanIdentifier{Person}\spanOperator{*} \spanOperator{=} \spanKeyword{object} \spanComment{\# the type is visible from other modules}
    \spanIdentifier{name}\spanOperator{*:} \spanIdentifier{string}  \spanComment{\# the field of this type is visible from other modules}
    \spanIdentifier{age}\spanOperator{*:} \spanIdentifier{int}
\end{rstpre}
\rsthB{Tuples}\label{advanced-types-tuples}
Tuples are very much like what you have seen so far from objects. They are value types where the assignment operator copies each component. Unlike object types though, tuple types are structurally typed, meaning different tuple-types are \emph{equivalent} if they specify fields of the same type and of the same name in the same order.

The constructor \texttt{()} can be used to construct tuples. The order of the fields in the constructor must match the order in the tuple's definition. But unlike objects, a name for the tuple type may not be used here.

Like the object type the notation \texttt{t.field} is used to access a tuple's field. Another notation that is not available for objects is \texttt{t\symbol{91}i\symbol{93}} to access the \texttt{i}'th field. Here \texttt{i} must be a constant integer.

\begin{rstpre}
\spanKeyword{type}
  \spanComment{\# type representing a person:}
  \spanComment{\# A person consists of a name and an age.}
  \spanIdentifier{Person} \spanOperator{=} \spanKeyword{tuple}
    \spanIdentifier{name}\spanPunctuation{:} \spanIdentifier{string}
    \spanIdentifier{age}\spanPunctuation{:} \spanIdentifier{int}
  
  \spanComment{\# Alternative syntax for an equivalent type.}
  \spanIdentifier{PersonX} \spanOperator{=} \spanKeyword{tuple}\spanPunctuation{\symbol{91}}\spanIdentifier{name}\spanPunctuation{:} \spanIdentifier{string}\spanPunctuation{,} \spanIdentifier{age}\spanPunctuation{:} \spanIdentifier{int}\spanPunctuation{\symbol{93}}
  
  \spanComment{\# anonymous field syntax}
  \spanIdentifier{PersonY} \spanOperator{=} \spanPunctuation{(}\spanIdentifier{string}\spanPunctuation{,} \spanIdentifier{int}\spanPunctuation{)}

\spanKeyword{var}
  \spanIdentifier{person}\spanPunctuation{:} \spanIdentifier{Person}
  \spanIdentifier{personX}\spanPunctuation{:} \spanIdentifier{PersonX}
  \spanIdentifier{personY}\spanPunctuation{:} \spanIdentifier{PersonY}

\spanIdentifier{person} \spanOperator{=} \spanPunctuation{(}\spanIdentifier{name}\spanPunctuation{:} \spanStringLit{"Peter"}\spanPunctuation{,} \spanIdentifier{age}\spanPunctuation{:} \spanDecNumber{30}\spanPunctuation{)}
\spanComment{\# Person and PersonX are equivalent}
\spanIdentifier{personX} \spanOperator{=} \spanIdentifier{person}

\spanComment{\# Create a tuple with anonymous fields:}
\spanIdentifier{personY} \spanOperator{=} \spanPunctuation{(}\spanStringLit{"Peter"}\spanPunctuation{,} \spanDecNumber{30}\spanPunctuation{)}

\spanComment{\# A tuple with anonymous fields is compatible with a tuple that has}
\spanComment{\# field names.}
\spanIdentifier{person} \spanOperator{=} \spanIdentifier{personY}
\spanIdentifier{personY} \spanOperator{=} \spanIdentifier{person}

\spanComment{\# Usually used for short tuple initialization syntax}
\spanIdentifier{person} \spanOperator{=} \spanPunctuation{(}\spanStringLit{"Peter"}\spanPunctuation{,} \spanDecNumber{30}\spanPunctuation{)}

\spanIdentifier{echo} \spanIdentifier{person}\spanOperator{.}\spanIdentifier{name} \spanComment{\# "Peter"}
\spanIdentifier{echo} \spanIdentifier{person}\spanOperator{.}\spanIdentifier{age}  \spanComment{\# 30}

\spanIdentifier{echo} \spanIdentifier{person}\spanPunctuation{\symbol{91}}\spanDecNumber{0}\spanPunctuation{\symbol{93}} \spanComment{\# "Peter"}
\spanIdentifier{echo} \spanIdentifier{person}\spanPunctuation{\symbol{91}}\spanDecNumber{1}\spanPunctuation{\symbol{93}} \spanComment{\# 30}

\spanComment{\# You don't need to declare tuples in a separate type section.}
\spanKeyword{var} \spanIdentifier{building}\spanPunctuation{:} \spanKeyword{tuple}\spanPunctuation{\symbol{91}}\spanIdentifier{street}\spanPunctuation{:} \spanIdentifier{string}\spanPunctuation{,} \spanIdentifier{number}\spanPunctuation{:} \spanIdentifier{int}\spanPunctuation{\symbol{93}}
\spanIdentifier{building} \spanOperator{=} \spanPunctuation{(}\spanStringLit{"Rue del Percebe"}\spanPunctuation{,} \spanDecNumber{13}\spanPunctuation{)}
\spanIdentifier{echo} \spanIdentifier{building}\spanOperator{.}\spanIdentifier{street}

\spanComment{\# The following line does not compile, they are different tuples!}
\spanComment{\#person = building}
\spanComment{\# --> Error: type mismatch: got (tuple\symbol{91}street: string, number: int\symbol{93})}
\spanComment{\#     but expected 'Person'}
\end{rstpre}
Even though you don't need to declare a type for a tuple to use it, tuples created with different field names will be considered different objects despite having the same field types.

Tuples can be \emph{unpacked} during variable assignment (and only then!). This can be handy to assign directly the fields of the tuples to individually named variables. An example of this is the \href{os.html\#splitFile,string}{splitFile} proc from the \href{os.html}{os module} which returns the directory, name and extension of a path at the same time. For tuple unpacking to work you must use parentheses around the values you want to assign the unpacking to, otherwise you will be assigning the same value to all the individual variables! For example:

\begin{rstpre}
\spanKeyword{import} \spanIdentifier{os}

\spanKeyword{let}
  \spanIdentifier{path} \spanOperator{=} \spanStringLit{"usr/local/nimc.html"}
  \spanPunctuation{(}\spanIdentifier{dir}\spanPunctuation{,} \spanIdentifier{name}\spanPunctuation{,} \spanIdentifier{ext}\spanPunctuation{)} \spanOperator{=} \spanIdentifier{splitFile}\spanPunctuation{(}\spanIdentifier{path}\spanPunctuation{)}
  \spanIdentifier{baddir}\spanPunctuation{,} \spanIdentifier{badname}\spanPunctuation{,} \spanIdentifier{badext} \spanOperator{=} \spanIdentifier{splitFile}\spanPunctuation{(}\spanIdentifier{path}\spanPunctuation{)}
\spanIdentifier{echo} \spanIdentifier{dir}      \spanComment{\# outputs \symbol{96}usr/local\symbol{96}}
\spanIdentifier{echo} \spanIdentifier{name}     \spanComment{\# outputs \symbol{96}nimc\symbol{96}}
\spanIdentifier{echo} \spanIdentifier{ext}      \spanComment{\# outputs \symbol{96}.html\symbol{96}}
\spanComment{\# All the following output the same line:}
\spanComment{\# \symbol{96}(dir: usr/local, name: nimc, ext: .html)\symbol{96}}
\spanIdentifier{echo} \spanIdentifier{baddir}
\spanIdentifier{echo} \spanIdentifier{badname}
\spanIdentifier{echo} \spanIdentifier{badext}
\end{rstpre}
Fields of tuples are always public, they don't need to be explicity marked to be exported, unlike for example fields in an object type.

\rsthB{Reference and pointer types}\label{advanced-types-reference-and-pointer-types}
References (similar to pointers in other programming languages) are a way to introduce many-to-one relationships. This means different references can point to and modify the same location in memory.

Nim distinguishes between traced\label{traced_1} and untraced\label{untraced_1} references. Untraced references are also called \emph{pointers}. Traced references point to objects in a garbage collected heap, untraced references point to manually allocated objects or to objects elsewhere in memory. Thus untraced references are \emph{unsafe}. However for certain low-level operations (e.g., accessing the hardware), untraced references are necessary.

Traced references are declared with the \textbf{ref} keyword; untraced references are declared with the \textbf{ptr} keyword.

The empty \texttt{\symbol{91}\symbol{93}} subscript notation can be used to \emph{derefer} a reference, meaning to retrieve the item the reference points to. The \texttt{.} (access a tuple/object field operator) and \texttt{\symbol{91}\symbol{93}} (array/string/sequence index operator) operators perform implicit dereferencing operations for reference types:

\begin{rstpre}
\spanKeyword{type}
  \spanIdentifier{Node} \spanOperator{=} \spanKeyword{ref} \spanKeyword{object}
    \spanIdentifier{le}\spanPunctuation{,} \spanIdentifier{ri}\spanPunctuation{:} \spanIdentifier{Node}
    \spanIdentifier{data}\spanPunctuation{:} \spanIdentifier{int}
\spanKeyword{var}
  \spanIdentifier{n}\spanPunctuation{:} \spanIdentifier{Node}
\spanIdentifier{new}\spanPunctuation{(}\spanIdentifier{n}\spanPunctuation{)}
\spanIdentifier{n}\spanOperator{.}\spanIdentifier{data} \spanOperator{=} \spanDecNumber{9}
\spanComment{\# no need to write n\symbol{91}\symbol{93}.data; in fact n\symbol{91}\symbol{93}.data is highly discouraged!}
\end{rstpre}
To allocate a new traced object, the built-in procedure \texttt{new} must be used. To deal with untraced memory, the procedures \texttt{alloc}, \texttt{dealloc} and \texttt{realloc} can be used. The \href{system.html}{system} module's documentation contains further details.

If a reference points to \emph{nothing}, it has the value \texttt{nil}.

\rsthB{Procedural type}\label{advanced-types-procedural-type}
A procedural type is a (somewhat abstract) pointer to a procedure. \texttt{nil} is an allowed value for a variable of a procedural type. Nim uses procedural types to achieve functional\label{functional_1} programming techniques.

Example:

\begin{rstpre}
\spanKeyword{proc} \spanIdentifier{echoItem}\spanPunctuation{(}\spanIdentifier{x}\spanPunctuation{:} \spanIdentifier{int}\spanPunctuation{)} \spanOperator{=} \spanIdentifier{echo} \spanIdentifier{x}

\spanKeyword{proc} \spanIdentifier{forEach}\spanPunctuation{(}\spanIdentifier{action}\spanPunctuation{:} \spanKeyword{proc} \spanPunctuation{(}\spanIdentifier{x}\spanPunctuation{:} \spanIdentifier{int}\spanPunctuation{)}\spanPunctuation{)} \spanOperator{=}
  \spanKeyword{const}
    \spanIdentifier{data} \spanOperator{=} \spanPunctuation{\symbol{91}}\spanDecNumber{2}\spanPunctuation{,} \spanDecNumber{3}\spanPunctuation{,} \spanDecNumber{5}\spanPunctuation{,} \spanDecNumber{7}\spanPunctuation{,} \spanDecNumber{11}\spanPunctuation{\symbol{93}}
  \spanKeyword{for} \spanIdentifier{d} \spanKeyword{in} \spanIdentifier{items}\spanPunctuation{(}\spanIdentifier{data}\spanPunctuation{)}\spanPunctuation{:}
    \spanIdentifier{action}\spanPunctuation{(}\spanIdentifier{d}\spanPunctuation{)}

\spanIdentifier{forEach}\spanPunctuation{(}\spanIdentifier{echoItem}\spanPunctuation{)}
\end{rstpre}
A subtle issue with procedural types is that the calling convention of the procedure influences the type compatibility: procedural types are only compatible if they have the same calling convention. The different calling conventions are listed in the \href{manual.html\#types-procedural-type}{manual}.

\rsthB{Distinct type}\label{advanced-types-distinct-type}
A Distinct type allows for the creation of new type that "does not imply a subtype relationship between it and its base type". You must \textbf{explicitly} define all behaviour for the distinct type. To help with this, both the distinct type and its base type can cast from one type to the other. Examples are provided in the \href{manual.html\#types-distinct-type}{manual}.

\rsthA{Modules}\label{modules}
Nim supports splitting a program into pieces with a module concept. Each module is in its own file. Modules enable information hiding\label{information-hiding_1} and separate compilation\label{separate-compilation_1}. A module may gain access to the symbols of another module by using the import\label{import_1} statement. Only top-level symbols that are marked with an asterisk (\texttt{*}) are exported:

\begin{rstpre}
\spanComment{\# Module A}
\spanKeyword{var}
  \spanIdentifier{x}\spanOperator{*}\spanPunctuation{,} \spanIdentifier{y}\spanPunctuation{:} \spanIdentifier{int}

\spanKeyword{proc} \spanPunctuation{\symbol{96}}\spanOperator{*}\spanPunctuation{\symbol{96}} \spanOperator{*}\spanPunctuation{(}\spanIdentifier{a}\spanPunctuation{,} \spanIdentifier{b}\spanPunctuation{:} \spanIdentifier{seq}\spanPunctuation{\symbol{91}}\spanIdentifier{int}\spanPunctuation{\symbol{93}}\spanPunctuation{)}\spanPunctuation{:} \spanIdentifier{seq}\spanPunctuation{\symbol{91}}\spanIdentifier{int}\spanPunctuation{\symbol{93}} \spanOperator{=}
  \spanComment{\# allocate a new sequence:}
  \spanIdentifier{newSeq}\spanPunctuation{(}\spanIdentifier{result}\spanPunctuation{,} \spanIdentifier{len}\spanPunctuation{(}\spanIdentifier{a}\spanPunctuation{)}\spanPunctuation{)}
  \spanComment{\# multiply two int sequences:}
  \spanKeyword{for} \spanIdentifier{i} \spanKeyword{in} \spanFloatNumber{0.}\spanOperator{.}\spanIdentifier{len}\spanPunctuation{(}\spanIdentifier{a}\spanPunctuation{)}\spanOperator{-}\spanDecNumber{1}\spanPunctuation{:} \spanIdentifier{result}\spanPunctuation{\symbol{91}}\spanIdentifier{i}\spanPunctuation{\symbol{93}} \spanOperator{=} \spanIdentifier{a}\spanPunctuation{\symbol{91}}\spanIdentifier{i}\spanPunctuation{\symbol{93}} \spanOperator{*} \spanIdentifier{b}\spanPunctuation{\symbol{91}}\spanIdentifier{i}\spanPunctuation{\symbol{93}}

\spanKeyword{when} \spanIdentifier{isMainModule}\spanPunctuation{:}
  \spanComment{\# test the new \symbol{96}\symbol{96}*\symbol{96}\symbol{96} operator for sequences:}
  \spanIdentifier{assert}\spanPunctuation{(}\spanOperator{\symbol{64}}\spanPunctuation{\symbol{91}}\spanDecNumber{1}\spanPunctuation{,} \spanDecNumber{2}\spanPunctuation{,} \spanDecNumber{3}\spanPunctuation{\symbol{93}} \spanOperator{*} \spanOperator{\symbol{64}}\spanPunctuation{\symbol{91}}\spanDecNumber{1}\spanPunctuation{,} \spanDecNumber{2}\spanPunctuation{,} \spanDecNumber{3}\spanPunctuation{\symbol{93}} \spanOperator{==} \spanOperator{\symbol{64}}\spanPunctuation{\symbol{91}}\spanDecNumber{1}\spanPunctuation{,} \spanDecNumber{4}\spanPunctuation{,} \spanDecNumber{9}\spanPunctuation{\symbol{93}}\spanPunctuation{)}
\end{rstpre}
The above module exports \texttt{x} and \texttt{*}, but not \texttt{y}.

A module's top-level statements are executed at the start of the program. This can be used to initialize complex data structures for example.

Each module has a special magic constant \texttt{isMainModule} that is true if the module is compiled as the main file. This is very useful to embed tests within the module as shown by the above example.

A symbol of a module \emph{can} be \emph{qualified} with the \texttt{module.symbol} syntax. And if a symbol is ambiguous, it \emph{must} be qualified. A symbol is ambiguous if it is defined in two (or more) different modules and both modules are imported by a third one:

\begin{rstpre}
\spanComment{\# Module A}
\spanKeyword{var} \spanIdentifier{x}\spanOperator{*:} \spanIdentifier{string}
\end{rstpre}
\begin{rstpre}
\spanComment{\# Module B}
\spanKeyword{var} \spanIdentifier{x}\spanOperator{*:} \spanIdentifier{int}
\end{rstpre}
\begin{rstpre}
\spanComment{\# Module C}
\spanKeyword{import} \spanIdentifier{A}\spanPunctuation{,} \spanIdentifier{B}
\spanIdentifier{write}\spanPunctuation{(}\spanIdentifier{stdout}\spanPunctuation{,} \spanIdentifier{x}\spanPunctuation{)} \spanComment{\# error: x is ambiguous}
\spanIdentifier{write}\spanPunctuation{(}\spanIdentifier{stdout}\spanPunctuation{,} \spanIdentifier{A}\spanOperator{.}\spanIdentifier{x}\spanPunctuation{)} \spanComment{\# okay: qualifier used}

\spanKeyword{var} \spanIdentifier{x} \spanOperator{=} \spanDecNumber{4}
\spanIdentifier{write}\spanPunctuation{(}\spanIdentifier{stdout}\spanPunctuation{,} \spanIdentifier{x}\spanPunctuation{)} \spanComment{\# not ambiguous: uses the module C's x}
\end{rstpre}
But this rule does not apply to procedures or iterators. Here the overloading rules apply:

\begin{rstpre}
\spanComment{\# Module A}
\spanKeyword{proc} \spanIdentifier{x}\spanOperator{*}\spanPunctuation{(}\spanIdentifier{a}\spanPunctuation{:} \spanIdentifier{int}\spanPunctuation{)}\spanPunctuation{:} \spanIdentifier{string} \spanOperator{=} \spanOperator{\$}\spanIdentifier{a}
\end{rstpre}
\begin{rstpre}
\spanComment{\# Module B}
\spanKeyword{proc} \spanIdentifier{x}\spanOperator{*}\spanPunctuation{(}\spanIdentifier{a}\spanPunctuation{:} \spanIdentifier{string}\spanPunctuation{)}\spanPunctuation{:} \spanIdentifier{string} \spanOperator{=} \spanOperator{\$}\spanIdentifier{a}
\end{rstpre}
\begin{rstpre}
\spanComment{\# Module C}
\spanKeyword{import} \spanIdentifier{A}\spanPunctuation{,} \spanIdentifier{B}
\spanIdentifier{write}\spanPunctuation{(}\spanIdentifier{stdout}\spanPunctuation{,} \spanIdentifier{x}\spanPunctuation{(}\spanDecNumber{3}\spanPunctuation{)}\spanPunctuation{)}   \spanComment{\# no error: A.x is called}
\spanIdentifier{write}\spanPunctuation{(}\spanIdentifier{stdout}\spanPunctuation{,} \spanIdentifier{x}\spanPunctuation{(}\spanStringLit{""}\spanPunctuation{)}\spanPunctuation{)}  \spanComment{\# no error: B.x is called}

\spanKeyword{proc} \spanIdentifier{x}\spanOperator{*}\spanPunctuation{(}\spanIdentifier{a}\spanPunctuation{:} \spanIdentifier{int}\spanPunctuation{)}\spanPunctuation{:} \spanIdentifier{string} \spanOperator{=} \spanKeyword{discard}
\spanIdentifier{write}\spanPunctuation{(}\spanIdentifier{stdout}\spanPunctuation{,} \spanIdentifier{x}\spanPunctuation{(}\spanDecNumber{3}\spanPunctuation{)}\spanPunctuation{)}   \spanComment{\# ambiguous: which \symbol{96}x\symbol{96} is to call?}
\end{rstpre}
\rsthB{Excluding symbols}\label{modules-excluding-symbols}
The normal \texttt{import} statement will bring in all exported symbols. These can be limited by naming symbols which should be excluded with the \texttt{except} qualifier.

\begin{rstpre}
\spanKeyword{import} \spanIdentifier{mymodule} \spanKeyword{except} \spanIdentifier{y}
\end{rstpre}
\rsthB{From statement}\label{modules-from-statement}
We have already seen the simple \texttt{import} statement that just imports all exported symbols. An alternative that only imports listed symbols is the \texttt{from import} statement:

\begin{rstpre}
\spanKeyword{from} \spanIdentifier{mymodule} \spanKeyword{import} \spanIdentifier{x}\spanPunctuation{,} \spanIdentifier{y}\spanPunctuation{,} \spanIdentifier{z}
\end{rstpre}
The \texttt{from} statement can also force namespace qualification on symbols, thereby making symbols available, but needing to be qualified to be used.

\begin{rstpre}
\spanKeyword{from} \spanIdentifier{mymodule} \spanKeyword{import} \spanIdentifier{x}\spanPunctuation{,} \spanIdentifier{y}\spanPunctuation{,} \spanIdentifier{z}

\spanIdentifier{x}\spanPunctuation{(}\spanPunctuation{)}           \spanComment{\# use x without any qualification}
\end{rstpre}
\begin{rstpre}
\spanKeyword{from} \spanIdentifier{mymodule} \spanKeyword{import} \spanKeyword{nil}

\spanIdentifier{mymodule}\spanOperator{.}\spanIdentifier{x}\spanPunctuation{(}\spanPunctuation{)}  \spanComment{\# must qualify x with the module name as prefix}

\spanIdentifier{x}\spanPunctuation{(}\spanPunctuation{)}           \spanComment{\# using x here without qualification is a compile error}
\end{rstpre}
Since module names are generally long to be descriptive, you can also define a shorter alias to use when qualifying symbols.

\begin{rstpre}
\spanKeyword{from} \spanIdentifier{mymodule} \spanKeyword{as} \spanIdentifier{m} \spanKeyword{import} \spanKeyword{nil}

\spanIdentifier{m}\spanOperator{.}\spanIdentifier{x}\spanPunctuation{(}\spanPunctuation{)}         \spanComment{\# m is aliasing mymodule}
\end{rstpre}
\rsthB{Include statement}\label{modules-include-statement}
The \texttt{include} statement does something fundamentally different than importing a module: it merely includes the contents of a file. The \texttt{include} statement is useful to split up a large module into several files:

\begin{rstpre}
\spanKeyword{include} \spanIdentifier{fileA}\spanPunctuation{,} \spanIdentifier{fileB}\spanPunctuation{,} \spanIdentifier{fileC}
\end{rstpre}
\rsthA{Part 2}\label{part-2}
So, now that we are done with the basics, let's see what Nim offers apart from a nice syntax for procedural programming: \href{tut2.html}{Part II}




\end{document}
