% This file was generated by Nim.
% Generated: 2020-08-22 06:37:23 UTC
\documentclass[a4paper]{article}
\usepackage[left=2cm,right=3cm,top=3cm,bottom=3cm]{geometry}
\usepackage[utf8]{inputenc}
\usepackage[T1]{fontenc}
\usepackage{graphicx}
\usepackage{lmodern}
\usepackage{fancyvrb, courier}
\usepackage{tabularx}
\usepackage{hyperref}

\begin{document}
\title{API naming design }
\author{}

\tolerance 1414 
\hbadness 1414 
\emergencystretch 1.5em 
\hfuzz 0.3pt 
\widowpenalty=10000 
\vfuzz \hfuzz 
\raggedbottom 

\maketitle

\newenvironment{rstpre}{\VerbatimEnvironment\begingroup\begin{Verbatim}[fontsize=\footnotesize , commandchars=\\\{\}]}{\end{Verbatim}\endgroup}

% to pack tabularx into a new environment, special syntax is needed :-(
\newenvironment{rsttab}[1]{\tabularx{\linewidth}{#1}}{\endtabularx}

\newcommand{\rstsub}[1]{\raisebox{-0.5ex}{\scriptsize{#1}}}
\newcommand{\rstsup}[1]{\raisebox{0.5ex}{\scriptsize{#1}}}

\newcommand{\rsthA}[1]{\section{#1}}
\newcommand{\rsthB}[1]{\subsection{#1}}
\newcommand{\rsthC}[1]{\subsubsection{#1}}
\newcommand{\rsthD}[1]{\paragraph{#1}}
\newcommand{\rsthE}[1]{\paragraph{#1}}

\newcommand{\rstovA}[1]{\section*{#1}}
\newcommand{\rstovB}[1]{\subsection*{#1}}
\newcommand{\rstovC}[1]{\subsubsection*{#1}}
\newcommand{\rstovD}[1]{\paragraph*{#1}}
\newcommand{\rstovE}[1]{\paragraph*{#1}}

% Syntax highlighting:
\newcommand{\spanDecNumber}[1]{#1}
\newcommand{\spanBinNumber}[1]{#1}
\newcommand{\spanHexNumber}[1]{#1}
\newcommand{\spanOctNumber}[1]{#1}
\newcommand{\spanFloatNumber}[1]{#1}
\newcommand{\spanIdentifier}[1]{#1}
\newcommand{\spanKeyword}[1]{\textbf{#1}}
\newcommand{\spanStringLit}[1]{#1}
\newcommand{\spanLongStringLit}[1]{#1}
\newcommand{\spanCharLit}[1]{#1}
\newcommand{\spanEscapeSequence}[1]{#1}
\newcommand{\spanOperator}[1]{#1}
\newcommand{\spanPunctuation}[1]{#1}
\newcommand{\spanComment}[1]{\emph{#1}}
\newcommand{\spanLongComment}[1]{\emph{#1}}
\newcommand{\spanRegularExpression}[1]{#1}
\newcommand{\spanTagStart}[1]{#1}
\newcommand{\spanTagEnd}[1]{#1}
\newcommand{\spanKey}[1]{#1}
\newcommand{\spanValue}[1]{#1}
\newcommand{\spanRawData}[1]{#1}
\newcommand{\spanAssembler}[1]{#1}
\newcommand{\spanPreprocessor}[1]{#1}
\newcommand{\spanDirective}[1]{#1}
\newcommand{\spanCommand}[1]{#1}
\newcommand{\spanRule}[1]{#1}
\newcommand{\spanHyperlink}[1]{#1}
\newcommand{\spanLabel}[1]{#1}
\newcommand{\spanReference}[1]{#1}
\newcommand{\spanOther}[1]{#1}
\newcommand{\spantok}[1]{\frame{#1}}

The API is designed to be \textbf{easy to use} and consistent. Ease of use is measured by the number of calls to achieve a concrete high level action.

\rsthA{Naming scheme}\label{naming-scheme}
The library uses a simple naming scheme that makes use of common abbreviations to keep the names short but meaningful. Since version 0.8.2 many symbols have been renamed to fit this scheme. The ultimate goal is that the programmer can \emph{guess} a name.

\begin{table}\begin{rsttab}{|X|X|X|X|X|X|X|X|X|X|X|X|X|X|X|X|X|X|X|X|X|X|X|X|X|X|X|X|X|X|X|X|X|X|X|X|X|X|X|X|X|X|X|X|X|X|X|X|}
\hline
\textbf{English word} & \textbf{To use} & \textbf{Notes}\\
\hline
initialize & initT  & \texttt{init} is used to create a value type \texttt{T}\\
\hline
new & newP  & \texttt{new} is used to create a reference type \texttt{P}\\
\hline
find & find & should return the position where something was found; for a bool result use \texttt{contains}\\
\hline
contains & contains & often short for \texttt{find() >= 0}\\
\hline
append & add & use \texttt{add} instead of \texttt{append}\\
\hline
compare & cmp & should return an int with the \texttt{< 0} \texttt{== 0} or \texttt{> 0} semantics; for a bool result use \texttt{sameXYZ}\\
\hline
put & put, \texttt{\symbol{91}\symbol{93}=} & consider overloading \texttt{\symbol{91}\symbol{93}=} for put\\
\hline
get & get, \texttt{\symbol{91}\symbol{93}} & consider overloading \texttt{\symbol{91}\symbol{93}} for get; consider to not use \texttt{get} as a prefix: \texttt{len} instead of \texttt{getLen}\\
\hline
length & len & also used for \emph{number of elements}\\
\hline
size & size, len & size should refer to a byte size\\
\hline
capacity & cap & \\
\hline
memory & mem & implies a low-level operation\\
\hline
items & items & default iterator over a collection\\
\hline
pairs & pairs & iterator over (key, value) pairs\\
\hline
delete & delete, del & del is supposed to be faster than delete, because it does not keep the order; delete keeps the order\\
\hline
remove & delete, del & inconsistent right now\\
\hline
remove-and-return & pop & \texttt{Table}/\texttt{TableRef} alias to \texttt{take}\\
\hline
include & incl & \\
\hline
exclude & excl & \\
\hline
command & cmd & \\
\hline
execute & exec & \\
\hline
environment & env & \\
\hline
variable & var & \\
\hline
value & value, val  & val is preferred, inconsistent right now\\
\hline
executable & exe & \\
\hline
directory & dir & \\
\hline
path & path & path is the string "/usr/bin" (for example), dir is the content of "/usr/bin"; inconsistent right now\\
\hline
extension & ext & \\
\hline
separator & sep & \\
\hline
column & col, column  & col is preferred, inconsistent right now\\
\hline
application & app & \\
\hline
configuration & cfg & \\
\hline
message & msg & \\
\hline
argument & arg & \\
\hline
object & obj & \\
\hline
parameter & param & \\
\hline
operator & opr & \\
\hline
procedure & proc & \\
\hline
function & func & \\
\hline
coordinate & coord & \\
\hline
rectangle & rect & \\
\hline
point & point & \\
\hline
symbol & sym & \\
\hline
literal & lit & \\
\hline
string & str & \\
\hline
identifier & ident & \\
\hline
indentation & indent & \\
\hline
\end{rsttab}\end{table}


\end{document}
