% This file was generated by Nim.
% Generated: 2020-08-22 06:37:34 UTC
\documentclass[a4paper]{article}
\usepackage[left=2cm,right=3cm,top=3cm,bottom=3cm]{geometry}
\usepackage[utf8]{inputenc}
\usepackage[T1]{fontenc}
\usepackage{graphicx}
\usepackage{lmodern}
\usepackage{fancyvrb, courier}
\usepackage{tabularx}
\usepackage{hyperref}

\begin{document}
\title{Nimfix User Guide 1.3.5}
\author{Andreas Rumpf}

\tolerance 1414 
\hbadness 1414 
\emergencystretch 1.5em 
\hfuzz 0.3pt 
\widowpenalty=10000 
\vfuzz \hfuzz 
\raggedbottom 

\maketitle

\newenvironment{rstpre}{\VerbatimEnvironment\begingroup\begin{Verbatim}[fontsize=\footnotesize , commandchars=\\\{\}]}{\end{Verbatim}\endgroup}

% to pack tabularx into a new environment, special syntax is needed :-(
\newenvironment{rsttab}[1]{\tabularx{\linewidth}{#1}}{\endtabularx}

\newcommand{\rstsub}[1]{\raisebox{-0.5ex}{\scriptsize{#1}}}
\newcommand{\rstsup}[1]{\raisebox{0.5ex}{\scriptsize{#1}}}

\newcommand{\rsthA}[1]{\section{#1}}
\newcommand{\rsthB}[1]{\subsection{#1}}
\newcommand{\rsthC}[1]{\subsubsection{#1}}
\newcommand{\rsthD}[1]{\paragraph{#1}}
\newcommand{\rsthE}[1]{\paragraph{#1}}

\newcommand{\rstovA}[1]{\section*{#1}}
\newcommand{\rstovB}[1]{\subsection*{#1}}
\newcommand{\rstovC}[1]{\subsubsection*{#1}}
\newcommand{\rstovD}[1]{\paragraph*{#1}}
\newcommand{\rstovE}[1]{\paragraph*{#1}}

% Syntax highlighting:
\newcommand{\spanDecNumber}[1]{#1}
\newcommand{\spanBinNumber}[1]{#1}
\newcommand{\spanHexNumber}[1]{#1}
\newcommand{\spanOctNumber}[1]{#1}
\newcommand{\spanFloatNumber}[1]{#1}
\newcommand{\spanIdentifier}[1]{#1}
\newcommand{\spanKeyword}[1]{\textbf{#1}}
\newcommand{\spanStringLit}[1]{#1}
\newcommand{\spanLongStringLit}[1]{#1}
\newcommand{\spanCharLit}[1]{#1}
\newcommand{\spanEscapeSequence}[1]{#1}
\newcommand{\spanOperator}[1]{#1}
\newcommand{\spanPunctuation}[1]{#1}
\newcommand{\spanComment}[1]{\emph{#1}}
\newcommand{\spanLongComment}[1]{\emph{#1}}
\newcommand{\spanRegularExpression}[1]{#1}
\newcommand{\spanTagStart}[1]{#1}
\newcommand{\spanTagEnd}[1]{#1}
\newcommand{\spanKey}[1]{#1}
\newcommand{\spanValue}[1]{#1}
\newcommand{\spanRawData}[1]{#1}
\newcommand{\spanAssembler}[1]{#1}
\newcommand{\spanPreprocessor}[1]{#1}
\newcommand{\spanDirective}[1]{#1}
\newcommand{\spanCommand}[1]{#1}
\newcommand{\spanRule}[1]{#1}
\newcommand{\spanHyperlink}[1]{#1}
\newcommand{\spanLabel}[1]{#1}
\newcommand{\spanReference}[1]{#1}
\newcommand{\spanOther}[1]{#1}
\newcommand{\spantok}[1]{\frame{#1}}

\textbf{WARNING}: Nimfix is currently beta-quality.

Nimfix is a tool to help you upgrade from Nimrod (<= version 0.9.6) to Nim (=> version 0.10.0).

It performs 3 different actions:

\begin{enumerate}\item It makes your code case consistent.
\item It renames every symbol that has a deprecation rule. So if a module has a rule \texttt{\symbol{123}.deprecated: \symbol{91}TFoo: Foo\symbol{93}.\symbol{125}} then \texttt{TFoo} is replaced by \texttt{Foo}.
\item It can also check that your identifiers adhere to the official style guide and optionally modify them to do so (via \texttt{--styleCheck:auto}).
\end{enumerate}
Note that \texttt{nimfix} defaults to \textbf{overwrite} your code unless you use \texttt{--overwriteFiles:off}! But hey, if you do not use a version control system by this day and age, your project is already in big trouble.

\rsthA{Installation}\label{installation}
Nimfix is part of the compiler distribution. Compile via:\begin{rstpre}

nim c compiler/nimfix/nimfix.nim
mv compiler/nimfix/nimfix bin
\end{rstpre}


Or on windows:\begin{rstpre}

nim c compiler\symbol{92}nimfix\symbol{92}nimfix.nim
move compiler\symbol{92}nimfix\symbol{92}nimfix.exe bin
\end{rstpre}


\rsthA{Usage}\label{usage}
\begin{description}\item[Usage:] nimfix \symbol{91}options\symbol{93} projectfile.nim
\end{description}
Options:

\begin{quote}\begin{description}
\item[--overwriteFiles:on|off] overwrite the original nim files. DEFAULT is ON!
\item[--wholeProject] overwrite every processed file.
\item[--checkExtern:on|off] style check also extern names
\item[--styleCheck:on|off|auto] performs style checking for identifiers and suggests an alternative spelling; 'auto' corrects the spelling.
\end{description}
In addition, all command line options of Nim are supported.

\end{quote}



\end{document}
