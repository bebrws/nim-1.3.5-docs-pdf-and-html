% This file was generated by Nim.
% Generated: 2020-08-22 06:37:34 UTC
\documentclass[a4paper]{article}
\usepackage[left=2cm,right=3cm,top=3cm,bottom=3cm]{geometry}
\usepackage[utf8]{inputenc}
\usepackage[T1]{fontenc}
\usepackage{graphicx}
\usepackage{lmodern}
\usepackage{fancyvrb, courier}
\usepackage{tabularx}
\usepackage{hyperref}

\begin{document}
\title{nimgrep User's manual 0.9}
\author{Andreas Rumpf}

\tolerance 1414 
\hbadness 1414 
\emergencystretch 1.5em 
\hfuzz 0.3pt 
\widowpenalty=10000 
\vfuzz \hfuzz 
\raggedbottom 

\maketitle

\newenvironment{rstpre}{\VerbatimEnvironment\begingroup\begin{Verbatim}[fontsize=\footnotesize , commandchars=\\\{\}]}{\end{Verbatim}\endgroup}

% to pack tabularx into a new environment, special syntax is needed :-(
\newenvironment{rsttab}[1]{\tabularx{\linewidth}{#1}}{\endtabularx}

\newcommand{\rstsub}[1]{\raisebox{-0.5ex}{\scriptsize{#1}}}
\newcommand{\rstsup}[1]{\raisebox{0.5ex}{\scriptsize{#1}}}

\newcommand{\rsthA}[1]{\section{#1}}
\newcommand{\rsthB}[1]{\subsection{#1}}
\newcommand{\rsthC}[1]{\subsubsection{#1}}
\newcommand{\rsthD}[1]{\paragraph{#1}}
\newcommand{\rsthE}[1]{\paragraph{#1}}

\newcommand{\rstovA}[1]{\section*{#1}}
\newcommand{\rstovB}[1]{\subsection*{#1}}
\newcommand{\rstovC}[1]{\subsubsection*{#1}}
\newcommand{\rstovD}[1]{\paragraph*{#1}}
\newcommand{\rstovE}[1]{\paragraph*{#1}}

% Syntax highlighting:
\newcommand{\spanDecNumber}[1]{#1}
\newcommand{\spanBinNumber}[1]{#1}
\newcommand{\spanHexNumber}[1]{#1}
\newcommand{\spanOctNumber}[1]{#1}
\newcommand{\spanFloatNumber}[1]{#1}
\newcommand{\spanIdentifier}[1]{#1}
\newcommand{\spanKeyword}[1]{\textbf{#1}}
\newcommand{\spanStringLit}[1]{#1}
\newcommand{\spanLongStringLit}[1]{#1}
\newcommand{\spanCharLit}[1]{#1}
\newcommand{\spanEscapeSequence}[1]{#1}
\newcommand{\spanOperator}[1]{#1}
\newcommand{\spanPunctuation}[1]{#1}
\newcommand{\spanComment}[1]{\emph{#1}}
\newcommand{\spanLongComment}[1]{\emph{#1}}
\newcommand{\spanRegularExpression}[1]{#1}
\newcommand{\spanTagStart}[1]{#1}
\newcommand{\spanTagEnd}[1]{#1}
\newcommand{\spanKey}[1]{#1}
\newcommand{\spanValue}[1]{#1}
\newcommand{\spanRawData}[1]{#1}
\newcommand{\spanAssembler}[1]{#1}
\newcommand{\spanPreprocessor}[1]{#1}
\newcommand{\spanDirective}[1]{#1}
\newcommand{\spanCommand}[1]{#1}
\newcommand{\spanRule}[1]{#1}
\newcommand{\spanHyperlink}[1]{#1}
\newcommand{\spanLabel}[1]{#1}
\newcommand{\spanReference}[1]{#1}
\newcommand{\spanOther}[1]{#1}
\newcommand{\spantok}[1]{\frame{#1}}

Nimgrep is a command line tool for search\&replace tasks. It can search for regex or peg patterns and can search whole directories at once. User confirmation for every single replace operation can be requested.

Nimgrep has particularly good support for Nim's eccentric \emph{style insensitivity}. Apart from that it is a generic text manipulation tool.

\rsthA{Installation}\label{installation}
Compile nimgrep with the command:\begin{rstpre}

nim c -d:release tools/nimgrep.nim
\end{rstpre}


And copy the executable somewhere in your \texttt{\$PATH}.

\rsthA{Command line switches}\label{command-line-switches}
\begin{description}\item[Usage:] nimgrep \symbol{91}options\symbol{93} \symbol{91}pattern\symbol{93} \symbol{91}replacement\symbol{93} (file/directory)*
\item[Options:] \begin{description}
\item[--find, -f] find the pattern (default)
\item[--replace, -r] replace the pattern
\item[--peg] pattern is a peg
\item[--re] pattern is a regular expression (default); extended syntax for the regular expression is always turned on
\item[--recursive] process directories recursively
\item[--confirm] confirm each occurrence/replacement; there is a chance to abort any time without touching the file
\item[--stdin] read pattern from stdin (to avoid the shell's confusing quoting rules)
\item[--word, -w] the match should have word boundaries (buggy for pegs!)
\item[--ignoreCase, -i] be case insensitive
\item[--ignoreStyle, -y] be style insensitive
\item[--ext:EX1|EX2|...] only search the files with the given extension(s)
\item[--verbose] be verbose: list every processed file
\item[--help, -h] shows this help
\item[--version, -v] shows the version
\end{description}

\end{description}



\end{document}
