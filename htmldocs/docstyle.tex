% This file was generated by Nim.
% Generated: 2020-08-22 06:37:23 UTC
\documentclass[a4paper]{article}
\usepackage[left=2cm,right=3cm,top=3cm,bottom=3cm]{geometry}
\usepackage[utf8]{inputenc}
\usepackage[T1]{fontenc}
\usepackage{graphicx}
\usepackage{lmodern}
\usepackage{fancyvrb, courier}
\usepackage{tabularx}
\usepackage{hyperref}

\begin{document}
\title{docstyle }
\author{}

\tolerance 1414 
\hbadness 1414 
\emergencystretch 1.5em 
\hfuzz 0.3pt 
\widowpenalty=10000 
\vfuzz \hfuzz 
\raggedbottom 

\maketitle

\newenvironment{rstpre}{\VerbatimEnvironment\begingroup\begin{Verbatim}[fontsize=\footnotesize , commandchars=\\\{\}]}{\end{Verbatim}\endgroup}

% to pack tabularx into a new environment, special syntax is needed :-(
\newenvironment{rsttab}[1]{\tabularx{\linewidth}{#1}}{\endtabularx}

\newcommand{\rstsub}[1]{\raisebox{-0.5ex}{\scriptsize{#1}}}
\newcommand{\rstsup}[1]{\raisebox{0.5ex}{\scriptsize{#1}}}

\newcommand{\rsthA}[1]{\section{#1}}
\newcommand{\rsthB}[1]{\subsection{#1}}
\newcommand{\rsthC}[1]{\subsubsection{#1}}
\newcommand{\rsthD}[1]{\paragraph{#1}}
\newcommand{\rsthE}[1]{\paragraph{#1}}

\newcommand{\rstovA}[1]{\section*{#1}}
\newcommand{\rstovB}[1]{\subsection*{#1}}
\newcommand{\rstovC}[1]{\subsubsection*{#1}}
\newcommand{\rstovD}[1]{\paragraph*{#1}}
\newcommand{\rstovE}[1]{\paragraph*{#1}}

% Syntax highlighting:
\newcommand{\spanDecNumber}[1]{#1}
\newcommand{\spanBinNumber}[1]{#1}
\newcommand{\spanHexNumber}[1]{#1}
\newcommand{\spanOctNumber}[1]{#1}
\newcommand{\spanFloatNumber}[1]{#1}
\newcommand{\spanIdentifier}[1]{#1}
\newcommand{\spanKeyword}[1]{\textbf{#1}}
\newcommand{\spanStringLit}[1]{#1}
\newcommand{\spanLongStringLit}[1]{#1}
\newcommand{\spanCharLit}[1]{#1}
\newcommand{\spanEscapeSequence}[1]{#1}
\newcommand{\spanOperator}[1]{#1}
\newcommand{\spanPunctuation}[1]{#1}
\newcommand{\spanComment}[1]{\emph{#1}}
\newcommand{\spanLongComment}[1]{\emph{#1}}
\newcommand{\spanRegularExpression}[1]{#1}
\newcommand{\spanTagStart}[1]{#1}
\newcommand{\spanTagEnd}[1]{#1}
\newcommand{\spanKey}[1]{#1}
\newcommand{\spanValue}[1]{#1}
\newcommand{\spanRawData}[1]{#1}
\newcommand{\spanAssembler}[1]{#1}
\newcommand{\spanPreprocessor}[1]{#1}
\newcommand{\spanDirective}[1]{#1}
\newcommand{\spanCommand}[1]{#1}
\newcommand{\spanRule}[1]{#1}
\newcommand{\spanHyperlink}[1]{#1}
\newcommand{\spanLabel}[1]{#1}
\newcommand{\spanReference}[1]{#1}
\newcommand{\spanOther}[1]{#1}
\newcommand{\spantok}[1]{\frame{#1}}

\rsthA{Documentation Style}\label{documentation-style}
\rsthB{General Guidelines}\label{general-guidelines}
\begin{itemize}\item See also \href{https://nim-lang.github.io/Nim/nep1.html}{nep1} which should probably be merged here.
\item Authors should document anything that is exported; documentation for private procs can be useful too (visible via \texttt{nim doc --docInternal foo.nim}).
\item Within documentation, a period (\texttt{.}) should follow each sentence (or sentence fragment) in a comment block. The documentation may be limited to one sentence fragment, but if multiple sentences are within the documentation, each sentence after the first should be complete and in present tense.
\item Documentation is parsed as a custom ReStructuredText (RST) with partial markdown support.
\item In nim sources, prefer single backticks to double backticks since it's simpler and \texttt{nim doc} supports it (even in rst files with \texttt{nim rst2html}).
\item In nim sources, for links, prefer \texttt{\symbol{91}link text\symbol{93}(link.html)} to \begin{rstpre}
 \symbol{96}link text<link.html>\symbol{96}\_ 
\end{rstpre}
 since the syntax is simpler and markdown is more common (likewise, \texttt{nim rst2html} also supports it in rst files).
\end{itemize}
\begin{rstpre}
\spanKeyword{proc} \spanIdentifier{someproc}\spanOperator{*}\spanPunctuation{(}\spanIdentifier{s}\spanPunctuation{:} \spanIdentifier{string}\spanPunctuation{,} \spanIdentifier{foo}\spanPunctuation{:} \spanIdentifier{int}\spanPunctuation{)} \spanOperator{=}
  \spanComment{\#\# Use single backticks for inline code, eg: \symbol{96}s\symbol{96} or \symbol{96}someExpr(true)\symbol{96}.}
  \spanComment{\#\# Use a backlash to follow with alphanumeric char: \symbol{96}int8\symbol{96}\symbol{92}s are great.}
\end{rstpre}
\rsthB{Module-level documentation}\label{moduleminuslevel-documentation}
Documentation of a module is placed at the top of the module itself. Each line of documentation begins with double hashes (\texttt{\#\#}). Sometimes \texttt{\#\#\symbol{91} multiline docs containing code \symbol{93}\#\#} is preferable, see \texttt{lib/pure/times.nim}. Code samples are encouraged, and should follow the general RST syntax:

\begin{rstpre}
\spanComment{\#\# The \symbol{96}universe\symbol{96} module computes the answer to life, the universe, and everything.}
\spanComment{\#\#}
\spanComment{\#\# .. code-block::}
\spanComment{\#\#  doAssert computeAnswerString() == 42}
\end{rstpre}
Within this top-level comment, you can indicate the authorship and copyright of the code, which will be featured in the produced documentation.

\begin{rstpre}
\spanComment{\#\# This is the best module ever. It provides answers to everything!}
\spanComment{\#\#}
\spanComment{\#\# :Author: Steve McQueen}
\spanComment{\#\# :Copyright: 1965}
\spanComment{\#\#}
\end{rstpre}
Leave a space between the last line of top-level documentation and the beginning of Nim code (the imports, etc.).

\rsthB{Procs, Templates, Macros, Converters, and Iterators}\label{procs-templates-macros-converters-and-iterators}
The documentation of a procedure should begin with a capital letter and should be in present tense. Variables referenced in the documentation should be surrounded by single tick marks:

\begin{rstpre}
\spanKeyword{proc} \spanIdentifier{example1}\spanOperator{*}\spanPunctuation{(}\spanIdentifier{x}\spanPunctuation{:} \spanIdentifier{int}\spanPunctuation{)} \spanOperator{=}
  \spanComment{\#\# Prints the value of \symbol{96}x\symbol{96}.}
  \spanIdentifier{echo} \spanIdentifier{x}
\end{rstpre}
Whenever an example of usage would be helpful to the user, you should include one within the documentation in RST format as below.

\begin{rstpre}
\spanKeyword{proc} \spanIdentifier{addThree}\spanOperator{*}\spanPunctuation{(}\spanIdentifier{x}\spanPunctuation{,} \spanIdentifier{y}\spanPunctuation{,} \spanIdentifier{z}\spanPunctuation{:} \spanIdentifier{int8}\spanPunctuation{)}\spanPunctuation{:} \spanIdentifier{int} \spanOperator{=}
  \spanComment{\#\# Adds three \symbol{96}int8\symbol{96} values, treating them as unsigned and}
  \spanComment{\#\# truncating the result.}
  \spanComment{\#\#}
  \spanComment{\#\# .. code-block::}
  \spanComment{\#\#  \# things that aren't suitable for a \symbol{96}runnableExamples\symbol{96} go in code-block:}
  \spanComment{\#\#  echo execCmdEx("git pull")}
  \spanComment{\#\#  drawOnScreen()}
  \spanIdentifier{runnableExamples}\spanPunctuation{:}
    \spanComment{\# \symbol{96}runnableExamples\symbol{96} is usually preferred to \symbol{96}code-block\symbol{96}, when possible.}
    \spanIdentifier{doAssert} \spanIdentifier{addThree}\spanPunctuation{(}\spanDecNumber{3}\spanPunctuation{,} \spanDecNumber{125}\spanPunctuation{,} \spanDecNumber{6}\spanPunctuation{)} \spanOperator{==} \spanOperator{-}\spanDecNumber{122}
  \spanIdentifier{result} \spanOperator{=} \spanIdentifier{x} \spanOperator{+\%} \spanIdentifier{y} \spanOperator{+\%} \spanIdentifier{z}
\end{rstpre}
The commands \texttt{nim doc} and \texttt{nim doc2} will then correctly syntax highlight the Nim code within the documentation.

\rsthB{Types}\label{types}
Exported types should also be documented. This documentation can also contain code samples, but those are better placed with the functions to which they refer.

\begin{rstpre}
\spanKeyword{type}
  \spanIdentifier{NamedQueue}\spanOperator{*}\spanPunctuation{\symbol{91}}\spanIdentifier{T}\spanPunctuation{\symbol{93}} \spanOperator{=} \spanKeyword{object} \spanComment{\#\# Provides a linked data structure with names}
                          \spanComment{\#\# throughout. It is named for convenience. I'm making}
                          \spanComment{\#\# this comment long to show how you can, too.}
    \spanIdentifier{name}\spanOperator{*:} \spanIdentifier{string} \spanComment{\#\# The name of the item}
    \spanIdentifier{val}\spanOperator{*:} \spanIdentifier{T} \spanComment{\#\# Its value}
    \spanIdentifier{next}\spanOperator{*:} \spanKeyword{ref} \spanIdentifier{NamedQueue}\spanPunctuation{\symbol{91}}\spanIdentifier{T}\spanPunctuation{\symbol{93}} \spanComment{\#\# The next item in the queue}
\end{rstpre}
You have some flexibility when placing the documentation:

\begin{rstpre}
\spanKeyword{type}
  \spanIdentifier{NamedQueue}\spanOperator{*}\spanPunctuation{\symbol{91}}\spanIdentifier{T}\spanPunctuation{\symbol{93}} \spanOperator{=} \spanKeyword{object}
    \spanComment{\#\# Provides a linked data structure with names}
    \spanComment{\#\# throughout. It is named for convenience. I'm making}
    \spanComment{\#\# this comment long to show how you can, too.}
    \spanIdentifier{name}\spanOperator{*:} \spanIdentifier{string} \spanComment{\#\# The name of the item}
    \spanIdentifier{val}\spanOperator{*:} \spanIdentifier{T} \spanComment{\#\# Its value}
    \spanIdentifier{next}\spanOperator{*:} \spanKeyword{ref} \spanIdentifier{NamedQueue}\spanPunctuation{\symbol{91}}\spanIdentifier{T}\spanPunctuation{\symbol{93}} \spanComment{\#\# The next item in the queue}
\end{rstpre}
Make sure to place the documentation beside or within the object.

\begin{rstpre}
\spanKeyword{type}
  \spanComment{\#\# Bad: this documentation disappears because it annotates the \symbol{96}\symbol{96}type\symbol{96}\symbol{96} keyword}
  \spanComment{\#\# above, not \symbol{96}\symbol{96}NamedQueue\symbol{96}\symbol{96}.}
  \spanIdentifier{NamedQueue}\spanOperator{*}\spanPunctuation{\symbol{91}}\spanIdentifier{T}\spanPunctuation{\symbol{93}} \spanOperator{=} \spanKeyword{object}
    \spanIdentifier{name}\spanOperator{*:} \spanIdentifier{string} \spanComment{\#\# This becomes the main documentation for the object, which}
                  \spanComment{\#\# is not what we want.}
    \spanIdentifier{val}\spanOperator{*:} \spanIdentifier{T} \spanComment{\#\# Its value}
    \spanIdentifier{next}\spanOperator{*:} \spanKeyword{ref} \spanIdentifier{NamedQueue}\spanPunctuation{\symbol{91}}\spanIdentifier{T}\spanPunctuation{\symbol{93}} \spanComment{\#\# The next item in the queue}
\end{rstpre}
\rsthB{Var, Let, and Const}\label{var-let-and-const}
When declaring module-wide constants and values, documentation is encouraged. The placement of doc comments is similar to the \texttt{type} sections.

\begin{rstpre}
\spanKeyword{const}
  \spanIdentifier{X}\spanOperator{*} \spanOperator{=} \spanDecNumber{42} \spanComment{\#\# An awesome number.}
  \spanIdentifier{SpreadArray}\spanOperator{*} \spanOperator{=} \spanPunctuation{\symbol{91}}
    \spanPunctuation{\symbol{91}}\spanDecNumber{1}\spanPunctuation{,}\spanDecNumber{2}\spanPunctuation{,}\spanDecNumber{3}\spanPunctuation{\symbol{93}}\spanPunctuation{,}
    \spanPunctuation{\symbol{91}}\spanDecNumber{2}\spanPunctuation{,}\spanDecNumber{3}\spanPunctuation{,}\spanDecNumber{1}\spanPunctuation{\symbol{93}}\spanPunctuation{,}
    \spanPunctuation{\symbol{91}}\spanDecNumber{3}\spanPunctuation{,}\spanDecNumber{1}\spanPunctuation{,}\spanDecNumber{2}\spanPunctuation{\symbol{93}}\spanPunctuation{,}
  \spanPunctuation{\symbol{93}} \spanComment{\#\# Doc comment for \symbol{96}\symbol{96}SpreadArray\symbol{96}\symbol{96}.}
\end{rstpre}
Placement of comments in other areas is usually allowed, but will not become part of the documentation output and should therefore be prefaced by a single hash (\texttt{\#}).

\begin{rstpre}
\spanKeyword{const}
  \spanIdentifier{BadMathVals}\spanOperator{*} \spanOperator{=} \spanPunctuation{\symbol{91}}
    \spanFloatNumber{3.14}\spanPunctuation{,} \spanComment{\# pi}
    \spanFloatNumber{2.72}\spanPunctuation{,} \spanComment{\# e}
    \spanFloatNumber{0.58}\spanPunctuation{,} \spanComment{\# gamma}
  \spanPunctuation{\symbol{93}} \spanComment{\#\# A bunch of badly rounded values.}
\end{rstpre}
Nim supports Unicode in comments, so the above can be replaced with the following:

\begin{rstpre}
\spanKeyword{const}
  \spanIdentifier{BadMathVals}\spanOperator{*} \spanOperator{=} \spanPunctuation{\symbol{91}}
    \spanFloatNumber{3.14}\spanPunctuation{,} \spanComment{\# π}
    \spanFloatNumber{2.72}\spanPunctuation{,} \spanComment{\# e}
    \spanFloatNumber{0.58}\spanPunctuation{,} \spanComment{\# γ}
  \spanPunctuation{\symbol{93}} \spanComment{\#\# A bunch of badly rounded values (including π!).}
\end{rstpre}



\end{document}
