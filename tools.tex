% This file was generated by Nim.
% Generated: 2020-08-22 06:37:34 UTC
\documentclass[a4paper]{article}
\usepackage[left=2cm,right=3cm,top=3cm,bottom=3cm]{geometry}
\usepackage[utf8]{inputenc}
\usepackage[T1]{fontenc}
\usepackage{graphicx}
\usepackage{lmodern}
\usepackage{fancyvrb, courier}
\usepackage{tabularx}
\usepackage{hyperref}

\begin{document}
\title{Tools available with Nim }
\author{}

\tolerance 1414 
\hbadness 1414 
\emergencystretch 1.5em 
\hfuzz 0.3pt 
\widowpenalty=10000 
\vfuzz \hfuzz 
\raggedbottom 

\maketitle

\newenvironment{rstpre}{\VerbatimEnvironment\begingroup\begin{Verbatim}[fontsize=\footnotesize , commandchars=\\\{\}]}{\end{Verbatim}\endgroup}

% to pack tabularx into a new environment, special syntax is needed :-(
\newenvironment{rsttab}[1]{\tabularx{\linewidth}{#1}}{\endtabularx}

\newcommand{\rstsub}[1]{\raisebox{-0.5ex}{\scriptsize{#1}}}
\newcommand{\rstsup}[1]{\raisebox{0.5ex}{\scriptsize{#1}}}

\newcommand{\rsthA}[1]{\section{#1}}
\newcommand{\rsthB}[1]{\subsection{#1}}
\newcommand{\rsthC}[1]{\subsubsection{#1}}
\newcommand{\rsthD}[1]{\paragraph{#1}}
\newcommand{\rsthE}[1]{\paragraph{#1}}

\newcommand{\rstovA}[1]{\section*{#1}}
\newcommand{\rstovB}[1]{\subsection*{#1}}
\newcommand{\rstovC}[1]{\subsubsection*{#1}}
\newcommand{\rstovD}[1]{\paragraph*{#1}}
\newcommand{\rstovE}[1]{\paragraph*{#1}}

% Syntax highlighting:
\newcommand{\spanDecNumber}[1]{#1}
\newcommand{\spanBinNumber}[1]{#1}
\newcommand{\spanHexNumber}[1]{#1}
\newcommand{\spanOctNumber}[1]{#1}
\newcommand{\spanFloatNumber}[1]{#1}
\newcommand{\spanIdentifier}[1]{#1}
\newcommand{\spanKeyword}[1]{\textbf{#1}}
\newcommand{\spanStringLit}[1]{#1}
\newcommand{\spanLongStringLit}[1]{#1}
\newcommand{\spanCharLit}[1]{#1}
\newcommand{\spanEscapeSequence}[1]{#1}
\newcommand{\spanOperator}[1]{#1}
\newcommand{\spanPunctuation}[1]{#1}
\newcommand{\spanComment}[1]{\emph{#1}}
\newcommand{\spanLongComment}[1]{\emph{#1}}
\newcommand{\spanRegularExpression}[1]{#1}
\newcommand{\spanTagStart}[1]{#1}
\newcommand{\spanTagEnd}[1]{#1}
\newcommand{\spanKey}[1]{#1}
\newcommand{\spanValue}[1]{#1}
\newcommand{\spanRawData}[1]{#1}
\newcommand{\spanAssembler}[1]{#1}
\newcommand{\spanPreprocessor}[1]{#1}
\newcommand{\spanDirective}[1]{#1}
\newcommand{\spanCommand}[1]{#1}
\newcommand{\spanRule}[1]{#1}
\newcommand{\spanHyperlink}[1]{#1}
\newcommand{\spanLabel}[1]{#1}
\newcommand{\spanReference}[1]{#1}
\newcommand{\spanOther}[1]{#1}
\newcommand{\spantok}[1]{\frame{#1}}

The standard distribution ships with the following tools:

\begin{itemize}\item \href{hcr.html}{Hot code reloading}\\
The "Hot code reloading" feature is built into the compiler but has its own document explaining how it works.\\



\item \href{docgen.html}{Documentation generator}\\
The builtin document generator \texttt{nim doc} generates HTML documentation from \texttt{.nim} source files.\\



\item \href{nimsuggest.html}{Nimsuggest for IDE support}\\
Through the \texttt{nimsuggest} tool, any IDE can query a \texttt{.nim} source file and obtain useful information like definition of symbols or suggestions for completion.\\



\item \href{https://github.com/nim-lang/c2nim/blob/master/doc/c2nim.rst}{C2nim}\\
C to Nim source converter. Translates C header files to Nim.\\



\item \href{niminst.html}{niminst}\\
niminst is a tool to generate an installer for a Nim program.\\



\item \href{nimgrep.html}{nimgrep}\\
Nim search and replace utility.\\



\item nimpretty\\
\texttt{nimpretty} is a Nim source code beautifier, to format code according to the official style guide.\\



\item testament\\
\texttt{testament} is an advanced automatic \emph{unittests runner} for Nim tests, is used for the development of Nim itself, offers process isolation for your tests, it can generate statistics about test cases, supports multiple targets (C, JS, etc), \href{https://en.wikipedia.org/wiki/Dry\_run\_(testing)}{simulated Dry-Runs}, has logging, can generate HTML reports, skip tests from a file and more, so can be useful to run your tests, even the most complex ones.\\



\end{itemize}



\end{document}
