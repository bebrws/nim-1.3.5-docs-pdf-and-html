\hypertarget{nimscript}{%
\section{NimScript}\label{nimscript}}

Strictly speaking, \texttt{NimScript} is the subset of Nim that can be
evaluated by Nim's builtin virtual machine (VM). This VM is used for
Nim's compiletime function evaluation features.

The \texttt{nim} executable processes the \texttt{.nims} configuration
files in the following directories (in this order; later files overwrite
previous settings):

\begin{enumerate}
\def\labelenumi{\arabic{enumi})}
\tightlist
\item
  If environment variable \texttt{XDG\_CONFIG\_HOME} is defined,
  \texttt{\$XDG\_CONFIG\_HOME/nim/config.nims} or
  \texttt{\textasciitilde{}/.config/nim/config.nims} (POSIX) or
  \texttt{\%APPDATA\%/nim/config.nims} (Windows). This file can be
  skipped with the \texttt{-\/-skipUserCfg} command line option.
\item
  \texttt{\$parentDir/config.nims} where \texttt{\$parentDir} stands for
  any parent directory of the project file's path. These files can be
  skipped with the \texttt{-\/-skipParentCfg} command line option.
\item
  \texttt{\$projectDir/config.nims} where \texttt{\$projectDir} stands
  for the project's path. This file can be skipped with the
  \texttt{-\/-skipProjCfg} command line option.
\item
  A project can also have a project specific configuration file named
  \texttt{\$project.nims} that resides in the same directory as
  \texttt{\$project.nim}. This file can be skipped with the same
  \texttt{-\/-skipProjCfg} command line option.
\end{enumerate}

For available procs and implementation details see
\href{nimscript.html}{nimscript}.

\hypertarget{limitations}{%
\subsection{Limitations}\label{limitations}}

NimScript is subject to some limitations caused by the implementation of
the VM (virtual machine):

\begin{itemize}
\tightlist
\item
  Nim's FFI (foreign function interface) is not available in NimScript.
  This means that any stdlib module which relies on \texttt{importc} can
  not be used in the VM.
\item
  \texttt{ptr} operations are are hard to emulate with the symbolic
  representation the VM uses. They are available and tested extensively
  but there are bugs left.
\item
  \texttt{var\ T} function arguments rely on \texttt{ptr} operations
  internally and might also be problematic in some cases.
\item
  More than one level of {ref} is generally not supported (for example,
  the type {ref ref int}).
\item
  Multimethods are not available.
\item
  \texttt{random.randomize()} requires an \texttt{int64} explicitly
  passed as argument, you \emph{must} pass a Seed integer.
\end{itemize}

\hypertarget{standard-library-modules}{%
\subsection{Standard library modules}\label{standard-library-modules}}

At least the following standard library modules are available:

\begin{itemize}
\tightlist
\item
  \href{macros.html}{macros}
\item
  \href{os.html}{os}
\item
  \href{strutils.html}{strutils}
\item
  \href{math.html}{math}
\item
  \href{distros.html}{distros}
\item
  \href{sugar.html}{sugar}
\item
  \href{algorithm.html}{algorithm}
\item
  \href{base64.html}{base64}
\item
  \href{bitops.html}{bitops}
\item
  \href{chains.html}{chains}
\item
  \href{colors.html}{colors}
\item
  \href{complex.html}{complex}
\item
  \href{htmlgen.html}{htmlgen}
\item
  \href{httpcore.html}{httpcore}
\item
  \href{lenientops.html}{lenientops}
\item
  \href{mersenne.html}{mersenne}
\item
  \href{options.html}{options}
\item
  \href{parseutils.html}{parseutils}
\item
  \href{punycode.html}{punycode}
\item
  \href{punycode.html}{random}
\item
  \href{stats.html}{stats}
\item
  \href{strformat.html}{strformat}
\item
  \href{strmisc.html}{strmisc}
\item
  \href{strscans.html}{strscans}
\item
  \href{unicode.html}{unicode}
\item
  \href{uri.html}{uri}
\item
  \href{editdistance.html}{std/editdistance}
\item
  \href{wordwrap.html}{std/wordwrap}
\item
  \href{sums.html}{std/sums}
\item
  \href{parsecsv.html}{parsecsv}
\item
  \href{parsecfg.html}{parsecfg}
\item
  \href{parsesql.html}{parsesql}
\item
  \href{xmlparser.html}{xmlparser}
\item
  \href{htmlparser.html}{htmlparser}
\item
  \href{ropes.html}{ropes}
\item
  \href{json.html}{json}
\item
  \href{parsejson.html}{parsejson}
\item
  \href{strtabs.html}{strtabs}
\item
  \href{unidecode.html}{unidecode}
\end{itemize}

In addition to the standard Nim syntax (\href{system.html}{system}
module), NimScripts support the procs and templates defined in the
\href{nimscript.html}{nimscript} module too.

See also: *
\href{https://github.com/nim-lang/Nim/blob/devel/tests/test_nimscript.nims}{Check
the tests for more information about modules compatible with NimScript.}

\hypertarget{nimscript-as-a-configuration-file}{%
\subsection{NimScript as a configuration
file}\label{nimscript-as-a-configuration-file}}

A command-line switch \texttt{-\/-FOO} is written as
\texttt{switch("FOO")} in NimScript. Similarly, command-line
\texttt{-\/-FOO:VAL} translates to \texttt{switch("FOO",\ "VAL")}.

Here are few examples of using the \texttt{switch} proc:

\begin{verbatim}
\end{verbatim}

NimScripts also support \texttt{-\/-} templates for convenience, which
look like command-line switches written as-is in the NimScript file. So
the above example can be rewritten as:

\begin{verbatim}
\end{verbatim}

\textbf{Note}: In general, the \emph{define} switches can also be set in
NimScripts using \texttt{switch} or \texttt{-\/-}, as shown in above
examples. Only the \texttt{release} define (\texttt{-d:release}) cannot
be set in NimScripts.

\hypertarget{nimscript-as-a-build-tool}{%
\subsection{NimScript as a build tool}\label{nimscript-as-a-build-tool}}

The \texttt{task} template that the \texttt{system} module defines
allows a NimScript file to be used as a build tool. The following
example defines a task \texttt{build} that is an alias for the
\texttt{c} command:

\begin{verbatim}
\end{verbatim}

In fact, as a convention the following tasks should be available:

\begin{longtable}[]{@{}ll@{}}
\toprule
Task & Description\tabularnewline
\midrule
\endhead
\texttt{help} & List all the available NimScript tasks along with their
docstrings.\tabularnewline
\texttt{build} & Build the project with the required backend
(\texttt{c}, \texttt{cpp} or \texttt{js}).\tabularnewline
\texttt{tests} & Runs the tests belonging to the project.\tabularnewline
\texttt{bench} & Runs benchmarks belonging to the
project.\tabularnewline
\bottomrule
\end{longtable}

Look at the module \href{distros.html}{distros} for some support of the
OS's native package managers.

\hypertarget{nimble-integration}{%
\subsection{Nimble integration}\label{nimble-integration}}

See the \href{https://github.com/nim-lang/nimble\#readme}{Nimble readme}
for more information.

\hypertarget{standalone-nimscript}{%
\subsection{Standalone NimScript}\label{standalone-nimscript}}

NimScript can also be used directly as a portable replacement for Bash
and Batch files. Use \texttt{nim\ myscript.nims} to run
\texttt{myscript.nims}. For example, installation of Nimble could be
accomplished with this simple script:

\begin{verbatim}
mode = ScriptMode.Verbose

var id = 0
while dirExists("nimble" & $id):
  inc id

exec "git clone https://github.com/nim-lang/nimble.git nimble" & $id

withDir "nimble" & $id & "/src":
  exec "nim c nimble"

mvFile "nimble" & $id & "/src/nimble".toExe, "bin/nimble".toExe
\end{verbatim}

On Unix, you can also use the shebang \texttt{\#!/usr/bin/env\ nim}, as
long as your filename ends with \texttt{.nims}:

\begin{verbatim}
#!/usr/bin/env nim
mode = ScriptMode.Silent

echo "hello world"
\end{verbatim}

Use \texttt{\#!/usr/bin/env\ -S\ nim\ -\/-hints:off} to disable hints.

\hypertarget{benefits}{%
\subsection{Benefits}\label{benefits}}

\hypertarget{cross-platform}{%
\subsubsection{Cross-Platform}\label{cross-platform}}

It is a cross-platform scripting language that can run where Nim can
run, e.g. you can not run Batch or PowerShell on Linux or Mac, the Bash
for Linux might not run on Mac, there are no unit tests tools for Batch,
etc.

NimScript can detect on which platform, operating system, architecture,
and even which Linux distribution is running on, allowing the same
script to support a lot of systems.

See the following (incomplete) example:

\begin{verbatim}
import distros

# Architectures.
if defined(amd64):
  echo "Architecture is x86 64Bits"
elif defined(i386):
  echo "Architecture is x86 32Bits"
elif defined(arm):
  echo "Architecture is ARM"

# Operating Systems.
if defined(linux):
  echo "Operating System is GNU Linux"
elif defined(windows):
  echo "Operating System is Microsoft Windows"
elif defined(macosx):
  echo "Operating System is Apple OS X"

# Distros.
if detectOs(Ubuntu):
  echo "Distro is Ubuntu"
elif detectOs(ArchLinux):
  echo "Distro is ArchLinux"
elif detectOs(Debian):
  echo "Distro is Debian"
\end{verbatim}

\hypertarget{uniform-syntax}{%
\subsubsection{Uniform Syntax}\label{uniform-syntax}}

The syntax, style, and rest of the ecosystem is the same as for compiled
Nim, that means there is nothing new to learn, no context switch for
developers.

\hypertarget{powerful-metaprogramming}{%
\subsubsection{Powerful
Metaprogramming}\label{powerful-metaprogramming}}

NimScript can use Nim's templates, macros, types, concepts, effect
tracking system, and more, you can create modules that work on compiled
Nim and also on interpreted NimScript.

\texttt{func} will still check for side effects, \texttt{debugEcho} also
works as expected, making it ideal for functional scripting
metaprogramming.

This is an example of a third party module that uses macros and
templates to translate text strings on unmodified NimScript:

\begin{verbatim}
import nimterlingua
nimterlingua("translations.cfg")
echo "cat"  # Run with -d:RU becomes "kot", -d:ES becomes "gato", ...
\end{verbatim}

translations.cfg

\begin{verbatim}
[cat]
ES = gato
PT = minino
RU = kot
FR = chat
\end{verbatim}

\begin{itemize}
\tightlist
\item
  \href{https://nimble.directory/pkg/nimterlingua}{Nimterlingua}
\end{itemize}

\hypertarget{graceful-fallback}{%
\subsubsection{Graceful Fallback}\label{graceful-fallback}}

Some features of compiled Nim may not work on NimScript, but often a
graceful and seamless fallback degradation is used.

See the following NimScript:

\begin{verbatim}
if likely(true):
  discard
elif unlikely(false):
  discard

proc foo() {.compiletime.} = echo NimVersion

static:
  echo CompileDate
\end{verbatim}

\texttt{likely()}, \texttt{unlikely()}, \texttt{static:} and
\texttt{\{.compiletime.\}} will produce no code at all when run on
NimScript, but still no error nor warning is produced and the code just
works.

\hypertarget{evolving-scripting-language}{%
\subsubsection{Evolving Scripting
language}\label{evolving-scripting-language}}

NimScript evolves together with Nim,
\href{https://github.com/nim-lang/Nim/pulls?utf8=\%E2\%9C\%93\&q=nimscript}{occasionally
new features might become available on NimScript} , adapted from
compiled Nim or added as new features on both.

\hypertarget{scripting-language-with-a-package-manager}{%
\subsubsection{Scripting Language with a Package
Manager}\label{scripting-language-with-a-package-manager}}

You can create your own modules to be compatible with NimScript, and
check \href{https://nimble.directory}{Nimble} to search for third party
modules that may work on NimScript.

\hypertarget{devops-scripting}{%
\subsubsection{DevOps Scripting}\label{devops-scripting}}

You can use NimScript to deploy to production, run tests, build
projects, do benchmarks, generate documentation, and all kinds of
DevOps/SysAdmin specific tasks.

\begin{itemize}
\tightlist
\item
  \href{https://github.com/kaushalmodi/nim_config\#list-available-tasks}{An
  example of a third party NimScript that can be used as a
  project-agnostic tool.}
\end{itemize}
