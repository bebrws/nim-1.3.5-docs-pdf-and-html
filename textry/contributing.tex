\hypertarget{contributing}{%
\section{Contributing}\label{contributing}}

Contributing happens via "Pull requests" (PR) on github. Every PR needs
to be reviewed before it can be merged and the Continuous Integration
should be green.

The PR has to be approved (and is often merged too) by one "code owner",
either by the code owner who is responsible for the subsystem the PR
belongs to or by two core developers or by Araq.

See \href{codeowners.html}{codeowners} for more details.

\hypertarget{writing-tests}{%
\subsection{Writing tests}\label{writing-tests}}

There are 4 types of tests:

\begin{enumerate}
\def\labelenumi{\arabic{enumi}.}
\tightlist
\item
  \texttt{runnableExamples} documentation comment tests, ran by
  \texttt{nim\ doc\ mymod.nim} These end up in documentation and ensure
  documentation stays in sync with code.
\item
  separate test files, e.g.: \texttt{tests/stdlib/tos.nim}. In nim repo,
  {testament} (see below) runs all {\$nim/tests/*/t*.nim} test files;
  for nimble packages, see
  \url{https://github.com/nim-lang/nimble\#tests}.
\item
  (deprecated) tests in \texttt{when\ isMainModule:} block, ran by
  \texttt{nim\ r\ mymod.nim}. \texttt{nimble\ test} can run those in
  nimble packages when specified in a {task "test"}.
\item
  (not preferred) {.. code-block:: nim} RST snippets; these should only
  be used in rst sources, in nim sources {runnableExamples} should now
  always be preferred to those for several reasons (cleaner syntax,
  syntax highlights, batched testing, and {rdoccmd} allows
  customization).
\end{enumerate}

Not all the tests follow the convention here, feel free to change the
ones that don't. Always leave the code cleaner than you found it.

\hypertarget{stdlib}{%
\subsubsection{Stdlib}\label{stdlib}}

Each stdlib module (anything under \texttt{lib/}, e.g.
\texttt{lib/pure/os.nim}) should preferably have a corresponding
separate test file, eg {tests/stdlib/tos.nim}. The old convention was to
add a \texttt{when\ isMainModule:} block in the source file, which only
gets executed when the tester is building the file.

Each test should be in a separate \texttt{block:} statement, such that
each has its own scope. Use boolean conditions and \texttt{doAssert} for
the testing by itself, don't rely on echo statements or similar; in
particular avoid things like {echo "done"}.

Sample test:

\begin{verbatim}
block: # bug #1234
  static: doAssert 1+1 == 2

block: # bug #1235
  var seq2D = newSeqWith(4, newSeq[bool](2))
  seq2D[0][0] = true
  seq2D[1][0] = true
  seq2D[0][1] = true
  doAssert seq2D == @[@[true, true], @[true, false],
                      @[false, false], @[false, false]]
  # doAssert with `not` can now be done as follows:
  doAssert not (1 == 2)
\end{verbatim}

Always refer to a github issue using the following exact syntax: {bug
\#1234} as shown above, so that it's consistent and easier to search or
for tooling. Some browser extensions (eg
\url{https://github.com/sindresorhus/refined-github}) will even turn
those in clickable links when it works.

Rationale for using a separate test file instead of {when isMainModule:}
block: * allows custom compiler flags or testing options (see details
below) * faster CI since they can be joined in {megatest} (combined into
a single test) * avoids making the parser do un-necessary work when a
source file is merely imported * avoids mixing source and test code when
reporting line of code statistics or code coverage

\hypertarget{compiler}{%
\subsubsection{Compiler}\label{compiler}}

The tests for the compiler use a testing tool called \texttt{testament}.
They are all located in \texttt{tests/} (e.g.:
\texttt{tests/destructor/tdestructor3.nim}). Each test has its own file.
All test files are prefixed with \texttt{t}. If you want to create a
file for import into another test only, use the prefix \texttt{m}.

At the beginning of every test is the expected behavior of the test.
Possible keys are:

\begin{itemize}
\tightlist
\item
  \texttt{cmd}: A compilation command template e.g.
  \texttt{nim\ \$target\ -\/-threads:on\ \$options\ \$file}
\item
  \texttt{output}: The expected output (stdout + stderr), most likely
  via \texttt{echo}
\item
  \texttt{exitcode}: Exit code of the test (via \texttt{exit(number)})
\item
  \texttt{errormsg}: The expected compiler error message
\item
  \texttt{file}: The file the errormsg was produced at
\item
  \texttt{line}: The line the errormsg was produced at
\end{itemize}

For a full spec, see here: \texttt{testament/specs.nim}

An example for a test:

\begin{verbatim}
discard """
  errormsg: "type mismatch: got (PTest)"
"""

type
  PTest = ref object

proc test(x: PTest, y: int) = nil

var buf: PTest
buf.test()
\end{verbatim}

\hypertarget{running-tests}{%
\subsection{Running tests}\label{running-tests}}

You can run the tests with

\begin{verbatim}
./koch tests
\end{verbatim}

which will run a good subset of tests. Some tests may fail. If you only
want to see the output of failing tests, go for

\begin{verbatim}
./koch tests --failing all
\end{verbatim}

You can also run only a single category of tests. A category is a
subdirectory in the \texttt{tests} directory. There are a couple of
special categories; for a list of these, see
\texttt{testament/categories.nim}, at the bottom.

\begin{verbatim}
./koch tests c lib # compiles/runs stdlib modules, including `isMainModule` tests
./koch tests c megatest # runs a set of tests that can be combined into 1
\end{verbatim}

To run a single test:

\begin{verbatim}
./koch test run <category>/<name>    # e.g.: tuples/ttuples_issues
./koch test run tests/stdlib/tos.nim # can also provide relative path
\end{verbatim}

For reproducible tests (to reproduce an environment more similar to the
one run by Continuous Integration on travis/appveyor), you may want to
disable your local configuration (e.g. in
\texttt{\textasciitilde{}/.config/nim/nim.cfg}) which may affect some
tests; this can also be achieved by using
\texttt{export\ XDG\_CONFIG\_HOME=pathtoAlternateConfig} before running
\texttt{./koch} commands.

\hypertarget{comparing-tests}{%
\subsection{Comparing tests}\label{comparing-tests}}

Test failures can be grepped using \texttt{Failure:}.

The tester can compare two test runs. First, you need to create the
reference test. You'll also need to the commit id, because that's what
the tester needs to know in order to compare the two.

\begin{verbatim}
git checkout devel
DEVEL_COMMIT=$(git rev-parse HEAD)
./koch tests
\end{verbatim}

Then switch over to your changes and run the tester again.

\begin{verbatim}
git checkout your-changes
./koch tests
\end{verbatim}

Then you can ask the tester to create a \texttt{testresults.html} which
will tell you if any new tests passed/failed.

\begin{verbatim}
./koch tests --print html $DEVEL_COMMIT
\end{verbatim}

\hypertarget{deprecation}{%
\subsection{Deprecation}\label{deprecation}}

Backward compatibility is important, so instead of a rename you need to
deprecate the old name and introduce a new name:

\begin{verbatim}
# for routines (proc/template/macro/iterator) and types:
proc oldProc(a: int, b: float): bool {.deprecated:
    "deprecated since v1.2.3; use `newImpl: string -> int` instead".} = discard

# for (const/var/let/fields) the msg is not yet supported:
const Foo {.deprecated.}  = 1

# for enum types, you can deprecate the type or some elements
# (likewise with object types and their fields):
type Bar {.deprecated.} = enum bar0, bar1
type Barz  = enum baz0, baz1 {.deprecated.}, baz2
\end{verbatim}

See also \href{manual.html\#pragmas-deprecated-pragma}{Deprecated}
pragma in the manual.

\hypertarget{documentation}{%
\subsection{Documentation}\label{documentation}}

When contributing new procs, be sure to add documentation, especially if
the proc is public. Even private procs benefit from documentation and
can be viewed using \texttt{nim\ doc\ -\/-docInternal\ foo.nim}.
Documentation begins on the line following the \texttt{proc} definition,
and is prefixed by \texttt{\#\#} on each line.

Runnable code examples are also encouraged, to show typical behavior
with a few test cases (typically 1 to 3 \texttt{assert} statements,
depending on complexity). These \texttt{runnableExamples} are
automatically run by \texttt{nim\ doc\ mymodule.nim} as well as
\texttt{testament} and guarantee they stay in sync.

\begin{verbatim}
\end{verbatim}

See \href{os.html\#parentDir,string}{parentDir} example.

The RestructuredText Nim uses has a special syntax for including code
snippets embedded in documentation; these are not run by
\texttt{nim\ doc} and therefore are not guaranteed to stay in sync, so
\texttt{runnableExamples} is usually preferred:

\begin{verbatim}
proc someproc*(): string =
  ## Return "something"
  ##
  ## .. code-block::
  ##  echo someproc() # "something"
  result = "something" # single-hash comments do not produce documentation
\end{verbatim}

The \texttt{..\ code-block::\ nim} followed by a newline and an
indentation instructs the \texttt{nim\ doc} command to produce
syntax-highlighted example code with the documentation
(\texttt{..\ code-block::} is sufficient from inside a nim module).

When forward declaration is used, the documentation should be included
with the first appearance of the proc.

\begin{verbatim}
proc hello*(): string
  ## Put documentation here
proc nothing() = discard
proc hello*(): string =
  ## ignore this
  echo "hello"
\end{verbatim}

The preferred documentation style is to begin with a capital letter and
use the imperative (command) form. That is, between:

\begin{verbatim}
proc hello*(): string =
  ## Return "hello"
  result = "hello"
\end{verbatim}

or

\begin{verbatim}
proc hello*(): string =
  ## says hello
  result = "hello"
\end{verbatim}

the first is preferred.

\hypertarget{best-practices}{%
\subsection{Best practices}\label{best-practices}}

Note: these are general guidelines, not hard rules; there are always
exceptions. Code reviews can just point to a specific section here to
save time and propagate best practices.

by {nim} for symbols in nim sources (e.g. compiler, standard library).
This is to avoid name conflicts across packages.

\begin{verbatim}
# if in nim sources
when defined(allocStats): discard # bad, can cause conflicts
when defined(nimAllocStats): discard # preferred
# if in a pacakge `cligen`:
when defined(debug): discard # bad, can cause conflicts
when defined(cligenDebug): discard # preferred
\end{verbatim}

\begin{verbatim}
doAssert isValid() == true
doAssert isValid() # preferred
\end{verbatim}

\begin{verbatim}
proc foo(cond: bool, lines: seq[string]) # bad
proc foo(lines: seq[string], cond: bool) # preferred
# can be called as: `getLines().foo(false)`
\end{verbatim}

rationale: \url{https://forum.nim-lang.org/t/4089}

\begin{verbatim}
quit() # bad in almost all cases
doAssert() # preferred
\end{verbatim}

be enabled even in release mode (except for tests in
\texttt{runnableExamples} blocks which for which \texttt{nim\ doc}
ignores \texttt{-d:release}).

\begin{verbatim}
when isMainModule:
  assert foo() # bad
  doAssert foo() # preferred
\end{verbatim}

rationale: it's more flexible (e.g. allows caller to call custom
printing, including prepending location info, writing to log files,
etc).

\begin{verbatim}
proc foo() = echo "bar" # bad
proc foo(): string = "bar" # preferred (usually)
\end{verbatim}

unless stack allocation is needed (e.g. for efficiency).

\begin{verbatim}
proc foo(a: var Bar): bool
proc foo(): Option[Bar]
\end{verbatim}

except when that's not possible. It's more precise, easier for readers
and maintaners to where expected values refer to. See for example
\url{https://github.com/nim-lang/Nim/pull/9335} and
\url{https://forum.nim-lang.org/t/4089}

\begin{verbatim}
echo foo() # adds a line for testament in `output:` block inside `discard`.
doAssert foo() == [1, 2] # preferred, except when not possible to do so.
\end{verbatim}

\hypertarget{the-git-stuff}{%
\subsection{The Git stuff}\label{the-git-stuff}}

\hypertarget{general-commit-rules}{%
\subsubsection{General commit rules}\label{general-commit-rules}}

\begin{enumerate}
\def\labelenumi{\arabic{enumi}.}
\item
  Important, critical bugfixes that have a tiny chance of breaking
  somebody's code should be backported to the latest stable release
  branch (currently 1.2.x) and maybe also to the 1.0 branch. The commit
  message should contain the tag \texttt{{[}backport{]}} for "backport
  to all stable releases" and the tag \texttt{{[}backport:\$VERSION{]}}
  for backporting to the given \$VERSION.
\item
  If you introduce changes which affect backwards compatibility, make
  breaking changes, or have PR which is tagged as
  \texttt{{[}feature{]}}, the changes should be mentioned in
  \href{https://github.com/nim-lang/Nim/blob/devel/changelog.md}{the
  changelog}.
\item
  All changes introduced by the commit (diff lines) must be related to
  the subject of the commit.

  If you change something unrelated to the subject parts of the file,
  because your editor reformatted automatically the code or whatever
  different reason, this should be excluded from the commit.

  \emph{Tip:} Never commit everything as is using
  \texttt{git\ commit\ -a}, but review carefully your changes with
  \texttt{git\ add\ -p}.
\item
  Changes should not introduce any trailing whitespace.

  Always check your changes for whitespace errors using
  \texttt{git\ diff\ -\/-check} or add following \texttt{pre-commit}
  hook:

\begin{Shaded}
\begin{Highlighting}[]
\CommentTok{\#!/bin/sh}
\FunctionTok{git}\NormalTok{ diff {-}{-}check {-}{-}cached }\KeywordTok{||} \BuiltInTok{exit} \VariableTok{$?}
\end{Highlighting}
\end{Shaded}
\item
  Describe your commit and use your common sense. Example commit
  message:

  \texttt{Fixes\ \#123;\ refs\ \#124}

  indicates that issue \texttt{\#123} is completely fixed (github may
  automatically close it when the PR is committed), wheres issue
  \texttt{\#124} is referenced (e.g.: partially fixed) and won't close
  the issue when committed.
\item
  Commits should be always be rebased against devel (so a fast forward
  merge can happen)

  e.g.: use \texttt{git\ pull\ -\/-rebase\ origin\ devel}. This is to
  avoid messing up git history. Exceptions should be very rare: when
  rebase gives too many conflicts, simply squash all commits using the
  script shown in \url{https://github.com/nim-lang/Nim/pull/9356}
\item
  Do not mix pure formatting changes (e.g. whitespace changes,
  nimpretty) or automated changes (e.g. nimfix) with other code changes:
  these should be in separate commits (and the merge on github should
  not squash these into 1).
\end{enumerate}

\hypertarget{continuous-integration-ci}{%
\subsubsection{Continuous Integration
(CI)}\label{continuous-integration-ci}}

\begin{enumerate}
\def\labelenumi{\arabic{enumi}.}
\tightlist
\item
  Continuous Integration is by default run on every push in a PR; this
  clogs the CI pipeline and affects other PR's; if you don't need it
  (e.g. for WIP or documentation only changes), add
  \texttt{{[}ci\ skip{]}} to your commit message title. This convention
  is supported by
  \href{https://www.appveyor.com/docs/how-to/filtering-commits/\#skip-directive-in-commit-message}{Appveyor}
  and
  \href{https://docs.travis-ci.com/user/customizing-the-build/\#skipping-a-build}{Travis}.
\item
  Consider enabling CI (travis and appveyor) in your own Nim fork, and
  waiting for CI to be green in that fork (fixing bugs as needed) before
  opening your PR in original Nim repo, so as to reduce CI congestion.
  Same applies for updates on a PR: you can test commits on a separate
  private branch before updating the main PR.
\end{enumerate}

\hypertarget{code-reviews}{%
\subsubsection{Code reviews}\label{code-reviews}}

\begin{enumerate}
\def\labelenumi{\arabic{enumi}.}
\item
  Whenever possible, use github's new 'Suggested change' in code
  reviews, which saves time explaining the change or applying it; see
  also \url{https://forum.nim-lang.org/t/4317}
\item
  When reviewing large diffs that may involve code moving around,
  github's interface doesn't help much as it doesn't highlight moves.
  Instead you can use something like this, see visual results
  \href{https://github.com/nim-lang/Nim/pull/10431\#issuecomment-456968196}{here}:

\begin{Shaded}
\begin{Highlighting}[]
\FunctionTok{git}\NormalTok{ fetch origin pull/10431/head }\KeywordTok{\&\&} \FunctionTok{git}\NormalTok{ checkout FETCH\_HEAD}
\FunctionTok{git}\NormalTok{ diff {-}{-}color{-}moved{-}ws=allow{-}indentation{-}change {-}{-}color{-}moved=blocks HEAD\^{}}
\end{Highlighting}
\end{Shaded}
\item
  In addition, you can view github-like diffs locally to identify what
  was changed within a code block using {diff-highlight} or
  {diff-so-fancy}, e.g.:

\begin{Shaded}
\begin{Highlighting}[]
\CommentTok{\# put this in \textasciitilde{}/.gitconfig:}
\NormalTok{[}\ExtensionTok{core}\NormalTok{]}
  \ExtensionTok{pager}\NormalTok{ = }\StringTok{"diff{-}so{-}fancy | less {-}R"}\NormalTok{ \# or: use: }\KeywordTok{\textasciigrave{}}\ExtensionTok{diff{-}highlight}\KeywordTok{\textasciigrave{}}
\end{Highlighting}
\end{Shaded}
\end{enumerate}

\hypertarget{documentation-style}{%
\subsection{Documentation Style}\label{documentation-style}}

\hypertarget{general-guidelines}{%
\subsubsection{General Guidelines}\label{general-guidelines}}

\begin{itemize}
\tightlist
\item
  See also \href{https://nim-lang.github.io/Nim/nep1.html}{nep1} which
  should probably be merged here.
\item
  Authors should document anything that is exported; documentation for
  private procs can be useful too (visible via
  \texttt{nim\ doc\ -\/-docInternal\ foo.nim}).
\item
  Within documentation, a period ({.}) should follow each sentence (or
  sentence fragment) in a comment block. The documentation may be
  limited to one sentence fragment, but if multiple sentences are within
  the documentation, each sentence after the first should be complete
  and in present tense.
\item
  Documentation is parsed as a custom ReStructuredText (RST) with
  partial markdown support.
\item
  In nim sources, prefer single backticks to double backticks since it's
  simpler and {nim doc} supports it (even in rst files with {nim
  rst2html}).
\item
  In nim sources, for links, prefer {{[}link text{]}(link.html)} to
  \texttt{\textasciigrave{}\ \textasciigrave{}link\ text\textless{}link.html\textgreater{}\textasciigrave{}\_}{
  since the syntax is simpler and markdown is more common (likewise,
  `nim rst2html} also supports it in rst files).
\end{itemize}

\begin{verbatim}
proc someproc*(s: string, foo: int) =
  ## Use single backticks for inline code, eg: `s` or `someExpr(true)`.
  ## Use a backlash to follow with alphanumeric char: `int8`\s are great.
\end{verbatim}

\hypertarget{module-level-documentation}{%
\subsubsection{Module-level
documentation}\label{module-level-documentation}}

Documentation of a module is placed at the top of the module itself.
Each line of documentation begins with double hashes (\texttt{\#\#}).
Sometimes \texttt{\#\#{[}\ multiline\ docs\ containing\ code\ {]}\#\#}
is preferable, see \texttt{lib/pure/times.nim}. Code samples are
encouraged, and should follow the general RST syntax:

\begin{verbatim}
## The `universe` module computes the answer to life, the universe, and everything.
##
## .. code-block::
##  doAssert computeAnswerString() == 42
\end{verbatim}

Within this top-level comment, you can indicate the authorship and
copyright of the code, which will be featured in the produced
documentation.

\begin{verbatim}
## This is the best module ever. It provides answers to everything!
##
## :Author: Steve McQueen
## :Copyright: 1965
##
\end{verbatim}

Leave a space between the last line of top-level documentation and the
beginning of Nim code (the imports, etc.).

\hypertarget{procs-templates-macros-converters-and-iterators}{%
\subsubsection{Procs, Templates, Macros, Converters, and
Iterators}\label{procs-templates-macros-converters-and-iterators}}

The documentation of a procedure should begin with a capital letter and
should be in present tense. Variables referenced in the documentation
should be surrounded by single tick marks:

\begin{verbatim}
proc example1*(x: int) =
  ## Prints the value of `x`.
  echo x
\end{verbatim}

Whenever an example of usage would be helpful to the user, you should
include one within the documentation in RST format as below.

\begin{verbatim}
proc addThree*(x, y, z: int8): int =
  ## Adds three `int8` values, treating them as unsigned and
  ## truncating the result.
  ##
  ## .. code-block::
  ##  # things that aren't suitable for a `runnableExamples` go in code-block:
  ##  echo execCmdEx("git pull")
  ##  drawOnScreen()
  runnableExamples:
    # `runnableExamples` is usually preferred to `code-block`, when possible.
    doAssert addThree(3, 125, 6) == -122
  result = x +% y +% z
\end{verbatim}

The commands \texttt{nim\ doc} and \texttt{nim\ doc2} will then
correctly syntax highlight the Nim code within the documentation.

\hypertarget{types}{%
\subsubsection{Types}\label{types}}

Exported types should also be documented. This documentation can also
contain code samples, but those are better placed with the functions to
which they refer.

\begin{verbatim}
type
  NamedQueue*[T] = object ## Provides a linked data structure with names
                          ## throughout. It is named for convenience. I'm making
                          ## this comment long to show how you can, too.
    name*: string ## The name of the item
    val*: T ## Its value
    next*: ref NamedQueue[T] ## The next item in the queue
\end{verbatim}

You have some flexibility when placing the documentation:

\begin{verbatim}
type
  NamedQueue*[T] = object
    ## Provides a linked data structure with names
    ## throughout. It is named for convenience. I'm making
    ## this comment long to show how you can, too.
    name*: string ## The name of the item
    val*: T ## Its value
    next*: ref NamedQueue[T] ## The next item in the queue
\end{verbatim}

Make sure to place the documentation beside or within the object.

\begin{verbatim}
type
  ## Bad: this documentation disappears because it annotates the ``type`` keyword
  ## above, not ``NamedQueue``.
  NamedQueue*[T] = object
    name*: string ## This becomes the main documentation for the object, which
                  ## is not what we want.
    val*: T ## Its value
    next*: ref NamedQueue[T] ## The next item in the queue
\end{verbatim}

\hypertarget{var-let-and-const}{%
\subsubsection{Var, Let, and Const}\label{var-let-and-const}}

When declaring module-wide constants and values, documentation is
encouraged. The placement of doc comments is similar to the
\texttt{type} sections.

\begin{verbatim}
const
  X* = 42 ## An awesome number.
  SpreadArray* = [
    [1,2,3],
    [2,3,1],
    [3,1,2],
  ] ## Doc comment for ``SpreadArray``.
\end{verbatim}

Placement of comments in other areas is usually allowed, but will not
become part of the documentation output and should therefore be prefaced
by a single hash (\texttt{\#}).

\begin{verbatim}
const
  BadMathVals* = [
    3.14, # pi
    2.72, # e
    0.58, # gamma
  ] ## A bunch of badly rounded values.
\end{verbatim}

Nim supports Unicode in comments, so the above can be replaced with the
following:

\begin{verbatim}
const
  BadMathVals* = [
    3.14, # π
    2.72, # e
    0.58, # γ
  ] ## A bunch of badly rounded values (including π!).
\end{verbatim}

\hypertarget{evolving-the-stdlib}{%
\subsection{Evolving the stdlib}\label{evolving-the-stdlib}}

As outlined in \url{https://github.com/nim-lang/RFCs/issues/173} there
are a couple of guidelines about what should go into the stdlib, what
should be added and what eventually should be removed.

\hypertarget{what-the-compiler-itself-needs-must-be-part-of-the-stdlib}{%
\subsubsection{What the compiler itself needs must be part of the
stdlib}\label{what-the-compiler-itself-needs-must-be-part-of-the-stdlib}}

Maybe in the future the compiler itself can depend on Nimble packages
but for the time being, we strive to have zero dependencies in the
compiler as the compiler is the root of the bootstrapping process and is
also used to build Nimble.

\hypertarget{vocabulary-types-must-be-part-of-the-stdlib}{%
\subsubsection{Vocabulary types must be part of the
stdlib}\label{vocabulary-types-must-be-part-of-the-stdlib}}

These are types most packages need to agree on for better
interoperability, for example \texttt{Option{[}T{]}}. This rule also
covers the existing collections like \texttt{Table}, \texttt{CountTable}
etc. "Sorted" containers based on a tree-like data structure are still
missing and should be added.

Time handling, especially the \texttt{Time} type are also covered by
this rule.

\hypertarget{existing-battle-tested-modules-stay}{%
\subsubsection{Existing, battle-tested modules
stay}\label{existing-battle-tested-modules-stay}}

Reason: There is no benefit in moving them around just to fullfill some
design fashion as in "Nim's core MUST BE SMALL". If you don't like an
existing module, don't import it. If a compilation target (e.g. JS)
cannot support a module, document this limitation.

This covers modules like \texttt{os}, \texttt{osproc},
\texttt{strscans}, \texttt{strutils}, \texttt{strformat}, etc.

\hypertarget{syntactic-helpers-can-start-as-experimental-stdlib-modules}{%
\subsubsection{Syntactic helpers can start as experimental stdlib
modules}\label{syntactic-helpers-can-start-as-experimental-stdlib-modules}}

Reason: Generally speaking as external dependencies they are not exposed
to enough users so that we can see if the shortcuts provide enough
benefit or not. Many programmers avoid external dependencies, even
moreso for "tiny syntactic improvements". However, this is only true for
really good syntactic improvements that have the potential to clean up
other parts of the Nim library substantially. If in doubt, new stdlib
modules should start as external, successful Nimble packages.

\hypertarget{other-new-stdlib-modules-do-not-start-as-stdlib-modules}{%
\subsubsection{Other new stdlib modules do not start as stdlib
modules}\label{other-new-stdlib-modules-do-not-start-as-stdlib-modules}}

As we strive for higher quality everywhere, it's easier to adopt
existing, battle-tested modules eventually rather than creating modules
from scratch.

\hypertarget{little-additions-are-acceptable}{%
\subsubsection{Little additions are
acceptable}\label{little-additions-are-acceptable}}

As long as they are documented and tested well, adding little helpers to
existing modules is acceptable. For two reasons:

\begin{enumerate}
\def\labelenumi{\arabic{enumi}.}
\tightlist
\item
  It makes Nim easier to learn and use in the long run. ("Why does
  sequtils lack a \texttt{countIt}? Because version 1.0 happens to have
  lacked it? Silly...")
\item
  To encourage contributions. Contributors often start with PRs that add
  simple things and then they stay and also fix bugs. Nim is an open
  source project and lives from people's contributions and involvement.
  Newly introduced issues have to be balanced against motivating new
  people. We know where to find perfectly designed pieces of software
  that have no bugs -\/- these are the systems that nobody uses.
\end{enumerate}
