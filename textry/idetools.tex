\hypertarget{nim-ide-integration-guide}{%
\section{Nim IDE Integration Guide}\label{nim-ide-integration-guide}}

\begin{description}
\item[Author]
Britney Spears
\item[Version]
1.3.5
\end{description}

Note: this is mostly outdated, see instead
\href{nimsuggest.html}{nimsuggest}

Nim differs from many other compilers in that it is really fast, and
being so fast makes it suited to provide external queries for text
editors about the source code being written. Through the
\texttt{idetools} command of \href{nimc.html}{the compiler}, any IDE can
query a \texttt{.nim} source file and obtain useful information like
definition of symbols or suggestions for completion.

This document will guide you through the available options. If you want
to look at practical examples of idetools support you can look at the
test files found in the \protect\hyperlink{test-suite}{Test suite} or
\href{https://github.com/Araq/Nim/wiki/Editor-Support}{various editor
integrations} already available.

\hypertarget{idetools-invocation}{%
\subsection{Idetools invocation}\label{idetools-invocation}}

\hypertarget{specifying-the-location-of-the-query}{%
\subsubsection{Specifying the location of the
query}\label{specifying-the-location-of-the-query}}

All of the available idetools commands require you to specify a query
location through the \texttt{-\/-track} or \texttt{-\/-trackDirty}
switches. The general idetools invocations are:

\begin{verbatim}
nim idetools --track:FILE,LINE,COL <switches> proj.nim
\end{verbatim}

Or:

\begin{verbatim}
nim idetools --trackDirty:DIRTY_FILE,FILE,LINE,COL <switches> proj.nim
\end{verbatim}

\begin{description}
\item[\texttt{proj.nim}]
This is the main \emph{project} filename. Most of the time you will pass
in the same as \textbf{FILE}, but for bigger projects this is the file
which is used as main entry point for the program, the one which users
compile to generate a final binary.
\item[\texttt{\textless{}switches\textgreater{}}]
This would be any of the other idetools available options, like
\texttt{-\/-def} or \texttt{-\/-suggest} explained in the following
sections.
\item[\texttt{COL}]
An integer with the column you are going to query. For the compiler
columns start at zero, so the first column will be \textbf{0} and the
last in an 80 column terminal will be \textbf{79}.
\item[\texttt{LINE}]
An integer with the line you are going to query. For the compiler lines
start at \textbf{1}.
\item[\texttt{FILE}]
The file you want to perform the query on. Usually you will pass in the
same value as \textbf{proj.nim}.
\item[\texttt{DIRTY\_FILE}]
The \textbf{FILE} parameter is enough for static analysis, but IDEs tend
to have \emph{unsaved buffers} where the user may still be in the middle
of typing a line. In such situations the IDE can save the current
contents to a temporary file and then use the \texttt{-\/-trackDirty}
switch.

Dirty files are likely to contain errors and they are usually compiled
partially only to the point needed to service the idetool request. The
compiler discriminates them to ensure that \textbf{a)} they won't be
cached and \textbf{b)} they won't invalidate the cached contents of the
original module.

The other reason is that the dirty file can appear anywhere on disk
(e.g. in tmpfs), but it must be treated as having a path matching the
original module when it comes to usage of relative paths, etc. Queries,
however, will refer to the dirty module name in their answers instead of
the normal filename.
\end{description}

\hypertarget{definitions}{%
\subsubsection{Definitions}\label{definitions}}

The \texttt{-\/-def} idetools switch performs a query about the
definition of a specific symbol. If available, idetools will answer with
the type, source file, line/column information and other accessory data
if available like a docstring. With this information an IDE can provide
the typical \emph{Jump to definition} where a user puts the cursor on a
symbol or uses the mouse to select it and is redirected to the place
where the symbol is located.

Since Nim is implemented in Nim, one of the nice things of this feature
is that any user with an IDE supporting it can quickly jump around the
standard library implementation and see exactly what a proc does,
learning about the language and seeing real life examples of how to
write/implement specific features.

Idetools will always answer with a single definition or none if it can't
find any valid symbol matching the position of the query.

\hypertarget{suggestions}{%
\subsubsection{Suggestions}\label{suggestions}}

The \texttt{-\/-suggest} idetools switch performs a query about possible
completion symbols at some point in the file. IDEs can easily provide an
autocompletion feature where the IDE scans the current file (and related
ones, if it knows about the language being edited and follows
includes/imports) and when the user starts typing something a completion
box with different options appears.

However such features are not context sensitive and work simply on
string matching, which can be problematic in Nim especially due to the
case insensitiveness of the language (plus underscores as separators!).

The typical usage scenario for this option is to call it after the user
has typed the dot character for
\href{tut2.html\#object-oriented-programming-method-call-syntax}{the
object oriented call syntax}. Idetools will try to return the
suggestions sorted first by scope (from innermost to outermost) and then
by item name.

\hypertarget{invocation-context}{%
\subsubsection{Invocation context}\label{invocation-context}}

The \texttt{-\/-context} idetools switch is very similar to the
suggestions switch, but instead of being used after the user has typed a
dot character, this one is meant to be used after the user has typed an
opening brace to start typing parameters.

\hypertarget{symbol-usages}{%
\subsubsection{Symbol usages}\label{symbol-usages}}

The \texttt{-\/-usages} idetools switch lists all usages of the symbol
at a position. IDEs can use this to find all the places in the file
where the symbol is used and offer the user to rename it in all places
at the same time. Again, a pure string based search and replace may
catch symbols out of the scope of a function/loop.

For this kind of query the IDE will most likely ignore all the
type/signature info provided by idetools and concentrate on the
filename, line and column position of the multiple returned answers.

\hypertarget{expression-evaluation}{%
\subsubsection{Expression evaluation}\label{expression-evaluation}}

This feature is still under development. In the future it will allow an
IDE to evaluate an expression in the context of the currently
running/debugged user project.

\hypertarget{compiler-as-a-service-caas}{%
\subsection{Compiler as a service
(CAAS)}\label{compiler-as-a-service-caas}}

The occasional use of idetools is acceptable for things like
definitions, where the user puts the cursor on a symbol or double clicks
it and after a second or two the IDE displays where that symbol is
defined. Such latencies would be terrible for features like symbol
suggestion, plus why wait at all if we can avoid it?

The idetools command can be run as a compiler service (CAAS), where you
first launch the compiler and it will stay online as a server, accepting
queries in a telnet like fashion. The advantage of staying on is that
for many queries the compiler can cache the results of the compilation,
and subsequent queries should be fast in the millisecond range, thus
being responsive enough for IDEs.

If you want to start the server using stdin/stdout as communication you
need to type:

\begin{verbatim}
nim serve --server.type:stdin proj.nim
\end{verbatim}

If you want to start the server using tcp and a port, you need to type:

\begin{verbatim}
nim serve --server.type:tcp --server.port:6000 \
  --server.address:hostname proj.nim
\end{verbatim}

In both cases the server will start up and await further commands. The
syntax of the commands you can now send to the server is practically the
same as running the nim compiler on the commandline, you only need to
remove the name of the compiler since you are already talking to it. The
server will answer with as many lines of text it thinks necessary plus
an empty line to indicate the end of the answer.

You can find examples of client/server communication in the idetools
tests found in the \protect\hyperlink{test-suite}{Test suite}.

\hypertarget{parsing-idetools-output}{%
\subsection{Parsing idetools output}\label{parsing-idetools-output}}

Idetools outputs is always returned on single lines separated by tab
characters (\texttt{\textbackslash{}t}). The values of each column are:

\begin{enumerate}
\def\labelenumi{\arabic{enumi}.}
\item
  Three characters indicating the type of returned answer (e.g. def for
  definition, \texttt{sug} for suggestion, etc).
\item
  Type of the symbol. This can be \texttt{skProc}, \texttt{skLet}, and
  just about any of the enums defined in the module
  \texttt{compiler/ast.nim}.
\item
  Full qualified path of the symbol. If you are querying a symbol
  defined in the \texttt{proj.nim} file, this would have the form
  \texttt{proj.symbolName}.
\item
  Type/signature. For variables and enums this will contain the type of
  the symbol, for procs, methods and templates this will contain the
  full unique signature (e.g. \texttt{proc\ (File)}).
\item
  Full path to the file containing the symbol.
\item
  Line where the symbol is located in the file. Lines start to count at
  \textbf{1}.
\item
  Column where the symbol is located in the file. Columns start to count
  at \textbf{0}.
\item
  Docstring for the symbol if available or the empty string. To
  differentiate the docstring from end of answer in server mode, the
  docstring is always provided enclosed in double quotes, and if the
  docstring spans multiple lines, all following lines of the docstring
  will start with a blank space to align visually with the starting
  quote.

  Also, you won't find raw \texttt{\textbackslash{}n} characters
  breaking the one answer per line format. Instead you will need to
  parse sequences in the form \texttt{\textbackslash{}xHH}, where
  \emph{HH} is a hexadecimal value (e.g. newlines generate the sequence
  \texttt{\textbackslash{}x0A}).
\end{enumerate}

The following sections define the expected output for each kind of
symbol for which idetools returns valid output.

\hypertarget{skconst}{%
\subsubsection{skConst}\label{skconst}}

\textbf{Third column}: module + {[}n scope nesting{]} + const name.\\
\textbf{Fourth column}: the type of the const value.\\
\textbf{Docstring}: always the empty string.

\begin{verbatim}
\end{verbatim}

\hypertarget{skenumfield}{%
\subsubsection{skEnumField}\label{skenumfield}}

\textbf{Third column}: module + {[}n scope nesting{]} + enum type + enum
field name.\\
\textbf{Fourth column}: enum type grouping other enum fields.\\
\textbf{Docstring}: always the empty string.

\begin{verbatim}
\end{verbatim}

\hypertarget{skforvar}{%
\subsubsection{skForVar}\label{skforvar}}

\textbf{Third column}: module + {[}n scope nesting{]} + var name.\\
\textbf{Fourth column}: type of the var.\\
\textbf{Docstring}: always the empty string.

\begin{verbatim}
\end{verbatim}

\hypertarget{skiterator-skclosureiterator}{%
\subsubsection{skIterator,
skClosureIterator}\label{skiterator-skclosureiterator}}

The fourth column will be the empty string if the iterator is being
defined, since at that point in the file the parser hasn't processed the
full line yet. The signature will be returned complete in posterior
instances of the iterator.

\textbf{Third column}: module + {[}n scope nesting{]} + iterator name.\\
\textbf{Fourth column}: signature of the iterator including return
type.\\
\textbf{Docstring}: docstring if available.

\begin{verbatim}
\end{verbatim}

\hypertarget{sklabel}{%
\subsubsection{skLabel}\label{sklabel}}

\textbf{Third column}: module + {[}n scope nesting{]} + name.\\
\textbf{Fourth column}: always the empty string.\\
\textbf{Docstring}: always the empty string.

\begin{verbatim}
\end{verbatim}

\hypertarget{sklet}{%
\subsubsection{skLet}\label{sklet}}

\textbf{Third column}: module + {[}n scope nesting{]} + let name.\\
\textbf{Fourth column}: the type of the let variable.\\
\textbf{Docstring}: always the empty string.

\begin{verbatim}
\end{verbatim}

\hypertarget{skmacro}{%
\subsubsection{skMacro}\label{skmacro}}

The fourth column will be the empty string if the macro is being
defined, since at that point in the file the parser hasn't processed the
full line yet. The signature will be returned complete in posterior
instances of the macro.

\textbf{Third column}: module + {[}n scope nesting{]} + macro name.\\
\textbf{Fourth column}: signature of the macro including return type.\\
\textbf{Docstring}: docstring if available.

\begin{verbatim}
\end{verbatim}

\hypertarget{skmethod}{%
\subsubsection{skMethod}\label{skmethod}}

The fourth column will be the empty string if the method is being
defined, since at that point in the file the parser hasn't processed the
full line yet. The signature will be returned complete in posterior
instances of the method.

Methods imply
\href{tut2.html\#object-oriented-programming-dynamic-dispatch}{dynamic
dispatch} and idetools performs a static analysis on the code. For this
reason idetools may not return the definition of the correct method you
are querying because it may be impossible to know until the code is
executed. It will try to return the method which covers the most
possible cases (i.e. for variations of different classes in a hierarchy
it will prefer methods using the base class).

While at the language level a method is differentiated from others by
the parameters and return value, the signature of the method returned by
idetools returns also the pragmas for the method.

Note that at the moment the word \texttt{proc} is returned for the
signature of the found method instead of the expected \texttt{method}.
This may change in the future.

\textbf{Third column}: module + {[}n scope nesting{]} + method name.\\
\textbf{Fourth column}: signature of the method including return type.\\
\textbf{Docstring}: docstring if available.

\begin{verbatim}
\end{verbatim}

\hypertarget{skparam}{%
\subsubsection{skParam}\label{skparam}}

\textbf{Third column}: module + {[}n scope nesting{]} + param name.\\
\textbf{Fourth column}: the type of the parameter.\\
\textbf{Docstring}: always the empty string.

\begin{verbatim}
\end{verbatim}

\hypertarget{skproc}{%
\subsubsection{skProc}\label{skproc}}

The fourth column will be the empty string if the proc is being defined,
since at that point in the file the parser hasn't processed the full
line yet. The signature will be returned complete in posterior instances
of the proc.

While at the language level a proc is differentiated from others by the
parameters and return value, the signature of the proc returned by
idetools returns also the pragmas for the proc.

\textbf{Third column}: module + {[}n scope nesting{]} + proc name.\\
\textbf{Fourth column}: signature of the proc including return type.\\
\textbf{Docstring}: docstring if available.

\begin{verbatim}
Default mode is readonly. Returns true iff the file could be opened.
This throws no exception if the file could not be opened."
\end{verbatim}

\hypertarget{skresult}{%
\subsubsection{skResult}\label{skresult}}

\textbf{Third column}: module + {[}n scope nesting{]} + result.\\
\textbf{Fourth column}: the type of the result.\\
\textbf{Docstring}: always the empty string.

\begin{verbatim}
\end{verbatim}

\hypertarget{sktemplate}{%
\subsubsection{skTemplate}\label{sktemplate}}

The fourth column will be the empty string if the template is being
defined, since at that point in the file the parser hasn't processed the
full line yet. The signature will be returned complete in posterior
instances of the template.

\textbf{Third column}: module + {[}n scope nesting{]} + template name.\\
\textbf{Fourth column}: signature of the template including return
type.\\
\textbf{Docstring}: docstring if available.

\begin{verbatim}
Example:

.. code-block:: nim
  let
    numeric = @[1, 2, 3, 4, 5, 6, 7, 8, 9]
    odd_numbers = toSeq(filter(numeric) do (x: int) -> bool:
      if x mod 2 == 1:
        result = true)
  assert odd_numbers == @[1, 3, 5, 7, 9]"
\end{verbatim}

\hypertarget{sktype}{%
\subsubsection{skType}\label{sktype}}

\textbf{Third column}: module + {[}n scope nesting{]} + type name.\\
\textbf{Fourth column}: the type.\\
\textbf{Docstring}: always the empty string.

\begin{verbatim}
\end{verbatim}

\hypertarget{skvar}{%
\subsubsection{skVar}\label{skvar}}

\textbf{Third column}: module + {[}n scope nesting{]} + var name.\\
\textbf{Fourth column}: the type of the var.\\
\textbf{Docstring}: always the empty string.

\begin{verbatim}
\end{verbatim}

\hypertarget{test-suite}{%
\subsection{Test suite}\label{test-suite}}

To verify that idetools is working properly there are files in the
\texttt{tests/caas/} directory which provide unit testing. If you find
odd idetools behaviour and are able to reproduce it, you are welcome to
report it as a bug and add a test to the suite to avoid future
regressions.

\hypertarget{running-the-test-suite}{%
\subsubsection{Running the test suite}\label{running-the-test-suite}}

At the moment idetools support is still in development so the test suite
is not integrated with the main test suite and you have to run it
manually. First you have to compile the tester:

\begin{verbatim}
$ cd my/nim/checkout/tests
$ nim c testament/caasdriver.nim
\end{verbatim}

Running the \texttt{caasdriver} without parameters will attempt to
process all the test cases in all three operation modes. If a test
succeeds nothing will be printed and the process will exit with zero. If
any test fails, the specific line of the test preceding the failure and
the failure itself will be dumped to stdout, along with a final
indicator of the success state and operation mode. You can pass the
parameter \texttt{verbose} to force all output even on successful tests.

The normal operation mode is called \texttt{ProcRun} and it involves
starting a process for each command or query, similar to running
manually the Nim compiler from the commandline. The \texttt{CaasRun}
mode starts a server process to answer all queries. The
\texttt{SymbolProcRun} mode is used by compiler developers. This means
that running all tests involves processing all \texttt{*.txt} files
three times, which can be quite time consuming.

If you don't want to run all the test case files you can pass any
substring as a parameter to \texttt{caasdriver}. Only files matching the
passed substring will be run. The filtering doesn't use any globbing
metacharacters, it's a plain match. For example, to run only
\texttt{*-compile*.txt} tests in verbose mode:

\begin{verbatim}
./caasdriver verbose -compile
\end{verbatim}

\hypertarget{test-case-file-format}{%
\subsubsection{Test case file format}\label{test-case-file-format}}

All the \texttt{tests/caas/*.txt} files encode a session with the
compiler:

\begin{itemize}
\tightlist
\item
  The first line indicates the main project file.
\item
  Lines starting with \texttt{\textgreater{}} indicate a command to be
  sent to the compiler and the lines following a command include checks
  for expected or forbidden output (\texttt{!} for forbidden).
\item
  If a line starts with \texttt{\#} it will be ignored completely, so
  you can use that for comments.
\item
  Since some cases are specific to either \texttt{ProcRun} or
  \texttt{CaasRun} modes, you can prefix a line with the mode and the
  line will be processed only in that mode.
\item
  The rest of the line is treated as a \href{re.html}{regular
  expression}, so be careful escaping metacharacters like parenthesis.
\end{itemize}

Before the line is processed as a regular expression, some basic
variables are searched for and replaced in the tests. The variables
which will be replaced are:

\begin{itemize}
\tightlist
\item
  \textbf{\$TESTNIM}: filename specified in the first line of the
  script.
\item
  \textbf{\$MODULE}: like \$TESTNIM but without extension, useful for
  expected output.
\end{itemize}

When adding a test case to the suite it is a good idea to write a few
comments about what the test is meant to verify.
