\hypertarget{nim-backend-integration}{%
\section{Nim Backend Integration}\label{nim-backend-integration}}

\begin{description}
\item[Author]
Puppet Master
\item[Version]
1.3.5
\end{description}

"Heresy grows from idleness." -\/- Unknown.

\hypertarget{introduction}{%
\subsection{Introduction}\label{introduction}}

The \href{nimc.html}{Nim Compiler User Guide} documents the typical
compiler invocation, using the \texttt{compile} or \texttt{c} command to
transform a \texttt{.nim} file into one or more \texttt{.c} files which
are then compiled with the platform's C compiler into a static binary.
However there are other commands to compile to C++, Objective-C or
JavaScript. This document tries to concentrate in a single place all the
backend and interfacing options.

The Nim compiler supports mainly two backend families: the C, C++ and
Objective-C targets and the JavaScript target.
\protect\hyperlink{backends-the-c-like-targets}{The C like targets}
creates source files which can be compiled into a library or a final
executable. \protect\hyperlink{backends-the-javascript-target}{The
JavaScript target} can generate a \texttt{.js} file which you reference
from an HTML file or create a \href{http://nodejs.org}{standalone nodejs
program}.

On top of generating libraries or standalone applications, Nim offers
bidirectional interfacing with the backend targets through generic and
specific pragmas.

\hypertarget{backends}{%
\subsection{Backends}\label{backends}}

\hypertarget{the-c-like-targets}{%
\subsubsection{The C like targets}\label{the-c-like-targets}}

The commands to compile to either C, C++ or Objective-C are:

\begin{quote}
//compileToC, cc compile project with C code generator //compileToCpp,
cpp compile project to C++ code //compileToOC, objc compile project to
Objective C code
\end{quote}

The most significant difference between these commands is that if you
look into the \texttt{nimcache} directory you will find \texttt{.c},
\texttt{.cpp} or \texttt{.m} files, other than that all of them will
produce a native binary for your project. This allows you to take the
generated code and place it directly into a project using any of these
languages. Here are some typical command line invocations:

\begin{verbatim}
$ nim c hallo.nim
$ nim cpp hallo.nim
$ nim objc hallo.nim
\end{verbatim}

The compiler commands select the target backend, but if needed you can
\href{nimc.html\#cross-compilation}{specify additional switches for
cross compilation} to select the target CPU, operative system or
compiler/linker commands.

\hypertarget{the-javascript-target}{%
\subsubsection{The JavaScript target}\label{the-javascript-target}}

Nim can also generate \texttt{JavaScript} code through the \texttt{js}
command.

Nim targets JavaScript 1.5 which is supported by any widely used
browser. Since JavaScript does not have a portable means to include
another module, Nim just generates a long \texttt{.js} file.

Features or modules that the JavaScript platform does not support are
not available. This includes:

\begin{itemize}
\tightlist
\item
  manual memory management (\texttt{alloc}, etc.)
\item
  casting and other unsafe operations (\texttt{cast} operator,
  \texttt{zeroMem}, etc.)
\item
  file management
\item
  OS-specific operations
\item
  threading, coroutines
\item
  some modules of the standard library
\item
  proper 64 bit integer arithmetic
\end{itemize}

To compensate, the standard library has modules
\href{lib.html\#pure-libraries-modules-for-js-backend}{catered to the JS
backend} and more support will come in the future (for instance, Node.js
bindings to get OS info).

To compile a Nim module into a \texttt{.js} file use the \texttt{js}
command; the default is a \texttt{.js} file that is supposed to be
referenced in an \texttt{.html} file. However, you can also run the code
with \texttt{nodejs} (\url{http://nodejs.org}):

\begin{verbatim}
nim js -d:nodejs -r examples/hallo.nim
\end{verbatim}

\hypertarget{interfacing}{%
\subsection{Interfacing}\label{interfacing}}

Nim offers bidirectional interfacing with the target backend. This means
that you can call backend code from Nim and Nim code can be called by
the backend code. Usually the direction of which calls which depends on
your software architecture (is Nim your main program or is Nim providing
a component?).

\hypertarget{nim-code-calling-the-backend}{%
\subsubsection{Nim code calling the
backend}\label{nim-code-calling-the-backend}}

Nim code can interface with the backend through the
\href{manual.html\#foreign-function-interface}{Foreign function
interface} mainly through the
\href{manual.html\#foreign-function-interface-importc-pragma}{importc
pragma}. The \texttt{importc} pragma is the \emph{generic} way of making
backend symbols available in Nim and is available in all the target
backends (JavaScript too). The C++ or Objective-C backends have their
respective
\href{manual.html\#implementation-specific-pragmas-importcpp-pragma}{ImportCpp}
and
\href{manual.html\#implementation-specific-pragmas-importobjc-pragma}{ImportObjC}
pragmas to call methods from classes.

Whenever you use any of these pragmas you need to integrate native code
into your final binary. In the case of JavaScript this is no problem at
all, the same html file which hosts the generated JavaScript will likely
provide other JavaScript functions which you are importing with
\texttt{importc}.

However, for the C like targets you need to link external code either
statically or dynamically. The preferred way of integrating native code
is to use dynamic linking because it allows you to compile Nim programs
without the need for having the related development libraries installed.
This is done through the
\href{manual.html\#foreign-function-interface-dynlib-pragma-for-import}{dynlib
pragma for import}, though more specific control can be gained using the
\href{dynlib.html}{dynlib module}.

The \href{nimc.html\#dynliboverride}{dynlibOverride} command line switch
allows to avoid dynamic linking if you need to statically link something
instead. Nim wrappers designed to statically link source files can use
the
\href{manual.html\#implementation-specific-pragmas-compile-pragma}{compile
pragma} if there are few sources or providing them along the Nim code is
easier than using a system library. Libraries installed on the host
system can be linked in with the
\href{manual.html\#implementation-specific-pragmas-passl-pragma}{PassL
pragma}.

To wrap native code, take a look at the
\href{https://github.com/nim-lang/c2nim/blob/master/doc/c2nim.rst}{c2nim
tool} which helps with the process of scanning and transforming header
files into a Nim interface.

\hypertarget{c-invocation-example}{%
\paragraph{C invocation example}\label{c-invocation-example}}

Create a \texttt{logic.c} file with the following content:

\begin{verbatim}
\end{verbatim}

Create a \texttt{calculator.nim} file with the following content:

\begin{verbatim}
{.compile: "logic.c".}
proc addTwoIntegers(a, b: cint): cint {.importc.}

when isMainModule:
  echo addTwoIntegers(3, 7)
\end{verbatim}

With these two files in place, you can run
\texttt{nim\ c\ -r\ calculator.nim} and the Nim compiler will compile
the \texttt{logic.c} file in addition to \texttt{calculator.nim} and
link both into an executable, which outputs \texttt{10} when run.
Another way to link the C file statically and get the same effect would
be remove the line with the \texttt{compile} pragma and run the
following typical Unix commands:

\begin{verbatim}
$ gcc -c logic.c
$ ar rvs mylib.a logic.o
$ nim c --passL:mylib.a -r calculator.nim
\end{verbatim}

Just like in this example we pass the path to the \texttt{mylib.a}
library (and we could as well pass \texttt{logic.o}) we could be passing
switches to link any other static C library.

\hypertarget{javascript-invocation-example}{%
\paragraph{JavaScript invocation
example}\label{javascript-invocation-example}}

Create a \texttt{host.html} file with the following content:

\begin{verbatim}
<html><body>
<script type="text/javascript">
function addTwoIntegers(a, b)
{
  return a + b;
}
</script>
<script type="text/javascript" src="calculator.js"></script>
</body></html>
\end{verbatim}

Create a \texttt{calculator.nim} file with the following content (or
reuse the one from the previous section):

\begin{verbatim}
proc addTwoIntegers(a, b: int): int {.importc.}

when isMainModule:
  echo addTwoIntegers(3, 7)
\end{verbatim}

Compile the Nim code to JavaScript with
\texttt{nim\ js\ -o:calculator.js\ calculator.nim} and open
\texttt{host.html} in a browser. If the browser supports javascript, you
should see the value \texttt{10} in the browser's console. Use the
\href{dom.html}{dom module} for specific DOM querying and modification
procs or take a look at \href{https://github.com/pragmagic/karax}{karax}
for how to develop browser based applications.

\hypertarget{backend-code-calling-nim}{%
\subsubsection{Backend code calling
Nim}\label{backend-code-calling-nim}}

Backend code can interface with Nim code exposed through the
\href{manual.html\#foreign-function-interface-exportc-pragma}{exportc
pragma}. The \texttt{exportc} pragma is the \emph{generic} way of making
Nim symbols available to the backends. By default the Nim compiler will
mangle all the Nim symbols to avoid any name collision, so the most
significant thing the \texttt{exportc} pragma does is maintain the Nim
symbol name, or if specified, use an alternative symbol for the backend
in case the symbol rules don't match.

The JavaScript target doesn't have any further interfacing
considerations since it also has garbage collection, but the C targets
require you to initialize Nim's internals, which is done calling a
\texttt{NimMain} function. Also, C code requires you to specify a
forward declaration for functions or the compiler will assume certain
types for the return value and parameters which will likely make your
program crash at runtime.

The Nim compiler can generate a C interface header through the
\texttt{-\/-header} command line switch. The generated header will
contain all the exported symbols and the \texttt{NimMain} proc which you
need to call before any other Nim code.

\hypertarget{nim-invocation-example-from-c}{%
\paragraph{Nim invocation example from
C}\label{nim-invocation-example-from-c}}

Create a \texttt{fib.nim} file with the following content:

\begin{verbatim}
proc fib(a: cint): cint {.exportc.} =
  if a <= 2:
    result = 1
  else:
    result = fib(a - 1) + fib(a - 2)
\end{verbatim}

Create a \texttt{maths.c} file with the following content:

\begin{Shaded}
\begin{Highlighting}[]
\PreprocessorTok{\#include }\ImportTok{"fib.h"}
\PreprocessorTok{\#include }\ImportTok{\textless{}stdio.h\textgreater{}}

\DataTypeTok{int}\NormalTok{ main(}\DataTypeTok{void}\NormalTok{)}
\NormalTok{\{}
\NormalTok{  NimMain();}
  \ControlFlowTok{for}\NormalTok{ (}\DataTypeTok{int}\NormalTok{ f = }\DecValTok{0}\NormalTok{; f \textless{} }\DecValTok{10}\NormalTok{; f++)}
\NormalTok{    printf(}\StringTok{"Fib of \%d is \%d}\SpecialCharTok{\textbackslash{}n}\StringTok{"}\NormalTok{, f, fib(f));}
  \ControlFlowTok{return} \DecValTok{0}\NormalTok{;}
\NormalTok{\}}
\end{Highlighting}
\end{Shaded}

Now you can run the following Unix like commands to first generate C
sources from the Nim code, then link them into a static binary along
your main C program:

\begin{verbatim}
$ nim c --noMain --noLinking --header:fib.h fib.nim
$ gcc -o m -I$HOME/.cache/nim/fib_d -Ipath/to/nim/lib $HOME/.cache/nim/fib_d/*.c maths.c
\end{verbatim}

The first command runs the Nim compiler with three special options to
avoid generating a \texttt{main()} function in the generated files,
avoid linking the object files into a final binary, and explicitly
generate a header file for C integration. All the generated files are
placed into the \texttt{nimcache} directory. That's why the next command
compiles the \texttt{maths.c} source plus all the \texttt{.c} files from
\texttt{nimcache}. In addition to this path, you also have to tell the C
compiler where to find Nim's \texttt{nimbase.h} header file.

Instead of depending on the generation of the individual \texttt{.c}
files you can also ask the Nim compiler to generate a statically linked
library:

\begin{verbatim}
$ nim c --app:staticLib --noMain --header fib.nim
$ gcc -o m -Inimcache -Ipath/to/nim/lib libfib.nim.a maths.c
\end{verbatim}

The Nim compiler will handle linking the source files generated in the
\texttt{nimcache} directory into the \texttt{libfib.nim.a} static
library, which you can then link into your C program. Note that these
commands are generic and will vary for each system. For instance, on
Linux systems you will likely need to use \texttt{-ldl} too to link in
required dlopen functionality.

\hypertarget{nim-invocation-example-from-javascript}{%
\paragraph{Nim invocation example from
JavaScript}\label{nim-invocation-example-from-javascript}}

Create a \texttt{mhost.html} file with the following content:

\begin{verbatim}
<html><body>
<script type="text/javascript" src="fib.js"></script>
<script type="text/javascript">
alert("Fib for 9 is " + fib(9));
</script>
</body></html>
\end{verbatim}

Create a \texttt{fib.nim} file with the following content (or reuse the
one from the previous section):

\begin{verbatim}
proc fib(a: cint): cint {.exportc.} =
  if a <= 2:
    result = 1
  else:
    result = fib(a - 1) + fib(a - 2)
\end{verbatim}

Compile the Nim code to JavaScript with
\texttt{nim\ js\ -o:fib.js\ fib.nim} and open \texttt{mhost.html} in a
browser. If the browser supports javascript, you should see an alert box
displaying the text \texttt{Fib\ for\ 9\ is\ 34}. As mentioned earlier,
JavaScript doesn't require an initialisation call to \texttt{NimMain} or
similar function and you can call the exported Nim proc directly.

\hypertarget{nimcache-naming-logic}{%
\subsubsection{Nimcache naming logic}\label{nimcache-naming-logic}}

The \texttt{nimcache} directory is generated during compilation and will
hold either temporary or final files depending on your backend target.
The default name for the directory depends on the used backend and on
your OS but you can use the \texttt{-\/-nimcache}
\href{nimc.html\#compiler-usage-command-line-switches}{compiler switch}
to change it.

\hypertarget{memory-management}{%
\subsection{Memory management}\label{memory-management}}

In the previous sections the \texttt{NimMain()} function reared its
head. Since JavaScript already provides automatic memory management, you
can freely pass objects between the two language without problems. In C
and derivate languages you need to be careful about what you do and how
you share memory. The previous examples only dealt with simple scalar
values, but passing a Nim string to C, or reading back a C string in Nim
already requires you to be aware of who controls what to avoid crashing.

\hypertarget{strings-and-c-strings}{%
\subsubsection{Strings and C strings}\label{strings-and-c-strings}}

The manual mentions that \href{manual.html\#types-cstring-type}{Nim
strings are implicitly convertible to cstrings} which makes interaction
usually painless. Most C functions accepting a Nim string converted to a
\texttt{cstring} will likely not need to keep this string around and by
the time they return the string won't be needed any more. However, for
the rare cases where a Nim string has to be preserved and made available
to the C backend as a \texttt{cstring}, you will need to manually
prevent the string data from being freed with
\href{system.html\#GC_ref,string}{GC\_ref} and
\href{system.html\#GC_unref,string}{GC\_unref}.

A similar thing happens with C code invoking Nim code which returns a
\texttt{cstring}. Consider the following proc:

\begin{verbatim}
proc gimme(): cstring {.exportc.} =
  result = "Hey there C code! " & $rand(100)
\end{verbatim}

Since Nim's garbage collector is not aware of the C code, once the
\texttt{gimme} proc has finished it can reclaim the memory of the
\texttt{cstring}. However, from a practical standpoint, the C code
invoking the \texttt{gimme} function directly will be able to use it
since Nim's garbage collector has not had a chance to run \emph{yet}.
This gives you enough time to make a copy for the C side of the program,
as calling any further Nim procs \emph{might} trigger garbage collection
making the previously returned string garbage. Or maybe you are
\href{gc.html}{yourself triggering the collection}.

\hypertarget{custom-data-types}{%
\subsubsection{Custom data types}\label{custom-data-types}}

Just like strings, custom data types that are to be shared between Nim
and the backend will need careful consideration of who controls who. If
you want to hand a Nim reference to C code, you will need to use
\href{system.html\#GC_ref,ref.T}{GC\_ref} to mark the reference as used,
so it does not get freed. And for the C backend you will need to expose
the \href{system.html\#GC_unref,ref.T}{GC\_unref} proc to clean up this
memory when it is not required any more.

Again, if you are wrapping a library which \emph{mallocs} and
\emph{frees} data structures, you need to expose the appropriate
\emph{free} function to Nim so you can clean it up. And of course, once
cleaned you should avoid accessing it from Nim (or C for that matter).
Typically C data structures have their own \texttt{malloc\_structure}
and \texttt{free\_structure} specific functions, so wrapping these for
the Nim side should be enough.

\hypertarget{thread-coordination}{%
\subsubsection{Thread coordination}\label{thread-coordination}}

When the \texttt{NimMain()} function is called Nim initializes the
garbage collector to the current thread, which is usually the main
thread of your application. If your C code later spawns a different
thread and calls Nim code, the garbage collector will fail to work
properly and you will crash.

As long as you don't use the threadvar emulation Nim uses native thread
variables, of which you get a fresh version whenever you create a
thread. You can then attach a GC to this thread via

\begin{verbatim}
system.setupForeignThreadGc()
\end{verbatim}

It is \textbf{not} safe to disable the garbage collector and enable it
after the call from your background thread even if the code you are
calling is short lived.

Before the thread exits, you should tear down the thread's GC to prevent
memory leaks by calling

\begin{verbatim}
system.tearDownForeignThreadGc()
\end{verbatim}
