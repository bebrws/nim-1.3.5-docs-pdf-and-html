\hypertarget{documentation-style}{%
\section{Documentation Style}\label{documentation-style}}

\hypertarget{general-guidelines}{%
\subsection{General Guidelines}\label{general-guidelines}}

\begin{itemize}
\tightlist
\item
  See also \href{https://nim-lang.github.io/Nim/nep1.html}{nep1} which
  should probably be merged here.
\item
  Authors should document anything that is exported; documentation for
  private procs can be useful too (visible via
  \texttt{nim\ doc\ -\/-docInternal\ foo.nim}).
\item
  Within documentation, a period ({.}) should follow each sentence (or
  sentence fragment) in a comment block. The documentation may be
  limited to one sentence fragment, but if multiple sentences are within
  the documentation, each sentence after the first should be complete
  and in present tense.
\item
  Documentation is parsed as a custom ReStructuredText (RST) with
  partial markdown support.
\item
  In nim sources, prefer single backticks to double backticks since it's
  simpler and {nim doc} supports it (even in rst files with {nim
  rst2html}).
\item
  In nim sources, for links, prefer {{[}link text{]}(link.html)} to
  \texttt{\textasciigrave{}\ \textasciigrave{}link\ text\textless{}link.html\textgreater{}\textasciigrave{}\_}{
  since the syntax is simpler and markdown is more common (likewise,
  `nim rst2html} also supports it in rst files).
\end{itemize}

\begin{verbatim}
proc someproc*(s: string, foo: int) =
  ## Use single backticks for inline code, eg: `s` or `someExpr(true)`.
  ## Use a backlash to follow with alphanumeric char: `int8`\s are great.
\end{verbatim}

\hypertarget{module-level-documentation}{%
\subsection{Module-level
documentation}\label{module-level-documentation}}

Documentation of a module is placed at the top of the module itself.
Each line of documentation begins with double hashes (\texttt{\#\#}).
Sometimes \texttt{\#\#{[}\ multiline\ docs\ containing\ code\ {]}\#\#}
is preferable, see \texttt{lib/pure/times.nim}. Code samples are
encouraged, and should follow the general RST syntax:

\begin{verbatim}
## The `universe` module computes the answer to life, the universe, and everything.
##
## .. code-block::
##  doAssert computeAnswerString() == 42
\end{verbatim}

Within this top-level comment, you can indicate the authorship and
copyright of the code, which will be featured in the produced
documentation.

\begin{verbatim}
## This is the best module ever. It provides answers to everything!
##
## :Author: Steve McQueen
## :Copyright: 1965
##
\end{verbatim}

Leave a space between the last line of top-level documentation and the
beginning of Nim code (the imports, etc.).

\hypertarget{procs-templates-macros-converters-and-iterators}{%
\subsection{Procs, Templates, Macros, Converters, and
Iterators}\label{procs-templates-macros-converters-and-iterators}}

The documentation of a procedure should begin with a capital letter and
should be in present tense. Variables referenced in the documentation
should be surrounded by single tick marks:

\begin{verbatim}
proc example1*(x: int) =
  ## Prints the value of `x`.
  echo x
\end{verbatim}

Whenever an example of usage would be helpful to the user, you should
include one within the documentation in RST format as below.

\begin{verbatim}
proc addThree*(x, y, z: int8): int =
  ## Adds three `int8` values, treating them as unsigned and
  ## truncating the result.
  ##
  ## .. code-block::
  ##  # things that aren't suitable for a `runnableExamples` go in code-block:
  ##  echo execCmdEx("git pull")
  ##  drawOnScreen()
  runnableExamples:
    # `runnableExamples` is usually preferred to `code-block`, when possible.
    doAssert addThree(3, 125, 6) == -122
  result = x +% y +% z
\end{verbatim}

The commands \texttt{nim\ doc} and \texttt{nim\ doc2} will then
correctly syntax highlight the Nim code within the documentation.

\hypertarget{types}{%
\subsection{Types}\label{types}}

Exported types should also be documented. This documentation can also
contain code samples, but those are better placed with the functions to
which they refer.

\begin{verbatim}
type
  NamedQueue*[T] = object ## Provides a linked data structure with names
                          ## throughout. It is named for convenience. I'm making
                          ## this comment long to show how you can, too.
    name*: string ## The name of the item
    val*: T ## Its value
    next*: ref NamedQueue[T] ## The next item in the queue
\end{verbatim}

You have some flexibility when placing the documentation:

\begin{verbatim}
type
  NamedQueue*[T] = object
    ## Provides a linked data structure with names
    ## throughout. It is named for convenience. I'm making
    ## this comment long to show how you can, too.
    name*: string ## The name of the item
    val*: T ## Its value
    next*: ref NamedQueue[T] ## The next item in the queue
\end{verbatim}

Make sure to place the documentation beside or within the object.

\begin{verbatim}
type
  ## Bad: this documentation disappears because it annotates the ``type`` keyword
  ## above, not ``NamedQueue``.
  NamedQueue*[T] = object
    name*: string ## This becomes the main documentation for the object, which
                  ## is not what we want.
    val*: T ## Its value
    next*: ref NamedQueue[T] ## The next item in the queue
\end{verbatim}

\hypertarget{var-let-and-const}{%
\subsection{Var, Let, and Const}\label{var-let-and-const}}

When declaring module-wide constants and values, documentation is
encouraged. The placement of doc comments is similar to the
\texttt{type} sections.

\begin{verbatim}
const
  X* = 42 ## An awesome number.
  SpreadArray* = [
    [1,2,3],
    [2,3,1],
    [3,1,2],
  ] ## Doc comment for ``SpreadArray``.
\end{verbatim}

Placement of comments in other areas is usually allowed, but will not
become part of the documentation output and should therefore be prefaced
by a single hash (\texttt{\#}).

\begin{verbatim}
const
  BadMathVals* = [
    3.14, # pi
    2.72, # e
    0.58, # gamma
  ] ## A bunch of badly rounded values.
\end{verbatim}

Nim supports Unicode in comments, so the above can be replaced with the
following:

\begin{verbatim}
const
  BadMathVals* = [
    3.14, # π
    2.72, # e
    0.58, # γ
  ] ## A bunch of badly rounded values (including π!).
\end{verbatim}
