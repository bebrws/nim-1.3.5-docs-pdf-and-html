\hypertarget{nimfix-user-guide}{%
\section{Nimfix User Guide}\label{nimfix-user-guide}}

\begin{description}
\item[Author]
Andreas Rumpf
\item[Version]
1.3.5
\end{description}

\textbf{WARNING}: Nimfix is currently beta-quality.

Nimfix is a tool to help you upgrade from Nimrod (\textless= version
0.9.6) to Nim (=\textgreater{} version 0.10.0).

It performs 3 different actions:

\begin{enumerate}
\def\labelenumi{\arabic{enumi}.}
\tightlist
\item
  It makes your code case consistent.
\item
  It renames every symbol that has a deprecation rule. So if a module
  has a rule \texttt{\{.deprecated:\ {[}TFoo:\ Foo{]}.\}} then
  \texttt{TFoo} is replaced by \texttt{Foo}.
\item
  It can also check that your identifiers adhere to the official style
  guide and optionally modify them to do so (via
  \texttt{-\/-styleCheck:auto}).
\end{enumerate}

Note that \texttt{nimfix} defaults to \textbf{overwrite} your code
unless you use \texttt{-\/-overwriteFiles:off}! But hey, if you do not
use a version control system by this day and age, your project is
already in big trouble.

\hypertarget{installation}{%
\subsection{Installation}\label{installation}}

Nimfix is part of the compiler distribution. Compile via:

\begin{verbatim}
nim c compiler/nimfix/nimfix.nim
mv compiler/nimfix/nimfix bin
\end{verbatim}

Or on windows:

\begin{verbatim}
nim c compiler\nimfix\nimfix.nim
move compiler\nimfix\nimfix.exe bin
\end{verbatim}

\hypertarget{usage}{%
\subsection{Usage}\label{usage}}

\begin{description}
\item[Usage:]
nimfix {[}options{]} projectfile.nim
\end{description}

Options:

\begin{quote}
-\/-overwriteFiles:onoff style check also extern names
-\/-styleCheck:onauto performs style checking for identifiers and
suggests an alternative spelling; 'auto' corrects the spelling.
\end{quote}

In addition, all command line options of Nim are supported.
