\hypertarget{nim-tutorial-part-iii}{%
\section{Nim Tutorial (Part III)}\label{nim-tutorial-part-iii}}

\begin{description}
\item[Author]
Arne Döring
\item[Version]
1.3.5
\end{description}

\hypertarget{introduction}{%
\subsection{Introduction}\label{introduction}}

\begin{quote}
"With Great Power Comes Great Responsibility." -\/- Spider Man's Uncle
\end{quote}

This document is a tutorial about Nim's macro system. A macro is a
function that is executed at compile time and transforms a Nim syntax
tree into a different tree.

Examples of things that can be implemented in macros:

\begin{itemize}
\tightlist
\item
  An assert macro that prints both sides of a comparison operator, if
  the assertion fails. \texttt{myAssert(a\ ==\ b)} is converted to
  \texttt{if\ a\ !=\ b:\ quit(\$a\ "\ !=\ "\ \$b)}
\item
  A debug macro that prints the value and the name of the symbol.
  \texttt{myDebugEcho(a)} is converted to \texttt{echo\ "a:\ ",\ a}
\item
  Symbolic differentiation of an expression.
  \texttt{diff(a*pow(x,3)\ +\ b*pow(x,2)\ +\ c*x\ +\ d,\ x)} is
  converted to \texttt{3*a*pow(x,2)\ +\ 2*b*x\ +\ c}
\end{itemize}

\hypertarget{macro-arguments}{%
\subsubsection{Macro Arguments}\label{macro-arguments}}

The types of macro arguments have two faces. One face is used for the
overload resolution, and the other face is used within the macro body.
For example, if \texttt{macro\ foo(arg:\ int)} is called in an
expression \texttt{foo(x)}, \texttt{x} has to be of a type compatible to
int, but \emph{within} the macro's body \texttt{arg} has the type
\texttt{NimNode}, not \texttt{int}! Why it is done this way will become
obvious later, when we have seen concrete examples.

There are two ways to pass arguments to a macro, an argument can be
either \texttt{typed} or \texttt{untyped}.

\hypertarget{untyped-arguments}{%
\subsubsection{Untyped Arguments}\label{untyped-arguments}}

Untyped macro arguments are passed to the macro before they are
semantically checked. This means the syntax tree that is passed down to
the macro does not need to make sense for Nim yet, the only limitation
is that it needs to be parsable. Usually the macro does not check the
argument either but uses it in the transformation's result somehow. The
result of a macro expansion is always checked by the compiler, so apart
from weird error messages nothing bad can happen.

The downside for an \texttt{untyped} argument is that these do not play
well with Nim's overloading resolution.

The upside for untyped arguments is that the syntax tree is quite
predictable and less complex compared to its \texttt{typed} counterpart.

\hypertarget{typed-arguments}{%
\subsubsection{Typed Arguments}\label{typed-arguments}}

For typed arguments, the semantic checker runs on the argument and does
transformations on it, before it is passed to the macro. Here identifier
nodes are resolved as symbols, implicit type conversions are visible in
the tree as calls, templates are expanded and probably most importantly,
nodes have type information. Typed arguments can have the type
\texttt{typed} in the arguments list. But all other types, such as
\texttt{int}, \texttt{float} or \texttt{MyObjectType} are typed
arguments as well, and they are passed to the macro as a syntax tree.

\hypertarget{static-arguments}{%
\subsubsection{Static Arguments}\label{static-arguments}}

Static arguments are a way to pass values as values and not as syntax
tree nodes to a macro. For example for
\texttt{macro\ foo(arg:\ static{[}int{]})} in the expression
\texttt{foo(x)}, \texttt{x} needs to be an integer constant, but in the
macro body \texttt{arg} is just like a normal parameter of type
\texttt{int}.

\begin{verbatim}
import macros

macro myMacro(arg: static[int]): untyped =
  echo arg # just an int (7), not ``NimNode``

myMacro(1 + 2 * 3)
\end{verbatim}

\hypertarget{code-blocks-as-arguments}{%
\subsubsection{Code Blocks as
Arguments}\label{code-blocks-as-arguments}}

It is possible to pass the last argument of a call expression in a
separate code block with indentation. For example the following code
example is a valid (but not a recommended) way to call \texttt{echo}:

\begin{verbatim}
echo "Hello ":
  let a = "Wor"
  let b = "ld!"
  a & b
\end{verbatim}

For macros this way of calling is very useful; syntax trees of arbitrary
complexity can be passed to macros with this notation.

\hypertarget{the-syntax-tree}{%
\subsubsection{The Syntax Tree}\label{the-syntax-tree}}

In order to build a Nim syntax tree one needs to know how Nim source
code is represented as a syntax tree, and how such a tree needs to look
like so that the Nim compiler will understand it. The nodes of the Nim
syntax tree are documented in the \href{macros.html}{macros} module. But
a more interactive way to explore the Nim syntax tree is with
\texttt{macros.treeRepr}, it converts a syntax tree into a multi line
string for printing on the console. It can be used to explore how the
argument expressions are represented in tree form and for debug printing
of generated syntax tree. \texttt{dumpTree} is a predefined macro that
just prints its argument in tree representation, but does nothing else.
Here is an example of such a tree representation:

\begin{verbatim}
dumpTree:
  var mt: MyType = MyType(a:123.456, b:"abcdef")

# output:
#   StmtList
#     VarSection
#       IdentDefs
#         Ident "mt"
#         Ident "MyType"
#         ObjConstr
#           Ident "MyType"
#           ExprColonExpr
#             Ident "a"
#             FloatLit 123.456
#           ExprColonExpr
#             Ident "b"
#             StrLit "abcdef"
\end{verbatim}

Custom Semantic Checking
-\/-\/-\/-\/-\/-\/-\/-\/-\/-\/-\/-\/-\/-\/-\/-\/-\/-\/-\/-\/-\/-\/-

The first thing that a macro should do with its arguments is to check if
the argument is in the correct form. Not every type of wrong input needs
to be caught here, but anything that could cause a crash during macro
evaluation should be caught and create a nice error message.
\texttt{macros.expectKind} and \texttt{macros.expectLen} are a good
start. If the checks need to be more complex, arbitrary error messages
can be created with the \texttt{macros.error} proc.

\begin{verbatim}
macro myAssert(arg: untyped): untyped =
  arg.expectKind nnkInfix
\end{verbatim}

\hypertarget{generating-code}{%
\subsubsection{Generating Code}\label{generating-code}}

There are two ways to generate the code. Either by creating the syntax
tree with expressions that contain a lot of calls to \texttt{newTree}
and \texttt{newLit}, or with \texttt{quote\ do:} expressions. The first
option offers the best low level control for the syntax tree generation,
but the second option is much less verbose. If you choose to create the
syntax tree with calls to \texttt{newTree} and \texttt{newLit} the macro
\texttt{macros.dumpAstGen} can help you with the verbosity.
\texttt{quote\ do:} allows you to write the code that you want to
generate literally, backticks are used to insert code from
\texttt{NimNode} symbols into the generated expression. This means that
you can't use backticks within \texttt{quote\ do:} for anything else
than injecting symbols. Make sure to inject only symbols of type
\texttt{NimNode} into the generated syntax tree. You can use
\texttt{newLit} to convert arbitrary values into expressions trees of
type \texttt{NimNode} so that it is safe to inject them into the tree.

\begin{verbatim}
import macros

type
  MyType = object
    a: float
    b: string

macro myMacro(arg: untyped): untyped =
  var mt: MyType = MyType(a:123.456, b:"abcdef")

  # ...

  let mtLit = newLit(mt)

  result = quote do:
    echo `arg`
    echo `mtLit`

myMacro("Hallo")
\end{verbatim}

The call to \texttt{myMacro} will generate the following code:

\begin{verbatim}
\end{verbatim}

\hypertarget{building-your-first-macro}{%
\subsubsection{Building Your First
Macro}\label{building-your-first-macro}}

To give a starting point to writing macros we will show now how to
implement the \texttt{myDebug} macro mentioned earlier. The first thing
to do is to build a simple example of the macro usage, and then just
print the argument. This way it is possible to get an idea of a correct
argument should be look like.

\begin{verbatim}
import macros

macro myAssert(arg: untyped): untyped =
  echo arg.treeRepr

let a = 1
let b = 2

myAssert(a != b)
\end{verbatim}

\begin{verbatim}
Infix
  Ident "!="
  Ident "a"
  Ident "b"
\end{verbatim}

From the output it is possible to see that the argument is an infix
operator (node kind is "Infix"), as well as that the two operands are at
index 1 and 2. With this information the actual macro can be written.

\begin{verbatim}
import macros

macro myAssert(arg: untyped): untyped =
  # all node kind identifiers are prefixed with "nnk"
  arg.expectKind nnkInfix
  arg.expectLen 3
  # operator as string literal
  let op  = newLit(" " & arg[0].repr & " ")
  let lhs = arg[1]
  let rhs = arg[2]

  result = quote do:
    if not `arg`:
      raise newException(AssertionDefect,$`lhs` & `op` & $`rhs`)

let a = 1
let b = 2

myAssert(a != b)
myAssert(a == b)
\end{verbatim}

This is the code that will be generated. To debug what the macro
actually generated, the statement \texttt{echo\ result.repr} can be
used, in the last line of the macro. It is also the statement that has
been used to get this output.

\begin{verbatim}
\end{verbatim}

\hypertarget{with-power-comes-responsibility}{%
\subsubsection{With Power Comes
Responsibility}\label{with-power-comes-responsibility}}

Macros are very powerful. A good advice is to use them as little as
possible, but as much as necessary. Macros can change the semantics of
expressions, making the code incomprehensible for anybody who does not
know exactly what the macro does with it. So whenever a macro is not
necessary and the same logic can be implemented using templates or
generics, it is probably better not to use a macro. And when a macro is
used for something, the macro should better have a well written
documentation. For all the people who claim to write only perfectly
self-explanatory code: when it comes to macros, the implementation is
not enough for documentation.

\hypertarget{limitations}{%
\subsubsection{Limitations}\label{limitations}}

Since macros are evaluated in the compiler in the NimVM, macros share
all the limitations of the NimVM. They have to be implemented in pure
Nim code. Macros can start external processes on the shell, but they
cannot call C functions except from those that are built in the
compiler.

\hypertarget{more-examples}{%
\subsection{More Examples}\label{more-examples}}

This tutorial can only cover the basics of the macro system. There are
macros out there that could be an inspiration for you of what is
possible with it.

\hypertarget{strformat}{%
\subsubsection{Strformat}\label{strformat}}

In the Nim standard library, the \texttt{strformat} library provides a
macro that parses a string literal at compile time. Parsing a string in
a macro like here is generally not recommended. The parsed AST cannot
have type information, and parsing implemented on the VM is generally
not very fast. Working on AST nodes is almost always the recommended
way. But still \texttt{strformat} is a good example for a practical use
case for a macro that is slightly more complex than the \texttt{assert}
macro.

\href{https://github.com/nim-lang/Nim/blob/5845716df8c96157a047c2bd6bcdd795a7a2b9b1/lib/pure/strformat.nim\#L280}{Strformat}

\hypertarget{ast-pattern-matching}{%
\subsubsection{Ast Pattern Matching}\label{ast-pattern-matching}}

Ast Pattern Matching is a macro library to aid in writing complex
macros. This can be seen as a good example of how to repurpose the Nim
syntax tree with new semantics.

\href{https://github.com/krux02/ast-pattern-matching}{Ast Pattern
Matching}

\hypertarget{opengl-sandbox}{%
\subsubsection{OpenGL Sandbox}\label{opengl-sandbox}}

This project has a working Nim to GLSL compiler written entirely in
macros. It scans recursively though all used function symbols to compile
them so that cross library functions can be executed on the GPU.

\href{https://github.com/krux02/opengl-sandbox}{OpenGL Sandbox}
