============================================== Nim Enhancement Proposal
\#1 - Standard Library Style Guide
============================================== :Author: Clay Sweetser,
Dominik Picheta :Version: 1.3.5

\hypertarget{introduction}{%
\section{Introduction}\label{introduction}}

Although Nim supports a variety of code and formatting styles, it is
nevertheless beneficial that certain community efforts, such as the
standard library, should follow a consistent set of style guidelines
when suitable. This enhancement proposal aims to list a series of
guidelines that the standard library should follow.

Note that there can be exceptions to these rules. Nim being as flexible
as it is, there will be parts of this style guide that don't make sense
in certain contexts. Furthermore, just as
\href{http://legacy.python.org/dev/peps/pep-0008/}{Python's style guide}
changes over time, this style guide will too.

These rules will only be enforced for contributions to the Nim codebase
and official projects, such as the Nim compiler, the standard library,
and the various official tools such as C2Nim.

\hypertarget{style-guidelines}{%
\subsection{Style Guidelines}\label{style-guidelines}}

\hypertarget{spacing-and-whitespace-conventions}{%
\subsubsection{Spacing and Whitespace
Conventions}\label{spacing-and-whitespace-conventions}}

\begin{itemize}
\item
  Lines should be no longer than 80 characters. Limiting the amount of
  information present on each line makes for more readable code - the
  reader has smaller chunks to process.
\item
  Two spaces should be used for indentation of blocks; tabstops are not
  allowed (the compiler enforces this). Using spaces means that the
  appearance of code is more consistent across editors. Unlike spaces,
  tabstop width varies across editors, and not all editors provide means
  of changing this width.
\item
  Although use of whitespace for stylistic reasons other than the ones
  endorsed by this guide are allowed, careful thought should be put into
  such practices. Not all editors support automatic alignment of code
  sections, and re-aligning long sections of code by hand can quickly
  become tedious.

\begin{verbatim}
\end{verbatim}
\end{itemize}

\hypertarget{naming-conventions}{%
\subsubsection{Naming Conventions}\label{naming-conventions}}

Note: While the rules outlined below are the \emph{current} naming
conventions, these conventions have not always been in place.
Previously, the naming conventions for identifiers followed the Pascal
tradition of prefixes which indicated the base type of the identifier -
PFoo for pointer and reference types, TFoo for value types, EFoo for
exceptions, etc. Though this has since changed, there are many places in
the standard library which still use this convention. Such style remains
in place purely for legacy reasons, and will be changed in the future.

\begin{itemize}
\item
  Type identifiers should be in PascalCase. All other identifiers should
  be in camelCase with the exception of constants which \textbf{may} use
  PascalCase but are not required to.

\begin{Shaded}
\begin{Highlighting}[]
\NormalTok{var aVariable = }\StringTok{"Meep"}\NormalTok{ \# Variables must start with a lowercase letter.}

\NormalTok{\# Types must start with an uppercase letter.}
\NormalTok{type}
\NormalTok{  FooBar = object}
\end{Highlighting}
\end{Shaded}

  For constants coming from a C/C++ wrapper, ALL\_UPPERCASE are allowed,
  but ugly. (Why shout CONSTANT? Constants do no harm, variables do!)
\item
  When naming types that come in value, pointer, and reference
  varieties, use a regular name for the variety that is to be used the
  most, and add a "Obj", "Ref", or "Ptr" suffix for the other varieties.
  If there is no single variety that will be used the most, add the
  suffixes to the pointer variants only. The same applies to C/C++
  wrappers.

\begin{verbatim}
\end{verbatim}
\item
  Exception and Error types should have the "Error" or "Defect" suffix.

\begin{verbatim}
\end{verbatim}
\item
  Unless marked with the {\{.pure.\}} pragma, members of enums should
  have an identifying prefix, such as an abbreviation of the enum's
  name.

\begin{verbatim}
\end{verbatim}
\item
  Non-pure enum values should use camelCase whereas pure enum values
  should use PascalCase.

\begin{verbatim}
\end{verbatim}
\item
  In the age of HTTP, HTML, FTP, TCP, IP, UTF, WWW it is foolish to
  pretend these are somewhat special words requiring all uppercase.
  Instead treat them as what they are: Real words. So it's
  \texttt{parseUrl} rather than \texttt{parseURL},
  \texttt{checkHttpHeader} instead of \texttt{checkHTTPHeader} etc.
\item
  Operations like \texttt{mitems} or \texttt{mpairs} (or the now
  deprecated \texttt{mget}) that allow a \emph{mutating view} into some
  data structure should start with an \texttt{m}.
\item
  When both in-place mutation and 'returns transformed copy' are
  available the latter is a past participle of the former:

  \begin{itemize}
  \tightlist
  \item
    reverse and reversed in algorithm
  \item
    sort and sorted
  \item
    rotate and rotated
  \end{itemize}
\item
  When the 'returns transformed copy' version already exists like
  \texttt{strutils.replace} an in-place version should get an
  \texttt{-In} suffix (\texttt{replaceIn} for this example).
\item
  Use {subjectVerb}, not {verbSubject}, eg: {fileExists}, not
  {existsFile}.
\end{itemize}

The stdlib API is designed to be \textbf{easy to use} and consistent.
Ease of use is measured by the number of calls to achieve a concrete
high level action. The ultimate goal is that the programmer can
\emph{guess} a name.

The library uses a simple naming scheme that makes use of common
abbreviations to keep the names short but meaningful.

-\/-\/-\/-\/-\/-\/-\/-\/-\/-\/-\/-\/-\/-\/-\/-\/-\/-\/-
-\/-\/-\/-\/-\/-\/-\/-\/-\/-\/-\/-
-\/-\/-\/-\/-\/-\/-\/-\/-\/-\/-\/-\/-\/-\/-\/-\/-\/-\/-\/-\/-\/-\/-\/-\/-\/-\/-\/-\/-\/-\/-\/-\/-\/-\/-\/-\/-\/-English
word To use Notes
-\/-\/-\/-\/-\/-\/-\/-\/-\/-\/-\/-\/-\/-\/-\/-\/-\/-\/-
-\/-\/-\/-\/-\/-\/-\/-\/-\/-\/-\/-
-\/-\/-\/-\/-\/-\/-\/-\/-\/-\/-\/-\/-\/-\/-\/-\/-\/-\/-\/-\/-\/-\/-\/-\/-\/-\/-\/-\/-\/-\/-\/-\/-\/-\/-\/-\/-\/-initialize
initT \texttt{init} is used to create a value type \texttt{T} new newP
\texttt{new} is used to create a reference type \texttt{P} find find
should return the position where something was found; for a bool result
use \texttt{contains} contains contains often short for
\texttt{find()\ \textgreater{}=\ 0} append add use \texttt{add} instead
of \texttt{append} compare cmp should return an int with the
\texttt{\textless{}\ 0} \texttt{==\ 0} or \texttt{\textgreater{}\ 0}
semantics; for a bool result use \texttt{sameXYZ} put put,
\texttt{{[}{]}=} consider overloading \texttt{{[}{]}=} for put get get,
\texttt{{[}{]}} consider overloading \texttt{{[}{]}} for get; consider
to not use \texttt{get} as a prefix: \texttt{len} instead of
\texttt{getLen} length len also used for \emph{number of elements} size
size, len size should refer to a byte size capacity cap memory mem
implies a low-level operation items items default iterator over a
collection pairs pairs iterator over (key, value) pairs delete delete,
del del is supposed to be faster than delete, because it does not keep
the order; delete keeps the order remove delete, del inconsistent right
now include incl exclude excl command cmd execute exec environment env
variable var value value, val val is preferred, inconsistent right now
executable exe directory dir path path path is the string "/usr/bin"
(for example), dir is the content of "/usr/bin"; inconsistent right now
extension ext separator sep column col, column col is preferred,
inconsistent right now application app configuration cfg message msg
argument arg object obj parameter param operator opr procedure proc
function func coordinate coord rectangle rect point point symbol sym
literal lit string str identifier ident indentation indent
-\/-\/-\/-\/-\/-\/-\/-\/-\/-\/-\/-\/-\/-\/-\/-\/-\/-\/-
-\/-\/-\/-\/-\/-\/-\/-\/-\/-\/-\/-
-\/-\/-\/-\/-\/-\/-\/-\/-\/-\/-\/-\/-\/-\/-\/-\/-\/-\/-\/-\/-\/-\/-\/-\/-\/-\/-\/-\/-\/-\/-\/-\/-\/-\/-\/-\/-\/-

\hypertarget{coding-conventions}{%
\subsubsection{Coding Conventions}\label{coding-conventions}}

\begin{itemize}
\item
  The 'return' statement should ideally be used when its control-flow
  properties are required. Use a procedure's implicit 'result' variable
  whenever possible. This improves readability.

\begin{verbatim}
for i in 0 .. x:
  result.add($i)
\end{verbatim}
\item
  Use a proc when possible, only using the more powerful facilities of
  macros, templates, iterators, and converters when necessary.
\item
  Use the \texttt{let} statement (not the \texttt{var} statement) when
  declaring variables that do not change within their scope. Using the
  \texttt{let} statement ensures that variables remain immutable, and
  gives those who read the code a better idea of the code's purpose.
\end{itemize}

\hypertarget{conventions-for-multi-line-statements-and-expressions}{%
\subsubsection{Conventions for multi-line statements and
expressions}\label{conventions-for-multi-line-statements-and-expressions}}

\begin{itemize}
\item
  Tuples which are longer than one line should indent their parameters
  to align with the parameters above it.

\begin{verbatim}
\end{verbatim}
\item
  Similarly, any procedure and procedure type declarations that are
  longer than one line should do the same thing.

\begin{verbatim}
proc lotsOfArguments(argOne: string, argTwo: int, argThree: float,
                     argFour: proc(), argFive: bool): int
                    {.heyLookALongPragma.} =
\end{verbatim}
\item
  Multi-line procedure calls should continue on the same column as the
  opening parenthesis (like multi-line procedure declarations).

\begin{verbatim}
\end{verbatim}
\end{itemize}
