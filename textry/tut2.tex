\hypertarget{nim-tutorial-part-ii}{%
\section{Nim Tutorial (Part II)}\label{nim-tutorial-part-ii}}

\begin{description}
\item[Author]
Andreas Rumpf
\item[Version]
1.3.5
\end{description}

\hypertarget{introduction}{%
\subsection{Introduction}\label{introduction}}

\begin{quote}
"Repetition renders the ridiculous reasonable." -\/- Norman Wildberger
\end{quote}

This document is a tutorial for the advanced constructs of the
\emph{Nim} programming language. \textbf{Note that this document is
somewhat obsolete as the} \href{manual.html}{manual} \textbf{contains
many more examples of the advanced language features.}

\hypertarget{pragmas}{%
\subsection{Pragmas}\label{pragmas}}

Pragmas are Nim's method to give the compiler additional information/
commands without introducing a massive number of new keywords. Pragmas
are enclosed in the special \texttt{\{.} and \texttt{.\}} curly dot
brackets. This tutorial does not cover pragmas. See the
\href{manual.html\#pragmas}{manual} or
\href{nimc.html\#additional-features}{user guide} for a description of
the available pragmas.

\hypertarget{object-oriented-programming}{%
\subsection{Object Oriented
Programming}\label{object-oriented-programming}}

While Nim's support for object oriented programming (OOP) is
minimalistic, powerful OOP techniques can be used. OOP is seen as
\emph{one} way to design a program, not \emph{the only} way. Often a
procedural approach leads to simpler and more efficient code. In
particular, preferring composition over inheritance is often the better
design.

\hypertarget{inheritance}{%
\subsubsection{Inheritance}\label{inheritance}}

Inheritance in Nim is entirely optional. To enable inheritance with
runtime type information the object needs to inherit from
\texttt{RootObj}. This can be done directly, or indirectly by inheriting
from an object that inherits from \texttt{RootObj}. Usually types with
inheritance are also marked as \texttt{ref} types even though this isn't
strictly enforced. To check at runtime if an object is of a certain
type, the \texttt{of} operator can be used.

\begin{verbatim}
\end{verbatim}

\begin{quote}
\begin{description}
\item[type]
\begin{description}
\item[Person = ref object of RootObj]
name*: string \# the * means that {name} is accessible from other
modules age: int \# no * means that the field is hidden from other
modules
\item[Student = ref object of Person \# Student inherits from Person]
id: int \# with an id field
\end{description}
\item[var]
student: Student person: Person
\end{description}

assert(student of Student) \# is true \# object construction: student =
Student(name: "Anton", age: 5, id: 2) echo student{[}{]}
\end{quote}

Inheritance is done with the \texttt{object\ of} syntax. Multiple
inheritance is currently not supported. If an object type has no
suitable ancestor, \texttt{RootObj} can be used as its ancestor, but
this is only a convention. Objects that have no ancestor are implicitly
\texttt{final}. You can use the \texttt{inheritable} pragma to introduce
new object roots apart from \texttt{system.RootObj}. (This is used in
the GTK wrapper for instance.)

Ref objects should be used whenever inheritance is used. It isn't
strictly necessary, but with non-ref objects assignments such as
\texttt{let\ person:\ Person\ =\ Student(id:\ 123)} will truncate
subclass fields.

\textbf{Note}: Composition (\emph{has-a} relation) is often preferable
to inheritance (\emph{is-a} relation) for simple code reuse. Since
objects are value types in Nim, composition is as efficient as
inheritance.

\hypertarget{mutually-recursive-types}{%
\subsubsection{Mutually recursive
types}\label{mutually-recursive-types}}

Objects, tuples and references can model quite complex data structures
which depend on each other; they are \emph{mutually recursive}. In Nim
these types can only be declared within a single type section. (Anything
else would require arbitrary symbol lookahead which slows down
compilation.)

Example:

\begin{verbatim}
\end{verbatim}

\begin{quote}
\begin{description}
\item[type]
\begin{description}
\item[Node = ref object \# a reference to an object with the following
field:]
le, ri: Node \# left and right subtrees sym: ref Sym \# leaves contain a
reference to a Sym
\item[Sym = object \# a symbol]
name: string \# the symbol's name line: int \# the line the symbol was
declared in code: Node \# the symbol's abstract syntax tree
\end{description}
\end{description}
\end{quote}

\hypertarget{type-conversions}{%
\subsubsection{Type conversions}\label{type-conversions}}

Nim distinguishes between \texttt{type\ casts} and
\texttt{type\ conversions}. Casts are done with the \texttt{cast}
operator and force the compiler to interpret a bit pattern to be of
another type.

Type conversions are a much more polite way to convert a type into
another: They preserve the abstract \emph{value}, not necessarily the
\emph{bit-pattern}. If a type conversion is not possible, the compiler
complains or an exception is raised.

The syntax for type conversions is
\texttt{destination\_type(expression\_to\_convert)} (like an ordinary
call):

\begin{verbatim}
\end{verbatim}

The \texttt{InvalidObjectConversionDefect} exception is raised if
\texttt{x} is not a \texttt{Student}.

\hypertarget{object-variants}{%
\subsubsection{Object variants}\label{object-variants}}

Often an object hierarchy is overkill in certain situations where simple
variant types are needed.

An example:

\begin{verbatim}
# This is an example how an abstract syntax tree could be modelled in Nim
type
  NodeKind = enum  # the different node types
    nkInt,          # a leaf with an integer value
    nkFloat,        # a leaf with a float value
    nkString,       # a leaf with a string value
    nkAdd,          # an addition
    nkSub,          # a subtraction
    nkIf            # an if statement
  Node = ref object
    case kind: NodeKind  # the ``kind`` field is the discriminator
    of nkInt: intVal: int
    of nkFloat: floatVal: float
    of nkString: strVal: string
    of nkAdd, nkSub:
      leftOp, rightOp: Node
    of nkIf:
      condition, thenPart, elsePart: Node

var n = Node(kind: nkFloat, floatVal: 1.0)
# the following statement raises an `FieldDefect` exception, because
# n.kind's value does not fit:
n.strVal = ""
\end{verbatim}

As can been seen from the example, an advantage to an object hierarchy
is that no conversion between different object types is needed. Yet,
access to invalid object fields raises an exception.

\hypertarget{method-call-syntax}{%
\subsubsection{Method call syntax}\label{method-call-syntax}}

There is a syntactic sugar for calling routines: The syntax
\texttt{obj.method(args)} can be used instead of
\texttt{method(obj,\ args)}. If there are no remaining arguments, the
parentheses can be omitted: \texttt{obj.len} (instead of
\texttt{len(obj)}).

This method call syntax is not restricted to objects, it can be used for
any type:

\begin{verbatim}
\end{verbatim}

\begin{quote}
import strutils

echo "abc".len \# is the same as echo len("abc") echo
"abc".toUpperAscii() echo(\{'a', 'b', 'c'\}.card)
stdout.writeLine("Hallo") \# the same as writeLine(stdout, "Hallo")
\end{quote}

(Another way to look at the method call syntax is that it provides the
missing postfix notation.)

So "pure object oriented" code is easy to write:

\begin{verbatim}
\end{verbatim}

\begin{quote}
import strutils, sequtils

stdout.writeLine("Give a list of numbers (separated by spaces): ")
stdout.write(stdin.readLine.splitWhitespace.map(parseInt).max.{\$})
stdout.writeLine(" is the maximum!")
\end{quote}

\hypertarget{properties}{%
\subsubsection{Properties}\label{properties}}

As the above example shows, Nim has no need for \emph{get-properties}:
Ordinary get-procedures that are called with the \emph{method call
syntax} achieve the same. But setting a value is different; for this a
special setter syntax is needed:

\begin{verbatim}
type
  Socket* = ref object of RootObj
    h: int # cannot be accessed from the outside of the module due to missing star

proc `host=`*(s: var Socket, value: int) {.inline.} =
  ## setter of host address
  s.h = value

proc host*(s: Socket): int {.inline.} =
  ## getter of host address
  s.h

var s: Socket
new s
s.host = 34  # same as `host=`(s, 34)
\end{verbatim}

(The example also shows \texttt{inline} procedures.)

The \texttt{{[}{]}} array access operator can be overloaded to provide
\texttt{array\ properties}:

\begin{verbatim}
\end{verbatim}

\begin{quote}
\begin{description}
\item[type]
\begin{description}
\item[Vector* = object]
x, y, z: float
\end{description}
\item[proc {{[}{]}=}* (v: var Vector, i: int, value: float) =]
\# setter case i of 0: v.x = value of 1: v.y = value of 2: v.z = value
else: assert(false)
\item[proc {{[}{]}}* (v: Vector, i: int): float =]
\# getter case i of 0: result = v.x of 1: result = v.y of 2: result =
v.z else: assert(false)
\end{description}
\end{quote}

The example is silly, since a vector is better modelled by a tuple which
already provides \texttt{v{[}{]}} access.

\hypertarget{dynamic-dispatch}{%
\subsubsection{Dynamic dispatch}\label{dynamic-dispatch}}

Procedures always use static dispatch. For dynamic dispatch replace the
\texttt{proc} keyword by \texttt{method}:

\begin{verbatim}
\end{verbatim}

\begin{quote}
\begin{description}
\item[type]
Expression = ref object of RootObj \#\# abstract base class for an
expression Literal = ref object of Expression x: int PlusExpr = ref
object of Expression a, b: Expression
\end{description}

\# watch out: 'eval' relies on dynamic binding method eval(e:
Expression): int \{.base.\} = \# override this base method quit "to
override!"

method eval(e: Literal): int = e.x method eval(e: PlusExpr): int =
eval(e.a) + eval(e.b)

proc newLit(x: int): Literal = Literal(x: x) proc newPlus(a, b:
Expression): PlusExpr = PlusExpr(a: a, b: b)

echo eval(newPlus(newPlus(newLit(1), newLit(2)), newLit(4)))
\end{quote}

Note that in the example the constructors \texttt{newLit} and
\texttt{newPlus} are procs because it makes more sense for them to use
static binding, but \texttt{eval} is a method because it requires
dynamic binding.

\textbf{Note:} Starting from Nim 0.20, to use multi-methods one must
explicitly pass \texttt{-\/-multimethods:on} when compiling.

In a multi-method all parameters that have an object type are used for
the dispatching:

\begin{verbatim}
type
  Thing = ref object of RootObj
  Unit = ref object of Thing
    x: int

method collide(a, b: Thing) {.inline.} =
  quit "to override!"

method collide(a: Thing, b: Unit) {.inline.} =
  echo "1"

method collide(a: Unit, b: Thing) {.inline.} =
  echo "2"

var a, b: Unit
new a
new b
collide(a, b) # output: 2
\end{verbatim}

As the example demonstrates, invocation of a multi-method cannot be
ambiguous: Collide 2 is preferred over collide 1 because the resolution
works from left to right. Thus \texttt{Unit,\ Thing} is preferred over
\texttt{Thing,\ Unit}.

\textbf{Performance note}: Nim does not produce a virtual method table,
but generates dispatch trees. This avoids the expensive indirect branch
for method calls and enables inlining. However, other optimizations like
compile time evaluation or dead code elimination do not work with
methods.

\hypertarget{exceptions}{%
\subsection{Exceptions}\label{exceptions}}

In Nim exceptions are objects. By convention, exception types are
suffixed with 'Error'. The \href{system.html}{system} module defines an
exception hierarchy that you might want to stick to. Exceptions derive
from \texttt{system.Exception}, which provides the common interface.

Exceptions have to be allocated on the heap because their lifetime is
unknown. The compiler will prevent you from raising an exception created
on the stack. All raised exceptions should at least specify the reason
for being raised in the \texttt{msg} field.

A convention is that exceptions should be raised in \emph{exceptional}
cases: For example, if a file cannot be opened, this should not raise an
exception since this is quite common (the file may not exist).

\hypertarget{raise-statement}{%
\subsubsection{Raise statement}\label{raise-statement}}

Raising an exception is done with the \texttt{raise} statement:

\begin{verbatim}
\end{verbatim}

\begin{quote}
\begin{description}
\item[var]
e: ref OSError
\end{description}

new(e) e.msg = "the request to the OS failed" raise e
\end{quote}

If the \texttt{raise} keyword is not followed by an expression, the last
exception is \emph{re-raised}. For the purpose of avoiding repeating
this common code pattern, the template \texttt{newException} in the
\texttt{system} module can be used:

\begin{verbatim}
\end{verbatim}

\hypertarget{try-statement}{%
\subsubsection{Try statement}\label{try-statement}}

The \texttt{try} statement handles exceptions:

\begin{verbatim}
\end{verbatim}

\begin{quote}
from strutils import parseInt

\# read the first two lines of a text file that should contain numbers
\# and tries to add them var f: File if open(f, "numbers.txt"): try: let
a = readLine(f) let b = readLine(f) echo "sum: ", parseInt(a) +
parseInt(b) except OverflowDefect: echo "overflow!" except ValueError:
echo "could not convert string to integer" except IOError: echo "IO
error!" except: echo "Unknown exception!" \# reraise the unknown
exception: raise finally: close(f)
\end{quote}

The statements after the \texttt{try} are executed unless an exception
is raised. Then the appropriate \texttt{except} part is executed.

The empty \texttt{except} part is executed if there is an exception that
is not explicitly listed. It is similar to an \texttt{else} part in
\texttt{if} statements.

If there is a \texttt{finally} part, it is always executed after the
exception handlers.

The exception is \emph{consumed} in an \texttt{except} part. If an
exception is not handled, it is propagated through the call stack. This
means that often the rest of the procedure - that is not within a
\texttt{finally} clause -is not executed (if an exception occurs).

If you need to \emph{access} the actual exception object or message
inside an \texttt{except} branch you can use the
\href{system.html\#getCurrentException}{getCurrentException()} and
\href{system.html\#getCurrentExceptionMsg}{getCurrentExceptionMsg()}
procs from the \href{system.html}{system} module. Example:

\begin{verbatim}
\end{verbatim}

\hypertarget{annotating-procs-with-raised-exceptions}{%
\subsubsection{Annotating procs with raised
exceptions}\label{annotating-procs-with-raised-exceptions}}

Through the use of the optional \texttt{\{.raises.\}} pragma you can
specify that a proc is meant to raise a specific set of exceptions, or
none at all. If the \texttt{\{.raises.\}} pragma is used, the compiler
will verify that this is true. For instance, if you specify that a proc
raises \texttt{IOError}, and at some point it (or one of the procs it
calls) starts raising a new exception the compiler will prevent that
proc from compiling. Usage example:

\begin{verbatim}
proc simpleProc() {.raises: [].} =
  ...
\end{verbatim}

Once you have code like this in place, if the list of raised exception
changes the compiler will stop with an error specifying the line of the
proc which stopped validating the pragma and the raised exception not
being caught, along with the file and line where the uncaught exception
is being raised, which may help you locate the offending code which has
changed.

If you want to add the \texttt{\{.raises.\}} pragma to existing code,
the compiler can also help you. You can add the \texttt{\{.effects.\}}
pragma statement to your proc and the compiler will output all inferred
effects up to that point (exception tracking is part of Nim's effect
system). Another more roundabout way to find out the list of exceptions
raised by a proc is to use the Nim \texttt{doc2} command which generates
documentation for a whole module and decorates all procs with the list
of raised exceptions. You can read more about Nim's
\href{manual.html\#effect-system}{effect system and related pragmas in
the manual}.

\hypertarget{generics}{%
\subsection{Generics}\label{generics}}

Generics are Nim's means to parametrize procs, iterators or types with
\texttt{type\ parameters}. Generic parameters are written within square
brackets, for example \texttt{Foo{[}T{]}}. They are most useful for
efficient type safe containers:

\begin{verbatim}
\end{verbatim}

\begin{quote}
\begin{description}
\item[type]
\begin{description}
\item[BinaryTree*{[}T{]} = ref object \# BinaryTree is a generic type
with]
\# generic param \texttt{T} le, ri: BinaryTree{[}T{]} \# left and right
subtrees; may be nil data: T \# the data stored in a node
\end{description}
\item[proc newNode*{[}T{]}(data: T): BinaryTree{[}T{]} =]
\# constructor for a node new(result) result.data = data
\item[proc add*{[}T{]}(root: var BinaryTree{[}T{]}, n:
BinaryTree{[}T{]}) =]
\# insert a node into the tree if root == nil: root = n else: var it =
root while it != nil: \# compare the data items; uses the generic
\texttt{cmp} proc \# that works for any type that has a \texttt{==} and
\texttt{\textless{}} operator var c = cmp(it.data, n.data) if c
\textless{} 0: if it.le == nil: it.le = n return it = it.le else: if
it.ri == nil: it.ri = n return it = it.ri
\item[proc add*{[}T{]}(root: var BinaryTree{[}T{]}, data: T) =]
\# convenience proc: add(root, newNode(data))
\item[iterator preorder*{[}T{]}(root: BinaryTree{[}T{]}): T =]
\# Preorder traversal of a binary tree. \# Since recursive iterators are
not yet implemented, \# this uses an explicit stack (which is more
efficient anyway): var stack: seq{[}BinaryTree{[}T{]}{]} = @{[}root{]}
while stack.len \textgreater{} 0: var n = stack.pop() while n != nil:
yield n.data add(stack, n.ri) \# push right subtree onto the stack n =
n.le \# and follow the left pointer
\item[var]
root: BinaryTree{[}string{]} \# instantiate a BinaryTree with
\texttt{string}
\end{description}

add(root, newNode("hello")) \# instantiates \texttt{newNode} and
\texttt{add} add(root, "world") \# instantiates the second \texttt{add}
proc for str in preorder(root): stdout.writeLine(str)
\end{quote}

The example shows a generic binary tree. Depending on context, the
brackets are used either to introduce type parameters or to instantiate
a generic proc, iterator or type. As the example shows, generics work
with overloading: the best match of \texttt{add} is used. The built-in
\texttt{add} procedure for sequences is not hidden and is used in the
\texttt{preorder} iterator.

\hypertarget{templates}{%
\subsection{Templates}\label{templates}}

Templates are a simple substitution mechanism that operates on Nim's
abstract syntax trees. Templates are processed in the semantic pass of
the compiler. They integrate well with the rest of the language and
share none of C's preprocessor macros flaws.

To \emph{invoke} a template, call it like a procedure.

Example:

\begin{verbatim}
assert(5 != 6) # the compiler rewrites that to: assert(not (5 == 6))
\end{verbatim}

The \texttt{!=}, \texttt{\textgreater{}}, \texttt{\textgreater{}=},
\texttt{in}, \texttt{notin}, \texttt{isnot} operators are in fact
templates: this has the benefit that if you overload the \texttt{==}
operator, the \texttt{!=} operator is available automatically and does
the right thing. (Except for IEEE floating point numbers - NaN breaks
basic boolean logic.)

\texttt{a\ \textgreater{}\ b} is transformed into
\texttt{b\ \textless{}\ a}. \texttt{a\ in\ b} is transformed into
\texttt{contains(b,\ a)}. \texttt{notin} and \texttt{isnot} have the
obvious meanings.

Templates are especially useful for lazy evaluation purposes. Consider a
simple proc for logging:

\begin{verbatim}
\end{verbatim}

\begin{quote}
\begin{description}
\item[const]
debug = true
\item[proc log(msg: string) \{.inline.\} =]
if debug: stdout.writeLine(msg)
\item[var]
x = 4
\end{description}

log("x has the value: " \& \$x)
\end{quote}

This code has a shortcoming: if \texttt{debug} is set to false someday,
the quite expensive \texttt{\$} and \texttt{\&} operations are still
performed! (The argument evaluation for procedures is \emph{eager}).

Turning the \texttt{log} proc into a template solves this problem:

\begin{verbatim}
\end{verbatim}

\begin{quote}
\begin{description}
\item[const]
debug = true
\item[template log(msg: string) =]
if debug: stdout.writeLine(msg)
\item[var]
x = 4
\end{description}

log("x has the value: " \& \$x)
\end{quote}

The parameters' types can be ordinary types or the meta types
\texttt{untyped}, \texttt{typed}, or \texttt{type}. \texttt{type}
suggests that only a type symbol may be given as an argument, and
\texttt{untyped} means symbol lookups and type resolution is not
performed before the expression is passed to the template.

If the template has no explicit return type, \texttt{void} is used for
consistency with procs and methods.

To pass a block of statements to a template, use \texttt{untyped} for
the last parameter:

\begin{verbatim}
template withFile(f: untyped, filename: string, mode: FileMode,
                  body: untyped) =
  let fn = filename
  var f: File
  if open(f, fn, mode):
    try:
      body
    finally:
      close(f)
  else:
    quit("cannot open: " & fn)

withFile(txt, "ttempl3.txt", fmWrite):
  txt.writeLine("line 1")
  txt.writeLine("line 2")
\end{verbatim}

In the example the two \texttt{writeLine} statements are bound to the
\texttt{body} parameter. The \texttt{withFile} template contains
boilerplate code and helps to avoid a common bug: to forget to close the
file. Note how the \texttt{let\ fn\ =\ filename} statement ensures that
\texttt{filename} is evaluated only once.

\hypertarget{example-lifting-procs}{%
\subsubsection{Example: Lifting Procs}\label{example-lifting-procs}}

\begin{verbatim}
\end{verbatim}

\begin{quote}
import math

\begin{description}
\item[template liftScalarProc(fname) =]
\#\# Lift a proc taking one scalar parameter and returning a \#\# scalar
value (eg \texttt{proc\ sssss{[}T{]}(x:\ T):\ float}), \#\# to provide
templated procs that can handle a single \#\# parameter of seq{[}T{]} or
nested seq{[}seq{[}{]}{]} or the same type \#\# \#\# .. code-block:: Nim
\#\# liftScalarProc(abs) \#\# \# now abs(@{[}@{[}1,-2{]},
@{[}-2,-3{]}{]}) == @{[}@{[}1,2{]}, @{[}2,3{]}{]} proc fname{[}T{]}(x:
openarray{[}T{]}): auto = var temp: T type outType = typeof(fname(temp))
result = newSeq{[}outType{]}(x.len) for i in 0..\textless x.len:
result{[}i{]} = fname(x{[}i{]})
\end{description}

liftScalarProc(sqrt) \# make sqrt() work for sequences echo
sqrt(@{[}4.0, 16.0, 25.0, 36.0{]}) \# =\textgreater{} @{[}2.0, 4.0, 5.0,
6.0{]}
\end{quote}

\hypertarget{compilation-to-javascript}{%
\subsection{Compilation to JavaScript}\label{compilation-to-javascript}}

Nim code can be compiled to JavaScript. However in order to write
JavaScript-compatible code you should remember the following: -
\texttt{addr} and \texttt{ptr} have slightly different semantic meaning
in JavaScript. It is recommended to avoid those if you're not sure how
they are translated to JavaScript. - \texttt{cast{[}T{]}(x)} in
JavaScript is translated to \texttt{(x)}, except for casting between
signed/unsigned ints, in which case it behaves as static cast in C
language. - \texttt{cstring} in JavaScript means JavaScript string. It
is a good practice to use \texttt{cstring} only when it is semantically
appropriate. E.g. don't use \texttt{cstring} as a binary data buffer.

\hypertarget{part-3}{%
\subsection{Part 3}\label{part-3}}

The next part is entirely about metaprogramming via macros:
\href{tut3.html}{Part III}
