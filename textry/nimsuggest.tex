\hypertarget{nim-ide-integration-guide}{%
\section{Nim IDE Integration Guide}\label{nim-ide-integration-guide}}

\begin{description}
\item[Author]
Unknown
\item[Version]
1.3.5
\end{description}

Nim differs from many other compilers in that it is really fast, and
being so fast makes it suited to provide external queries for text
editors about the source code being written. Through the
\texttt{nimsuggest} tool, any IDE can query a \texttt{.nim} source file
and obtain useful information like definition of symbols or suggestions
for completion.

This document will guide you through the available options. If you want
to look at practical examples of nimsuggest support you can look at the
\href{https://github.com/Araq/Nim/wiki/Editor-Support}{various editor
integrations} already available.

\hypertarget{installation}{%
\subsection{Installation}\label{installation}}

Nimsuggest is part of Nim's core. Build it via:

\begin{verbatim}
koch nimsuggest
\end{verbatim}

\hypertarget{nimsuggest-invocation}{%
\subsection{Nimsuggest invocation}\label{nimsuggest-invocation}}

Run it via \texttt{nimsuggest\ -\/-stdin\ -\/-debug\ myproject.nim}.
Nimsuggest is a server that takes queries that are related to
\texttt{myproject}. There is some support so that you can throw random
\texttt{.nim} files which are not part of \texttt{myproject} at
Nimsuggest too, but usually the query refer to modules/files that are
part of \texttt{myproject}.

\texttt{-\/-stdin} means that Nimsuggest reads the query from
\texttt{stdin}. This is great for testing things out and playing with it
but for an editor communication via sockets is more reasonable so that
is the default. It listens to port 6000 by default.

\hypertarget{specifying-the-location-of-the-query}{%
\subsubsection{Specifying the location of the
query}\label{specifying-the-location-of-the-query}}

Nimsuggest then waits for queries to process. A query consists of a
cryptic 3 letter "command" \texttt{def} or \texttt{con} or \texttt{sug}
or \texttt{use} followed by a location. A query location consists of:

\begin{description}
\item[\texttt{file.nim}]
This is the name of the module or include file the query refers to.
\item[\texttt{dirtyfile.nim}]
This is optional.

The \texttt{file} parameter is enough for static analysis, but IDEs tend
to have \emph{unsaved buffers} where the user may still be in the middle
of typing a line. In such situations the IDE can save the current
contents to a temporary file and then use the \texttt{dirtyfile.nim}
option to tell Nimsuggest that \texttt{foobar.nim} should be taken from
\texttt{temporary/foobar.nim}.
\item[\texttt{line}]
An integer with the line you are going to query. For the compiler lines
start at \textbf{1}.
\item[\texttt{col}]
An integer with the column you are going to query. For the compiler
columns start at \textbf{0}.
\end{description}

\hypertarget{definitions}{%
\subsubsection{Definitions}\label{definitions}}

The \texttt{def} Nimsuggest command performs a query about the
definition of a specific symbol. If available, Nimsuggest will answer
with the type, source file, line/column information and other accessory
data if available like a docstring. With this information an IDE can
provide the typical \emph{Jump to definition} where a user puts the
cursor on a symbol or uses the mouse to select it and is redirected to
the place where the symbol is located.

Since Nim is implemented in Nim, one of the nice things of this feature
is that any user with an IDE supporting it can quickly jump around the
standard library implementation and see exactly what a proc does,
learning about the language and seeing real life examples of how to
write/implement specific features.

Nimsuggest will always answer with a single definition or none if it
can't find any valid symbol matching the position of the query.

\hypertarget{suggestions}{%
\subsubsection{Suggestions}\label{suggestions}}

The \texttt{sug} Nimsuggest command performs a query about possible
completion symbols at some point in the file.

The typical usage scenario for this option is to call it after the user
has typed the dot character for
\href{tut2.html\#object-oriented-programming-method-call-syntax}{the
object oriented call syntax}. Nimsuggest will try to return the
suggestions sorted first by scope (from innermost to outermost) and then
by item name.

\hypertarget{invocation-context}{%
\subsubsection{Invocation context}\label{invocation-context}}

The \texttt{con} Nimsuggest command is very similar to the suggestions
command, but instead of being used after the user has typed a dot
character, this one is meant to be used after the user has typed an
opening brace to start typing parameters.

\hypertarget{symbol-usages}{%
\subsubsection{Symbol usages}\label{symbol-usages}}

The \texttt{use} Nimsuggest command lists all usages of the symbol at a
position. IDEs can use this to find all the places in the file where the
symbol is used and offer the user to rename it in all places at the same
time.

For this kind of query the IDE will most likely ignore all the
type/signature info provided by Nimsuggest and concentrate on the
filename, line and column position of the multiple returned answers.

\hypertarget{parsing-nimsuggest-output}{%
\subsection{Parsing nimsuggest output}\label{parsing-nimsuggest-output}}

Nimsuggest output is always returned on single lines separated by tab
characters (\texttt{\textbackslash{}t}). The values of each column are:

\begin{enumerate}
\def\labelenumi{\arabic{enumi}.}
\item
  Three characters indicating the type of returned answer (e.g.
  \texttt{def} for definition, \texttt{sug} for suggestion, etc).
\item
  Type of the symbol. This can be \texttt{skProc}, \texttt{skLet}, and
  just about any of the enums defined in the module
  \texttt{compiler/ast.nim}.
\item
  Fully qualified path of the symbol. If you are querying a symbol
  defined in the \texttt{proj.nim} file, this would have the form
  \texttt{proj.symbolName}.
\item
  Type/signature. For variables and enums this will contain the type of
  the symbol, for procs, methods and templates this will contain the
  full unique signature (e.g. \texttt{proc\ (File)}).
\item
  Full path to the file containing the symbol.
\item
  Line where the symbol is located in the file. Lines start to count at
  \textbf{1}.
\item
  Column where the symbol is located in the file. Columns start to count
  at \textbf{0}.
\item
  Docstring for the symbol if available or the empty string. To
  differentiate the docstring from end of answer, the docstring is
  always provided enclosed in double quotes, and if the docstring spans
  multiple lines, all following lines of the docstring will start with a
  blank space to align visually with the starting quote.

  Also, you won't find raw \texttt{\textbackslash{}n} characters
  breaking the one answer per line format. Instead you will need to
  parse sequences in the form \texttt{\textbackslash{}xHH}, where
  \emph{HH} is a hexadecimal value (e.g. newlines generate the sequence
  \texttt{\textbackslash{}x0A}).
\end{enumerate}
