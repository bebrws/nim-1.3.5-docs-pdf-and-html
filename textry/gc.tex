\hypertarget{nims-memory-management}{%
\section{Nim's Memory Management}\label{nims-memory-management}}

\begin{description}
\item[Author]
Andreas Rumpf
\item[Version]
1.3.5
\end{description}

\hypertarget{introduction}{%
\subsection{Introduction}\label{introduction}}

This document describes how the multi-paradigm memory management
strategies work. How to tune the garbage collectors for your needs, like
(soft) \texttt{realtime\ systems}, and how the memory management
strategies that are not garbage collectors work.

\hypertarget{multi-paradigm-memory-management-strategies}{%
\subsection{Multi-paradigm Memory Management
Strategies}\label{multi-paradigm-memory-management-strategies}}

To choose the memory management strategy use the \texttt{-\/-gc:}
switch.

\begin{itemize}
\tightlist
\item
  \texttt{-\/-gc:refc}. This is the default GC. It's a deferred
  reference counting based garbage collector with a simple Mark\&Sweep
  backup GC in order to collect cycles. Heaps are thread local.
\item
  \texttt{-\/-gc:markAndSweep}. Simple Mark-And-Sweep based garbage
  collector. Heaps are thread local.
\item
  \texttt{-\/-gc:boehm}. Boehm based garbage collector, it offers a
  shared heap.
\item
  \texttt{-\/-gc:go}. Go's garbage collector, useful for
  interoperability with Go. Offers a shared heap.
\item
  \texttt{-\/-gc:arc}. Plain reference counting with
  \href{destructors.html\#move-semantics}{move semantic optimizations},
  offers a shared heap. It offers deterministic performance for
  \texttt{hard\ realtime} systems. Reference cycles cause memory leaks,
  beware.
\item
  \texttt{-\/-gc:orc}. Same as \texttt{-gc:arc} but adds a cycle
  collector based on "trial deletion". Unfortunately that makes its
  performance profile hard to reason about so it is less useful for hard
  realtime systems.
\item
  \texttt{-\/-gc:none}. No memory management strategy nor garbage
  collector. Allocated memory is simply never freed. You should use
  \texttt{-\/-gc:arc} instead.
\end{itemize}

\begin{longtable}[]{@{}lllll@{}}
\toprule
Memory Management & Heap & Reference Cycles & Stop-The-World & Command
line switch\tabularnewline
\midrule
\endhead
RefC & Local & Cycle Collector & No &
\texttt{-\/-gc:refc}\tabularnewline
Mark \& Sweep & Local & Cycle Collector & No &
\texttt{-\/-gc:markAndSweep}\tabularnewline
ARC & Shared & Leak & No & \texttt{-\/-gc:arc}\tabularnewline
ORC & Shared & Cycle Collector & No & \texttt{-\/-gc:orc}\tabularnewline
Boehm & Shared & Cycle Collector & Yes &
\texttt{-\/-gc:boehm}\tabularnewline
Go & Shared & Cycle Collector & Yes & \texttt{-\/-gc:go}\tabularnewline
None & Manual & Manual & Manual & \texttt{-\/-gc:none}\tabularnewline
\bottomrule
\end{longtable}

JavaScript's garbage collector is used for the
\href{backends.html\#backends-the-javascript-target}{JavaScript and
NodeJS} compilation targets. The \href{nims.html}{NimScript} target uses
the memory management strategy built into the Nim compiler.

\hypertarget{tweaking-the-refc-gc}{%
\subsection{Tweaking the refc GC}\label{tweaking-the-refc-gc}}

\hypertarget{cycle-collector}{%
\subsubsection{Cycle collector}\label{cycle-collector}}

The cycle collector can be en-/disabled independently from the other
parts of the garbage collector with \texttt{GC\_enableMarkAndSweep} and
\texttt{GC\_disableMarkAndSweep}.

\hypertarget{soft-realtime-support}{%
\subsubsection{Soft realtime support}\label{soft-realtime-support}}

To enable realtime support, the symbol \texttt{useRealtimeGC} needs to
be defined via \texttt{-\/-define:useRealtimeGC} (you can put this into
your config file as well). With this switch the garbage collector
supports the following operations:

\begin{verbatim}
\end{verbatim}

The unit of the parameters \texttt{maxPauseInUs} and \texttt{us} is
microseconds.

These two procs are the two modus operandi of the realtime garbage
collector:

\begin{enumerate}
\def\labelenumi{(\arabic{enumi})}
\item
  GC\_SetMaxPause Mode

  You can call \texttt{GC\_SetMaxPause} at program startup and then each
  triggered garbage collector run tries to not take longer than
  \texttt{maxPause} time. However, it is possible (and common) that the
  work is nevertheless not evenly distributed as each call to
  \texttt{new} can trigger the garbage collector and thus take
  \texttt{maxPause} time.
\item
  GC\_step Mode

  This allows the garbage collector to perform some work for up to
  \texttt{us} time. This is useful to call in a main loop to ensure the
  garbage collector can do its work. To bind all garbage collector
  activity to a \texttt{GC\_step} call, deactivate the garbage collector
  with \texttt{GC\_disable} at program startup. If \texttt{strongAdvice}
  is set to \texttt{true}, then the garbage collector will be forced to
  perform collection cycle. Otherwise, the garbage collector may decide
  not to do anything, if there is not much garbage to collect. You may
  also specify the current stack size via \texttt{stackSize} parameter.
  It can improve performance, when you know that there are no unique Nim
  references below certain point on the stack. Make sure the size you
  specify is greater than the potential worst case size.
\end{enumerate}

These procs provide a "best effort" realtime guarantee; in particular
the cycle collector is not aware of deadlines. Deactivate it to get more
predictable realtime behaviour. Tests show that a 1ms max pause time
will be met in almost all cases on modern CPUs (with the cycle collector
disabled).

\hypertarget{time-measurement-with-garbage-collectors}{%
\subsubsection{Time measurement with garbage
collectors}\label{time-measurement-with-garbage-collectors}}

The garbage collectors's way of measuring time uses (see
\texttt{lib/system/timers.nim} for the implementation):

\begin{enumerate}
\def\labelenumi{\arabic{enumi})}
\tightlist
\item
  \texttt{QueryPerformanceCounter} and
  \texttt{QueryPerformanceFrequency} on Windows.
\item
  \texttt{mach\_absolute\_time} on Mac OS X.
\item
  \texttt{gettimeofday} on Posix systems.
\end{enumerate}

As such it supports a resolution of nanoseconds internally; however the
API uses microseconds for convenience.

Define the symbol \texttt{reportMissedDeadlines} to make the garbage
collector output whenever it missed a deadline. The reporting will be
enhanced and supported by the API in later versions of the collector.

\hypertarget{tweaking-the-garbage-collector}{%
\subsubsection{Tweaking the garbage
collector}\label{tweaking-the-garbage-collector}}

The collector checks whether there is still time left for its work after
every \texttt{workPackage}'th iteration. This is currently set to 100
which means that up to 100 objects are traversed and freed before it
checks again. Thus \texttt{workPackage} affects the timing granularity
and may need to be tweaked in highly specialized environments or for
older hardware.

\hypertarget{keeping-track-of-memory}{%
\subsection{Keeping track of memory}\label{keeping-track-of-memory}}

If you need to pass around memory allocated by Nim to C, you can use the
procs \texttt{GC\_ref} and \texttt{GC\_unref} to mark objects as
referenced to avoid them being freed by the garbage collector. Other
useful procs from \href{system.html}{system} you can use to keep track
of memory are:

\begin{itemize}
\tightlist
\item
  \texttt{getTotalMem()} Returns the amount of total memory managed by
  the garbage collector.
\item
  \texttt{getOccupiedMem()} Bytes reserved by the garbage collector and
  used by objects.
\item
  \texttt{getFreeMem()} Bytes reserved by the garbage collector and not
  in use.
\item
  \texttt{GC\_getStatistics()} Garbage collector statistics as a
  human-readable string.
\end{itemize}

These numbers are usually only for the running thread, not for the whole
heap, with the exception of \texttt{-\/-gc:boehm} and
\texttt{-\/-gc:go}.

In addition to \texttt{GC\_ref} and \texttt{GC\_unref} you can avoid the
garbage collector by manually allocating memory with procs like
\texttt{alloc}, \texttt{alloc0}, \texttt{allocShared},
\texttt{allocShared0} or \texttt{allocCStringArray}. The garbage
collector won't try to free them, you need to call their respective
\emph{dealloc} pairs (\texttt{dealloc}, \texttt{deallocShared},
\texttt{deallocCStringArray}, etc) when you are done with them or they
will leak.

\hypertarget{heap-dump}{%
\subsection{Heap dump}\label{heap-dump}}

The heap dump feature is still in its infancy, but it already proved
useful for us, so it might be useful for you. To get a heap dump,
compile with \texttt{-d:nimTypeNames} and call
\texttt{dumpNumberOfInstances} at a strategic place in your program.
This produces a list of used types in your program and for every type
the total amount of object instances for this type as well as the total
amount of bytes these instances take up. This list is currently
unsorted! You need to use external shell script hacking to sort it.

The numbers count the number of objects in all garbage collector heaps,
they refer to all running threads, not only to the current thread. (The
current thread would be the thread that calls
\texttt{dumpNumberOfInstances}.) This might change in later versions.
