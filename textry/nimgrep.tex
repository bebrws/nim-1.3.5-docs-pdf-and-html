\hypertarget{nimgrep-users-manual}{%
\section{nimgrep User's manual}\label{nimgrep-users-manual}}

\begin{description}
\item[Author]
Andreas Rumpf
\item[Version]
0.9
\end{description}

Nimgrep is a command line tool for search\&replace tasks. It can search
for regex or peg patterns and can search whole directories at once. User
confirmation for every single replace operation can be requested.

Nimgrep has particularly good support for Nim's eccentric \emph{style
insensitivity}. Apart from that it is a generic text manipulation tool.

\hypertarget{installation}{%
\subsection{Installation}\label{installation}}

Compile nimgrep with the command:

\begin{verbatim}
nim c -d:release tools/nimgrep.nim
\end{verbatim}

And copy the executable somewhere in your \texttt{\$PATH}.

\hypertarget{command-line-switches}{%
\subsection{Command line switches}\label{command-line-switches}}

\begin{description}
\item[Usage:]
nimgrep {[}options{]} {[}pattern{]} {[}replacement{]} (file/directory)*
\item[Options:]
-\/-find, -f find the pattern (default) -\/-replace, -r replace the
pattern -\/-peg pattern is a peg -\/-re pattern is a regular expression
(default); extended syntax for the regular expression is always turned
on -\/-recursive process directories recursively -\/-confirm confirm
each occurrence/replacement; there is a chance to abort any time without
touching the file -\/-stdin read pattern from stdin (to avoid the
shell's confusing quoting rules) -\/-word, -w the match should have word
boundaries (buggy for pegs!) -\/-ignoreCase, -i be case insensitive
-\/-ignoreStyle, -y be style insensitive -\/-ext:EX1... only search the
files with the given extension(s) -\/-verbose be verbose: list every
processed file -\/-help, -h shows this help -\/-version, -v shows the
version
\end{description}
