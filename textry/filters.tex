\hypertarget{source-code-filters}{%
\section{Source Code Filters}\label{source-code-filters}}

A {Source Code Filter (SCF)} transforms the input character stream to an
in-memory output stream before parsing. A filter can be used to provide
templating systems or preprocessors.

To use a filter for a source file the \texttt{\#?} notation is used:

\begin{verbatim}
#? stdtmpl(subsChar = '$', metaChar = '#')
#proc generateXML(name, age: string): string =
#  result = ""
<xml>
  <name>$name</name>
  <age>$age</age>
</xml>
\end{verbatim}

As the example shows, passing arguments to a filter can be done just
like an ordinary procedure call with named or positional arguments. The
available parameters depend on the invoked filter. Before version 0.12.0
of the language \texttt{\#!} was used instead of \texttt{\#?}.

\textbf{Hint:} With \texttt{-\/-hint{[}codeBegin{]}:on} or
\texttt{-\/-verbosity:2} (or higher) while compiling or {nim check}, Nim
lists the processed code after each filter application.

\hypertarget{usage}{%
\subsection{Usage}\label{usage}}

First, put your SCF code in a separate file with filters specified in
the first line. \textbf{Note:} You can name your SCF file with any file
extension you want, but the conventional extension is \texttt{.nimf} (it
used to be \texttt{.tmpl} but that was too generic, for example
preventing github to recognize it as Nim source file).

If we use {generateXML} code shown above and call the SCF file
{xmlGen.nimf} In your `main.nim`:

\begin{verbatim}
echo generateXML("John Smith","42")
\end{verbatim}

\hypertarget{pipe-operator}{%
\subsection{Pipe operator}\label{pipe-operator}}

Filters can be combined with the \texttt{\textbar{}} pipe operator:

\begin{verbatim}
#? strip(startswith="<") | stdtmpl
#proc generateXML(name, age: string): string =
#  result = ""
<xml>
  <name>$name</name>
  <age>$age</age>
</xml>
\end{verbatim}

\hypertarget{available-filters}{%
\subsection{Available filters}\label{available-filters}}

\hypertarget{replace-filter}{%
\subsubsection{Replace filter}\label{replace-filter}}

The replace filter replaces substrings in each line.

Parameters and their defaults:

\begin{quote}
\begin{description}
\item[\texttt{sub:\ string\ =\ ""}]
the substring that is searched for
\item[\texttt{by:\ string\ =\ ""}]
the string the substring is replaced with
\end{description}
\end{quote}

\hypertarget{strip-filter}{%
\subsubsection{Strip filter}\label{strip-filter}}

The strip filter simply removes leading and trailing whitespace from
each line.

Parameters and their defaults:

\begin{quote}
\begin{description}
\item[\texttt{startswith:\ string\ =\ ""}]
strip only the lines that start with \emph{startswith} (ignoring leading
whitespace). If empty every line is stripped.
\item[\texttt{leading:\ bool\ =\ true}]
strip leading whitespace
\item[\texttt{trailing:\ bool\ =\ true}]
strip trailing whitespace
\end{description}
\end{quote}

\hypertarget{stdtmpl-filter}{%
\subsubsection{StdTmpl filter}\label{stdtmpl-filter}}

The stdtmpl filter provides a simple templating engine for Nim. The
filter uses a line based parser: Lines prefixed with a \emph{meta
character} (default: \texttt{\#}) contain Nim code, other lines are
verbatim. Because indentation-based parsing is not suited for a
templating engine, control flow statements need \texttt{end\ X}
delimiters.

Parameters and their defaults:

\begin{quote}
\begin{description}
\item[\texttt{metaChar:\ char\ =\ \textquotesingle{}\#\textquotesingle{}}]
prefix for a line that contains Nim code
\item[\texttt{subsChar:\ char\ =\ \textquotesingle{}\$\textquotesingle{}}]
prefix for a Nim expression within a template line
\item[\texttt{conc:\ string\ =\ "\ \&\ "}]
the operation for concatenation
\item[\texttt{emit:\ string\ =\ "result.add"}]
the operation to emit a string literal
\item[\texttt{toString:\ string\ =\ "\$"}]
the operation that is applied to each expression
\end{description}
\end{quote}

Example:

\begin{verbatim}
#? stdtmpl | standard
#proc generateHTMLPage(title, currentTab, content: string,
#                      tabs: openArray[string]): string =
#  result = ""
<head><title>$title</title></head>
<body>
  <div id="menu">
    <ul>
  #for tab in items(tabs):
    #if currentTab == tab:
    <li><a id="selected"
    #else:
    <li><a
    #end if
    href="${tab}.html">$tab</a></li>
  #end for
    </ul>
  </div>
  <div id="content">
    $content
    A dollar: $$.
  </div>
</body>
\end{verbatim}

The filter transforms this into:

\begin{verbatim}
\end{verbatim}

Each line that does not start with the meta character (ignoring leading
whitespace) is converted to a string literal that is added to
\texttt{result}.

The substitution character introduces a Nim expression \emph{e} within
the string literal. \emph{e} is converted to a string with the
\emph{toString} operation which defaults to \texttt{\$}. For strong type
checking, set \texttt{toString} to the empty string. \emph{e} must match
this PEG pattern:

\begin{verbatim}
e <- [a-zA-Z\128-\255][a-zA-Z0-9\128-\255_.]* / '{' x '}'
x <- '{' x+ '}' / [^}]*
\end{verbatim}

To produce a single substitution character it has to be doubled:
\texttt{\$\$} produces \texttt{\$}.

The template engine is quite flexible. It is easy to produce a procedure
that writes the template code directly to a file:

\begin{verbatim}
#? stdtmpl(emit="f.write") | standard
#proc writeHTMLPage(f: File, title, currentTab, content: string,
#                   tabs: openArray[string]) =
<head><title>$title</title></head>
<body>
  <div id="menu">
    <ul>
  #for tab in items(tabs):
    #if currentTab == tab:
    <li><a id="selected"
    #else:
    <li><a
    #end if
    href="${tab}.html" title = "$title - $tab">$tab</a></li>
  #end for
    </ul>
  </div>
  <div id="content">
    $content
    A dollar: $$.
  </div>
</body>
\end{verbatim}
