\hypertarget{tools-available-with-nim}{%
\section{Tools available with Nim}\label{tools-available-with-nim}}

The standard distribution ships with the following tools:

\begin{itemize}
\item
  \href{hcr.html}{Hot code reloading}\\
  The "Hot code reloading" feature is built into the compiler but has
  its own document explaining how it works.
\item
  \href{docgen.html}{Documentation generator}\\
  The builtin document generator \texttt{nim\ doc} generates HTML
  documentation from \texttt{.nim} source files.
\item
  \href{nimsuggest.html}{Nimsuggest for IDE support}\\
  Through the \texttt{nimsuggest} tool, any IDE can query a
  \texttt{.nim} source file and obtain useful information like
  definition of symbols or suggestions for completion.
\item
  \href{https://github.com/nim-lang/c2nim/blob/master/doc/c2nim.rst}{C2nim}\\
  C to Nim source converter. Translates C header files to Nim.
\item
  \href{niminst.html}{niminst}\\
  niminst is a tool to generate an installer for a Nim program.
\item
  \href{nimgrep.html}{nimgrep}\\
  Nim search and replace utility.
\item
  nimpretty\\
  \texttt{nimpretty} is a Nim source code beautifier, to format code
  according to the official style guide.
\item
  testament\\
  \texttt{testament} is an advanced automatic \emph{unittests runner}
  for Nim tests, is used for the development of Nim itself, offers
  process isolation for your tests, it can generate statistics about
  test cases, supports multiple targets (C, JS, etc),
  \href{https://en.wikipedia.org/wiki/Dry_run_(testing)}{simulated
  Dry-Runs}, has logging, can generate HTML reports, skip tests from a
  file and more, so can be useful to run your tests, even the most
  complex ones.
\end{itemize}
