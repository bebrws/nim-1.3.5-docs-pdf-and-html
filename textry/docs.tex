The documentation consists of several documents:

\begin{itemize}
\item
  \href{tut1.html}{Tutorial (part I)}\\
  The Nim tutorial part one deals with the basics.
\item
  \href{tut2.html}{Tutorial (part II)}\\
  The Nim tutorial part two deals with the advanced language constructs.
\item
  \href{tut3.html}{Tutorial (part III)}\\
  The Nim tutorial part three about Nim's macro system.
\item
  \href{manual.html}{Language Manual}\\
  The Nim manual is a draft that will evolve into a proper
  specification.
\item
  \href{lib.html}{Library documentation}\\
  This document describes Nim's standard library.
\item
  \href{nimc.html}{Compiler user guide}\\
  The user guide lists command line arguments, special features of the
  compiler, etc.
\item
  \href{tools.html}{Tools documentation}\\
  Description of some tools that come with the standard distribution.
\item
  \href{gc.html}{GC}\\
  Additional documentation about Nim's multi-paradigm memory management
  strategies\\
  and how to operate them in a realtime setting.
\item
  \href{filters.html}{Source code filters}\\
  The Nim compiler supports source code filters as a simple yet powerful
  builtin templating system.
\item
  \href{intern.html}{Internal documentation}\\
  The internal documentation describes how the compiler is implemented.
  Read this if you want to hack the compiler.
\item
  \href{theindex.html}{Index}\\
  The generated index. \textbf{Index + (Ctrl+F) == Joy}
\end{itemize}
