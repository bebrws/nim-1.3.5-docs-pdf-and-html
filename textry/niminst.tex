\hypertarget{niminst-users-manual}{%
\section{niminst User's manual}\label{niminst-users-manual}}

\begin{description}
\item[Author]
Andreas Rumpf
\item[Version]
1.3.5
\end{description}

\hypertarget{introduction}{%
\subsection{Introduction}\label{introduction}}

niminst is a tool to generate an installer for a Nim program. Currently
it can create an installer for Windows via
\href{http://www.jrsoftware.org/isinfo.php}{Inno Setup} as well as
installation/deinstallation scripts for UNIX. Later versions will
support Linux' package management systems.

niminst works by reading a configuration file that contains all the
information that it needs to generate an installer for the different
operating systems.

\hypertarget{configuration-file}{%
\subsection{Configuration file}\label{configuration-file}}

niminst uses the Nim \href{parsecfg.html}{parsecfg} module to parse the
configuration file. Here's an example of how the syntax looks like:

The value of a key-value pair can reference user-defined variables via
the \texttt{\$variable} notation: They can be defined in the command
line with the \texttt{-\/-var:name=value} switch. This is useful to not
hard-coding the program's version number into the configuration file,
for instance.

It follows a description of each possible section and how it affects the
generated installers.

\hypertarget{project-section}{%
\subsubsection{Project section}\label{project-section}}

The project section gathers general information about your project. It
must contain the following key-value pairs:

\begin{longtable}[]{@{}ll@{}}
\toprule
Key & description\tabularnewline
\midrule
\endhead
\texttt{Name} & the project's name; this needs to be a single
word\tabularnewline
\texttt{DisplayName} & the project's long name; this can contain spaces.
If not specified, this is the same as \texttt{Name}.\tabularnewline
\texttt{Version} & the project's version\tabularnewline
\texttt{OS} & the OSes to generate C code for; for example:
\texttt{"windows;linux;macosx"}\tabularnewline
\texttt{CPU} & the CPUs to generate C code for; for example:
\texttt{"i386;amd64;powerpc"}\tabularnewline
\texttt{Authors} & the project's authors\tabularnewline
\texttt{Description} & the project's description\tabularnewline
\texttt{App} & the application's type: "Console" or "GUI". If "Console",
niminst generates a special batch file for Windows to open up the
command line shell.\tabularnewline
\texttt{License} & the filename of the application's
license\tabularnewline
\bottomrule
\end{longtable}

\hypertarget{files-key}{%
\subsubsection{\texorpdfstring{\texttt{files}
key}{files key}}\label{files-key}}

Many sections support the \texttt{files} key. Listed filenames can be
separated by semicolon or the \texttt{files} key can be repeated.
Wildcards in filenames are supported. If it is a directory name, all
files in the directory are used:

\begin{verbatim}
[Config]
Files: "configDir"
Files: "otherconfig/*.conf;otherconfig/*.cfg"
\end{verbatim}

\hypertarget{config-section}{%
\subsubsection{Config section}\label{config-section}}

The \texttt{config} section currently only supports the \texttt{files}
key. Listed files will be installed into the OS's configuration
directory.

\hypertarget{documentation-section}{%
\subsubsection{Documentation section}\label{documentation-section}}

The \texttt{documentation} section supports the \texttt{files} key.
Listed files will be installed into the OS's native documentation
directory (which might be \texttt{\$appdir/doc}).

There is a \texttt{start} key which determines whether the Windows
installer generates a link to e.g. the \texttt{index.html} of your
documentation.

\hypertarget{other-section}{%
\subsubsection{Other section}\label{other-section}}

The \texttt{other} section currently only supports the \texttt{files}
key. Listed files will be installed into the application installation
directory (\texttt{\$appdir}).

\hypertarget{lib-section}{%
\subsubsection{Lib section}\label{lib-section}}

The \texttt{lib} section currently only supports the \texttt{files} key.
Listed files will be installed into the OS's native library directory
(which might be \texttt{\$appdir/lib}).

\hypertarget{windows-section}{%
\subsubsection{Windows section}\label{windows-section}}

The \texttt{windows} section supports the \texttt{files} key for Windows
specific files. Listed files will be installed into the application
installation directory (\texttt{\$appdir}).

Other possible options are:

\begin{longtable}[]{@{}ll@{}}
\toprule
Key & description\tabularnewline
\midrule
\endhead
\texttt{BinPath} & paths to add to the Windows \texttt{\%PATH\%}
environment variable. Example:
\texttt{BinPath:\ r"bin;dist\textbackslash{}mingw\textbackslash{}bin"}\tabularnewline
\texttt{InnoSetup} & boolean flag whether an Inno Setup installer should
be generated for Windows. Example:
\texttt{InnoSetup:\ "Yes"}\tabularnewline
\bottomrule
\end{longtable}

\hypertarget{unixbin-section}{%
\subsubsection{UnixBin section}\label{unixbin-section}}

The \texttt{UnixBin} section currently only supports the \texttt{files}
key. Listed files will be installed into the OS's native bin directory
(e.g. \texttt{/usr/local/bin}). The exact location depends on the
installation path the user specifies when running the
\texttt{install.sh} script.

\hypertarget{unix-section}{%
\subsubsection{Unix section}\label{unix-section}}

Possible options are:

\begin{longtable}[]{@{}ll@{}}
\toprule
Key & description\tabularnewline
\midrule
\endhead
\texttt{InstallScript} & boolean flag whether an installation shell
script should be generated. Example:
\texttt{InstallScript:\ "Yes"}\tabularnewline
\texttt{UninstallScript} & boolean flag whether a deinstallation shell
script should be generated. Example:
\texttt{UninstallScript:\ "Yes"}\tabularnewline
\bottomrule
\end{longtable}

\hypertarget{innosetup-section}{%
\subsubsection{InnoSetup section}\label{innosetup-section}}

Possible options are:

\begin{longtable}[]{@{}ll@{}}
\toprule
Key & description\tabularnewline
\midrule
\endhead
\texttt{path} & Path to Inno Setup. Example:
\texttt{path\ =\ r"c:\textbackslash{}inno\ setup\ 5\textbackslash{}iscc.exe"}\tabularnewline
\texttt{flags} & Flags to pass to Inno Setup. Example:
\texttt{flags\ =\ "/Q"}\tabularnewline
\bottomrule
\end{longtable}

\hypertarget{c_compiler-section}{%
\subsubsection{C\_Compiler section}\label{c_compiler-section}}

Possible options are:

\begin{longtable}[]{@{}ll@{}}
\toprule
Key & description\tabularnewline
\midrule
\endhead
\texttt{path} & Path to the C compiler.\tabularnewline
\texttt{flags} & Flags to pass to the C Compiler. Example:
\texttt{flags\ =\ "-w"}\tabularnewline
\bottomrule
\end{longtable}

\hypertarget{real-world-example}{%
\subsection{Real world example}\label{real-world-example}}

The installers for the Nim compiler itself are generated by niminst.
Have a look at its configuration file:
