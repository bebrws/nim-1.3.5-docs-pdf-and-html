\hypertarget{drnim-user-guide}{%
\section{DrNim User Guide}\label{drnim-user-guide}}

\begin{description}
\item[Author]
Andreas Rumpf
\item[Version]
1.3.5
\end{description}

\hypertarget{introduction}{%
\subsection{Introduction}\label{introduction}}

This document describes the usage of the \emph{DrNim} tool. DrNim
combines the Nim frontend with the
\href{https://github.com/Z3Prover/z3}{Z3} proof engine in order to allow
verify / validate software written in Nim. DrNim's command line options
are the same as the Nim compiler's.

DrNim currently only checks the sections of your code that are marked
via \texttt{staticBoundChecks:\ on}:

\begin{verbatim}
{.push staticBoundChecks: on.}
# <--- code section here ---->
{.pop.}
\end{verbatim}

DrNim currently only tries to prove array indexing or subrange checks,
overflow errors are \emph{not} prevented. Overflows will be checked for
in the future.

Later versions of the \textbf{Nim compiler} will \textbf{assume} that
the checks inside the \texttt{staticBoundChecks:\ on} environment have
been proven correct and so it will \textbf{omit} the runtime checks. If
you do not want this behavior, use instead
\texttt{\{.push\ staticBoundChecks:\ defined(nimDrNim).\}}. This way the
Nim compiler remains unaware of the performed proofs but DrNim will
prove your code.

\hypertarget{installation}{%
\subsection{Installation}\label{installation}}

Run \texttt{koch\ drnim}, the executable will afterwards be in
\texttt{\$nim/bin/drnim}.

\hypertarget{motivating-example}{%
\subsection{Motivating Example}\label{motivating-example}}

The follow example highlights what DrNim can easily do, even without
additional annotations:

\begin{verbatim}
{.push staticBoundChecks: on.}

proc sum(a: openArray[int]): int =
  for i in 0..a.len:
    result += a[i]

{.pop.}

echo sum([1, 2, 3])
\end{verbatim}

This program contains a famous "index out of bounds" bug. DrNim detects
it and produces the following error message:

\begin{verbatim}
cannot prove: i <= len(a) + -1; counter example: i -> 0 a.len -> 0 [IndexCheck]
\end{verbatim}

In other words for \texttt{i\ ==\ 0} and \texttt{a.len\ ==\ 0} (for
example!) there would be an index out of bounds error.

\hypertarget{pre--postconditions-and-invariants}{%
\subsection{Pre-, postconditions and
invariants}\label{pre--postconditions-and-invariants}}

DrNim adds 4 additional annotations (pragmas) to Nim:

\begin{itemize}
\tightlist
\item
  \texttt{requires}
\item
  \texttt{ensures}
\item
  \texttt{invariant}
\item
  \texttt{assume}
\end{itemize}

These pragmas are ignored by the Nim compiler so that they don't have to
be disabled via \texttt{when\ defined(nimDrNim)}.

\hypertarget{invariant}{%
\subsubsection{Invariant}\label{invariant}}

An \texttt{invariant} is a proposition that must be true after every
loop iteration, it's tied to the loop body it's part of.

\hypertarget{requires}{%
\subsubsection{Requires}\label{requires}}

A \texttt{requires} annotation describes what the function expects to be
true before it's called so that it can perform its operation. A
\texttt{requires} annotation is also called a \texttt{precondition}.

\hypertarget{ensures}{%
\subsubsection{Ensures}\label{ensures}}

An \texttt{ensures} annotation describes what will be true after the
function call. An \texttt{ensures} annotation is also called a
\texttt{postcondition}.

\hypertarget{assume}{%
\subsubsection{Assume}\label{assume}}

An \texttt{assume} annotation describes what DrNim should
\textbf{assume} to be true in this section of the program. It is an
unsafe escape mechanism comparable to Nim's \texttt{cast} statement. Use
it only when you really know better than DrNim. You should add a comment
to a paper that proves the proposition you assume.

\hypertarget{example-insertionsort}{%
\subsection{Example: insertionSort}\label{example-insertionsort}}

\textbf{Note}: This example does not yet work with DrNim.

\begin{verbatim}
import std / logic

proc insertionSort(a: var openArray[int]) {.
    ensures: forall(i in 1..<a.len, a[i-1] <= a[i]).} =

  for k in 1 ..< a.len:
    {.invariant: 1 <= k and k <= a.len.}
    {.invariant: forall(j in 1..<k, i in 0..<j, a[i] <= a[j]).}
    var t = k
    while t > 0 and a[t-1] > a[t]:
      {.invariant: k < a.len.}
      {.invariant: 0 <= t and t <= k.}
      {.invariant: forall(j in 1..k, i in 0..<j, j == t or a[i] <= a[j]).}
      swap a[t], a[t-1]
      dec t
\end{verbatim}

Unfortunately the invariants required to prove this code correct take
more code than the imperative instructions. However this effort can be
compensated by the fact that the result needs very little testing. Be
aware though that DrNim only proves that after \texttt{insertionSort}
this condition holds:

\begin{verbatim}
forall(i in 1..<a.len, a[i-1] <= a[i])
\end{verbatim}

This is required, but not sufficient to describe that a \texttt{sort}
operation was performed. For example, the same postcondition is true for
this proc which doesn't sort at all:

\begin{verbatim}
import std / logic

proc insertionSort(a: var openArray[int]) {.
    ensures: forall(i in 1..<a.len, a[i-1] <= a[i]).} =
  # does not sort, overwrites `a`'s contents!
  for i in 0..<a.len: a[i] = i
\end{verbatim}

\hypertarget{syntax-of-propositions}{%
\subsection{Syntax of propositions}\label{syntax-of-propositions}}

The basic syntax is
\texttt{ensures\textbar{}requires\textbar{}invariant:\ \textless{}prop\textgreater{}}.
A \texttt{prop} is either a comparison or a compound:

\begin{verbatim}
prop = nim_bool_expression
     | prop 'and' prop
     | prop 'or' prop
     | prop '->' prop # implication
     | prop '<->' prop
     | 'not' prop
     | '(' prop ')' # you can group props via ()
     | forallProp
     | existsProp

forallProp = 'forall' '(' quantifierList ',' prop ')'
existsProp = 'exists' '(' quantifierList ',' prop ')'

quantifierList = quantifier (',' quantifier)*
quantifier = <new identifier> 'in' nim_iteration_expression
\end{verbatim}

\texttt{nim\_iteration\_expression} here is an ordinary expression of
Nim code that describes an iteration space, for example \texttt{1..4} or
\texttt{1..\textless{}a.len}.

\texttt{nim\_bool\_expression} here is an ordinary expression of Nim
code of type \texttt{bool} like \texttt{a\ ==\ 3} or
\texttt{23\ \textgreater{}\ a.len}.

The supported subset of Nim code that can be used in these expressions
is currently underspecified but \texttt{let} variables, function
parameters and \texttt{result} (which represents the function's final
result) are amenable for verification. The expressions must not have any
side-effects and must terminate.

The operators \texttt{forall}, \texttt{exists},
\texttt{-\textgreater{}}, \texttt{\textless{}-\textgreater{}} have to
imported from \texttt{std\ /\ logic}.
