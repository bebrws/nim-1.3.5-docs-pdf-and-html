\hypertarget{nim-compiler-user-guide}{%
\section{Nim Compiler User Guide}\label{nim-compiler-user-guide}}

\begin{description}
\item[Author]
Andreas Rumpf
\item[Version]
1.3.5
\end{description}

"Look at you, hacker. A pathetic creature of meat and bone, panting and
sweating as you run through my corridors. How can you challenge a
perfect, immortal machine?"

\hypertarget{introduction}{%
\subsection{Introduction}\label{introduction}}

This document describes the usage of the \emph{Nim compiler} on the
different supported platforms. It is not a definition of the Nim
programming language (which is covered in the
\href{manual.html}{manual}).

Nim is free software; it is licensed under the
\href{http://www.opensource.org/licenses/mit-license.php}{MIT License}.

\hypertarget{compiler-usage}{%
\subsection{Compiler Usage}\label{compiler-usage}}

\hypertarget{command-line-switches}{%
\subsubsection{Command line switches}\label{command-line-switches}}

Basic command line switches are:

Usage:

\begin{verbatim}
nim command [options] [projectfile] [arguments]
\end{verbatim}

\begin{description}
\item[Command:]
//compile, c compile project with default code generator (C) //r compile
to \$nimcache/projname, run with {[}arguments{]} using backend specified
by {-\/-backend} (default: c) //doc generate the documentation for
inputfile for backend specified by {-\/-backend} (default: c)
\item[Arguments:]
arguments are passed to the program being run (if -\/-run option is
selected)
\item[Options:]
-p, -\/-path:PATH add path to search paths -d, -\/-define:SYMBOL(:VAL)
define a conditional symbol (Optionally: Define the value for that
symbol, see: "compile time define pragmas") -u, -\/-undef:SYMBOL
undefine a conditional symbol -f, -\/-forceBuild:onoff turn stack
tracing onoff turn line tracing onoff turn support for multi-threading
onoff turn all runtime checks onoff turn assertions onsize Note: use
-d:release for a release build! -\/-debugger:native Use native debugger
(gdb) -\/-app:consolelibGUI appstatic library -r, -\/-run run the
compiled program with given arguments -\/-fullhelp show all command line
switches -h, -\/-help show this help -v, -\/-version show detailed
version information
\end{description}

Note, single letter options that take an argument require a colon. E.g.
-p:PATH.

\begin{center}\rule{0.5\linewidth}{0.5pt}\end{center}

Advanced command line switches are:

\begin{description}
\item[Advanced commands:]
//compileToC, cc compile project with C code generator //compileToCpp,
cpp compile project to C++ code //compileToOC, objc compile project to
Objective C code //js compile project to Javascript //e run a Nimscript
file //rst2html convert a reStructuredText file to HTML use
{-\/-docCmd:skip} to skip compiling snippets //rst2tex convert a
reStructuredText file to TeX //jsondoc extract the documentation to a
json file //ctags create a tags file //buildIndex build an index for the
whole documentation //genDepend generate a DOT file containing the
module dependency graph //dump dump all defined conditionals and search
paths see also: -\/-dump.format:json (useful with: { \textbar{} jq})
//check checks the project for syntax and semantic
\item[Runtime checks (see -x):]
-\/-objChecks:onoff -\/-fieldChecks:onoff -\/-rangeChecks:onoff
-\/-boundChecks:onoff -\/-overflowChecks:onoff -\/-floatChecks:onoff
-\/-nanChecks:onoff -\/-infChecks:onoff -\/-refChecks:onoff (only for
-\/-newruntime)
\item[Advanced options:]
-o:FILE, -\/-out:FILE set the output filename -\/-outdir:DIR set the
path where the output file will be written -\/-usenimcache will use
{outdir=\$\$nimcache}, whichever it resolves to after all options have
been processed -\/-stdout:onoff turn compiler messages coloring onoff
list full paths in messages -w:onlist, -\/-warnings:onlist turn all
warnings onoff turn specific warning X onoffoff or list all available
-\/-hint{[}X{]}:onoff -\/-warningAsError{[}X{]}:onoff
-\/-styleCheck:offerror produce hints or errors for Nim identifiers that
do not adhere to Nim's official style guide
\url{https://nim-lang.org/docs/nep1.html} -\/-showAllMismatches:onoff
compile Nim files only; do not assemble or link -\/-noLinking:onoff do
not generate a main procedure -\/-genScript:onoff generate a '.deps'
file containing the dependencies -\/-os:SYMBOL set the target operating
system (cross-compilation) -\/-cpu:SYMBOL set the target processor
(cross-compilation) -\/-debuginfo:oncppobjc sets backend to use with
commands like {nim doc} or {nim r} -\/-docCmd:cmd if {cmd == skip},
skips runnableExamples else, runs runnableExamples with given options,
eg: {-\/-docCmd:"-d:foo -\/-threads:on"} -\/-docSeeSrcUrl:url activate
'see source' for doc and doc2 commands (see doc.item.seesrc in
config/nimdoc.cfg) -\/-docInternal also generate documentation for
non-exported symbols -\/-lineDir:onoff -\/-embedsrc:onoff turn thread
analysis onoff turn thread local storage emulation onoff turn taint mode
onoff turn implicit compile time evaluation onoff turn term rewriting
macros onoff turn multi-methods onoff turn memory tracker onoff turn
support for hot code reloading onoff stack traces use full file paths
-\/-stackTraceMsgs:onoff allow 'nil' for strings/seqs for backwards
compatibility -\/-seqsv2:onoff do not read the nim installation's
configuration file -\/-skipUserCfg:onoff do not read the parent dirs'
configuration files -\/-skipProjCfg:onarcmarkAndSweepgoregions select
the GC to use; default is 'refc' -\/-exceptions:setjmpgoto select the
exception handling implementation -\/-index:onoff -\/-putenv:key=value
set an environment variable -\/-NimblePath:PATH add a path for Nimble
support -\/-noNimblePath deactivate the Nimble path -\/-clearNimblePath
empty the list of Nimble package search paths
-\/-cppCompileToNamespace:namespace use the provided namespace for the
generated C++ code, if no namespace is provided "Nim" will be used
-\/-expandMacro:MACRO dump every generated AST from MACRO
-\/-expandArc:PROCNAME show how PROCNAME looks like after diverse
optimizations before the final backend phase (mostly ARC/ORC specific)
-\/-excludePath:PATH exclude a path from the list of search paths
-\/-dynlibOverride:SYMBOL marks SYMBOL so that dynlib:SYMBOL has no
effect and can be statically linked instead; symbol matching is fuzzy so
that -\/-dynlibOverride:lua matches dynlib: "liblua.so.3"
-\/-dynlibOverrideAll disables the effects of the dynlib pragma
-\/-listCmd list the compilation commands; can be combined with
{-\/-hint:exec:on} and {-\/-hint:link:on} -\/-asm produce assembler code
-\/-parallelBuild:0... perform a parallel build value = number of
processors (0 for auto-detect) -\/-incremental:on13 set Nim's verbosity
level (1 is default) -\/-errorMax:N stop compilation after N errors; 0
means unlimited -\/-maxLoopIterationsVM:N set max iterations for all VM
loops -\/-experimental:\$1 enable experimental language feature
-\/-legacy:\$2 enable obsolete/legacy language feature
-\/-useVersion:1.0 emulate Nim version X of the Nim compiler
-\/-profiler:onoff enable benchmarking of VM code with cpuTime()
-\/-profileVM:onoff en-/disable sink parameter inference (default: on)
-\/-panics:on\textbar off turn panics into process terminations
(default: off)
\end{description}

\hypertarget{list-of-warnings}{%
\subsubsection{List of warnings}\label{list-of-warnings}}

Each warning can be activated individually with
\texttt{-\/-warning{[}NAME{]}:on\textbar{}off} or in a \texttt{push}
pragma.

\begin{longtable}[]{@{}ll@{}}
\toprule
Name & Description\tabularnewline
\midrule
\endhead
CannotOpenFile & Some file not essential for the compiler's working
could not be opened.\tabularnewline
OctalEscape & The code contains an unsupported octal
sequence.\tabularnewline
Deprecated & The code uses a deprecated symbol.\tabularnewline
ConfigDeprecated & The project makes use of a deprecated config
file.\tabularnewline
SmallLshouldNotBeUsed & The letter 'l' should not be used as an
identifier.\tabularnewline
EachIdentIsTuple & The code contains a confusing \texttt{var}
declaration.\tabularnewline
User & Some user defined warning.\tabularnewline
\bottomrule
\end{longtable}

\hypertarget{list-of-hints}{%
\subsubsection{List of hints}\label{list-of-hints}}

Each hint can be activated individually with
\texttt{-\/-hint{[}NAME{]}:on\textbar{}off} or in a \texttt{push}
pragma.

\begin{longtable}[]{@{}ll@{}}
\toprule
Name & Description\tabularnewline
\midrule
\endhead
CC CodeBegin CodeEnd CondTrue & Shows when the C compiler is
called.\tabularnewline
Conf ConvToBaseNotNeeded ConvFromXtoItselfNotNeeded Dependency & A
config file was loaded.\tabularnewline
Exec ExprAlwaysX ExtendedContext & Program is executed.\tabularnewline
GCStats & Dumps statistics about the Garbage Collector.\tabularnewline
GlobalVar & Shows global variables declarations.\tabularnewline
LineTooLong & Line exceeds the maximum length.\tabularnewline
Link Name & Linking phase.\tabularnewline
Path Pattern Performance & Search paths modifications.\tabularnewline
Processing QuitCalled & Artifact being compiled.\tabularnewline
\begin{minipage}[t]{0.47\columnwidth}\raggedright
Source

StackTrace\strut
\end{minipage} & \begin{minipage}[t]{0.47\columnwidth}\raggedright
The source line that triggered a diagnostic message.\strut
\end{minipage}\tabularnewline
Success, SuccessX User UserRaw & Successful compilation of a library or
a binary.\tabularnewline
XDeclaredButNotUsed & Unused symbols in the code.\tabularnewline
\bottomrule
\end{longtable}

\hypertarget{verbosity-levels}{%
\subsubsection{Verbosity levels}\label{verbosity-levels}}

\begin{longtable}[]{@{}ll@{}}
\toprule
Level & Description\tabularnewline
\midrule
\endhead
0 & Minimal output level for the compiler.\tabularnewline
1 & Displays compilation of all the compiled files, including those
imported by other modules or through the
\href{manual.html\#implementation-specific-pragmas-compile-pragma}{compile
pragma}. This is the default level.\tabularnewline
2 & Displays compilation statistics, enumerates the dynamic libraries
that will be loaded by the final binary and dumps to standard output the
result of applying \href{filters.html}{a filter to the source code} if
any filter was used during compilation.\tabularnewline
3 & In addition to the previous levels dumps a debug stack trace for
compiler developers.\tabularnewline
\bottomrule
\end{longtable}

\hypertarget{compile-time-symbols}{%
\subsubsection{Compile time symbols}\label{compile-time-symbols}}

Through the \texttt{-d:x} or \texttt{-\/-define:x} switch you can define
compile time symbols for conditional compilation. The defined switches
can be checked in source code with the
\href{manual.html\#statements-and-expressions-when-statement}{when
statement} and \href{system.html\#defined,untyped}{defined proc}. The
typical use of this switch is to enable builds in release mode
(\texttt{-d:release}) where optimizations are enabled for better
performance. Another common use is the \texttt{-d:ssl} switch to
activate SSL sockets.

Additionally, you may pass a value along with the symbol:
\texttt{-d:x=y} which may be used in conjunction with the
\href{manual.html\#implementation-specific-pragmas-compile-time-define-pragmas}{compile
time define pragmas} to override symbols during build time.

Compile time symbols are completely \textbf{case insensitive} and
underscores are ignored too. \texttt{-\/-define:FOO} and
\texttt{-\/-define:foo} are identical.

Compile time symbols starting with the \texttt{nim} prefix are reserved
for the implementation and should not be used elsewhere.

\hypertarget{configuration-files}{%
\subsubsection{Configuration files}\label{configuration-files}}

\textbf{Note:} The \emph{project file name} is the name of the
\texttt{.nim} file that is passed as a command line argument to the
compiler.

The \texttt{nim} executable processes configuration files in the
following directories (in this order; later files overwrite previous
settings):

\begin{enumerate}
\def\labelenumi{\arabic{enumi})}
\tightlist
\item
  \texttt{\$nim/config/nim.cfg}, \texttt{/etc/nim/nim.cfg} (UNIX) or
  \texttt{\textless{}Nim\textquotesingle{}s\ installation\ directory\textgreater{}\textbackslash{}config\textbackslash{}nim.cfg}
  (Windows). This file can be skipped with the \texttt{-\/-skipCfg}
  command line option.
\item
  If environment variable \texttt{XDG\_CONFIG\_HOME} is defined,
  \texttt{\$XDG\_CONFIG\_HOME/nim/nim.cfg} or
  \texttt{\textasciitilde{}/.config/nim/nim.cfg} (POSIX) or
  \texttt{\%APPDATA\%/nim/nim.cfg} (Windows). This file can be skipped
  with the \texttt{-\/-skipUserCfg} command line option.
\item
  \texttt{\$parentDir/nim.cfg} where \texttt{\$parentDir} stands for any
  parent directory of the project file's path. These files can be
  skipped with the \texttt{-\/-skipParentCfg} command line option.
\item
  \texttt{\$projectDir/nim.cfg} where \texttt{\$projectDir} stands for
  the project file's path. This file can be skipped with the
  \texttt{-\/-skipProjCfg} command line option.
\item
  A project can also have a project specific configuration file named
  \texttt{\$project.nim.cfg} that resides in the same directory as
  \texttt{\$project.nim}. This file can be skipped with the
  \texttt{-\/-skipProjCfg} command line option.
\end{enumerate}

Command line settings have priority over configuration file settings.

The default build of a project is a \texttt{debug\ build}. To compile a
\texttt{release\ build} define the \texttt{release} symbol:

\begin{verbatim}
nim c -d:release myproject.nim

To compile a `dangerous release build`:idx: define the ``danger`` symbol::

nim c -d:danger myproject.nim
\end{verbatim}

\hypertarget{search-path-handling}{%
\subsubsection{Search path handling}\label{search-path-handling}}

Nim has the concept of a global search path (PATH) that is queried to
determine where to find imported modules or include files. If multiple
files are found an ambiguity error is produced.

\texttt{nim\ dump} shows the contents of the PATH.

However before the PATH is used the current directory is checked for the
file's existence. So if PATH contains \texttt{\$lib} and
\texttt{\$lib/bar} and the directory structure looks like this:

\begin{verbatim}
$lib/x.nim
$lib/bar/x.nim
foo/x.nim
foo/main.nim
other.nim
\end{verbatim}

And \texttt{main} imports \texttt{x}, \texttt{foo/x} is imported. If
\texttt{other} imports \texttt{x} then both \texttt{\$lib/x.nim} and
\texttt{\$lib/bar/x.nim} match but \texttt{\$lib/x.nim} is used as it is
the first match.

\hypertarget{generated-c-code-directory}{%
\subsubsection{Generated C code
directory}\label{generated-c-code-directory}}

The generated files that Nim produces all go into a subdirectory called
\texttt{nimcache}. Its full path is

\begin{itemize}
\tightlist
\item
  \texttt{\$XDG\_CACHE\_HOME/nim/\$projectname(\_r\textbar{}\_d)} or
  \texttt{\textasciitilde{}/.cache/nim/\$projectname(\_r\textbar{}\_d)}
  on Posix
\item
  \texttt{\$HOME/nimcache/\$projectname(\_r\textbar{}\_d)} on Windows.
\end{itemize}

The \texttt{\_r} suffix is used for release builds, \texttt{\_d} is for
debug builds.

This makes it easy to delete all generated files.

The \texttt{-\/-nimcache}
\protect\hyperlink{compiler-usage-command-line-switches}{compiler
switch} can be used to to change the \texttt{nimcache} directory.

However, the generated C code is not platform independent. C code
generated for Linux does not compile on Windows, for instance. The
comment on top of the C file lists the OS, CPU and CC the file has been
compiled for.

\hypertarget{compiler-selection}{%
\subsection{Compiler Selection}\label{compiler-selection}}

To change the compiler from the default compiler (at the command line):

\begin{verbatim}
nim c --cc:llvm_gcc --compile_only myfile.nim
\end{verbatim}

This uses the configuration defined in
\texttt{config\textbackslash{}nim.cfg} for \texttt{lvm\_gcc}.

If nimcache already contains compiled code from a different compiler for
the same project, add the \texttt{-f} flag to force all files to be
recompiled.

The default compiler is defined at the top of
\texttt{config\textbackslash{}nim.cfg}. Changing this setting affects
the compiler used by \texttt{koch} to (re)build Nim.

To use the \texttt{CC} environment variable, use
\texttt{nim\ c\ -\/-cc:env\ myfile.nim}. To use the \texttt{CXX}
environment variable, use \texttt{nim\ cpp\ -\/-cc:env\ myfile.nim}.
\texttt{-\/-cc:env} is available since Nim version 1.4.

\hypertarget{cross-compilation}{%
\subsection{Cross compilation}\label{cross-compilation}}

To cross compile, use for example:

\begin{verbatim}
nim c --cpu:i386 --os:linux --compileOnly --genScript myproject.nim
\end{verbatim}

Then move the C code and the compile script
\texttt{compile\_myproject.sh} to your Linux i386 machine and run the
script.

Another way is to make Nim invoke a cross compiler toolchain:

\begin{verbatim}
nim c --cpu:arm --os:linux myproject.nim
\end{verbatim}

For cross compilation, the compiler invokes a C compiler named like
\texttt{\$cpu.\$os.\$cc} (for example arm.linux.gcc) and the
configuration system is used to provide meaningful defaults. For example
for \texttt{ARM} your configuration file should contain something like:

\begin{verbatim}
arm.linux.gcc.path = "/usr/bin"
arm.linux.gcc.exe = "arm-linux-gcc"
arm.linux.gcc.linkerexe = "arm-linux-gcc"
\end{verbatim}

\hypertarget{cross-compilation-for-windows}{%
\subsection{Cross compilation for
Windows}\label{cross-compilation-for-windows}}

To cross compile for Windows from Linux or macOS using the MinGW-w64
toolchain:

\begin{verbatim}
nim c -d:mingw myproject.nim
\end{verbatim}

Use \texttt{-\/-cpu:i386} or \texttt{-\/-cpu:amd64} to switch the CPU
architecture.

The MinGW-w64 toolchain can be installed as follows:

\begin{verbatim}
Ubuntu: apt install mingw-w64
CentOS: yum install mingw32-gcc | mingw64-gcc - requires EPEL
OSX: brew install mingw-w64
\end{verbatim}

\hypertarget{cross-compilation-for-android}{%
\subsection{Cross compilation for
Android}\label{cross-compilation-for-android}}

There are two ways to compile for Android: terminal programs (Termux)
and with the NDK (Android Native Development Kit).

First one is to treat Android as a simple Linux and use
\href{https://wiki.termux.com}{Termux} to connect and run the Nim
compiler directly on android as if it was Linux. These programs are
console only programs that can't be distributed in the Play Store.

Use regular \texttt{nim\ c} inside termux to make Android terminal
programs.

Normal Android apps are written in Java, to use Nim inside an Android
app you need a small Java stub that calls out to a native library
written in Nim using the \href{https://developer.android.com/ndk}{NDK}.
You can also use
\href{https://developer.android.com/ndk/samples/sample_na}{native-acitivty}
to have the Java stub be auto generated for you.

Use
\texttt{nim\ c\ -c\ -\/-cpu:arm\ -\/-os:android\ -d:androidNDK\ -\/-noMain:on}
to generate the C source files you need to include in your Android
Studio project. Add the generated C files to CMake build script in your
Android project. Then do the final compile with Android Studio which
uses Gradle to call CMake to compile the project.

Because Nim is part of a library it can't have its own c style
\texttt{main()} so you would need to define your own
\texttt{android\_main} and init the Java environment, or use a library
like SDL2 or GLFM to do it. After the Android stuff is done, it's very
important to call \texttt{NimMain()} in order to initialize Nim's
garbage collector and to run the top level statements of your program.

\begin{verbatim}
proc NimMain() {.importc.}
proc glfmMain*(display: ptr GLFMDisplay) {.exportc.} =
  NimMain() # initialize garbage collector memory, types and stack
\end{verbatim}

\hypertarget{cross-compilation-for-ios}{%
\subsection{Cross compilation for iOS}\label{cross-compilation-for-ios}}

To cross compile for iOS you need to be on a MacOS computer and use
XCode. Normal languages for iOS development are Swift and Objective C.
Both of these use LLVM and can be compiled into object files linked
together with C, C++ or Objective C code produced by Nim.

Use \texttt{nim\ c\ -c\ -\/-os:ios\ -\/-noMain:on} to generate C files
and include them in your XCode project. Then you can use XCode to
compile, link, package and sign everything.

Because Nim is part of a library it can't have its own c style
\texttt{main()} so you would need to define {main} that calls
\texttt{autoreleasepool} and \texttt{UIApplicationMain} to do it, or use
a library like SDL2 or GLFM. After the iOS setup is done, it's very
important to call \texttt{NimMain()} in order to initialize Nim's
garbage collector and to run the top level statements of your program.

\begin{verbatim}
proc NimMain() {.importc.}
proc glfmMain*(display: ptr GLFMDisplay) {.exportc.} =
  NimMain() # initialize garbage collector memory, types and stack
\end{verbatim}

Note: XCode's "make clean" gets confused about the generated nim.c
files, so you need to clean those files manually to do a clean build.

\hypertarget{cross-compilation-for-nintendo-switch}{%
\subsection{Cross compilation for Nintendo
Switch}\label{cross-compilation-for-nintendo-switch}}

Simply add -\/-os:nintendoswitch to your usual \texttt{nim\ c} or
\texttt{nim\ cpp} command and set the \texttt{passC} and \texttt{passL}
command line switches to something like:

\begin{verbatim}
\end{verbatim}

or setup a nim.cfg file like so:

\begin{verbatim}
\end{verbatim}

The DevkitPro setup must be the same as the default with their new
installer \href{https://github.com/devkitPro/pacman/releases}{here for
Mac/Linux} or
\href{https://github.com/devkitPro/installer/releases}{here for
Windows}.

For example, with the above mentioned config:

\begin{verbatim}
nim c --os:nintendoswitch switchhomebrew.nim
\end{verbatim}

This will generate a file called \texttt{switchhomebrew.elf} which can
then be turned into an nro file with the \texttt{elf2nro} tool in the
DevkitPro release. Examples can be found at
\href{https://github.com/jyapayne/nim-libnx.git}{the nim-libnx github
repo}.

There are a few things that don't work because the DevkitPro libraries
don't support them. They are:

\begin{enumerate}
\def\labelenumi{\arabic{enumi}.}
\tightlist
\item
  Waiting for a subprocess to finish. A subprocess can be started, but
  right now it can't be waited on, which sort of makes subprocesses a
  bit hard to use
\item
  Dynamic calls. DevkitPro libraries have no dlopen/dlclose functions.
\item
  Command line parameters. It doesn't make sense to have these for a
  console anyways, so no big deal here.
\item
  mqueue. Sadly there are no mqueue headers.
\item
  ucontext. No headers for these either. No coroutines for now :(
\item
  nl\_types. No headers for this.
\end{enumerate}

\hypertarget{dll-generation}{%
\subsection{DLL generation}\label{dll-generation}}

Nim supports the generation of DLLs. However, there must be only one
instance of the GC per process/address space. This instance is contained
in \texttt{nimrtl.dll}. This means that every generated Nim DLL depends
on \texttt{nimrtl.dll}. To generate the "nimrtl.dll" file, use the
command:

\begin{verbatim}
nim c -d:release lib/nimrtl.nim
\end{verbatim}

To link against \texttt{nimrtl.dll} use the command:

\begin{verbatim}
nim c -d:useNimRtl myprog.nim
\end{verbatim}

\textbf{Note}: Currently the creation of \texttt{nimrtl.dll} with thread
support has never been tested and is unlikely to work!

\hypertarget{additional-compilation-switches}{%
\subsection{Additional compilation
switches}\label{additional-compilation-switches}}

The standard library supports a growing number of \texttt{useX}
conditional defines affecting how some features are implemented. This
section tries to give a complete list.

\begin{longtable}[]{@{}ll@{}}
\toprule
Define & Effect\tabularnewline
\midrule
\endhead
\texttt{release} & Turns on the optimizer. More aggressive optimizations
are possible, eg: \texttt{-\/-passC:-ffast-math} (but see issue
\#10305)\tabularnewline
\texttt{danger} & Turns off all runtime checks and turns on the
optimizer.\tabularnewline
\texttt{useFork} & Makes \texttt{osproc} use \texttt{fork} instead of
\texttt{posix\_spawn}.\tabularnewline
\texttt{useNimRtl} & Compile and link against
\texttt{nimrtl.dll}.\tabularnewline
\texttt{useMalloc} & Makes Nim use C's \texttt{malloc} instead of Nim's
own memory manager, albeit prefixing each allocation with its size to
support clearing memory on reallocation. This only works with
\texttt{gc:none} and with \texttt{-\/-newruntime}.\tabularnewline
\texttt{useRealtimeGC} & Enables support of Nim's GC for \emph{soft}
realtime systems. See the documentation of the \href{gc.html}{gc} for
further information.\tabularnewline
\texttt{logGC} & Enable GC logging to stdout.\tabularnewline
\texttt{nodejs} & The JS target is actually
\texttt{node.js}.\tabularnewline
\texttt{ssl} & Enables OpenSSL support for the sockets
module.\tabularnewline
\texttt{memProfiler} & Enables memory profiling for the native
GC.\tabularnewline
\texttt{uClibc} & Use uClibc instead of libc. (Relevant for Unix-like
OSes)\tabularnewline
\texttt{checkAbi} & When using types from C headers, add checks that
compare what's in the Nim file with what's in the C header. This may
become enabled by default in the future.\tabularnewline
\texttt{tempDir} & This symbol takes a string as its value, like
\texttt{-\/-define:tempDir:/some/temp/path} to override the temporary
directory returned by \texttt{os.getTempDir()}. The value
\textbf{should} end with a directory separator character. (Relevant for
the Android platform)\tabularnewline
\texttt{useShPath} & This symbol takes a string as its value, like
\texttt{-\/-define:useShPath:/opt/sh/bin/sh} to override the path for
the \texttt{sh} binary, in cases where it is not located in the default
location \texttt{/bin/sh}.\tabularnewline
\texttt{noSignalHandler} & Disable the crash handler from
\texttt{system.nim}.\tabularnewline
\texttt{globalSymbols} & Load all \texttt{\{.dynlib.\}} libraries with
the \texttt{RTLD\_GLOBAL} flag on Posix systems to resolve symbols in
subsequently loaded libraries.\tabularnewline
\bottomrule
\end{longtable}

\hypertarget{additional-features}{%
\subsection{Additional Features}\label{additional-features}}

This section describes Nim's additional features that are not listed in
the Nim manual. Some of the features here only make sense for the C code
generator and are subject to change.

\hypertarget{linedir-option}{%
\subsubsection{LineDir option}\label{linedir-option}}

The \texttt{lineDir} option can be turned on or off. If turned on the
generated C code contains \texttt{\#line} directives. This may be
helpful for debugging with GDB.

\hypertarget{stacktrace-option}{%
\subsubsection{StackTrace option}\label{stacktrace-option}}

If the \texttt{stackTrace} option is turned on, the generated C contains
code to ensure that proper stack traces are given if the program crashes
or an uncaught exception is raised.

\hypertarget{linetrace-option}{%
\subsubsection{LineTrace option}\label{linetrace-option}}

The \texttt{lineTrace} option implies the \texttt{stackTrace} option. If
turned on, the generated C contains code to ensure that proper stack
traces with line number information are given if the program crashes or
an uncaught exception is raised.

\hypertarget{dynliboverride}{%
\subsection{DynlibOverride}\label{dynliboverride}}

By default Nim's \texttt{dynlib} pragma causes the compiler to generate
\texttt{GetProcAddress} (or their Unix counterparts) calls to bind to a
DLL. With the \texttt{dynlibOverride} command line switch this can be
prevented and then via \texttt{-\/-passL} the static library can be
linked against. For instance, to link statically against Lua this
command might work on Linux:

\begin{verbatim}
nim c --dynlibOverride:lua --passL:liblua.lib program.nim
\end{verbatim}

\hypertarget{backend-language-options}{%
\subsection{Backend language options}\label{backend-language-options}}

The typical compiler usage involves using the \texttt{compile} or
\texttt{c} command to transform a \texttt{.nim} file into one or more
\texttt{.c} files which are then compiled with the platform's C compiler
into a static binary. However there are other commands to compile to
C++, Objective-C or JavaScript. More details can be read in the
\href{backends.html}{Nim Backend Integration document}.

\hypertarget{nim-documentation-tools}{%
\subsection{Nim documentation tools}\label{nim-documentation-tools}}

Nim provides the \texttt{doc} and \texttt{doc2} commands to generate
HTML documentation from \texttt{.nim} source files. Only exported
symbols will appear in the output. For more details
\href{docgen.html}{see the docgen documentation}.

\hypertarget{nim-idetools-integration}{%
\subsection{Nim idetools integration}\label{nim-idetools-integration}}

Nim provides language integration with external IDEs through the
idetools command. See the documentation of
\href{idetools.html}{idetools} for further information.

\hypertarget{nim-for-embedded-systems}{%
\subsection{Nim for embedded systems}\label{nim-for-embedded-systems}}

While the default Nim configuration is targeted for optimal performance
on modern PC hardware and operating systems with ample memory, it is
very well possible to run Nim code and a good part of the Nim standard
libraries on small embedded microprocessors with only a few kilobytes of
memory.

A good start is to use the \texttt{any} operating target together with
the \texttt{malloc} memory allocator and the \texttt{arc} garbage
collector. For example:

\texttt{nim\ c\ -\/-os:any\ -\/-gc:arc\ -d:useMalloc\ {[}...{]}\ x.nim}

\begin{itemize}
\tightlist
\item
  \texttt{-\/-gc:arc} will enable the reference counting memory
  management instead of the default garbage collector. This enables Nim
  to use heap memory which is required for strings and seqs, for
  example.
\item
  The \texttt{-\/-os:any} target makes sure Nim does not depend on any
  specific operating system primitives. Your platform should support
  only some basic ANSI C library \texttt{stdlib} and \texttt{stdio}
  functions which should be available on almost any platform.
\item
  The \texttt{-d:useMalloc} option configures Nim to use only the
  standard C memory manage primitives \texttt{malloc()},
  \texttt{free()}, \texttt{realloc()}.
\end{itemize}

If your platform does not provide these functions it should be trivial
to provide an implementation for them and link these to your program.

For targets with very restricted memory, it might be beneficial to pass
some additional flags to both the Nim compiler and the C compiler and/or
linker to optimize the build for size. For example, the following flags
can be used when targeting a gcc compiler:

\texttt{-\/-opt:size\ -\/-passC:-flto\ -\/-passL:-flto}

The \texttt{-\/-opt:size} flag instructs Nim to optimize code generation
for small size (with the help of the C compiler), the \texttt{flto}
flags enable link-time optimization in the compiler and linker.

Check the {Cross compilation} section for instructions how to compile
the program for your target.

\hypertarget{nim-for-realtime-systems}{%
\subsection{Nim for realtime systems}\label{nim-for-realtime-systems}}

See the documentation of Nim's soft realtime \href{gc.html}{GC} for
further information.

\hypertarget{signal-handling-in-nim}{%
\subsection{Signal handling in Nim}\label{signal-handling-in-nim}}

The Nim programming language has no concept of Posix's signal handling
mechanisms. However, the standard library offers some rudimentary
support for signal handling, in particular, segmentation faults are
turned into fatal errors that produce a stack trace. This can be
disabled with the \texttt{-d:noSignalHandler} switch.

\hypertarget{optimizing-for-nim}{%
\subsection{Optimizing for Nim}\label{optimizing-for-nim}}

Nim has no separate optimizer, but the C code that is produced is very
efficient. Most C compilers have excellent optimizers, so usually it is
not needed to optimize one's code. Nim has been designed to encourage
efficient code: The most readable code in Nim is often the most
efficient too.

However, sometimes one has to optimize. Do it in the following order:

\begin{enumerate}
\def\labelenumi{\arabic{enumi}.}
\tightlist
\item
  switch off the embedded debugger (it is \textbf{slow}!)
\item
  turn on the optimizer and turn off runtime checks
\item
  profile your code to find where the bottlenecks are
\item
  try to find a better algorithm
\item
  do low-level optimizations
\end{enumerate}

This section can only help you with the last item.

\hypertarget{optimizing-string-handling}{%
\subsubsection{Optimizing string
handling}\label{optimizing-string-handling}}

String assignments are sometimes expensive in Nim: They are required to
copy the whole string. However, the compiler is often smart enough to
not copy strings. Due to the argument passing semantics, strings are
never copied when passed to subroutines. The compiler does not copy
strings that are a result from a procedure call, because the callee
returns a new string anyway. Thus it is efficient to do:

\begin{verbatim}
\end{verbatim}

However it is not efficient to do:

\begin{verbatim}
\end{verbatim}

For \texttt{let} symbols a copy is not always necessary:

\begin{verbatim}
\end{verbatim}

If you know what you're doing, you can also mark single string (or
sequence) objects as \texttt{shallow}:

\begin{verbatim}
\end{verbatim}

Usage of \texttt{shallow} is always safe once you know the string won't
be modified anymore, similar to Ruby's \texttt{freeze}.

The compiler optimizes string case statements: A hashing scheme is used
for them if several different string constants are used. So code like
this is reasonably efficient:

\begin{verbatim}
\end{verbatim}
